$if(has-frontmatter)$
\frontmatter
$endif$

$if(title)$
\maketitle
$endif$

\clearpage
\newpage
\thispagestyle{empty} % Hide header and footer on this page
\mbox{~}
\clearpage
\newpage

%----------------------------------------------
%   Abstract
%----------------------------------------------

\thispagestyle{plain}

\begin{flushleft}
% Manually add a section to the table of contents
\pagenumbering{roman}
\addcontentsline{toc}{chapter}{Abstract}
\huge\textbf{Abstract}
\end{flushleft}

\vspace*{\baselineskip}

The ability to detect volatile organic compounds in a highly sensitive and selective manner is desirable for applications as varied as diagnoses of illnesses at a remote clinic, monitoring of air in an industrial setting, or identification of invasive organisms at a biosecurity checkpoint. Historically, animal noses have been used for such tasks, as their combined sensitivity and selectivity are superior to traditional artificial sensors. However, training and deploying animals in such situations is both time and cost intensive. In recent years, an improved understanding of \textit{in vivo} biological sensing has driven efforts to mimic these highly efficient processes in an artificial sensor format. \\[5pt] To this end, a "bioelectronic nose" was developed. This sensor uses an artificial transducer to amplify responses of an insect odorant receptor protein to specific volatile compounds. Thin-film transistors were used as the amplifier element, given their low cost, small size and extreme sensitivity. Various thin-film morphologies were compared, and their suitability for bioelectronic nose development assessed. Transducers made using a novel steam-assisted thin-film deposition technique were found to have highly consistent device-to-device electrical properties relative to other films. Films made using this process typically showed more surface contamination than other morphologies, but their high sensitivity was confirmed with a non-specific sensing series in an aqueous environment. \\[5pt] One of the major challenges encountered in this thesis was variability in the quality of sensor functionalisation. Raman spectroscopy and fluorescence microscopy confirmed an existing non-covalent attachment method could successfully immobilise nanodiscs onto the transistor channel region. However, various sensors functionalised using the same procedure often exhibited no sensing activity. Extensive electrical characterisation indicated the presence of an unidentified contamination layer which prevented electrical interaction between the insect odorant receptors and the transducer thin-film. It was shown that this layer was unlikely to be directly associated with the thin-film morphology used for the transducer. \\[5pt] Subsequently, an alternative biotin-based non-covalent method was used for functionalisation of the proteins, which eliminated several possible sources of contamination. This alternative biotin-based method was used to demonstrate successful aqueous sensing of femtomolar concentrations of methyl salicylate by an iOR10a-functionalised device. When tested in a custom-built vapour delivery system, a similar bioelectronic sensor was shown to be highly sensitive to the target vapour. However, consistent reproduction of the biotin-based method was challenging due to the harsh cleaning method involved. It was therefore difficult to determine conclusively whether the vapour-phase sensor responses were selective. By finding new, systematic approaches to address the barriers to sensor success carefully identified in this work, there are promising signs that a highly reliable vapour-phase bioelectronic nose can be produced.

%\fancyhf{} %clear all headers and footers fields
%\thispagestyle{fancy} % Change header and footer on this page
%\renewcommand{\headrulewidth}{0pt}
%\fancyhead[L]{\textit{Abstract}} % Set header content
%\fancyfoot[L]{\thepage} %prints the page number on the right side of the header

\clearpage
\newpage
\thispagestyle{empty} % Hide header and footer on this page
\mbox{~}
\clearpage
\newpage


%----------------------------------------------
%   Acknowledgement
%----------------------------------------------

\thispagestyle{plain}

\begin{flushleft}
% Manually add a section to the table of contents
\addcontentsline{toc}{chapter}{Acknowledgements}
\huge\textbf{Acknowledgements}
\end{flushleft}

\vspace*{\baselineskip}

I would first like to acknowledge the lands of my ancestors, and the lands of the sovereign first peoples to which my ancestors travelled. We each come from the land, live off the land and return to the land.\\[5pt]
\textit{Noon of Essex to Warrang, on the Friends, Autumn 1811} \\[5pt]
\textit{Cave of Cambridgeshire to Warrang, on the Royal Charlotte, Autumn 1825} \\[5pt]
\textit{Boyce of Suffolk to Warrang, 1832} \\[5pt] 
\textit{Charlton of Northumberland to Warrang, on the Clyde, Spring 1834} \\[5pt]
\textit{Prouse of Devonshire to Pito-one, on the Duke of Roxburgh, Summer 1840} \\[5pt]
\textit{Ebden of Devonshire to Pito-one, on the Tyne, Winter 1841} \\[5pt]
\textit{Collis of Hampshire to Pito-one, on the Birman, Autumn 1842} \\[5pt]
\textit{Swann of Loch Garman to Te Whanganui-a-Tara, 1844} \\[5pt] 
\textit{Blythe of Berkshire to Whakatū, circa 1846} \\[5pt]
\textit{Innes of Berkshire to Naarm, on the Sacramento, Autumn 1853} \\[5pt]
\textit{Sheppard of Gloucestershire to Naarm, 1853} \\[5pt] 
\textit{Bruce of London to Naarm, on the Omega, Autumn 1855} \\[5pt]
\textit{Quennell of Surrey to Warrang, on the Asiatic, Winter 1855} \\[5pt]
\textit{Barr of Glasgow to Kōpūtai, on the Sir Edward Paget, Winter 1856} \\[5pt] 
\textit{Perkins of London to Te Whanganui-a-Tara, on the Matoaka, Spring 1859} \\[5pt]
\textit{McKee of Antrim to Tāmaki Makaurau, on the Indian Empire, Spring 1862} \\[5pt]
\textit{Sandilands of Peeblesshire to Ōtepoti, circa 1864} \\[5pt] 
\textit{Treacher of Berkshire to Te Whanganui-a-Tara, on the Wild Duck, Summer 1865} \\[5pt]
\textit{McTaggart of Argyllshire to Kōpūtai, on the Edward P. Bouverie, Autumn 1869} \\[5pt] 
\textit{Chapman of Kent to Whakatū, on the Adamant, Winter 1874} \\[5pt]
\textit{Cheel of London to Whakatū, on the Queen Bee, Winter 1877} \\[5pt]  
\textit{Hutchison of Aberdeen to Tarntanya, before 1882.} \\[5pt] 
I chose to start my doctoral studies just a few months into a global pandemic. Completing a challenging project with a worldwide crisis in the background might have been impossible without the supervision of AProf. Natalie Plank. Her ability to adapt to and overcome any problem has taught me that there is no situation which is truly unmanageable. I am deeply grateful for her leadership throughout a time of particular chaos. \newpage
\fancyhf{} %clear all headers and footers fields
\thispagestyle{fancy} % Change header and footer on this page
\renewcommand{\headrulewidth}{0pt}
\fancyhead[L]{\textit{Acknowledgements}} % Set header content
\fancyfoot[L]{\thepage} %prints the page number on the left side of the header 
I started this project with minimal formal training in biological science, coming from a primarily physics and engineering background. \\[5pt] The immense support I received from Melissa Jordan and Colm Carraher from the Institute for Plant and Food Research (PFR) to complete this project meant that this was not an issue, and I thank them both immensely for this. \\[5pt] I would not have been able to begin this thesis without the financial backing and support I received from PFR and the Better Border Biosecurity (B3) programme. In particular, I am very grateful to Andrew Kralicek, formerly with PFR and now at Scentian Bio, and the ex-Director of B3, David Teulon, for helping to secure funding for my project. I would also like to thank the donor of the Ernest Marsden Scholarship in Physics for their significant financial support. \\[5pt] There are many incredibly supportive people who I worked alongside during my project. I would like to start off by thanking Rifat Ullah, whose mentoring and kindness encouraged me to pursue further study. His work on the initial design and setup of the vapour delivery system was invaluable to me throughout this project. I am also especially grateful to Alex Puglisi, for constructing the mechanical elements of the vapour delivery system and giving me extensive feedback on the system design. I would like to thank Peter Coard, for his advice and guidance when constructing the electrical elements of the vapour delivery system. I thank Selvan Murugathas, too, for his advice on constructing the insect odorant receptor sensors, as well as Damon Colbert and Valentina Lucarelli, who provided the insect odorant receptor nanodiscs used in this work. \\[5pt] Thank you to AProf. Ben Ruck, my supportive secondary supervisor, and to AProf. Franck Natali, for always asking about my thesis in the tearoom. Thank you to Gideon Gouws for his friendly encouragement and advice. For their substantial technical assistance and mentoring during this project, I thank Alan Rennie, Grant Franklin, Chris Lepper, Rashika Gunasekara, Pete Jebson and Sushila Pillai from VUW, Andrew Chan from PFR, AProf. Charles Unsworth from the University of Auckland, and Prof. Simon Brown and his nanomaterials group from the University of Canterbury. \\[5pt] I was lucky enough to start my doctoral program just as a group of supportive and talented senior students were finishing, and finished just as a group of enthusiastic and talented new doctoral students were starting. A special thanks to Jenna Nyugen, Erica Happe and Erica Cassie for teaching me the fabrication processes and characterisation procedures that made this thesis happen; and a special thanks to Marissa Dierkes, Danica Fontein, Sangar Begzaad and Alireza Zare, for their incredible support throughout the thesis writing process. I am also thankful for the assistance of the cleanroom group interns over the course of my PhD, including Liam, Hayden and Lotte. I would further like to thank everyone else I shared an office with and worked alongside, including Jackson, Will, Roshni, Ali, Sam, Kira, Catherine, Martin, Janani, Ted, Kiri and Joe. \\[5pt] A massive thank you to Openstar Technologies. It has been an honour to work on a cutting-edge plasma physics project right here in Te Whanganui-a-Tara. A particularly big thank you to Ratu, Darren and Thomas for having me as part of the plasma physics team. Thank you also to the other Openstar interns, in particular the other plasma physics interns, Valentina, Benjy and Chris. I wish you success in all your dipole-confined plasma related endeavours. \\[5pt] I want to thank Shodokan Aikido New Zealand for their support throughout this thesis, in particular for the once-in-a-lifetime opportunity to travel to Osaka to be graded for first-dan by Nariyama Shihan. Thanks for all the training and support, Ian. \\[5pt] Thank you to all the friends and whānau, old and new, who have supported me over these wild past few years. You know who you are. \\[5pt] Thank you to my brother, Keeson, and to my parents, Hilary and Phillip. Your support means everything to me, and I would not be where I am today without you. Our Friday lunchtime cafe visits kept me motivated and inspired throughout the doctoral program. Thank you, thank you, thank you for your love, your compassion, and for being there for me. \\[5pt] Finally, thank you Nina. Your incredible love has kept me going through the most difficult and most wonderful times over the last four years. You are the light of my life, and I am so happy to have taken on this challenge with you by my side. \\[5pt] Arohanui and peace to you all, Eddyn (Ned)

\fancyhf{} %clear all headers and footers fields
\thispagestyle{fancy} % Change header and footer on this page
\renewcommand{\headrulewidth}{0pt}
\fancyhead[R]{\textit{Acknowledgements}} % Set header content
\fancyfoot[R]{\thepage} %prints the page number on the right side of the header

\clearpage
\newpage
\thispagestyle{empty} % Hide header and footer on this page
\mbox{~}
\clearpage
\newpage

\pagestyle{headings}


