$if(has-frontmatter)$
\frontmatter
$endif$

$if(title)$
\maketitle
$endif$

\clearpage
\newpage
\thispagestyle{empty} % Hide header and footer on this page
\mbox{~}
\clearpage
\newpage

%----------------------------------------------
%   Dedication
%----------------------------------------------

\begin{center}
    \thispagestyle{empty}
    \vspace*{\fill}
    \begin{large}
    To Gran, Marjorie Jean, and your love of discovery.
    \end{large}
    \vspace*{\fill}
\end{center}

\clearpage
\newpage
\thispagestyle{empty} % Hide header and footer on this page
\mbox{~}
\clearpage
\newpage

%----------------------------------------------
%   Abstract
%----------------------------------------------

\begin{flushleft}
% Manually add a section to the table of contents
\pagenumbering{roman}
\addcontentsline{toc}{chapter}{Abstract}
\huge\textbf{Abstract}
\end{flushleft}

\vspace*{\baselineskip}

The ability to detect volatile organic compounds in a highly sensitive and selective manner is desirable for a variety of different applications, including diagnosing illnesses at a remote clinic, monitoring air in an industrial workplace, or identifying biomarkers from invasive organisms at a biosecurity checkpoint. This thesis therefore investigates the coupling of insect odorant receptors (iORs) with carbon nanotube network field-effect transistors (CNT FETs) and graphene field-effect transistors (GFETs) to create a sensitive and selective vapour-phase sensor or ‘bioelectronic nose’. \\[5pt] The properties of three distinct carbon nanotube network morphologies were compared to understand their relative suitability as transducer thin films for the bioelectronic nose. The first morphology was obtained by depositing ultrasonicated carbon nanotubes in solvent, the second by depositing carbon nanotubes in surfactant solution, and the third by depositing with surfactant in the presence of steam. Compared to the other morphologies used, the steam-assisted deposition devices were found to have highly consistent device-to-device electrical properties due to their relatively low bundling and dense coverage of the substrate (>70\%). The steam-assisted deposition devices, however, were found to be more highly \textit{p}-doped than the other morphologies, with a relatively large liquid-gated threshold voltage (0.37 V). The sensitivity of the transducer was verified using an aqueous phosphate buffered saline (PBS) dilution series which followed a logarithmic trend. To account for baseline drift due to hysteresis, the device was left at a constant voltage for 30 minutes before the dilution series. By allowing baseline drift to settle over this period, it could be reasonably approximated as linear, estimated using a linear fit and removed from the data. \\[5pt] A vapour delivery system was also adapted for the purpose of testing the bioelectronic nose in a vapour environment. A photoionisation detector (PID) and a relative humidity and temperature indicator (RHI) were added to the existing system, which could be used to non-selectively detect vapours for comparison with measurements taken by the bioelectronic nose. Two new mass flow controllers (MFCs) were also added to the system to allow for a greater range of flow rates. A new electronic control system was also created to integrate the new MFCs and RHI. In the upgraded vapour delivery system, MFC-controlled nitrogen flow bubbles through analyte, carrying vapour into a device characterisation chamber, and then into a manifold leading to the PID and RHI. A steam-assisted deposition CNT FET was found to be sensitive to ppb concentrations of ethyl hexanoate (EtHex) and \textit{trans}-2-hexen-1-al (E2Hex). Baseline drift from hysteresis was found to be significantly larger when backgated in the vapour delivery system, and a 40 minute period was therefore used when waiting for baseline drift to settle. \\[5pt] One of the major challenges encountered in this thesis was the variability in quality of the non-covalent functionalisation method used. 1-pyrenebutanoic acid N-hydroxysuccinimide ester (PBASE) was primarily used as a linker molecule to attach the iORs to carbon nanotubes and graphene. This linker is widely used in the literature, but there is little agreement on the optimal process variables to use. It was verified that PBASE and iOR nanodiscs were successfully attaching to the transistor channel with a method previously used for iOR functionalisation. However, devices functionalised in an identical manner would rarely operate as expected and usually showed no sensing activity. It was found that devices which showed no sensing activity also exhibited no change in transfer characteristics after functionalisation, despite iOR nanodiscs being present. A possible explanation for this behaviour is the presence of a contamination layer which electrically isolates iOR nanodiscs from the transducer material. A device functionalised with pyrene-PEG-biotin (PPB) and avi-tagged OR10a after the use of a destructive oxygen plasma cleaning step responded to low fM concentrations of methyl salicylate (MeSal), further supporting this hypothesis. It was concluded that successful and reproducible vapour sensing using non-covalent functionalisation may require a robust, non-destructive cleaning method for the transducer platform.

\fancyhf{} %clear all headers and footers fields
\thispagestyle{fancy} % Change header and footer on this page
\renewcommand{\headrulewidth}{0pt}
\fancyhead[L]{\textit{Abstract}} % Set header content
\fancyfoot[L]{\thepage} %prints the page number on the right side of the header

\clearpage
\newpage
\thispagestyle{empty} % Hide header and footer on this page
\mbox{~}
\clearpage
\newpage

%----------------------------------------------
%   Acknowledgement
%----------------------------------------------

\thispagestyle{plain}

\begin{flushleft}
% Manually add a section to the table of contents
\addcontentsline{toc}{chapter}{Acknowledgements}
\huge\textbf{Acknowledgements}
\end{flushleft}

\vspace*{\baselineskip}

\begin{normalsize}

    Noon of Essex to Warrane, on the Friends, Autumn 1811 \\[5pt]
    Cave of Cambridgeshire to Warrane, on the Royal Charlotte, Autumn 1825 \\[5pt]
    Charlton of Northumberland to Warrane, on the Clyde, Spring 1834 \\[5pt]
    Prouse of Devonshire to Pito-One, on the Duke of Roxburgh, Summer 1840 \\[5pt]
    Collis of Hampshire to Pito-One, on the Birman, Autumn 1842 \\[5pt]
    Treacher of Berkshire to Whakatū, on the Wild Duck, Spring 1850 \\[5pt]
    Innes of Berkshire to Narrm, on the Sacramento, Autumn 1853 \\[5pt]
    Bruce of Lambeth to Narrm, on the Omega, Autumn 1855 \\[5pt]
    Quennell of Surrey to Warrane, on the Asiatic, Winter 1855 \\[5pt]
    Barr of Glasgow to Kōpūtai, on the Sir Edward Paget, Winter 1856 \\[5pt] 
    Perkins of London to Te Whanganui-a-Tara, on the Matoaka, Spring 1859 \\[5pt]
    McKee of Antrim to Tāmaki Makaurau, on the Indian Empire, Spring 1862 \\[5pt]
    McTaggart of Argyllshire to Kōpūtai, on the Edward P. Bouverie, Autumn 1869 \\[5pt] 
    Chapman of Surrey to Whakatū, on the Adamant, Winter 1874 \\[5pt]
    Cheel of Marylebone to Whakatū, on the Queen Bee, Winter 1877 \\[5pt] 
    Hutchison of Aberdeenshire to Te Whanganui-a-Tara, on the Wakatipu, Autumn 1889 \\[5pt]
    
\end{normalsize}

I chose to start my doctoral studies just a few months into a global pandemic. Completing a challenging project with a worldwide crisis in the background might have been impossible without the supervision of AProf. Natalie Plank. Her ability to adapt to and overcome any problem has taught me that there is no situation which is truly unmanageable. I am deeply grateful for her leadership throughout a time of mass chaos. \\[5pt] I started this project with minimal formal training in biological science, coming from a primarily physics and engineering background. The immense support I received from Dr Melissa Jordan and Dr Colm Carraher from Plant and Food Research to complete this project meant that this was not an issue, and I thank them both immensely for this. \\[5pt] I would not have been able to begin this thesis without the financial backing and support I received from the Institute for Plant and Food Research, in association with the Better Border Biosecurity (B3) programme. In particular, I would like to thank the ex-Director of B3, Dr David Teulon. I would also like to thank the donor of the Ernest Marsden Scholarship in Physics for their significant financial support during my project. \\[5pt] There are many incredibly supportive people who I worked alongside at Te Herenga Waka – Victoria University of Wellington. I would like to start off by thanking Dr Rifat Ullah, whose mentoring and kindness encouraged me to pursue further study. His work on the initial design and setup of the vapour delivery system was invaluable to me throughout this project. I am also especially grateful to Alex Puglisi, for constructing the mechanical elements of the vapour delivery system and giving me extensive feedback on the system design. I would like to thank Peter Coard, for his advice and guidance when constructing the electrical elements of the vapour delivery system. Thank you to AProf. Ben Ruck, my supportive secondary supervisor, and to AProf. Franck Natali, for always asking about my thesis in the tearoom. Thank you to Dr. Gideon Gouws for his friendly encouragement and advice. I would also like to thank Alan Rennie, Grant Franklin, Chris Lepper, Rashika Gunasekara and Pete Jebson for their substantial technical and administrative assistance with this project. \\[5pt] I was lucky enough to start my doctoral program just as a group of supportive and talented senior students were finishing, and finished just as a group of enthusiastic and talented new doctoral students were starting. A special thanks to Jenna Nyugen, Erica Happe and Erica Cassie for teaching me the fabrication processes and characterisation procedures that made this thesis happen; and a special thanks to Marissa Dierkes, Danica Fontein, Sangar Begzaad and Alireza Zare, for their incredible support throughout the thesis writing process. \\[5pt] I would like to thank everyone else I shared an office with and worked alongside, including Jackson, Will, Ali, Kira, Catherine, Janani, Ted, Kiri and Joe. I would also like to thank all the interns and summer students who contributed to this project. \\[5pt] A massive thank you to Openstar Technologies. It was an honour to work on a cutting-edge plasma physics project right here in Wellington. A particularly big thank you to Thomas, Darren, Ben and Ratu for having me as part of the team. I sincerely hope to see fusion with net energy gain in Aotearoa New Zealand within the next few years. \\[5pt] I want to thank Shodokan Aikido New Zealand for their support throughout this thesis, in particular for the once-in-a-lifetime opportunity to travel to Osaka, Japan to be graded for my first dan by Nariyama Shihan. Thanks for all the training and support, Ian. Thank you to all the friends, old and new, and whānau, blood-related or not, who have supported me over these wild past few years. You know who you are. \\[5pt] Thank you to my brother, Keeson, and to my parents, Hilary and Phillip. Your support means everything to me, and I would not be where I am today without you. In particular, our Friday lunchtime cafe visits have kept me inspired and motivated me throughout the doctoral program. Thank you, thank you, thank you for your love, your compassion, and for being there for me. \\[5pt] Finally, thank you Nina. Your love has kept me going through the most difficult and most wonderful times over the last three and a half years. You are the light of my life, and I am so happy to have taken on this challenge with you by my side. \\[5pt] Arohanui and peace to you all, Eddyn (Ned)

\clearpage
\newpage
\thispagestyle{empty} % Hide header and footer on this page
\mbox{~}
\clearpage
\newpage

\pagestyle{headings}


