% Options for packages loaded elsewhere
\PassOptionsToPackage{unicode}{hyperref}
\PassOptionsToPackage{hyphens}{url}
%
\documentclass[
  a4paper,
]{scrbook}

\usepackage{amsmath,amssymb}
\usepackage{iftex}
\ifPDFTeX
  \usepackage[T1]{fontenc}
  \usepackage[utf8]{inputenc}
  \usepackage{textcomp} % provide euro and other symbols
\else % if luatex or xetex
  \usepackage{unicode-math}
  \defaultfontfeatures{Scale=MatchLowercase}
  \defaultfontfeatures[\rmfamily]{Ligatures=TeX,Scale=1}
\fi
\usepackage{lmodern}
\ifPDFTeX\else  
    % xetex/luatex font selection
  \setmainfont[]{Latin Modern Roman}
  \setsansfont[]{Latin Modern Roman}
\fi
% Use upquote if available, for straight quotes in verbatim environments
\IfFileExists{upquote.sty}{\usepackage{upquote}}{}
\IfFileExists{microtype.sty}{% use microtype if available
  \usepackage[]{microtype}
  \UseMicrotypeSet[protrusion]{basicmath} % disable protrusion for tt fonts
}{}
\makeatletter
\@ifundefined{KOMAClassName}{% if non-KOMA class
  \IfFileExists{parskip.sty}{%
    \usepackage{parskip}
  }{% else
    \setlength{\parindent}{0pt}
    \setlength{\parskip}{6pt plus 2pt minus 1pt}}
}{% if KOMA class
  \KOMAoptions{parskip=half}}
\makeatother
\usepackage{xcolor}
\setlength{\emergencystretch}{3em} % prevent overfull lines
\setcounter{secnumdepth}{5}
% Make \paragraph and \subparagraph free-standing
\ifx\paragraph\undefined\else
  \let\oldparagraph\paragraph
  \renewcommand{\paragraph}[1]{\oldparagraph{#1}\mbox{}}
\fi
\ifx\subparagraph\undefined\else
  \let\oldsubparagraph\subparagraph
  \renewcommand{\subparagraph}[1]{\oldsubparagraph{#1}\mbox{}}
\fi


\providecommand{\tightlist}{%
  \setlength{\itemsep}{0pt}\setlength{\parskip}{0pt}}\usepackage{longtable,booktabs,array}
\usepackage{calc} % for calculating minipage widths
% Correct order of tables after \paragraph or \subparagraph
\usepackage{etoolbox}
\makeatletter
\patchcmd\longtable{\par}{\if@noskipsec\mbox{}\fi\par}{}{}
\makeatother
% Allow footnotes in longtable head/foot
\IfFileExists{footnotehyper.sty}{\usepackage{footnotehyper}}{\usepackage{footnote}}
\makesavenoteenv{longtable}
\usepackage{graphicx}
\makeatletter
\def\maxwidth{\ifdim\Gin@nat@width>\linewidth\linewidth\else\Gin@nat@width\fi}
\def\maxheight{\ifdim\Gin@nat@height>\textheight\textheight\else\Gin@nat@height\fi}
\makeatother
% Scale images if necessary, so that they will not overflow the page
% margins by default, and it is still possible to overwrite the defaults
% using explicit options in \includegraphics[width, height, ...]{}
\setkeys{Gin}{width=\maxwidth,height=\maxheight,keepaspectratio}
% Set default figure placement to htbp
\makeatletter
\def\fps@figure{htbp}
\makeatother

\usepackage{booktabs}
\usepackage{longtable}
\usepackage{array}
\usepackage{multirow}
\usepackage{wrapfig}
\usepackage{float}
\usepackage{colortbl}
\usepackage{pdflscape}
\usepackage{tabu}
\usepackage{threeparttable}
\usepackage{threeparttablex}
\usepackage[normalem]{ulem}
\usepackage{makecell}
\usepackage{xcolor}
\usepackage{fancyhdr}
\usepackage{titling}
\setlength{\droptitle}{-2cm}
\preauthor{
  \begin{center}
  \Large
  \vspace{10mm}
  by

  \vspace{20mm}
}
\postauthor{
  \end{center}
  \vfill
}

\predate{
  \begin{center}
  A thesis 
  submitted in partial fulfilment of the \\
  requirements of the degree of \\
  Doctor of Philosophy in Physics\\               % Degree
  School of Physical and Chemical Sciences\\          % Department
  Te Herenga Waka - Victoria University of Wellington\\                       % University 
  \vspace{5mm}
}
\postdate{
  \\
  \includegraphics[width=3in,height=1.5in]{figures/VUW-logo.png}\\
  \end{center}
  }
\makeatletter
\makeatother
\makeatletter
\@ifpackageloaded{bookmark}{}{\usepackage{bookmark}}
\makeatother
\makeatletter
\@ifpackageloaded{caption}{}{\usepackage{caption}}
\AtBeginDocument{%
\ifdefined\contentsname
  \renewcommand*\contentsname{Table of contents}
\else
  \newcommand\contentsname{Table of contents}
\fi
\ifdefined\listfigurename
  \renewcommand*\listfigurename{List of Figures}
\else
  \newcommand\listfigurename{List of Figures}
\fi
\ifdefined\listtablename
  \renewcommand*\listtablename{List of Tables}
\else
  \newcommand\listtablename{List of Tables}
\fi
\ifdefined\figurename
  \renewcommand*\figurename{Figure}
\else
  \newcommand\figurename{Figure}
\fi
\ifdefined\tablename
  \renewcommand*\tablename{Table}
\else
  \newcommand\tablename{Table}
\fi
}
\@ifpackageloaded{float}{}{\usepackage{float}}
\floatstyle{ruled}
\@ifundefined{c@chapter}{\newfloat{codelisting}{h}{lop}}{\newfloat{codelisting}{h}{lop}[chapter]}
\floatname{codelisting}{Listing}
\newcommand*\listoflistings{\listof{codelisting}{List of Listings}}
\makeatother
\makeatletter
\@ifpackageloaded{caption}{}{\usepackage{caption}}
\@ifpackageloaded{subcaption}{}{\usepackage{subcaption}}
\makeatother
\makeatletter
\@ifpackageloaded{tcolorbox}{}{\usepackage[skins,breakable]{tcolorbox}}
\makeatother
\makeatletter
\@ifundefined{shadecolor}{\definecolor{shadecolor}{rgb}{.97, .97, .97}}
\makeatother
\makeatletter
\makeatother
\makeatletter
\makeatother
\ifLuaTeX
  \usepackage{selnolig}  % disable illegal ligatures
\fi
\usepackage[citestyle = ieee,urldate = iso8601]{biblatex}
\addbibresource{references.bib}
\IfFileExists{bookmark.sty}{\usepackage{bookmark}}{\usepackage{hyperref}}
\IfFileExists{xurl.sty}{\usepackage{xurl}}{} % add URL line breaks if available
\urlstyle{same} % disable monospaced font for URLs
\hypersetup{
  pdftitle={Developing an Insect Odorant Receptor Bioelectronic Nose for Vapour-Phase Detection},
  pdfauthor={Eddyn Oswald Perkins Treacher},
  hidelinks,
  pdfcreator={LaTeX via pandoc}}

\title{Developing an Insect Odorant Receptor Bioelectronic Nose for
Vapour-Phase Detection}
\author{Eddyn Oswald Perkins Treacher}
\date{May 2024}

\begin{document}
\frontmatter

\maketitle

\clearpage
\newpage
\thispagestyle{empty} % Hide header and footer on this page
\mbox{~}
\clearpage
\newpage

%----------------------------------------------
%   Abstract
%----------------------------------------------

\begin{flushleft}
% Manually add a section to the table of contents
\pagenumbering{roman}
\addcontentsline{toc}{chapter}{Abstract}
\huge\textbf{Abstract}
\end{flushleft}

\vspace*{\baselineskip}

The ability to detect volatile organic compounds in a highly sensitive and selective manner is desirable for a variety of different applications, including diagnosing illnesses at a remote clinic, monitoring air in an industrial workplace, or identifying biomarkers from invasive organisms at a biosecurity checkpoint. This thesis therefore investigates the coupling of insect odorant receptors (iORs) with carbon nanotube network field-effect transistors (CNT FETs) and graphene field-effect transistors (GFETs) to create a sensitive and selective vapour-phase sensor or ‘bioelectronic nose’. \\[5pt] The properties of three distinct carbon nanotube network morphologies were compared to understand their relative suitability as transducer thin films for the bioelectronic nose. The first morphology was obtained by depositing ultrasonicated carbon nanotubes in solvent, the second by depositing carbon nanotubes in surfactant solution, and the third by depositing with surfactant in the presence of steam. Compared to the other morphologies used, the steam-assisted deposition devices were found to have highly consistent device-to-device electrical properties due to their relatively low bundling and dense coverage of the substrate (>70\%). The steam-assisted deposition devices, however, were found to be more highly \textit{p}-doped than the other morphologies, with a relatively large liquid-gated threshold voltage (0.37 V). The sensitivity of the transducer was verified using an aqueous phosphate buffered saline (PBS) dilution series which followed a logarithmic trend. To account for baseline drift due to hysteresis, the device was left at a constant voltage for 30 minutes before the dilution series. By allowing baseline drift to settle over this period, it could be reasonably approximated as linear, estimated using a linear fit and removed from the data. \\[5pt] A vapour delivery system was also adapted for the purpose of testing the bioelectronic nose in a vapour environment. A photoionisation detector (PID) and a relative humidity and temperature indicator (RHI) were added to the existing system, which could be used to non-selectively detect vapours for comparison with measurements taken by the bioelectronic nose. Two new mass flow controllers (MFCs) were also added to the system to allow for a greater range of flow rates. A new electronic control system was also created to integrate the new MFCs and RHI. In the upgraded vapour delivery system, MFC-controlled nitrogen flow bubbles through analyte, carrying vapour into a device characterisation chamber, and then into a manifold leading to the PID and RHI. A steam-assisted deposition CNT FET was found to be sensitive to ppb concentrations of ethyl hexanoate (EtHex) and \textit{trans}-2-hexen-1-al (E2Hex). Baseline drift from hysteresis was found to be significantly larger when backgated in the vapour delivery system, and a 40 minute period was therefore used when waiting for baseline drift to settle. \\[5pt] One of the major challenges encountered in this thesis was the variability in quality of the non-covalent functionalisation method used. 1-pyrenebutanoic acid N-hydroxysuccinimide ester (PBASE) was primarily used as a linker molecule to attach the iORs to carbon nanotubes and graphene. This linker is widely used in the literature, but there is little agreement on the optimal process variables to use. It was verified that PBASE and iOR nanodiscs were successfully attaching to the transistor channel with a method previously used for iOR functionalisation. However, devices functionalised in an identical manner would rarely operate as expected and usually showed no sensing activity. It was found that devices which showed no sensing activity also exhibited no change in transfer characteristics after functionalisation, despite iOR nanodiscs being present. A possible explanation for this behaviour is the presence of a contamination layer which electrically isolates iOR nanodiscs from the transducer material. A device functionalised with pyrene-PEG-biotin (PPB) and avi-tagged OR10a after the use of a destructive oxygen plasma cleaning step responded to low fM concentrations of methyl salicylate (MeSal), further supporting this hypothesis. It was concluded that successful and reproducible vapour sensing using non-covalent functionalisation may require a robust, non-destructive cleaning method for the transducer platform.

\fancyhf{} %clear all headers and footers fields
\thispagestyle{fancy} % Change header and footer on this page
\renewcommand{\headrulewidth}{0pt}
\fancyhead[L]{\textit{Abstract}} % Set header content
\fancyfoot[L]{\thepage} %prints the page number on the right side of the header

\clearpage
\newpage
\thispagestyle{empty} % Hide header and footer on this page
\mbox{~}
\clearpage
\newpage

%----------------------------------------------
%   Acknowledgement
%----------------------------------------------

\thispagestyle{plain}

\begin{flushleft}
% Manually add a section to the table of contents
\addcontentsline{toc}{chapter}{Acknowledgements}
\huge\textbf{Acknowledgements}
\end{flushleft}

\vspace*{\baselineskip}

Acknowledgements go here

\clearpage
\newpage
\thispagestyle{empty} % Hide header and footer on this page
\mbox{~}
\clearpage
\newpage

\pagestyle{headings}

\ifdefined\Shaded\renewenvironment{Shaded}{\begin{tcolorbox}[boxrule=0pt, borderline west={3pt}{0pt}{shadecolor}, interior hidden, sharp corners, enhanced, breakable, frame hidden]}{\end{tcolorbox}}\fi

\renewcommand*\contentsname{Table of Contents}
{
\setcounter{tocdepth}{2}
\addcontentsline{toc}{chapter}{Table of Contents}
\tableofcontents
}
\listoffigures
\addcontentsline{toc}{chapter}{List of Figures}
\listoftables
\addcontentsline{toc}{chapter}{List of Tables}

\clearpage
\newpage
\thispagestyle{empty} % Hide header and footer on this page
\mbox{~}
\clearpage
\newpage

%----------------------------------------------
%   List of Abbreviations
%----------------------------------------------

\thispagestyle{plain}

\begin{flushleft}
% Manually add a section to the table of contents
\addcontentsline{toc}{chapter}{List of Abbreviations}
\huge\textbf{List of Abbreviations}
\end{flushleft}

\vspace*{\baselineskip}

\begin{table}[H]
  \begin{tabular}{@{}p{0.25\textwidth} p{0.75\textwidth}@{}}  % Adjust the width as needed
    2D  & 2-Dimensional  \\[5pt]
    Ab  & Antibody  \\[5pt]
    AB  & Amyl Butyrate  \\[5pt]
    AB-NTA  & N$\alpha$,N$\alpha$-Bis(carboxymethyl)-\textit{L}-lysine hydrate  \\[5pt]
    AFM  & Atomic Force Microscope/Microscopy  \\[5pt]
    AH  & Absolute Humidity  \\[5pt]
    Avi-tag  & Avidin-tag  \\[5pt]
    BMIM  & 1-Butyl-3-methylimidazolium bis(trifluoromethylsulfonyl)imide  \\[5pt]
    CAD  & Computer Aided Design \\[5pt]
    CNT  & Carbon Nanotube  \\[5pt]
    CVD  & Chemical Vapour Deposition  \\[5pt]
    Cy3  & Cyanine 3  \\[5pt]
    DAN  & 1,5-diaminonaphthalene  \\[5pt]
    DAQ  & Data Acquisition Input/Output Module  \\[5pt]
    DCB  & 1,2-dichlorobenzene  \\[5pt]
    DI  & Deionised  \\[5pt]
    DMF  & Dimethylformamide   \\[5pt]
    DMSO  & Dimethylsulfoxide   \\[5pt]
    DMT-MM   & 4-(4,6-dimethoxy-1,3,5-triazin-2-yl)-4 methylmorpholinium chloride \\[5pt]
    DMMP  & Dimethyl Methylphosphonate  \\[5pt]
    DNA  & Deoxyribonucleic Acid  \\[5pt]
    E2Hex  & \textit{trans}-2-hexan-1-al  \\[5pt]
    EB  & Ethyl Butyrate  \\[5pt]
    EDC  & 1-Ethyl-3-(3-dimethylaminopropyl)carbodiimide  \\[5pt]
    EDL  & Electric Double Layer  \\[5pt]
    EtHex  & Ethyl Hexanoate  \\[5pt]
    EtOH  & Ethanol  \\[5pt]
    FET  & Field-Effect Transistor  \\[5pt]
    FITC  & Fluorescein isothiocyanate  \\[5pt]
  \end{tabular}
\end{table}

\newpage
\fancyhf{} %clear all headers and footers fields
\thispagestyle{fancy} % Change header and footer on this page
\renewcommand{\headrulewidth}{0pt}
\fancyhead[L]{\textit{List of Abbreviations}} % Set header content
\fancyfoot[L]{\thepage} %prints the page number on the right side of the header
\begin{table}[H]
  \begin{tabular}{@{}p{0.25\textwidth} p{0.75\textwidth}@{}}  % Adjust the width as needed
    GA  & Glutaraldehyde  \\[5pt]
    GFET  & Graphene Field-Effect Transistor  \\[5pt]
    GFP  & Green Fluorescent Protein  \\[5pt]
    GPCR  & G-protein Coupled Receptor  \\[5pt]
    HEK  & Human Embryonic Kidney  \\[5pt]
    His-tag  & Histidine-tag  \\[5pt]
    hOR  & Human Odorant Receptor  \\[5pt]
    HPLC  & High-performance Liquid Chromatography   \\[5pt]
    iOR  & Insect Odorant Receptor  \\[5pt]
    IPA  & Isopropanol  \\[5pt]
    LOD  & Limit of Detection  \\[5pt]
    m-CNT  & Metallic Carbon Nanotube   \\[5pt]
    MeOH  & Methanol   \\[5pt]
    MeSal  & Methyl Salicylate   \\[5pt]
    MFC  & Mass Flow Controller   \\[5pt]
    mOR  & Mouse Odorant Receptor  \\[5pt]
    MOSFET  & Metal-Oxide-Semiconductor Field-Effect Transistor  \\[5pt]
    MSP  & Membrane Scaffold Protein  \\[5pt]
    MWCNT  & Multi-Walled Carbon Nanotube  \\[5pt]
    ND  & Nanodisc  \\[5pt]
    NHS  & N-Hydroxysuccinimide  \\[5pt]
    NMR  & Nuclear Magnetic Resonance  \\[5pt]
    NSB  & Non-Specific Binding   \\[5pt]
    NTA  & Nitrilotriacetic Acid   \\[5pt]
    OBP  & Odorant Binding Protein  \\[5pt]
    OR  & Odorant Receptor  \\[5pt]
    ORCO  & Odorant Receptor Co-Receptor  \\[5pt]
    PBA  & 1-Pyrenebutyric Acid  \\[5pt]
    PBASE  & 1-Pyrenebutanoic Acid N-hydroxysuccinimide Ester  \\[5pt]
    PBS  & Phosphate-Buffered Saline  \\[5pt]
    PCB  & Printed Circuit Board   \\[5pt]
    PDL & Poly-\textit{D}-lysine  \\[5pt]
  \end{tabular}
\end{table}

\newpage
\fancyhf{} %clear all headers and footers fields
\thispagestyle{fancy} % Change header and footer on this page
\renewcommand{\headrulewidth}{0pt}
\fancyhead[R]{\textit{List of Abbreviations}} % Set header content
\fancyfoot[R]{\thepage} %prints the page number on the right side of the header
\begin{table}[H]
  \begin{tabular}{@{}p{0.25\textwidth} p{0.75\textwidth}@{}}  % Adjust the width as needed
    PDMS  & Polydimethylsiloxane   \\  [5pt]
    PEG  & Polyethylene Glycol  \\[5pt] 
    PID  & Photoionisation Detector  \\[5pt] 
    PLL  & Poly-\textit{L}-lysine  \\[5pt]
    PPB  & Pyrene-PEG-Biotin  \\[5pt]
    PPF  & Pyrene-PEG-FITC  \\[5pt]
    PPN  & Pyrene-PEG-NTA  \\[5pt]
    PPR  & Pyrene-PEG-Rhodamine  \\[5pt]
    PTFE  & Polytetrafluoroethylene (Teflon™)  \\[5pt]
    PVC  & Polyvinyl chloride  \\[5pt]
    RH  & Relative Humidity  \\[5pt]
    RHI  & Relative Humidity and Temperature Indicator  \\[5pt] 
    RNA  & Ribonucleic Acid   \\[5pt]
    s-CNT  & Semiconducting Carbon Nanotube   \\[5pt]
    SEM  & Scanning Electron Microscope/Microscopy   \\[5pt]
    SMU  & Source Measure Unit   \\[5pt]
    Sulfo-NHS  & N-hydroxysulfosuccinimide   \\[5pt]
    SWCNT  & Single-Walled Carbon Nanotube   \\[5pt]
    TFTFET  & Thin-Film Field-Effect Transistor  \\[5pt]
    TMAH  & Tetramethylammonium hydroxide  \\[5pt]
    TX  & Transfer Characteristics  \\[5pt]
    UV  & Ultraviolet  \\[5pt]
    VI  & Virtual Instrument  \\[5pt]
    VUAA1  & N-(4-Ethylphenyl)-2-{[4-ethyl-5-(pyridin-3-yl)-4H-1,2,4-triazol-3-yl]sulfanyl}acetamide  \\[5pt] 
  \end{tabular}
\end{table}

\clearpage
\newpage
\thispagestyle{empty} % Hide header and footer on this page
\mbox{~}
\clearpage
\newpage

% Adjust the top and bottom margins of float pages to center floats
\makeatletter
\setlength{\@fptop}{0pt plus 1fil}
\setlength{\@fpbot}{0pt plus 1fil}
\makeatother

\pagestyle{headings}
\mainmatter
\bookmarksetup{startatroot}

\hypertarget{introduction}{%
\chapter{Introduction}\label{introduction}}

The `bioelectronic nose', an electronic transducer modified with
elements of the animal olfactory system, has the potential to allow
specific detection of vapour-phase volatile compounds at concentrations
in the air as low as parts per trillion. Insect odorant receptors (iORs)
enable simple invertebrates, such as the vinegar fruit fly
\emph{Drosophila melanogaster}, to forage for survival by carefully
distinguishing between a huge number of specific volatile compounds. In
the past five years, a range of iORs have been successfully coupled with
highly sensitive low-dimensional thin-film transistors (TFTs) for
specific detection of fruit-like odors in an aqueous environment. These
thin films include random networks of carbon nanotubes (CNTs) and
monolayer graphene, nanomaterials which are both extremely sensitive due
to their high surface-to-volume ratio. In this thesis, my aim was to
investigate whether a bioelectronic nose capable of odorant detection in
a vapour-phase environment could be constructed by coupling iORs with
TFTs in a similar manner.

My aim is to develop a `bioelectronic nose', a biosensor device which
couples sensitive biological recognition elements with an electronic
transducer for the detection of vapour phase compounds
\autocite{Lee2010,Dung2018,Moon2020}. The transducer converts the
interaction or interactions between the recognition element and analyte
or analytes of interest into a measureable electronic signal. The
sensitive biological component used here are \emph{Drosophila
melanogaster} insect odorant receptors (iORs), while the electronic
transducer element is a carbon nanotube- or graphene-based field effect
transistor (CNTFET or GFET). Carbon-based 2D nanomaterials are promising
for use in novel biosensors as they are highly sensitive, biocompatible
and cheap to fabricate \autocite{Shkodra2021}. I created a purpose-built
vapour delivery system apparatus in order to test these devices.
Initially, however, iOR-functionalised CNTFETs and GFETs (iOR-FETs) were
first tested in the liquid phase to corroborate previous findings within
my research group \autocite{Murugathas2019a,Murugathas2020}.

There has been a significant amount of work done towards creating
bioelectronic noses over the last twenty years. This is largely due to
their promisingly high level of sensitivity and specificity in real-time
in the gas phase, with the ability to signal the presence of volatile
organic compound (VOC) traces at lower concentrations than traditional
chemical sensors or the human nose in a timescale of seconds
\autocite{Lee2010,Moon2020,Terutsuki2020}. The implications of
successful development of a portable and robust bioelectronic nose are
significant and varied. Applications could be found in high-importance
fields such as biosecurity, medicine, environmental protection and food
or water safety \autocite{Dung2018,Arakawa2019,Yang2017,Son2017}. It has
been demonstrated that it is possible to detect invasive brown
marmorated stinkbugs based on their volatile trace \autocite{Moser2020}.
A bioelectronic nose could potentially accomplish this task far more
cheaply and efficiently than trained sniffer dogs.

As well as a variety of practical applications, development of a
bioelectronic nose may give us a greater understanding of the mechanisms
underlying insect olfaction, as well as novel understandings of the
transducer devices used to register the electronic response to VOCs
\autocite{Lee2010}. The transduction mechanism of nanomaterial-based iOR
sensors is still unknown, and I hope to shed further light on the
biological and electronic processes underpinning this mechanism
\autocite{Murugathas2020,Khadka2019}.

\cleardoublepage
\phantomsection
\addcontentsline{toc}{part}{Appendices}
\appendix

\hypertarget{vapour-system-hardware}{%
\chapter{Vapour System Hardware}\label{vapour-system-hardware}}

\hypertarget{tbl-vapour-sensor-components}{}
\begin{longtable}[t]{>{\raggedright\arraybackslash}p{5.5cm}>{\raggedright\arraybackslash}p{4.5cm}>{\raggedright\arraybackslash}p{3.75cm}}
\caption{\label{tbl-vapour-sensor-components}Major components used in construction of the vapour delivery system
described in this thesis. }\tabularnewline

\toprule
Description & Part No. & Manufacturer\\
\midrule
Mass flow controller, 20 sccm full scale & GE50A-013201SBV020 & MKS Instruments\\
Mass flow controller, 200 sccm full scale & GE50A-013202SBV020 & MKS Instruments\\
Mass flow controller, 500 sccm full scale & FC-2901V & Tylan\\
Analogue flowmeter, 240 sccm max. flow & 116261-30 & Dwyer\\
Micro diaphragm pump & P200-B3C5V-35000 & Xavitech\\
\addlinespace
Analogue flow controller, for micro diaphragm pump & X3000450 & Xavitech\\
10 mL Schott bottle & 218010802 & Duran\\
PTFE connection cap system & Z742273 & Duran\\
Baseline VOC-TRAQ flow cell, purple & 043-950 & Ametek Mocon\\
Baseline VOC-TRAQ flow cell, red & 043-951 & Ametek Mocon\\
\addlinespace
Humidity and temperature sensor & T9602-5-A & Telaire\\
Enclosure, for humidity and temperature sensor & MC001189 & Multicomp Pro\\
\bottomrule
\end{longtable}

\hypertarget{python-code-for-data-analysis}{%
\chapter{Python Code for Data
Analysis}\label{python-code-for-data-analysis}}

\hypertarget{code-repository}{%
\section{Code Repository}\label{code-repository}}

The code used for general analysis of field-effect transistor devices in
this thesis was written with Python 3.8.8. Contributors to the code used
include Erica Cassie, Erica Happe, Marissa Dierkes and Leo Browning. The
code is located on GitHub and the research group OneDrive, and is
available on request.

\hypertarget{sec-histogram-analysis}{%
\section{Atomic Force Microscope Histogram
Analysis}\label{sec-histogram-analysis}}

The purpose of this code is to analyse atomic force microscope (AFM)
images of carbon nanotube networks in .xyz format taken using an atomic
force microscope and processed in Gwyddion (see
\textbf{?@sec-afm-characterisation}). It was originally designed by
Erica Happe in Matlab, and adapted by Marissa Dierkes and myself for use
in Python. The code imports the .xyz data and sorts it into bins 0.15 nm
in size for processing. To perform skew-normal distribution fits, both
\emph{scipy.optimize.curve\_fit} and \emph{scipy.stats.skewnorm} modules
are used in this code.

\hypertarget{sec-raman-analysis}{%
\section{Raman Spectroscopy Analysis}\label{sec-raman-analysis}}

The purpose of this code is to analyse a series of Raman spectra taken
at different points on a single film (see
\textbf{?@sec-raman-characterisation}). Data is imported in a series of
tab-delimited text files, with the low wavenumber spectrum (100
cm\(^{-1} - 650\) cm\(^{-1}\)) and high wavenumber spectrum (1300
cm\(^{-1} - 1650\) cm\(^{-1}\)) imported in separate datafiles for each
scan location.

\hypertarget{sec-field-effect-transistor-analysis}{%
\section{Field-Effect Transistor
Analysis}\label{sec-field-effect-transistor-analysis}}

The purpose of this code is to analyse electrical measurements taken of
field-effect transistor (FET) devices. Electrical measurements were
either taken from the Keysight 4156C Semiconductor Parameter Analyser,
National Instruments NI-PXIe or Keysight B1500A Semiconductor Device
Analyser as discussed in \textbf{?@sec-electrical-characterisation}; the
code is able to analyse data in .csv format taken from all three
measurement setups. The main Python file in the code base consists of
three related but independent modules: the first analyses and plots
sensing data from the FET devices, the second analyses and plots
transfer characteristics from channels across a device, and the third
compares individual channel characteristics before and after a
modification or after each of several modifications. The code base also
features a separate config file and style sheet which govern the
behaviour of the main code. The code base was designed collaboratively
by myself and Erica Cassie over GitHub using the Sourcetree Git GUI.

\hypertarget{references}{%
\chapter*{References}\label{references}}
\addcontentsline{toc}{chapter}{References}

\markboth{References}{References}

\printbibliography[heading=none]


\backmatter

\end{document}
