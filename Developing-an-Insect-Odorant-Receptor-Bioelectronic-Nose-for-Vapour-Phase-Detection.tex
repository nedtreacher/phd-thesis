% Options for packages loaded elsewhere
\PassOptionsToPackage{unicode}{hyperref}
\PassOptionsToPackage{hyphens}{url}
%
\documentclass[
  a4paper,
]{scrbook}

\usepackage{amsmath,amssymb}
\usepackage{iftex}
\ifPDFTeX
  \usepackage[T1]{fontenc}
  \usepackage[utf8]{inputenc}
  \usepackage{textcomp} % provide euro and other symbols
\else % if luatex or xetex
  \usepackage{unicode-math}
  \defaultfontfeatures{Scale=MatchLowercase}
  \defaultfontfeatures[\rmfamily]{Ligatures=TeX,Scale=1}
\fi
\usepackage{lmodern}
\ifPDFTeX\else  
    % xetex/luatex font selection
  \setmainfont[]{Latin Modern Roman}
  \setsansfont[]{Latin Modern Roman}
\fi
% Use upquote if available, for straight quotes in verbatim environments
\IfFileExists{upquote.sty}{\usepackage{upquote}}{}
\IfFileExists{microtype.sty}{% use microtype if available
  \usepackage[]{microtype}
  \UseMicrotypeSet[protrusion]{basicmath} % disable protrusion for tt fonts
}{}
\makeatletter
\@ifundefined{KOMAClassName}{% if non-KOMA class
  \IfFileExists{parskip.sty}{%
    \usepackage{parskip}
  }{% else
    \setlength{\parindent}{0pt}
    \setlength{\parskip}{6pt plus 2pt minus 1pt}}
}{% if KOMA class
  \KOMAoptions{parskip=half}}
\makeatother
\usepackage{xcolor}
\setlength{\emergencystretch}{3em} % prevent overfull lines
\setcounter{secnumdepth}{5}
% Make \paragraph and \subparagraph free-standing
\ifx\paragraph\undefined\else
  \let\oldparagraph\paragraph
  \renewcommand{\paragraph}[1]{\oldparagraph{#1}\mbox{}}
\fi
\ifx\subparagraph\undefined\else
  \let\oldsubparagraph\subparagraph
  \renewcommand{\subparagraph}[1]{\oldsubparagraph{#1}\mbox{}}
\fi


\providecommand{\tightlist}{%
  \setlength{\itemsep}{0pt}\setlength{\parskip}{0pt}}\usepackage{longtable,booktabs,array}
\usepackage{calc} % for calculating minipage widths
% Correct order of tables after \paragraph or \subparagraph
\usepackage{etoolbox}
\makeatletter
\patchcmd\longtable{\par}{\if@noskipsec\mbox{}\fi\par}{}{}
\makeatother
% Allow footnotes in longtable head/foot
\IfFileExists{footnotehyper.sty}{\usepackage{footnotehyper}}{\usepackage{footnote}}
\makesavenoteenv{longtable}
\usepackage{graphicx}
\makeatletter
\def\maxwidth{\ifdim\Gin@nat@width>\linewidth\linewidth\else\Gin@nat@width\fi}
\def\maxheight{\ifdim\Gin@nat@height>\textheight\textheight\else\Gin@nat@height\fi}
\makeatother
% Scale images if necessary, so that they will not overflow the page
% margins by default, and it is still possible to overwrite the defaults
% using explicit options in \includegraphics[width, height, ...]{}
\setkeys{Gin}{width=\maxwidth,height=\maxheight,keepaspectratio}
% Set default figure placement to htbp
\makeatletter
\def\fps@figure{htbp}
\makeatother

\usepackage{booktabs}
\usepackage{longtable}
\usepackage{array}
\usepackage{multirow}
\usepackage{wrapfig}
\usepackage{float}
\usepackage{colortbl}
\usepackage{pdflscape}
\usepackage{tabu}
\usepackage{threeparttable}
\usepackage{threeparttablex}
\usepackage[normalem]{ulem}
\usepackage{makecell}
\usepackage{xcolor}
\usepackage{fancyhdr}
\usepackage{titling}
\usepackage{pdflscape}
\usepackage{geometry}
\setlength{\droptitle}{-2cm}
\preauthor{
  \begin{center}
  \Large
  \vspace{10mm}
  by
  \vspace{20mm}
}
\postauthor{
  \end{center}
  \vfill
}

\predate{
  \begin{center}
  A thesis 
  submitted in partial fulfilment of the \\
  requirements of the degree of \\
  Doctor of Philosophy in Physics\\               % Degree
  School of Physical and Chemical Sciences\\          % Department
  Te Herenga Waka - Victoria University of Wellington\\                       % University 
  \vspace{5mm}
}
\postdate{
  \\
  \includegraphics[width=3in,height=1.5in]{figures/VUW-logo.png}\\
  \end{center}
  }

\renewcommand{\topfraction}{.8}
\renewcommand{\bottomfraction}{.7}
\renewcommand{\textfraction}{.15}
\renewcommand{\floatpagefraction}{.8}
\setcounter{topnumber}{3}
\setcounter{bottomnumber}{3}
\setcounter{totalnumber}{4}

\clubpenalty=9996
\widowpenalty=9999
\makeatletter
\makeatother
\makeatletter
\@ifpackageloaded{bookmark}{}{\usepackage{bookmark}}
\makeatother
\makeatletter
\@ifpackageloaded{caption}{}{\usepackage{caption}}
\AtBeginDocument{%
\ifdefined\contentsname
  \renewcommand*\contentsname{Table of contents}
\else
  \newcommand\contentsname{Table of contents}
\fi
\ifdefined\listfigurename
  \renewcommand*\listfigurename{List of Figures}
\else
  \newcommand\listfigurename{List of Figures}
\fi
\ifdefined\listtablename
  \renewcommand*\listtablename{List of Tables}
\else
  \newcommand\listtablename{List of Tables}
\fi
\ifdefined\figurename
  \renewcommand*\figurename{Figure}
\else
  \newcommand\figurename{Figure}
\fi
\ifdefined\tablename
  \renewcommand*\tablename{Table}
\else
  \newcommand\tablename{Table}
\fi
}
\@ifpackageloaded{float}{}{\usepackage{float}}
\floatstyle{ruled}
\@ifundefined{c@chapter}{\newfloat{codelisting}{h}{lop}}{\newfloat{codelisting}{h}{lop}[chapter]}
\floatname{codelisting}{Listing}
\newcommand*\listoflistings{\listof{codelisting}{List of Listings}}
\makeatother
\makeatletter
\@ifpackageloaded{caption}{}{\usepackage{caption}}
\@ifpackageloaded{subcaption}{}{\usepackage{subcaption}}
\makeatother
\makeatletter
\@ifpackageloaded{tcolorbox}{}{\usepackage[skins,breakable]{tcolorbox}}
\makeatother
\makeatletter
\@ifundefined{shadecolor}{\definecolor{shadecolor}{rgb}{.97, .97, .97}}
\makeatother
\makeatletter
\makeatother
\makeatletter
\makeatother
\ifLuaTeX
  \usepackage{selnolig}  % disable illegal ligatures
\fi
\usepackage[citestyle = ieee,urldate = iso8601]{biblatex}
\addbibresource{references.bib}
\IfFileExists{bookmark.sty}{\usepackage{bookmark}}{\usepackage{hyperref}}
\IfFileExists{xurl.sty}{\usepackage{xurl}}{} % add URL line breaks if available
\urlstyle{same} % disable monospaced font for URLs
\hypersetup{
  pdftitle={Developing an Insect Odorant Receptor Bioelectronic Nose for Vapour-Phase Detection},
  pdfauthor={Eddyn Oswald Perkins Treacher},
  hidelinks,
  pdfcreator={LaTeX via pandoc}}

\title{Developing an Insect Odorant Receptor Bioelectronic Nose for
Vapour-Phase Detection}
\author{Eddyn Oswald Perkins Treacher}
\date{Dec 2024}

\begin{document}
\frontmatter

\maketitle

\clearpage
\newpage
\thispagestyle{empty} % Hide header and footer on this page
\mbox{~}
\clearpage
\newpage

%----------------------------------------------
%   Abstract
%----------------------------------------------

\thispagestyle{plain}

\begin{flushleft}
% Manually add a section to the table of contents
\pagenumbering{roman}
\addcontentsline{toc}{chapter}{Abstract}
\huge\textbf{Abstract}
\end{flushleft}

\vspace*{\baselineskip}

The ability to detect volatile organic compounds in a highly sensitive and selective manner is desirable for applications as varied as diagnosis of illnesses at a remote clinic, monitoring of air in an industrial setting, or identification of invasive organisms at a biosecurity checkpoint. Historically, animal noses have been used for such tasks, as their combined sensitivity and selectivity are superior to traditional artificial sensors. However, training and deploying animals in such situations is both time and cost intensive. In recent years, an improved understanding of \textit{in vivo} biological sensing has driven efforts to mimic these highly efficient processes in an artificial sensor format. \\[5pt] To this end, a “bioelectronic nose” was developed. This sensor uses an artificial transducer to amplify responses of an insect odorant receptor protein to specific volatile compounds. Thin-film transistors were used as the amplifier element, given their low cost, small size and extreme sensitivity. Various thin-film morphologies were compared, and their suitability for bioelectronic nose development assessed. Transducers made using a novel steam-assisted thin-film deposition technique were found to have highly consistent device-to-device electrical properties relative to other films. Films made using this process typically showed more surface contamination than other morphologies, but their high sensitivity was nonetheless confirmed with a non-specific sensing series in an aqueous environment. \\[5pt] One of the major challenges encountered in this thesis was variability in the quality of sensor functionalisation. Raman spectroscopy and fluorescence microscopy confirmed an existing non-covalent attachment method could successfully immobilise nanodiscs onto the transistor channel region. However, various sensors functionalised using the same procedure often exhibited no sensing activity. Extensive electrical characterisation indicated the presence of an unidentified contamination layer preventing electrical interaction between the insect odorant receptors and transducer thin-film. It was shown that this layer was unlikely to be directly associated with the thin-film morphology used for the transducer. \\[5pt] Subsequently, an alternative biotin-based non-covalent method was used for functionalisation of the proteins, which eliminated several possible contamination sources. This alternative biotin-based method was used to demonstrate successful aqueous sensing of femtomolar concentrations of methyl salicylate by an iOR10a-functionalised device. When tested in a custom-built vapour delivery system, a similar bioelectronic sensor was shown to be highly sensitive to the target vapour. However, consistent reproduction of the biotin-based method was challenging due to the harsh cleaning method involved. It was therefore difficult to determine conclusively whether sensor responses were selective. By finding new, systematic approaches to address the major barriers to sensor success carefully identified in this work, there are promising signs that a highly reliable vapour-phase bioelectronic nose can be produced.

%\fancyhf{} %clear all headers and footers fields
%\thispagestyle{fancy} % Change header and footer on this page
%\renewcommand{\headrulewidth}{0pt}
%\fancyhead[L]{\textit{Abstract}} % Set header content
%\fancyfoot[L]{\thepage} %prints the page number on the right side of the header

\clearpage
\newpage
\thispagestyle{empty} % Hide header and footer on this page
\mbox{~}
\clearpage
\newpage


%----------------------------------------------
%   Acknowledgement
%----------------------------------------------

\thispagestyle{plain}

\begin{flushleft}
% Manually add a section to the table of contents
\addcontentsline{toc}{chapter}{Acknowledgements}
\huge\textbf{Acknowledgements}
\end{flushleft}

\vspace*{\baselineskip}

I would first like to acknowledge the lands of my ancestors, and the lands of the sovereign first peoples to which my ancestors travelled. We each come from the land, live off the land and return to the land.\\[5pt]
\textit{Noon of Essex to Warrang, on the Friends, Autumn 1811} \\[5pt]
\textit{Cave of Cambridgeshire to Warrang, on the Royal Charlotte, Autumn 1825} \\[5pt]
\textit{Boyce of Suffolk to Warrang, 1832} \\[5pt] 
\textit{Charlton of Northumberland to Warrang, on the Clyde, Spring 1834} \\[5pt]
\textit{Prouse of Devonshire to Pito-one, on the Duke of Roxburgh, Summer 1840} \\[5pt]
\textit{Ebden of Devonshire to Pito-one, on the Tyne, Winter 1841} \\[5pt]
\textit{Collis of Hampshire to Pito-one, on the Birman, Autumn 1842} \\[5pt]
\textit{Swann of Loch Garman to Te Whanganui-a-Tara, 1844} \\[5pt] 
\textit{Blythe of Berkshire to Whakatū, circa 1846} \\[5pt]
\textit{Innes of Berkshire to Naarm, on the Sacramento, Autumn 1853} \\[5pt]
\textit{Sheppard of Gloucestershire to Naarm, 1853} \\[5pt] 
\textit{Bruce of London to Naarm, on the Omega, Autumn 1855} \\[5pt]
\textit{Quennell of Surrey to Warrang, on the Asiatic, Winter 1855} \\[5pt]
\textit{Barr of Glasgow to Kōpūtai, on the Sir Edward Paget, Winter 1856} \\[5pt] 
\textit{Perkins of London to Te Whanganui-a-Tara, on the Matoaka, Spring 1859} \\[5pt]
\textit{McKee of Antrim to Tāmaki Makaurau, on the Indian Empire, Spring 1862} \\[5pt]
\textit{Sandilands of Peeblesshire to Ōtepoti, circa 1864} \\[5pt] 
\textit{Treacher of Berkshire to Te Whanganui-a-Tara, on the Wild Duck, Summer 1865} \\[5pt]
\textit{McTaggart of Argyllshire to Kōpūtai, on the Edward P. Bouverie, Autumn 1869} \\[5pt] 
\textit{Chapman of Kent to Whakatū, on the Adamant, Winter 1874} \\[5pt]
\textit{Cheel of London to Whakatū, on the Queen Bee, Winter 1877} \\[5pt]  
\textit{Hutchison of Aberdeen to Tarntanya, before 1882.} \\[5pt] 
I chose to start my doctoral studies just a few months into a global pandemic. Completing a challenging project with a worldwide crisis in the background might have been impossible without the supervision of AProf. Natalie Plank. Her ability to adapt to and overcome any problem has taught me that there is no situation which is truly unmanageable. I am deeply grateful for her leadership throughout a time of particular chaos. \newpage
\fancyhf{} %clear all headers and footers fields
\thispagestyle{fancy} % Change header and footer on this page
\renewcommand{\headrulewidth}{0pt}
\fancyhead[L]{\textit{Acknowledgements}} % Set header content
\fancyfoot[L]{\thepage} %prints the page number on the left side of the header 
I started this project with minimal formal training in biological science, coming from a primarily physics and engineering background. \\[5pt] The immense support I received from Melissa Jordan and Colm Carraher from the Institute for Plant and Food Research (PFR) to complete this project meant that this was not an issue, and I thank them both immensely for this. \\[5pt] I would not have been able to begin this thesis without the financial backing and support I received from PFR and the Better Border Biosecurity (B3) programme. In particular, I am very grateful to Andrew Kralicek, formerly with PFR and now at Scentian Bio, and the ex-Director of B3, David Teulon, for helping to secure funding for my project. I would also like to thank the donor of the Ernest Marsden Scholarship in Physics for their significant financial support. \\[5pt] There are many incredibly supportive people who I worked alongside during my project. I would like to start off by thanking Rifat Ullah, whose mentoring and kindness encouraged me to pursue further study. His work on the initial design and setup of the vapour delivery system was invaluable to me throughout this project. I am also especially grateful to Alex Puglisi, for constructing the mechanical elements of the vapour delivery system and giving me extensive feedback on the system design. I would like to thank Peter Coard, for his advice and guidance when constructing the electrical elements of the vapour delivery system. I thank Selvan Murugathas, too, for his advice on constructing the insect odorant receptor sensors, as well as Damon Colbert and Valentina Lucarelli, who provided the insect odorant receptor nanodiscs used in this work. \\[5pt] Thank you to AProf. Ben Ruck, my supportive secondary supervisor, and to AProf. Franck Natali, for always asking about my thesis in the tearoom. Thank you to Gideon Gouws for his friendly encouragement and advice. For their substantial technical assistance and mentoring during this project, I thank Alan Rennie, Grant Franklin, Chris Lepper, Rashika Gunasekara, Pete Jebson and Sushila Pillai from VUW, Andrew Chan from PFR, AProf. Charles Unsworth from the University of Auckland, and Prof. Simon Brown and his nanomaterials group from the University of Canterbury. \\[5pt] I was lucky enough to start my doctoral program just as a group of supportive and talented senior students were finishing, and finished just as a group of enthusiastic and talented new doctoral students were starting. A special thanks to Jenna Nyugen, Erica Happe and Erica Cassie for teaching me the fabrication processes and characterisation procedures that made this thesis happen; and a special thanks to Marissa Dierkes, Danica Fontein, Sangar Begzaad and Alireza Zare, for their incredible support throughout the thesis writing process. I am also thankful for the assistance of the cleanroom group interns over the course of my PhD, including Liam, Hayden and Lotte. I would further like to thank everyone else I shared an office with and worked alongside, including Jackson, Will, Roshni, Ali, Sam, Kira, Catherine, Martin, Janani, Ted, Kiri and Joe. \\[5pt] A massive thank you to Openstar Technologies. It has been an honour to work on a cutting-edge plasma physics project right here in Te Whanganui-a-Tara. A particularly big thank you to Ratu, Darren and Thomas for having me as part of the plasma physics team. Thank you also to the other Openstar interns, in particular the other plasma physics interns, Valentina, Benjy and Chris. I wish you success in all your dipole-confined plasma related endeavours. \\[5pt] I want to thank Shodokan Aikido New Zealand for their support throughout this thesis, in particular for the once-in-a-lifetime opportunity to travel to Osaka to be graded for first-dan by Nariyama Shihan. Thanks for all the training and support, Ian. \\[5pt] Thank you to all the friends and whānau, old and new, who have supported me over these wild past few years. You know who you are. \\[5pt] Thank you to my brother, Keeson, and to my parents, Hilary and Phillip. Your support means everything to me, and I would not be where I am today without you. Our Friday lunchtime cafe visits kept me motivated and inspired throughout the doctoral program. Thank you, thank you, thank you for your love, your compassion, and for being there for me. \\[5pt] Finally, thank you Nina. Your incredible love has kept me going through the most difficult and most wonderful times over the last four years. You are the light of my life, and I am so happy to have taken on this challenge with you by my side. \\[5pt] Arohanui and peace to you all, Eddyn (Ned)

\fancyhf{} %clear all headers and footers fields
\thispagestyle{fancy} % Change header and footer on this page
\renewcommand{\headrulewidth}{0pt}
\fancyhead[R]{\textit{Acknowledgements}} % Set header content
\fancyfoot[R]{\thepage} %prints the page number on the right side of the header

\clearpage
\newpage
\thispagestyle{empty} % Hide header and footer on this page
\mbox{~}
\clearpage
\newpage

\pagestyle{headings}

\ifdefined\Shaded\renewenvironment{Shaded}{\begin{tcolorbox}[breakable, enhanced, boxrule=0pt, borderline west={3pt}{0pt}{shadecolor}, frame hidden, interior hidden, sharp corners]}{\end{tcolorbox}}\fi

\renewcommand*\contentsname{Table of Contents}
{
\setcounter{tocdepth}{2}
\addcontentsline{toc}{chapter}{Table of Contents}
\tableofcontents
}
\listoffigures
\addcontentsline{toc}{chapter}{List of Figures}
\listoftables
\addcontentsline{toc}{chapter}{List of Tables}

\clearpage
\newpage
\thispagestyle{empty} % Hide header and footer on this page
\mbox{~}
\clearpage
\newpage

%----------------------------------------------
%   List of Abbreviations
%----------------------------------------------

\thispagestyle{plain}

\begin{flushleft}
% Manually add a section to the table of contents
\addcontentsline{toc}{chapter}{List of Abbreviations}
\huge\textbf{List of Abbreviations}
\end{flushleft}

\vspace*{\baselineskip}

\begin{table}[H]
  \begin{tabular}{@{}p{0.25\textwidth} p{0.75\textwidth}@{}}  % Adjust the width as needed
    2D  & 2-Dimensional  \\[5pt]
    Ab  & Antibody  \\[5pt]
    AB  & Amyl Butyrate  \\[5pt]
    AB-NTA  & N$\alpha$,N$\alpha$-Bis(carboxymethyl)-\textit{L}-lysine hydrate  \\[5pt]
    AFM  & Atomic Force Microscope/Microscopy  \\[5pt]
    AH  & Absolute Humidity  \\[5pt]
    Avi-tag  & Avidin-tag  \\[5pt]
    BMIM  & 1-butyl-3-methylimidazolium bis(trifluoromethylsulfonyl)imide  \\[5pt]
    BWF  & Breit-Wigner-Fano  \\[5pt]
    CAD  & Computer Aided Design \\[5pt]
    CNT  & Carbon Nanotube  \\[5pt]
    CVD  & Chemical Vapour Deposition  \\[5pt]
    Cy3  & Cyanine 3  \\[5pt]
    DAN  & 1,5-diaminonaphthalene  \\[5pt]
    DAQ  & Data Acquisition Input/Output Module  \\[5pt]
    DCB  & 1,2-dichlorobenzene  \\[5pt]
    DI  & Deionised  \\[5pt]
    DMF  & Dimethylformamide   \\[5pt]
    DMSO  & Dimethylsulfoxide   \\[5pt]
    DMT-MM   & 4-(4,6-dimethoxy-1,3,5-triazin-2-yl)-4 methylmorpholinium chloride \\[5pt]
    DMMP  & Dimethyl Methylphosphonate  \\[5pt]
    DNA  & Deoxyribonucleic Acid  \\[5pt]
    E2Hex  & \textit{trans}-2-hexan-1-al  \\[5pt]
    EB  & Ethyl Butyrate  \\[5pt]
    EDC  & 1-Ethyl-3-(3-dimethylaminopropyl)carbodiimide  \\[5pt]
    EDL  & Electric Double Layer  \\[5pt]
    EIS  & Electrochemical Impedance Spectroscopy  \\[5pt]
    EtHex  & Ethyl Hexanoate  \\[5pt]
  \end{tabular}
\end{table}

\newpage
\fancyhf{} %clear all headers and footers fields
\thispagestyle{fancy} % Change header and footer on this page
\renewcommand{\headrulewidth}{0pt}
\fancyhead[L]{\textit{List of Abbreviations}} % Set header content
\fancyfoot[L]{\thepage} %prints the page number on the right side of the header
\begin{table}[H]
  \begin{tabular}{@{}p{0.25\textwidth} p{0.75\textwidth}@{}}  % Adjust the width as needed
    EtOH  & Ethanol  \\[5pt]
    FET  & Field-Effect Transistor  \\[5pt]
    FITC  & Fluorescein isothiocyanate  \\[5pt]
    GA  & Glutaraldehyde  \\[5pt]
    GFET  & Graphene Field-Effect Transistor  \\[5pt]
    GFP  & Green Fluorescent Protein  \\[5pt]
    GPCR  & G-protein Coupled Receptor  \\[5pt]
    HEK  & Human Embryonic Kidney  \\[5pt]
    His-tag  & Histidine-tag  \\[5pt]
    hOR  & Human Odorant Receptor  \\[5pt]
    HPLC  & High-performance Liquid Chromatography   \\[5pt]
    iOR  & Insect Odorant Receptor  \\[5pt]
    IPA  & Isopropanol  \\[5pt]
    LOD  & Limit of Detection  \\[5pt]
    m-CNT  & Metallic Carbon Nanotube   \\[5pt]
    MeOH  & Methanol   \\[5pt]
    MeSal  & Methyl Salicylate   \\[5pt]
    MFC  & Mass Flow Controller   \\[5pt]
    mOR  & Mouse Odorant Receptor  \\[5pt]
    MOSFET  & Metal-Oxide-Semiconductor Field-Effect Transistor  \\[5pt]
    MSP  & Membrane Scaffold Protein  \\[5pt]
    MWCNT  & Multi-Walled Carbon Nanotube  \\[5pt]
    ND  & Nanodisc  \\[5pt]
    NHS  & N-Hydroxysuccinimide  \\[5pt]
    NHSS  & N-hydroxysulfosuccinimide   \\[5pt]
    NMR  & Nuclear Magnetic Resonance  \\[5pt]
    NSB  & Non-Specific Binding   \\[5pt]
    NTA  & Nitrilotriacetic Acid   \\[5pt]
    OBP  & Odorant Binding Protein  \\[5pt]
    OR  & Odorant Receptor  \\[5pt]
    ORCO  & Odorant Receptor Co-Receptor  \\[5pt]
    PBA  & 1-Pyrenebutyric Acid  \\[5pt]
  \end{tabular}
\end{table}

\newpage
\fancyhf{} %clear all headers and footers fields
\thispagestyle{fancy} % Change header and footer on this page
\renewcommand{\headrulewidth}{0pt}
\fancyhead[R]{\textit{List of Abbreviations}} % Set header content
\fancyfoot[R]{\thepage} %prints the page number on the right side of the header
\begin{table}[H]
  \begin{tabular}{@{}p{0.25\textwidth} p{0.75\textwidth}@{}}  % Adjust the width as needed
    PBASE  & 1-Pyrenebutanoic Acid N-hydroxysuccinimide Ester  \\[5pt]
    PBS  & Phosphate-Buffered Saline  \\[5pt]
    PCB  & Printed Circuit Board   \\[5pt]
    PDL & Poly-\textit{D}-lysine  \\[5pt]
    PDMS  & Polydimethylsiloxane   \\  [5pt]
    PEG  & Polyethylene Glycol  \\[5pt] 
    PID  & Photoionisation Detector  \\[5pt]
    P\&ID & Process \& Instrumentation Diagram  \\[5pt]
    PLL  & Poly-\textit{L}-lysine  \\[5pt]
    PPB  & Pyrene-PEG-Biotin  \\[5pt]
    PPF  & Pyrene-PEG-FITC  \\[5pt]
    PPN  & Pyrene-PEG-NTA  \\[5pt]
    PPR  & Pyrene-PEG-Rhodamine  \\[5pt]
    PTFE  & Polytetrafluoroethylene (Teflon™)  \\[5pt]
    PVC  & Polyvinyl chloride  \\[5pt]
    QCM  & Quartz Crystal Microbalance  \\[5pt]
    RH  & Relative Humidity  \\[5pt]
    RHI  & Relative Humidity and Temperature Indicator  \\[5pt] 
    RNA  & Ribonucleic Acid   \\[5pt]
    SAW  & Surface Acoustic Wave   \\[5pt]
    s-CNT  & Semiconducting Carbon Nanotube   \\[5pt]
    SEM  & Scanning Electron Microscope/Microscopy   \\[5pt]
    SMU  & Source Measure Unit   \\[5pt]
    SPR  & Surface Plasmon Resonance   \\[5pt]
    SWCNT  & Single-Walled Carbon Nanotube   \\[5pt]
    TFTFET  & Thin-Film Field-Effect Transistor  \\[5pt]
    TMAH  & Tetramethylammonium hydroxide  \\[5pt]
    TX  & Transfer Characteristics  \\[5pt]
    UV  & Ultraviolet  \\[5pt]
    VI  & Virtual Instrument  \\[5pt]
    VUAA1  & N-(4-Ethylphenyl)-2-{[4-ethyl-5-(pyridin-3-yl)-4H-1,2,4-triazol-3-yl]sulfanyl}acetamide  \\[5pt] 
  \end{tabular}
\end{table}

\clearpage
\newpage
\thispagestyle{empty} % Hide header and footer on this page
\mbox{~}
\clearpage
\newpage

% Adjust the top and bottom margins of float pages to center floats
\makeatletter
\setlength{\@fptop}{0pt plus 1fil}
\setlength{\@fpbot}{0pt plus 1fil}
\makeatother

\pagestyle{headings}
\mainmatter
\bookmarksetup{startatroot}

\hypertarget{introduction}{%
\chapter{Introduction}\label{introduction}}

\hypertarget{background}{%
\section{Background}\label{background}}

The `bioelectronic nose', an electronic transducer modified with
elements of the animal olfactory system, has the potential to allow
specific detection of airborne volatile compounds at concentrations as
low as parts per trillion
\autocite{Glatz2011,Kwon2015,Dung2018,Kim2022a}. An ideal transducer
platform is the thin-film transistor (TFT) which is particularly
portable, simple to use, small and robust
\autocite{Kauffman2008,Khan2020}. The thin films used in these
field-effect transistors (FETs) include carbon nanotube networks and
graphene, low-dimensional nanomaterials which are both highly sensitive
and biocompatible \autocite{Shkodra2021}. The implications of successful
development of such a portable and robust bioelectronic nose are
significant. Applications could be found in high-importance fields such
as biosecurity, medicine, environmental protection and food safety
\autocite{Dung2018,Arakawa2019,Yang2017,Son2017}. For example, it has
been demonstrated that it is possible to detect invasive brown
marmorated stinkbugs based on their volatile trace \autocite{Moser2020}.
A bioelectronic nose could potentially accomplish this biosecurity task
far more cheaply and efficiently than trained sniffer dogs
\autocite{Lee2010,Moon2020,Terutsuki2020}. There has been rapid progress
in the development of bioelectronic noses using carbon nanotube
field-effect transistors (CNT FETs) and graphene field-effect
transistors (GFETs) over the past 15-20 years
\autocite{Yoon2009,Lee2010,Yang2018}.

Insect odorant receptors (iORs) enable simple invertebrates, such as the
vinegar fruit fly \emph{Drosophila melanogaster}, to distinguish between
a huge number of specific volatile compounds
\autocite{Hallem2004,Smart2008,Wicher2008,Munch2016,Bohbot2020}. Within
the past five years, a variety of \emph{Drosophila melanogaster} iORs
have been successfully coupled with highly sensitive low-dimensional
thin-film transistors (TFTs) for specific detection of fruit-like odors
in an aqueous environment \autocite{Murugathas2019a,Murugathas2020}.
iORs have also been used for sensitive and selective volatile detection
in a lipid bilayer format, but not in a portable bioelectronic nose
format \autocite{Yamada2021}. In this thesis, my aim was to investigate
whether a bioelectronic nose capable of odorant detection in a
vapour-phase environment could be constructed by coupling iORs with
TFTs. Alongside practical applications, development of a vapour-phase
bioelectronic nose using iORs may give us a greater understanding of the
mechanisms underlying insect olfaction \autocite{Lee2010}. The
transduction mechanism of nanomaterial-based iOR sensors is still
unknown, and I hope to shed further light on the biological and
electronic processes underpinning this mechanism
\autocite{Murugathas2020,Khadka2019,Cheema2021}.

\hypertarget{thesis-outline}{%
\section{Thesis Outline}\label{thesis-outline}}

This thesis consists of nine chapters. The first three chapters,
including this one, are background chapters introducing the general
topics of this thesis. The fourth and fifth chapters are methods
chapters, while the next three chapters (sixth, seventh and eight)
describe the results obtained. The ninth chapter concludes the thesis
and discusses possible next steps for future research.

\textbf{Chapter 2} gives a broad description of carbon nanotube and
graphene field-effect transistors with a focus on their use in sensing
applications. The chapter begins by looking at the general structure and
properties of thin-film transistors, where key figures of merit such as
transconductance, on-off ratio, gate current and hysteresis are
described. Graphene field-effect transistors (GFETs) and carbon nanotube
network field-effect transistors (CNT FETs) are then discussed in
greater detail. These descriptions include the chemical composition of
each nanomaterial, their conduction behaviour and their unique sensor
properties when integrated into a field-effect transistor as a
thin-film.

\textbf{Chapter 3} investigates existing odorant receptor-coupled
thin-film field-effect transistors in the literature. First, the
biological structure of odorant receptors and membrane formats for their
protection \emph{in vitro} are discussed. Details are then provided
regarding the construction and operation of existing vertebrate odorant
receptor TFT biosensors. The structure and function of the insect
odorant receptor is then contrasted with the vertebrate odorant
receptor, and existing insect odorant receptor TFT biosensors in the
literature are discussed. The chapter finishes with a brief discussion
of non-specific binding and its role in hindering biosensor activity.

\textbf{Chapter 4} describes the fabrication of the CNT FET and GFET
transducers used in this thesis and the characterisation techniques used
to probe their behaviour. The chapter starts with an introduction to
photolithography for thin-film transistor device fabrication. Various
techniques are described for random deposition of carbon nanotube
networks to act as channels for these thin-film transistors.
Characterisation techniques described in this chapter include atomic
force microscopy (AFM), fluorescence microscopy, Raman spectroscopy and
electrical characterisation with various semiconductor device analysers.

\textbf{Chapter 5} presents the results obtained from the use of
characterisation techniques on the pristine GFETs and CNT FETs. Various
carbon nanotube (CNT) network morphologies are displayed and analysed.
The Raman spectra and electrical device parameters of these CNT network
morphologies are then discussed, along with electrical parameters from
graphene devices. The sensitivity of a dense CNT network morphology
device is then verified in the aqueous phase.

\textbf{Chapter 6} explores the non-covalent functionalisation of GFETs
and CNT FETs with various linker molecules for insect odorant receptor
attachment. The linker molecules tested were 1-pyrenebutanoic acid
N-hydroxysuccinimide ester (PBASE) and 1-pyrenebutyric acid (PBA) with
1-Ethyl-3-(3-dimethylaminopropyl)carbodiimide (EDC). Pyrene-NTA and
pyrene-biotin were also investigated as other possible linker molecules.
The quality of various functionalisation approaches was then explored
with various fluorescently-tagged linker molecules and biomolecules. In
this process, various potential obstacles to successful biosensor
functionalisation were identified.

\textbf{Chapter 7} maps out progress made towards the creation of an
insect odorant receptor functionalised TFT biosensor for use in a
vapour-phase environment. Two different approaches are described that
gave rise to working aqueous-phase biosensors. The first
functionalisation approach, which used PBASE in methanol, led to
irreproducible results when biosensing. Possible factors causing the
unreliability of this method were then investigated. A second approach
was then designed to avoid the malign influence of any identified
factors.

\textbf{Chapter 8} outlines the development of a vapour delivery system
for characterisation of the insect odorant receptor functionalised TFT
biosensors in a vapour-phase environment. The vapour delivery system was
upgraded from an existing system to include new mass flow controllers,
to have greater control of flow through the system, and off the shelf
vapour sensors, to collect vapour flow data that could be used for
comparison against biosensor activity. The chapter also describes the
design and construction of an electronic interface to monitor and
control the components of the vapour delivery system, and calibration of
the system.

\textbf{Chapter 9} details the use of the vapour delivery system for
testing the functionalised biosensors in the vapour phase. First, the
flow behaviour of volatile organic vapours through the system was
validated using onboard reference sensors. The response of a pristine
carbon nanotube device to two volatile compounds is then compared to the
response of carbon nanotube devices functionalised using the second
functionalisation approach discussed in the previous chapter.

\textbf{Chapter 10} summarises the conclusions drawn from this work, and
proposes various related studies which can be undertaken to continue the
work described in this thesis.

\cleardoublepage
\phantomsection
\addcontentsline{toc}{part}{Appendices}
\appendix

\hypertarget{vapour-system-hardware}{%
\chapter{Vapour System Hardware}\label{vapour-system-hardware}}

\hypertarget{tbl-vapour-sensor-components}{}
\begin{longtable}[]{@{}
  >{\raggedright\arraybackslash}p{(\columnwidth - 4\tabcolsep) * \real{0.5930}}
  >{\raggedright\arraybackslash}p{(\columnwidth - 4\tabcolsep) * \real{0.2209}}
  >{\raggedright\arraybackslash}p{(\columnwidth - 4\tabcolsep) * \real{0.1860}}@{}}
\caption{\label{tbl-vapour-sensor-components}Major components used in
construction of the vapour delivery system described in this
thesis.}\tabularnewline
\toprule\noalign{}
\begin{minipage}[b]{\linewidth}\raggedright
Description
\end{minipage} & \begin{minipage}[b]{\linewidth}\raggedright
Part No.
\end{minipage} & \begin{minipage}[b]{\linewidth}\raggedright
Manufacturer
\end{minipage} \\
\midrule\noalign{}
\endfirsthead
\toprule\noalign{}
\begin{minipage}[b]{\linewidth}\raggedright
Description
\end{minipage} & \begin{minipage}[b]{\linewidth}\raggedright
Part No.
\end{minipage} & \begin{minipage}[b]{\linewidth}\raggedright
Manufacturer
\end{minipage} \\
\midrule\noalign{}
\endhead
\bottomrule\noalign{}
\endlastfoot
Mass flow controller, 20 sccm full scale & GE50A-013201SBV020 & MKS
Instruments \\
Mass flow controller, 200 sccm full scale & GE50A-013202SBV020 & MKS
Instruments \\
Mass flow controller, 500 sccm full scale & FC-2901V & Tylan \\
Analogue flowmeter, 240 sccm max. flow & 116261-30 & Dwyer \\
Micro diaphragm pump & P200-B3C5V-35000 & Xavitech \\
Analogue flow controller, for micro diaphragm pump & X3000450 &
Xavitech \\
10 mL Schott bottle & 218010802 & Duran \\
PTFE connection cap system & Z742273 & Duran \\
Baseline VOC-TRAQ flow cell, purple & 043-950 & Ametek Mocon \\
Baseline VOC-TRAQ flow cell, red & 043-951 & Ametek Mocon \\
Humidity and temperature sensor & T9602-5-A & Telaire \\
\end{longtable}

\hypertarget{python-code-for-data-analysis}{%
\chapter{Python Code for Data
Analysis}\label{python-code-for-data-analysis}}

\hypertarget{code-repository}{%
\section{Code Repository}\label{code-repository}}

The code used for general analysis of field-effect transistor devices in
this thesis was written with Python 3.8.8. Contributors to the code used
include Erica Cassie, Erica Happe, Marissa Dierkes and Leo Browning. The
code is located on GitHub and the research group OneDrive, and is
available on request.

\hypertarget{sec-histogram-analysis}{%
\section{Atomic Force Microscope Histogram
Analysis}\label{sec-histogram-analysis}}

The purpose of this code is to analyse atomic force microscope (AFM)
images of carbon nanotube networks in .xyz format taken using an atomic
force microscope and processed in Gwyddion (see
\textbf{?@sec-afm-characterisation}). It was originally designed by
Erica Happe in Matlab, and adapted by Marissa Dierkes and myself for use
in Python. The code imports the .xyz data and sorts it into bins 0.15 nm
in size for processing. To perform skew-normal distribution fits, both
\emph{scipy.optimize.curve\_fit} and \emph{scipy.stats.skewnorm} modules
are used in this code.

\hypertarget{sec-raman-analysis}{%
\section{Raman Spectroscopy Analysis}\label{sec-raman-analysis}}

The purpose of this code is to analyse a series of Raman spectra taken
at different points on a single film (see
\textbf{?@sec-raman-characterisation}). Data is imported in a series of
tab-delimited text files, with the low wavenumber spectrum (100
cm\(^{-1} - 650\) cm\(^{-1}\)) and high wavenumber spectrum (1300
cm\(^{-1} - 1650\) cm\(^{-1}\)) imported in separate datafiles for each
scan location.

\hypertarget{sec-field-effect-transistor-analysis}{%
\section{Field-Effect Transistor
Analysis}\label{sec-field-effect-transistor-analysis}}

The purpose of this code is to analyse electrical measurements taken of
field-effect transistor (FET) devices. Electrical measurements were
either taken from the Keysight 4156C Semiconductor Parameter Analyser,
National Instruments NI-PXIe or Keysight B1500A Semiconductor Device
Analyser as discussed in \textbf{?@sec-electrical-characterisation}; the
code is able to analyse data in .csv format taken from all three
measurement setups. The main Python file in the code base consists of
three related but independent modules: the first analyses and plots
sensing data from the FET devices, the second analyses and plots
transfer characteristics from channels across a device, and the third
compares individual channel characteristics before and after a
modification or after each individual modification in a series of
modifications. The code base also features a separate config file and
style sheet which govern the behaviour of the main code. The code base
was designed collaboratively by myself and Erica Cassie over GitHub
using the Sourcetree Git GUI.

\hypertarget{references}{%
\chapter*{References}\label{references}}
\addcontentsline{toc}{chapter}{References}

\markboth{References}{References}

\printbibliography[heading=none]


\backmatter

\end{document}
