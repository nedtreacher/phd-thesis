% Options for packages loaded elsewhere
\PassOptionsToPackage{unicode}{hyperref}
\PassOptionsToPackage{hyphens}{url}
%
\documentclass[
  a4paper,
]{scrbook}

\usepackage{amsmath,amssymb}
\usepackage{setspace}
\usepackage{iftex}
\ifPDFTeX
  \usepackage[T1]{fontenc}
  \usepackage[utf8]{inputenc}
  \usepackage{textcomp} % provide euro and other symbols
\else % if luatex or xetex
  \usepackage{unicode-math}
  \defaultfontfeatures{Scale=MatchLowercase}
  \defaultfontfeatures[\rmfamily]{Ligatures=TeX,Scale=1}
\fi
\usepackage{lmodern}
\ifPDFTeX\else  
    % xetex/luatex font selection
  \setmainfont[]{Latin Modern Roman}
  \setsansfont[]{Latin Modern Roman}
\fi
% Use upquote if available, for straight quotes in verbatim environments
\IfFileExists{upquote.sty}{\usepackage{upquote}}{}
\IfFileExists{microtype.sty}{% use microtype if available
  \usepackage[]{microtype}
  \UseMicrotypeSet[protrusion]{basicmath} % disable protrusion for tt fonts
}{}
\makeatletter
\@ifundefined{KOMAClassName}{% if non-KOMA class
  \IfFileExists{parskip.sty}{%
    \usepackage{parskip}
  }{% else
    \setlength{\parindent}{0pt}
    \setlength{\parskip}{6pt plus 2pt minus 1pt}}
}{% if KOMA class
  \KOMAoptions{parskip=half}}
\makeatother
\usepackage{xcolor}
\setlength{\emergencystretch}{3em} % prevent overfull lines
\setcounter{secnumdepth}{5}
% Make \paragraph and \subparagraph free-standing
\ifx\paragraph\undefined\else
  \let\oldparagraph\paragraph
  \renewcommand{\paragraph}[1]{\oldparagraph{#1}\mbox{}}
\fi
\ifx\subparagraph\undefined\else
  \let\oldsubparagraph\subparagraph
  \renewcommand{\subparagraph}[1]{\oldsubparagraph{#1}\mbox{}}
\fi


\providecommand{\tightlist}{%
  \setlength{\itemsep}{0pt}\setlength{\parskip}{0pt}}\usepackage{longtable,booktabs,array}
\usepackage{calc} % for calculating minipage widths
% Correct order of tables after \paragraph or \subparagraph
\usepackage{etoolbox}
\makeatletter
\patchcmd\longtable{\par}{\if@noskipsec\mbox{}\fi\par}{}{}
\makeatother
% Allow footnotes in longtable head/foot
\IfFileExists{footnotehyper.sty}{\usepackage{footnotehyper}}{\usepackage{footnote}}
\makesavenoteenv{longtable}
\usepackage{graphicx}
\makeatletter
\def\maxwidth{\ifdim\Gin@nat@width>\linewidth\linewidth\else\Gin@nat@width\fi}
\def\maxheight{\ifdim\Gin@nat@height>\textheight\textheight\else\Gin@nat@height\fi}
\makeatother
% Scale images if necessary, so that they will not overflow the page
% margins by default, and it is still possible to overwrite the defaults
% using explicit options in \includegraphics[width, height, ...]{}
\setkeys{Gin}{width=\maxwidth,height=\maxheight,keepaspectratio}
% Set default figure placement to htbp
\makeatletter
\def\fps@figure{htbp}
\makeatother

\usepackage{booktabs}
\usepackage{longtable}
\usepackage{array}
\usepackage{multirow}
\usepackage{wrapfig}
\usepackage{float}
\usepackage{colortbl}
\usepackage{pdflscape}
\usepackage{tabu}
\usepackage{threeparttable}
\usepackage{threeparttablex}
\usepackage[normalem]{ulem}
\usepackage{makecell}
\usepackage{xcolor}
\usepackage{fancyhdr}
\usepackage{textcomp}
\usepackage{titling}
\usepackage{pdflscape}
\usepackage{rotating}
\usepackage{geometry}
\usepackage{setspace}
\setlength{\droptitle}{-2cm}
\preauthor{
  \begin{center}
  \Large
  \vspace{15mm}
  by
  \vspace{10mm}
  
}
\postauthor{
  \end{center}
  
}

\predate{
  \begin{spacing}{1.2}
  \begin{center}
  \vspace{22mm}
  
  A thesis \\
  submitted to the Victoria University of Wellington \\
  in partial fulfilment of the requirements for the  \\
  degree of Doctor of Philosophy\\               % Degree
  \vspace{24mm}
  Te Herenga Waka $-$ Victoria University of Wellington\\
}
\postdate{
  \\
  \includegraphics[width=3in,height=1.5in]{figures/VUW-logo.png}\\
  \end{center}
  \end{spacing}
  }

\renewcommand{\topfraction}{.8}
\renewcommand{\bottomfraction}{.7}
\renewcommand{\textfraction}{.15}
\renewcommand{\floatpagefraction}{.8}
\setcounter{topnumber}{3}
\setcounter{bottomnumber}{3}
\setcounter{totalnumber}{4}

\clubpenalty=9996
\widowpenalty=9999
\makeatletter
\makeatother
\makeatletter
\@ifpackageloaded{bookmark}{}{\usepackage{bookmark}}
\makeatother
\makeatletter
\@ifpackageloaded{caption}{}{\usepackage{caption}}
\AtBeginDocument{%
\ifdefined\contentsname
  \renewcommand*\contentsname{Table of contents}
\else
  \newcommand\contentsname{Table of contents}
\fi
\ifdefined\listfigurename
  \renewcommand*\listfigurename{List of Figures}
\else
  \newcommand\listfigurename{List of Figures}
\fi
\ifdefined\listtablename
  \renewcommand*\listtablename{List of Tables}
\else
  \newcommand\listtablename{List of Tables}
\fi
\ifdefined\figurename
  \renewcommand*\figurename{Figure}
\else
  \newcommand\figurename{Figure}
\fi
\ifdefined\tablename
  \renewcommand*\tablename{Table}
\else
  \newcommand\tablename{Table}
\fi
}
\@ifpackageloaded{float}{}{\usepackage{float}}
\floatstyle{ruled}
\@ifundefined{c@chapter}{\newfloat{codelisting}{h}{lop}}{\newfloat{codelisting}{h}{lop}[chapter]}
\floatname{codelisting}{Listing}
\newcommand*\listoflistings{\listof{codelisting}{List of Listings}}
\makeatother
\makeatletter
\@ifpackageloaded{caption}{}{\usepackage{caption}}
\@ifpackageloaded{subcaption}{}{\usepackage{subcaption}}
\makeatother
\makeatletter
\@ifpackageloaded{tcolorbox}{}{\usepackage[skins,breakable]{tcolorbox}}
\makeatother
\makeatletter
\@ifundefined{shadecolor}{\definecolor{shadecolor}{rgb}{.97, .97, .97}}
\makeatother
\makeatletter
\makeatother
\makeatletter
\makeatother
\ifLuaTeX
  \usepackage{selnolig}  % disable illegal ligatures
\fi
\usepackage[citestyle = ieee,urldate = iso8601]{biblatex}
\addbibresource{references.bib}
\IfFileExists{bookmark.sty}{\usepackage{bookmark}}{\usepackage{hyperref}}
\IfFileExists{xurl.sty}{\usepackage{xurl}}{} % add URL line breaks if available
\urlstyle{same} % disable monospaced font for URLs
\hypersetup{
  pdftitle={Developing an Insect Odorant Receptor Bioelectronic Nose for Vapour-Phase Detection},
  pdfauthor={Eddyn Oswald Perkins Treacher},
  hidelinks,
  pdfcreator={LaTeX via pandoc}}

\title{Developing an Insect Odorant Receptor Bioelectronic Nose for
Vapour-Phase Detection}
\author{Eddyn Oswald Perkins Treacher}
\date{Dec 2024}

\begin{document}
\frontmatter

\maketitle

\begin{spacing}{1.2}

\clearpage
\newpage
\thispagestyle{empty} % Hide header and footer on this page
\mbox{~}
\clearpage
\newpage

%----------------------------------------------
%   Abstract
%----------------------------------------------

\thispagestyle{plain}

\begin{flushleft}
% Manually add a section to the table of contents
\pagenumbering{roman}
\addcontentsline{toc}{chapter}{Abstract}
\huge\textbf{Abstract}
\end{flushleft}

\vspace*{\baselineskip}

The ability to detect volatile organic compounds in a highly sensitive and selective manner could be used for applications as varied as diagnoses of illnesses at a remote clinic, monitoring of air in an industrial setting, or identification of invasive organisms at a biosecurity checkpoint. Historically, animal noses have been used for such tasks, as their combined sensitivity and selectivity are superior to traditional artificial sensors. However, training and deploying animals in such situations is both time and cost intensive. In recent years, an improved understanding of \textit{in vivo} biological sensing has driven efforts to mimic these highly efficient processes in an artificial sensor format. \\[5pt] To this end, a ``bioelectronic nose'' was developed. This sensor uses an artificial transducer to amplify responses of an insect odorant receptor protein to specific volatile compounds. Thin-film transistors were used as the amplifier element, given their low cost, small size and extreme sensitivity. Various thin-film morphologies were compared, and their suitability for bioelectronic nose development assessed. Transducers made using a novel steam-assisted thin-film deposition technique were found to have highly consistent device-to-device electrical properties relative to other films. Films made using this process typically showed more surface contamination than other morphologies, but their high sensitivity was confirmed with a non-specific sensing series in an aqueous environment. \\[5pt] One of the major challenges encountered in this thesis was variability in the quality of sensor functionalisation. Raman spectroscopy and fluorescence microscopy were used to confirm an existing non-covalent attachment method could successfully immobilise nanodiscs onto the transistor channel region. However, various sensors functionalised using the same procedure often exhibited no sensing activity. Extensive electrical characterisation indicated the presence of an unidentified contamination layer which prevented electrical interaction between the insect odorant receptors and the transducer thin-film. It was shown that this layer was unlikely to be directly associated with the thin-film morphology used for the transducer. \\[5pt] Subsequently, an alternative biotin-based non-covalent method was used for functionalisation of the proteins, which eliminated several possible sources of contamination. This alternative biotin-based method was used to demonstrate successful aqueous sensing \newpage
\fancyhf{} %clear all headers and footers fields
\thispagestyle{fancy} % Change header and footer on this page
\renewcommand{\headrulewidth}{0pt}
\fancyhead[L]{\textit{Abstract}} % Set header content
\fancyfoot[L]{\thepage} %prints the page number on the left side of the header
of femtomolar concentrations of methyl salicylate by an iOR10a-functionalised device. When tested in a custom-built vapour delivery system, a similar bioelectronic sensor was shown to be highly sensitive to the target vapour. However, consistent reproduction of the biotin-based method was challenging due to the harsh cleaning method involved. It was therefore difficult to determine conclusively whether the vapour-phase sensor responses were selective. By finding new, systematic approaches to address the barriers to sensor success carefully identified in this work, there are promising signs that a highly reliable vapour-phase bioelectronic nose can be produced.

\clearpage
\newpage

%----------------------------------------------
%   Acknowledgement
%----------------------------------------------

\thispagestyle{plain}

\begin{flushleft}
% Manually add a section to the table of contents
\addcontentsline{toc}{chapter}{Acknowledgements}
\huge\textbf{Acknowledgements}
\end{flushleft}

\vspace*{\baselineskip}

I would first like to acknowledge the lands of my ancestors, and the lands of the sovereign first peoples to which my ancestors travelled. We each come from the land, live off the land and return to the land.
\end{spacing}
\begin{spacing}{1.2} 
\textit{Noon of Essex to Warrang, on the Friends, Autumn 1811} \\[5pt]
\textit{Cave of Cambridgeshire to Warrang, on the Royal Charlotte, Autumn 1825} \\[5pt]
\textit{Boyce of Suffolk to Warrang, 1832} \\[5pt] 
\textit{Charlton of Northumberland to Warrang, on the Clyde, Spring 1834} \\[5pt]
\textit{Prouse of Devonshire to Pito-one, on the Duke of Roxburgh, Summer 1840} \\[5pt]
\textit{Ebden of Devonshire to Pito-one, on the Tyne, Winter 1841} \\[5pt]
\textit{Collis of Hampshire to Pito-one, on the Birman, Autumn 1842} \\[5pt]
\textit{Swann of Loch Garman to Te Whanganui-a-Tara, 1844} \\[5pt] 
\textit{Blythe of Berkshire to Whakatū, circa 1846} \\[5pt]
\textit{Innes of Berkshire to Naarm, on the Sacramento, Autumn 1853} \\[5pt]
\textit{Sheppard of Gloucestershire to Naarm, 1853} \\[5pt] 
\textit{Bruce of London to Naarm, on the Omega, Autumn 1855} \\[5pt]
\textit{Quennell of Surrey to Warrang, on the Asiatic, Winter 1855} \\[5pt]
\textit{Barr of Glasgow to Kōpūtai, on the Sir Edward Paget, Winter 1856} \\[5pt] 
\textit{Perkins of London to Te Whanganui-a-Tara, on the Matoaka, Spring 1859} \\[5pt]
\textit{McKee of Antrim to Tāmaki Makaurau, on the Indian Empire, Spring 1862} \\[5pt]
\textit{Sandilands of Peeblesshire to Ōtepoti, circa 1864} \\[5pt] 
\textit{Treacher of Berkshire to Te Whanganui-a-Tara, on the Wild Duck, Summer 1865} \\[5pt]
\textit{McTaggart of Argyllshire to Kōpūtai, on the Edward P. Bouverie, Autumn 1869} \\[5pt] 
\textit{Chapman of Kent to Whakatū, on the Adamant, Winter 1874} \\[5pt]
\textit{Cheel of London to Whakatū, on the Queen Bee, Winter 1877} \\[5pt]  
\textit{Hutchison of Aberdeen to Tarntanya, before 1882.}
\end{spacing}
\begin{spacing}{1.2} 
\newpage
\fancyhf{} %clear all headers and footers fields
\thispagestyle{fancy} % Change header and footer on this page
\renewcommand{\headrulewidth}{0pt}
\fancyhead[L]{\textit{Acknowledgements}} % Set header content
\fancyfoot[L]{\thepage} %prints the page number on the left side of the header 
I chose to start my doctoral studies just a few months into a global pandemic. Completing a challenging project with a worldwide crisis in the background might have been impossible without the supervision of AProf. Natalie Plank. Her ability to adapt to and overcome any problem has taught me that there is no situation which is truly unmanageable. I am deeply grateful for her leadership throughout a time of particular chaos. \\[5pt] I started this project with minimal formal training in biological science, coming from a primarily physics and engineering background. The immense support I received from Melissa Jordan and Colm Carraher from the Institute for Plant and Food Research (PFR) to complete this project meant that this was not an issue, and I thank them both for this. \\[5pt] I would not have been able to begin this thesis without the financial backing and support I received from PFR and the Better Border Biosecurity (B3) programme. In particular, I am very grateful to Andrew Kralicek, formerly with PFR and now at Scentian Bio, and the ex-Director of B3, David Teulon, for helping to secure funding for my project. I would also like to thank the donor of the Ernest Marsden Scholarship in Physics for their significant financial support. \\[5pt] There are many incredibly supportive people who I worked alongside during my project. I would like to start off by thanking Rifat Ullah, whose mentoring and kindness encouraged me to pursue further study. His work on the initial design and setup of the vapour delivery system was invaluable to me throughout this project. I am also especially grateful to Alex Puglisi, for constructing the mechanical elements of the vapour delivery system and giving me extensive feedback on the system design. I would like to thank Peter Coard, for his advice and guidance when constructing the electrical elements of the vapour delivery system. I thank Selvan Murugathas, too, for his advice on constructing the insect odorant receptor sensors, as well as Damon Colbert and Valentina Lucarelli, who provided the insect odorant receptor nanodiscs used in this work. \\[5pt] Thank you to Prof. Ben Ruck, my supportive secondary supervisor, and to AProf. Franck Natali, for always asking about my thesis in the tearoom. Thank you to Gideon Gouws for his friendly encouragement and advice. For their substantial technical assistance and mentoring during this project, I thank Alan Rennie, Grant Franklin, Chris Lepper, Rashika Gunasekara, Pete Jebson and Sushila Pillai from VUW, Andrew Chan from PFR, AProf. Charles Unsworth from the University of Auckland, and Prof. Simon Brown and his nanomaterials group from the University of Canterbury. \\[5pt] I was lucky enough to start my doctoral program just as a group of supportive and talented senior students were finishing, and finished just as a group of enthusiastic and talented new doctoral students were starting. A special thanks to Jenna Nyugen, Erica Happe and Erica Cassie for teaching me the fabrication processes and characterisation procedures that made this thesis happen; and a special thanks to Marissa Dierkes, Danica Fontein, Sangar Begzaad and Alireza Zare, for their incredible support throughout the thesis writing process. I am also thankful for the assistance of the cleanroom group interns over the course of my PhD, including Liam, Hayden and Lotte. I would further like to thank everyone else I shared an office with and worked alongside, including Jackson, Will, Roshni, Ali, Sam, Kira, Catherine, Martin, Janani, Ted, Kiri and Joe. \\[5pt] A massive thank you to Openstar Technologies. It has been an honour to work on a cutting-edge plasma physics project right here in Te Whanganui-a-Tara. A particularly big thank you to Ratu, Darren and Thomas for having me as part of the plasma physics team. Thank you also to the other Openstar interns, in particular the other plasma physics interns, Valentina, Benjy and Chris. I wish you success in all your dipole-confined plasma related endeavours. \\[5pt] I want to thank Shodokan Aikido New Zealand for their support throughout this thesis, in particular for the once-in-a-lifetime opportunity to travel to Osaka to be graded for first-dan by Nariyama Shihan. Thanks for all the training and support, Ian. \\[5pt] Thank you to all the friends and whānau, old and new, who have supported me over these wild past few years. You know who you are. \\[5pt] Thank you to my brother, Keeson, and to my parents, Hilary and Phillip. Your support means everything to me, and I would not be where I am today without you. Our Friday lunchtime cafe visits kept me motivated and inspired throughout the doctoral program. Thank you, thank you, thank you for your love, your compassion, and for being there for me. \\[5pt] Finally, thank you Nina. Your incredible love has kept me going through the most difficult and most wonderful times over the last four years. You are the light of my life, and I am so happy to have taken on this challenge with you by my side. \\[5pt] Arohanui and peace to you all, Eddyn (Ned)

\fancyhf{} %clear all headers and footers fields
\thispagestyle{fancy} % Change header and footer on this page
\renewcommand{\headrulewidth}{0pt}
\fancyhead[R]{\textit{Acknowledgements}} % Set header content
\fancyfoot[R]{\thepage} %prints the page number on the right side of the header

\clearpage
\newpage
\thispagestyle{empty} % Hide header and footer on this page
\mbox{~}
\clearpage
\newpage

\pagestyle{headings}

\end{spacing}\ifdefined\Shaded\renewenvironment{Shaded}{\begin{tcolorbox}[borderline west={3pt}{0pt}{shadecolor}, enhanced, interior hidden, boxrule=0pt, breakable, sharp corners, frame hidden]}{\end{tcolorbox}}\fi

\begin{spacing}{1.2}

\renewcommand*\contentsname{Table of Contents}
{
\setcounter{tocdepth}{2}
\addcontentsline{toc}{chapter}{Table of Contents}
\tableofcontents
}
\listoffigures
\addcontentsline{toc}{chapter}{List of Figures}
\listoftables
\addcontentsline{toc}{chapter}{List of Tables}

\clearpage
\newpage

%----------------------------------------------
%   List of Abbreviations
%----------------------------------------------

\thispagestyle{plain}

\begin{flushleft}
% Manually add a section to the table of contents
\addcontentsline{toc}{chapter}{List of Abbreviations}
\huge\textbf{List of Abbreviations}
\end{flushleft}

\vspace*{\baselineskip}

\begin{table}[H]
  \begin{tabular}{@{}p{0.25\textwidth} p{0.75\textwidth}@{}}  % Adjust the width as needed
    2D  & 2-Dimensional  \\[5pt]
    Ab  & Antibody  \\[5pt]
    AB  & Amyl Butyrate  \\[5pt]
    AB-NTA  & N$\alpha$,N$\alpha$-Bis(carboxymethyl)-\textit{L}-lysine hydrate  \\[5pt]
    AFM  & Atomic Force Microscope/Microscopy  \\[5pt]
    AH  & Absolute Humidity  \\[5pt]
    Avi-tag  & Avidin-tag  \\[5pt]
    BMIM  & 1-butyl-3-methylimidazolium bis(trifluoromethylsulfonyl)imide  \\[5pt]
    BWF  & Breit-Wigner-Fano  \\[5pt]
    CAD  & Computer Aided Design \\[5pt]
    CNT  & Carbon Nanotube  \\[5pt]
    CVD  & Chemical Vapour Deposition  \\[5pt]
    Cy3  & Cyanine 3  \\[5pt]
    DAN  & 1,5-diaminonaphthalene  \\[5pt]
    DAQ  & Data Acquisition Input/Output Module  \\[5pt]
    DCB  & 1,2-dichlorobenzene  \\[5pt]
    DI  & Deionised  \\[5pt]
    DMF  & Dimethylformamide   \\[5pt]
    DMSO  & Dimethylsulfoxide   \\[5pt]
    DMT-MM   & 4-(4,6-dimethoxy-1,3,5-triazin-2-yl)-4 methylmorpholinium chloride \\[5pt]
    DMMP  & Dimethyl Methylphosphonate  \\[5pt]
    DNA  & Deoxyribonucleic Acid  \\[5pt]
    E2Hex  & \textit{trans}-2-hexan-1-al  \\[5pt]
    EB  & Ethyl Butyrate  \\[5pt]
  \end{tabular}
\end{table}

\newpage
\fancyhf{} %clear all headers and footers fields
\thispagestyle{fancy} % Change header and footer on this page
\renewcommand{\headrulewidth}{0pt}
\fancyhead[L]{\textit{List of Abbreviations}} % Set header content
\fancyfoot[L]{\thepage} %prints the page number on the right side of the header
\begin{table}[H]
  \begin{tabular}{@{}p{0.25\textwidth} p{0.75\textwidth}@{}}  % Adjust the width as needed
    EDC  & 1-Ethyl-3-(3-dimethylaminopropyl)carbodiimide  \\[5pt]
    EDL  & Electric Double Layer  \\[5pt]
    EIS  & Electrochemical Impedance Spectroscopy  \\[5pt]
    EtHex  & Ethyl Hexanoate  \\[5pt]
    EtOH  & Ethanol  \\[5pt]
    FET  & Field-Effect Transistor  \\[5pt]
    FITC  & Fluorescein isothiocyanate  \\[5pt]
    GA  & Glutaraldehyde  \\[5pt]
    GFET  & Graphene Field-Effect Transistor  \\[5pt]
    GFP  & Green Fluorescent Protein  \\[5pt]
    GPCR  & G-protein Coupled Receptor  \\[5pt]
    HEK  & Human Embryonic Kidney  \\[5pt]
    His-tag  & Histidine-tag  \\[5pt]
    hOR  & Human Odorant Receptor  \\[5pt]
    HPLC  & High-performance Liquid Chromatography   \\[5pt]
    iOR  & Insect Odorant Receptor  \\[5pt]
    IPA  & Isopropanol  \\[5pt]
    LOD  & Limit of Detection  \\[5pt]
    m-CNT  & Metallic Carbon Nanotube   \\[5pt]
    MeOH  & Methanol   \\[5pt]
    MeSal  & Methyl Salicylate   \\[5pt]
    MFC  & Mass Flow Controller   \\[5pt]
    mOR  & Mouse Odorant Receptor  \\[5pt]
    MOSFET  & Metal-Oxide-Semiconductor Field-Effect Transistor  \\[5pt]
    MSP  & Membrane Scaffold Protein  \\[5pt]
    MWCNT  & Multi-Walled Carbon Nanotube  \\[5pt]
    ND  & Nanodisc  \\[5pt]
  \end{tabular}
\end{table}

\newpage
\fancyhf{} %clear all headers and footers fields
\thispagestyle{fancy} % Change header and footer on this page
\renewcommand{\headrulewidth}{0pt}
\fancyhead[R]{\textit{List of Abbreviations}} % Set header content
\fancyfoot[R]{\thepage} %prints the page number on the right side of the header
\begin{table}[H]
  \begin{tabular}{@{}p{0.25\textwidth} p{0.75\textwidth}@{}}  % Adjust the width as needed
    NHS  & N-Hydroxysuccinimide  \\[5pt]
    NHSS  & N-hydroxysulfosuccinimide   \\[5pt]
    NMR  & Nuclear Magnetic Resonance  \\[5pt]
    NSB  & Non-Specific Binding   \\[5pt]
    NTA  & Nitrilotriacetic Acid   \\[5pt]
    OBP  & Odorant Binding Protein  \\[5pt]
    OR  & Odorant Receptor  \\[5pt]
    ORCO  & Odorant Receptor Co-Receptor  \\[5pt]
    PBA  & 1-Pyrenebutyric Acid  \\[5pt]
    PBASE  & 1-Pyrenebutanoic Acid N-hydroxysuccinimide Ester  \\[5pt]
    PBS  & Phosphate-Buffered Saline  \\[5pt]
    PCB  & Printed Circuit Board   \\[5pt]
    PDL & Poly-\textit{D}-lysine  \\[5pt]
    PDMS  & Polydimethylsiloxane   \\  [5pt]
    PEG  & Polyethylene Glycol  \\[5pt] 
    PID  & Photoionisation Detector  \\[5pt]
    P\&ID & Process \& Instrumentation Diagram  \\[5pt]
    PLL  & Poly-\textit{L}-lysine  \\[5pt]
    PPB  & Pyrene-PEG-Biotin  \\[5pt]
    PPF  & Pyrene-PEG-FITC  \\[5pt]
    PPN  & Pyrene-PEG-NTA  \\[5pt]
    PPR  & Pyrene-PEG-Rhodamine  \\[5pt]
    PTFE  & Polytetrafluoroethylene (Teflon™)  \\[5pt]
    PVC  & Polyvinyl chloride  \\[5pt]
    QCM  & Quartz Crystal Microbalance  \\[5pt]
    RH  & Relative Humidity  \\[5pt]
    RHI  & Relative Humidity and Temperature Indicator  \\[5pt] 
  \end{tabular}
\end{table}

\newpage
\fancyhf{} %clear all headers and footers fields
\thispagestyle{fancy} % Change header and footer on this page
\renewcommand{\headrulewidth}{0pt}
\fancyhead[L]{\textit{List of Abbreviations}} % Set header content
\fancyfoot[L]{\thepage} %prints the page number on the right side of the header
\begin{table}[H]
  \begin{tabular}{@{}p{0.25\textwidth} p{0.75\textwidth}@{}}  % Adjust the width as needed
    RNA  & Ribonucleic Acid   \\[5pt]
    SAW  & Surface Acoustic Wave   \\[5pt]
    s-CNT  & Semiconducting Carbon Nanotube   \\[5pt]
    SEM  & Scanning Electron Microscope/Microscopy   \\[5pt]
    SMU  & Source Measure Unit   \\[5pt]
    SPR  & Surface Plasmon Resonance   \\[5pt]
    SWCNT  & Single-Walled Carbon Nanotube   \\[5pt]
    TFTFET  & Thin-Film Field-Effect Transistor  \\[5pt]
    TMAH  & Tetramethylammonium hydroxide  \\[5pt]
    TX  & Transfer Characteristics  \\[5pt]
    UV  & Ultraviolet  \\[5pt]
    VI  & Virtual Instrument  \\[5pt]
    VUAA1  & N-(4-Ethylphenyl)-2-{[4-ethyl-5-(pyridin-3-yl)-4H-1,2,4-triazol-3-yl]sulfanyl}acetamide  \\[5pt] 
  \end{tabular}
\end{table}

% Adjust the top and bottom margins of float pages to center floats
\makeatletter
\setlength{\@fptop}{0pt plus 1fil}
\setlength{\@fpbot}{0pt plus 1fil}
\makeatother

\fancyhf{} %clear all headers and footers fields
\thispagestyle{fancy} % Change header and footer on this page
\renewcommand{\headrulewidth}{0pt}
\fancyhead[L]{\textit{List of Abbreviations}} % Set header content
\fancyfoot[L]{\thepage} %prints the page number on the right side of the header

\clearpage
\newpage

\end{spacing}

\pagestyle{headings}
\setstretch{1.2}
\mainmatter
\bookmarksetup{startatroot}

\hypertarget{introduction}{%
\chapter{Introduction}\label{introduction}}

\hypertarget{background}{%
\section{Background}\label{background}}

The ``bioelectronic nose'', an electronic transducer modified with
elements of the animal olfactory system, has the potential to allow
specific detection of airborne volatile compounds at concentrations as
low as parts per trillion
\autocite{Glatz2011,Kwon2015,Dung2018,Kim2022a}. The thin-film
transistor (TFT), a type of field-effect transistor (FET), is a
particularly portable, simple to use, small and robust option for the
transducer component \autocite{Kauffman2008,Khan2020}. The thin films
used in these transistors include carbon nanotube networks and graphene,
low-dimensional nanomaterials which are both highly sensitive and
biocompatible \autocite{Shkodra2021}. The implications of successful
development of such a portable and robust bioelectronic nose are
significant. Applications could be found in high-importance fields such
as biosecurity, medicine, environmental protection and food safety
\autocite{Dung2018,Arakawa2019,Yang2017,Son2017}. For example, it has
been demonstrated that it is possible to specifically detect invasive
brown marmorated stinkbugs based on their volatile trace
\autocite{Moser2020}. A bioelectronic nose could potentially accomplish
this biosecurity task far more cheaply and efficiently than trained
sniffer dogs \autocite{Lee2010,Moon2020,Terutsuki2020}. There has been
rapid progress in the development of bioelectronic noses using carbon
nanotube field-effect transistors (CNT FETs) and graphene field-effect
transistors (GFETs) over the past 15-20 years
\autocite{Yoon2009,Lee2010,Yang2018}.

Insect odorant receptors (iORs) enable simple invertebrates, such as the
vinegar fruit fly \emph{Drosophila melanogaster}, to distinguish between
a huge number of specific volatile compounds
\autocite{Hallem2004,Smart2008,Wicher2008,Munch2016,Bohbot2020}. Within
the past five years, a variety of \emph{Drosophila melanogaster} iORs
have been successfully coupled with highly sensitive low-dimensional
thin-film transistors for specific detection of fruit-like odors in an
aqueous environment \autocite{Murugathas2019a,Murugathas2020}. Insect
odorant receptors have also been used for sensitive and selective
volatile detection of volatile compounds in a lipid bilayer format
\autocite{Yamada2021}. In this thesis, the primary aim was to verify
whether a bioelectronic nose capable of odorant detection in a
vapour-phase environment could be constructed by coupling iORs with
thin-film transistors. Alongside the future use of this technology in
targeted sensor applications, development of a vapour-phase
bioelectronic nose using iORs may give us a greater understanding of the
mechanisms underlying insect olfaction \autocite{Lee2010}. The
transduction mechanism of nanomaterial-based iOR sensors is still
unknown, and it is hoped this work sheds further light on the biological
and electronic processes underpinning this mechanism
\autocite{Murugathas2020,Khadka2019,Cheema2021}.

To answer the question of whether selective vapour-phase detection could
be achieved using the architecture described above, it was first
important to verify that each of the component parts of this highly
integrated system functioned as expected and were mutually compatible.
As the thin-film transistor is a relatively recent technology, the
design and fabrication of these devices continues to evolve, and it was
important to ensure that the methods used gave rise to devices of
consistent quality which were sufficiently sensitive. Insect odorant
receptor biofunctionalisation is also an emerging field; it was
important to review and test existing functionalisation approaches,
identifying their relative strengths and weaknesses for biosensor
applications. Finally, to ensure that both functioning parts worked
together once integrated, the functionalised transducer was tested in an
aqueous environment to compare its sensitivity with previous research
\autocite{Murugathas2019a,Murugathas2020}. A suitable vapour delivery
system for testing the working sensors was developed in parallel, with
reference sensors installed to compare their responses with the activity
of the biosensor. Once this system was confirmed to behave in a
controllable manner, the bioelectronic nose was tested in the
vapour-phase environment.

\hypertarget{thesis-outline}{%
\section{Thesis Outline}\label{thesis-outline}}

This thesis consists of ten chapters. The first three chapters,
including this one, are background chapters introducing the general
topics of this thesis. The fourth chapter is a methods chapter,
describing device fabrication and characterisation in detail. The fifth,
sixth and seventh chapters describe the results obtained when verifying
the sensitivity of the novel biosensor in an aqueous environment, while
the eighth and ninth chapters relate the results of testing the
biosensing platform in a vapour-phase environment. The tenth chapter
concludes the thesis and discusses possible next steps for future
research.

\textbf{Chapter 2} gives a broad description of carbon nanotube and
graphene field-effect transistors with a focus on their use in sensing
applications. The chapter begins by looking at the general structure and
properties of thin-film transistors, where key figures of merit such as
transconductance, on-off ratio, gate current and hysteresis are
described. The chemical compositions of graphene and carbon nanotube
networks are discussed, as well as their conduction behaviour and unique
sensing properties when integrated into a field-effect transistor as the
thin-film element.

\textbf{Chapter 3} investigates previous odorant receptor-coupled
thin-film field-effect transistors examined in the literature. First,
the biological structure of odorant receptors and membrane formats for
their protection \emph{in vitro} are discussed. Details are then
provided regarding the construction and operation of existing vertebrate
odorant receptor biosensors. The structure and function of the insect
odorant receptor is then contrasted with the vertebrate odorant
receptor, and existing insect odorant receptor TFT biosensors in the
literature are discussed. The chapter finishes with a brief discussion
of non-specific binding and its role in hindering biosensor activity.

\textbf{Chapter 4} describes the fabrication of the CNT FET and GFET
transducers used in this thesis and the characterisation techniques used
to probe their behaviour. The chapter starts with an introduction to
photolithography for thin-film transistor device fabrication. Various
techniques are described for random deposition of carbon nanotube
networks to act as channels for these thin-film transistors.
Characterisation techniques described in this chapter include atomic
force microscopy (AFM), fluorescence microscopy, Raman spectroscopy, and
electrical characterisation with various semiconductor device analysers.

\textbf{Chapter 5} presents the results obtained from the use of
characterisation techniques on the pristine GFETs and CNT FETs. Various
carbon nanotube (CNT) network morphologies are displayed and analysed.
The Raman spectra and electrical device parameters of these CNT network
morphologies are then discussed, along with electrical parameters from
graphene devices. The sensitivity of a dense CNT network morphology
device is then verified in the aqueous phase.

\textbf{Chapter 6} explores the non-covalent functionalisation of GFETs
and CNT FETs with various linker molecules for insect odorant receptor
attachment. The linker molecules tested were 1-pyrenebutanoic acid
N-hydroxysuccinimide ester (PBASE) and 1-pyrenebutyric acid (PBA) with
1-Ethyl-3-(3-dimethylaminopropyl)carbodiimide (EDC). Pyrene-NTA and
pyrene-biotin were also investigated as other possible linker molecules.
The quality of various functionalisation approaches was then explored
with various fluorescent-tagged linker molecules and biomolecules. In
this process, various potential obstacles to successful biosensor
functionalisation were identified.

\textbf{Chapter 7} maps out progress made towards the creation of an
insect odorant receptor functionalised thin-film biosensor which is
responsive to target analyte in an aqueous environment. Two different
approaches are described that gave rise to working aqueous-phase
biosensors. The first functionalisation approach, which used PBASE in
methanol, led to results that were difficult to reproduce. Possible
factors underlying the unreliability of this method were then
investigated. A second approach was then designed to avoid the
undesirable influence of the confounding factors identified.

\textbf{Chapter 8} outlines the development of a vapour delivery system
for characterisation of the insect odorant receptor functionalised TFT
biosensors in a vapour-phase environment. The vapour delivery system was
upgraded from an existing system to include new mass flow controllers
(to have greater control of flow through the system) and off the shelf
vapour sensors (to collect vapour flow data that could be used for
comparison against biosensor activity). The chapter also describes the
design and construction of an electronic interface to monitor and
control the components of the vapour delivery system, as well as
calibration of the mass flow controllers.

\textbf{Chapter 9} details the use of the vapour delivery system for
testing the functionalised biosensors in the vapour phase. First, the
flow behaviour of volatile organic vapours through the system was
validated using onboard reference sensors. The response of a pristine
carbon nanotube device to two volatile compounds was investigated and
compared to signals from the reference sensors. Finally, a device
functionalised using the second functionalisation approach from chapter
7 was tested in the vapour delivery system, and its responses to vapour
were compared to those of a pristine device.

\textbf{Chapter 10} summarises the conclusions drawn from this work, and
proposes various related studies which can be undertaken to continue the
work described in this thesis.

\bookmarksetup{startatroot}

\hypertarget{sec-pristine-characteristics}{%
\chapter{Characteristics of Pristine Thin-Film
Transistors}\label{sec-pristine-characteristics}}

\hypertarget{introduction-1}{%
\section{Introduction}\label{introduction-1}}

Several different approaches were followed to fabricate carbon nanotube
network and graphene field-effect transistors for biosensor use. The
three carbon nanotube film types used for devices were the
solvent-deposited, surfactant-deposited and steam-assisted
surfactant-deposited (steam-deposited) films discussed in
\textbf{?@sec-fabrication}. As minor changes were made to fabrication
processes throughout the thesis, the fabrication dates of devices used
are stated, alongside a brief description of the process used at the
time. This chapter uses the characterisation techniques outlined in
\textbf{?@sec-fabrication} to compare and contrast the device channel
morphologies and electrical characteristics resulting from the various
methods used. The other aim of this chapter is to show the electrical
behaviour of the transistors when exposed to vapour in the vapour
delivery system discussed in \textbf{?@sec-vapour-sensing-biosensors}.

Atomic force microscopy and Raman spectroscopy were performed on the
carbon nanotube networks to identify the distribution of carbon nanotube
diameters and the extent to which defects were present on the carbon
nanotube networks. Electrical characterisation was then used to see how
the morphology of each film type affected the performance of the
completed devices. Both back-gated and liquid-gated transfer
characteristics were compared, and figures of merit from the
liquid-gated characteristics were examined. Control measurements were
also taken to verify the behaviour of the pristine device as a sensor in
an aqueous environment. A salt concentration sensing series was
performed with a steam-deposited carbon nanotube network device. The
device characteristics were taken and device drift was examined and
modelled. The sensing series was performed by successively diluting
\(1 \times\) PBS in the PDMS ``well'' (electrolyte container) while
placing a voltage across the device channels, and measuring the current
response to dilutions on each channel. Various filters were applied to
the collected data to better understand the signal change.

\hypertarget{sec-pristine-morphology}{%
\section{Carbon Nanotube Network Morphology and
Composition}\label{sec-pristine-morphology}}

\hypertarget{sec-pristine-AFM}{%
\subsection{Atomic Force Microscopy}\label{sec-pristine-AFM}}

Figure~\ref{fig-afm-morphology} shows a side-by-side comparison of the
surface morphology of carbon nanotube films fabricated using the methods
described in \textbf{?@sec-dep-carbon-nanotubes}. These images were
collected using an atomic force microscope and processed in the manner
described in \textbf{?@sec-afm-characterisation}. As discussed in
previous works using solvent-based deposition techniques for depositing
carbon nanotubes, in each network multi-tube bundles form due to strong
mutual attraction between nanotubes
\autocite{Zheng2017,Murugathas2018,Murugathas2019,Nguyen2021}. However,
when surfactant is present, the surfactant is adsorbed by the carbon
nanotubes and forms a highly repulsive structure able to overcome the
strong attraction between nanotubes. This repulsion keeps the individual
carbon nanotubes relatively isolated
\autocite{Wenseleers2004,Gavrel2013,Hermanson2013-16,Shimizu2013,DiCrescenzo2014,Yang2023}.
The diameter range provided by the supplier for the individual carbon
nanotubes used is \(1.2-1.7\) nm, while the length range is \(0.3-5.0\)
µm (Nanointegris).

\begin{figure}

\begin{minipage}[t]{0.03\linewidth}

{\centering 

\raisebox{-\height}{

\includegraphics{figures/(a).png}

}

}

\end{minipage}%
%
\begin{minipage}[t]{0.01\linewidth}

{\centering 

~

}

\end{minipage}%
%
\begin{minipage}[t]{0.45\linewidth}

{\centering 

\raisebox{-\height}{

\includegraphics{figures/ch5/Ned_NTQ24_20220125_00235.png}

}

}

\end{minipage}%
%
\begin{minipage}[t]{0.01\linewidth}

{\centering 

~

}

\end{minipage}%
%
\begin{minipage}[t]{0.03\linewidth}

{\centering 

\raisebox{-\height}{

\includegraphics{figures/(b).png}

}

}

\end{minipage}%
%
\begin{minipage}[t]{0.01\linewidth}

{\centering 

~

}

\end{minipage}%
%
\begin{minipage}[t]{0.45\linewidth}

{\centering 

\raisebox{-\height}{

\includegraphics{figures/ch5/Ned_NTQ24_20220125_00235_histogram_initialguess_inset.png}

}

}

\end{minipage}%
%
\begin{minipage}[t]{0.01\linewidth}

{\centering 

~

}

\end{minipage}%
\newline
\begin{minipage}[t]{0.03\linewidth}

{\centering 

\raisebox{-\height}{

\includegraphics{figures/(c).png}

}

}

\end{minipage}%
%
\begin{minipage}[t]{0.01\linewidth}

{\centering 

~

}

\end{minipage}%
%
\begin{minipage}[t]{0.45\linewidth}

{\centering 

\raisebox{-\height}{

\includegraphics{figures/ch5/Ned_NTQ8C7_w4_pristine_00084_20210428(2).png}

}

}

\end{minipage}%
%
\begin{minipage}[t]{0.01\linewidth}

{\centering 

~

}

\end{minipage}%
%
\begin{minipage}[t]{0.03\linewidth}

{\centering 

\raisebox{-\height}{

\includegraphics{figures/(d).png}

}

}

\end{minipage}%
%
\begin{minipage}[t]{0.01\linewidth}

{\centering 

~

}

\end{minipage}%
%
\begin{minipage}[t]{0.45\linewidth}

{\centering 

\raisebox{-\height}{

\includegraphics{figures/ch5/Ned_NTQ8C7_w4_pristine_00084_20210428(2)_histogram_initialguess_inset.png}

}

}

\end{minipage}%
%
\begin{minipage}[t]{0.01\linewidth}

{\centering 

~

}

\end{minipage}%
\newline
\begin{minipage}[t]{0.03\linewidth}

{\centering 

\raisebox{-\height}{

\includegraphics{figures/(e).png}

}

}

\end{minipage}%
%
\begin{minipage}[t]{0.01\linewidth}

{\centering 

~

}

\end{minipage}%
%
\begin{minipage}[t]{0.45\linewidth}

{\centering 

\raisebox{-\height}{

\includegraphics{figures/ch5/Ned_NGQ14D2_W4_pristine_20220713_00567.png}

}

}

\end{minipage}%
%
\begin{minipage}[t]{0.01\linewidth}

{\centering 

~

}

\end{minipage}%
%
\begin{minipage}[t]{0.03\linewidth}

{\centering 

\raisebox{-\height}{

\includegraphics{figures/(f).png}

}

}

\end{minipage}%
%
\begin{minipage}[t]{0.01\linewidth}

{\centering 

~

}

\end{minipage}%
%
\begin{minipage}[t]{0.45\linewidth}

{\centering 

\raisebox{-\height}{

\includegraphics{figures/ch5/Ned_NGQ14D2_W4_pristine_20220713_00567_histogram_initialguess_inset.png}

}

}

\end{minipage}%
%
\begin{minipage}[t]{0.01\linewidth}

{\centering 

~

}

\end{minipage}%

\caption[Atomic force microscope images of carbon nanotube films
deposited using various methods, shown side-by-side with histogram
height distributions and kernel density estimate (KDE) plots
corresponding to each image.]{\label{fig-afm-morphology}2.5 µm
\(\times\) 2.5 µm atomic force microscope (AFM) images of carbon
nanotube films deposited using various methods, shown side-by-side with
histogram height distributions and kernel density estimate (KDE) plots
corresponding to each image. The network shown in (a) with height
distribution shown in (b) was deposited in solvent, the network shown in
(c) with height distribution shown in (d) was dropcast in surfactant,
and the network shown in (e) with height distribution shown in (f) was
dropcast in surfactant with steam present.}

\end{figure}

It has previously been demonstrated that the diameter range of deposited
single-walled carbon nanotubes can be modelled via a normal or Gaussian
distribution \autocite{LeMieux2008,Liu2013,Vobornik2023}. However, when
the height profiles from the 2.5 µm \(\times\) 2.5 µm AFM images are
directly extracted and binned, as shown in
Figure~\ref{fig-afm-morphology}, the histograms obtained do not follow a
normal distribution. One reason for this result is that the carbon
nanotubes do not lie perfectly flat on the substrate surface, as the
SiO\(_2\) substrate and the carbon nanotubes each possess some surface
roughness. To find the contribution of substrate surface roughness to
the height profile histogram corresponding to each network deposition
method, SiO\(_2\) substrates were modified using the same processes as
in Figure~\ref{fig-afm-morphology} but without carbon nanotubes present
in the solutions used. 2.5 µm \(\times\) 2.5 µm AFM images of the
modified surfaces are shown in Figure~\ref{fig-afm-substrate}.

\begin{figure}

\begin{minipage}[t]{0.03\linewidth}

{\centering 

\raisebox{-\height}{

\includegraphics{figures/(a).png}

}

}

\end{minipage}%
%
\begin{minipage}[t]{0.01\linewidth}

{\centering 

~

}

\end{minipage}%
%
\begin{minipage}[t]{0.45\linewidth}

{\centering 

\raisebox{-\height}{

\includegraphics{figures/ch5/Ned_SiO2_00351_20231016.png}

}

}

\end{minipage}%
%
\begin{minipage}[t]{0.01\linewidth}

{\centering 

~

}

\end{minipage}%
%
\begin{minipage}[t]{0.03\linewidth}

{\centering 

\raisebox{-\height}{

\includegraphics{figures/(b).png}

}

}

\end{minipage}%
%
\begin{minipage}[t]{0.01\linewidth}

{\centering 

~

}

\end{minipage}%
%
\begin{minipage}[t]{0.45\linewidth}

{\centering 

\raisebox{-\height}{

\includegraphics{figures/ch5/Ned_SiO2_00351_20231016_histogram_initialguess.png}

}

}

\end{minipage}%
%
\begin{minipage}[t]{0.01\linewidth}

{\centering 

~

}

\end{minipage}%
\newline
\begin{minipage}[t]{0.03\linewidth}

{\centering 

\raisebox{-\height}{

\includegraphics{figures/(c).png}

}

}

\end{minipage}%
%
\begin{minipage}[t]{0.01\linewidth}

{\centering 

~

}

\end{minipage}%
%
\begin{minipage}[t]{0.45\linewidth}

{\centering 

\raisebox{-\height}{

\includegraphics{figures/ch5/Ned_SiO2_s_surfactant_nosteam_00355_20231016.png}

}

}

\end{minipage}%
%
\begin{minipage}[t]{0.01\linewidth}

{\centering 

~

}

\end{minipage}%
%
\begin{minipage}[t]{0.03\linewidth}

{\centering 

\raisebox{-\height}{

\includegraphics{figures/(d).png}

}

}

\end{minipage}%
%
\begin{minipage}[t]{0.01\linewidth}

{\centering 

~

}

\end{minipage}%
%
\begin{minipage}[t]{0.45\linewidth}

{\centering 

\raisebox{-\height}{

\includegraphics{figures/ch5/Ned_SiO2_s_surfactant_nosteam_00355_20231016_histogram_initialguess.png}

}

}

\end{minipage}%
%
\begin{minipage}[t]{0.01\linewidth}

{\centering 

~

}

\end{minipage}%
\newline
\begin{minipage}[t]{0.03\linewidth}

{\centering 

\raisebox{-\height}{

\includegraphics{figures/(e).png}

}

}

\end{minipage}%
%
\begin{minipage}[t]{0.01\linewidth}

{\centering 

~

}

\end{minipage}%
%
\begin{minipage}[t]{0.45\linewidth}

{\centering 

\raisebox{-\height}{

\includegraphics{figures/ch5/Ned_SiO2_s_surfactant_steam_00357_20231016.png}

}

}

\end{minipage}%
%
\begin{minipage}[t]{0.01\linewidth}

{\centering 

~

}

\end{minipage}%
%
\begin{minipage}[t]{0.03\linewidth}

{\centering 

\raisebox{-\height}{

\includegraphics{figures/(f).png}

}

}

\end{minipage}%
%
\begin{minipage}[t]{0.01\linewidth}

{\centering 

~

}

\end{minipage}%
%
\begin{minipage}[t]{0.45\linewidth}

{\centering 

\raisebox{-\height}{

\includegraphics{figures/ch5/Ned_SiO2_s_surfactant_steam_00357_20231016_histogram_initialguess_inset.png}

}

}

\end{minipage}%
%
\begin{minipage}[t]{0.01\linewidth}

{\centering 

~

}

\end{minipage}%

\caption[Atomic force microscope images of SiO\(_2\) substrates
alongside histogram height distributions and KDE plots corresponding to
each image.]{\label{fig-afm-substrate}2.5 µm \(\times\) 2.5 µm atomic
force microscope images of SiO\(_2\) substrates alongside histogram
height distributions and KDE plots corresponding to each image. The
substrate in (a) and (b) was exposed to solvent, the substrate in (c)
and (d) was exposed to surfactant, and the substrate in (e) and (f) was
exposed to surfactant with steam present. The inset in (f) shows a
close-up view of the long tail of the distribution.}

\end{figure}

In Figure~\ref{fig-afm-substrate}, it appears that each substrate
surface has a roughness that follows a normal distribution with some
degree of skewness. Figure~\ref{fig-afm-substrate} (b) and
Figure~\ref{fig-afm-substrate} (d) are negatively skewed distributions.
The equation for a skew-normal distribution is given in
Equation~\ref{eq-skew}.

\begin{equation}\protect\hypertarget{eq-skew}{}{
f(x) = 2\phi(x)\Phi(\alpha x)
}\label{eq-skew}\end{equation}

Within Equation~\ref{eq-skew}, \(\alpha\) is the ``slant parameter''
which indicates the skewness of the distribution, \(\phi(x)\) is the
standard normal distribution and \(\Phi(x)\) is the corresponding
cumulative distribution function. In the most general case,
Equation~\ref{eq-skew} can be modified with the ``location parameter''
\(\xi\) and ``scale parameter'' \(\omega\). These variables respectively
correspond to the mean and standard deviation of the skew-free normal
distribution when \(\alpha\) is set equal to zero. These parameters are
inserted into Equation~\ref{eq-skew} by setting
\(x \rightarrow (x-\xi)/\omega\) and replacing the factor 2 with
\(2/\omega\) \autocite{Azzalini2013}. The fitted skew-normal
distribution in Figure~\ref{fig-afm-substrate} (b) has
\(\alpha = -3.2\), \(\xi = 2.2\) nm and \(\omega = 0.5\) nm, while in
Figure~\ref{fig-afm-substrate} (d) \(\alpha = -2.2\), \(\xi = 2.2\) nm
and \(\omega = 0.5\) nm. The close correspondence of the values found
for \(\xi\) and \(\omega\) between these distributions despite the
difference in \(\alpha\) implies that the skewness is a variable imaging
or processing artifact rather than a physical property of the surface.
Without distortion, the roughness of a clean SiO\(_2\) surface should
follow a normal distribution \autocite{Velicky2015}.

Figure~\ref{fig-afm-substrate} (f) has a pronounced positive skew
\(\sim\) 5.5 nm in length. The tail results from the droplets observed
in Figure~\ref{fig-afm-substrate} (e), which may be condensation from
the steam or surfactant residue trapped by the steam environment
\autocite{Christensen2022,Vobornik2023}. If the contamination identified
here is surfactant, this could have negative effects on both sensitivity
of carbon nanotubes and also could damage the attached biological
elements. Attempting to fit a skew-normal distribution to this histogram
fails when all three variables are allowed to vary, which results from
the presence of the tail. Instead, previous values obtained for \(\xi\)
and \(\omega\) can be fixed at \(\xi\) and \(\omega\) at 2.2 nm and 0.5
nm respectively in the fitting process, assuming that these values do
not vary significantly between images of the substrate. In this fitting
process, therefore, only \(\alpha\) was allowed to change. The resulting
fitted distribution is shown in Figure~\ref{fig-afm-substrate} (f), with
\(\alpha = -2.4\). The distribution closely fits the negative tail of
the histogram, but deviates from the positive tail due to the presence
of the droplets. This deviation is small and the fit is otherwise good
quality, with an R-squared value of 0.98.

In the atomic force microscope images of the more highly bundled
networks, the carbon nanotube bundles can be distinguished from the
substrate background by eye. Therefore, it was reasonably
straightforward to use network morphology analysis techniques similar to
those previously outlined by Vobornik \emph{et al.}
\autocite{Vobornik2023}. In Gwyddion, two coloured height thresholds or
``masks'' were defined, one which covered all exposed substrate along
with small bundles, as shown in Figure~\ref{fig-cnt-mask} (a)-(b), and
one which covered as much substrate as possible without covering any
bundles, shown in Figure~\ref{fig-cnt-mask} (c)-(d). The median
substrate height across the set of both substrate mask regions was
\(8.4 \pm 0.5\) nm for the solvent-deposited film, and \(4.3 \pm 0.1\)
nm for the surfactant-deposited film, where the medians of the
individual masks for each film were used to give the upper and lower
bounds of uncertainty. These values are dramatically different from the
median substrate height of \(1.9 \pm 0.3\) nm obtained from the
distributions in Figure~\ref{fig-afm-substrate}. From comparing the
atomic force microscope images in Figure~\ref{fig-afm-substrate} with
Figure~\ref{fig-cnt-mask}, it does not appear a dramatic increase in
substrate surface roughness has occurred after carbon nanotube
deposition. Instead, it appears that the discrepancy between these
values results from the imaging process.

\begin{figure}

\begin{minipage}[t]{0.03\linewidth}

{\centering 

\raisebox{-\height}{

\includegraphics{figures/(a).png}

}

}

\end{minipage}%
%
\begin{minipage}[t]{0.01\linewidth}

{\centering 

~

}

\end{minipage}%
%
\begin{minipage}[t]{0.45\linewidth}

{\centering 

\raisebox{-\height}{

\includegraphics{figures/ch5/Ned_NTQ24_20220125_00235_1.png}

}

}

\end{minipage}%
%
\begin{minipage}[t]{0.01\linewidth}

{\centering 

~

}

\end{minipage}%
%
\begin{minipage}[t]{0.03\linewidth}

{\centering 

\raisebox{-\height}{

\includegraphics{figures/(b).png}

}

}

\end{minipage}%
%
\begin{minipage}[t]{0.01\linewidth}

{\centering 

~

}

\end{minipage}%
%
\begin{minipage}[t]{0.45\linewidth}

{\centering 

\raisebox{-\height}{

\includegraphics{figures/ch5/Ned_NTQ8C7_w4_pristine_00084_20210428(2)_1.png}

}

}

\end{minipage}%
%
\begin{minipage}[t]{0.01\linewidth}

{\centering 

~

}

\end{minipage}%
\newline
\begin{minipage}[t]{0.03\linewidth}

{\centering 

\raisebox{-\height}{

\includegraphics{figures/(c).png}

}

}

\end{minipage}%
%
\begin{minipage}[t]{0.01\linewidth}

{\centering 

~

}

\end{minipage}%
%
\begin{minipage}[t]{0.45\linewidth}

{\centering 

\raisebox{-\height}{

\includegraphics{figures/ch5/Ned_NTQ24_20220125_00235_2.png}

}

}

\end{minipage}%
%
\begin{minipage}[t]{0.01\linewidth}

{\centering 

~

}

\end{minipage}%
%
\begin{minipage}[t]{0.03\linewidth}

{\centering 

\raisebox{-\height}{

\includegraphics{figures/(d).png}

}

}

\end{minipage}%
%
\begin{minipage}[t]{0.01\linewidth}

{\centering 

~

}

\end{minipage}%
%
\begin{minipage}[t]{0.45\linewidth}

{\centering 

\raisebox{-\height}{

\includegraphics{figures/ch5/Ned_NTQ8C7_w4_pristine_00084_20210428(2)_2.png}

}

}

\end{minipage}%
%
\begin{minipage}[t]{0.01\linewidth}

{\centering 

~

}

\end{minipage}%
\newline
\begin{minipage}[t]{0.03\linewidth}

{\centering 

\raisebox{-\height}{

\includegraphics{figures/(e).png}

}

}

\end{minipage}%
%
\begin{minipage}[t]{0.01\linewidth}

{\centering 

~

}

\end{minipage}%
%
\begin{minipage}[t]{0.45\linewidth}

{\centering 

\raisebox{-\height}{

\includegraphics{figures/ch5/Ned_NTQ24_20220125_00235_3.png}

}

}

\end{minipage}%
%
\begin{minipage}[t]{0.01\linewidth}

{\centering 

~

}

\end{minipage}%
%
\begin{minipage}[t]{0.03\linewidth}

{\centering 

\raisebox{-\height}{

\includegraphics{figures/(f).png}

}

}

\end{minipage}%
%
\begin{minipage}[t]{0.01\linewidth}

{\centering 

~

}

\end{minipage}%
%
\begin{minipage}[t]{0.45\linewidth}

{\centering 

\raisebox{-\height}{

\includegraphics{figures/ch5/Ned_NTQ8C7_w4_pristine_00084_20210428(2)_3.png}

}

}

\end{minipage}%
%
\begin{minipage}[t]{0.01\linewidth}

{\centering 

~

}

\end{minipage}%

\caption[Atomic force microscope images which have been masked
(highlighted green) below various threshold
heights.]{\label{fig-cnt-mask}A mask, shaded green, sets a height
threshold so that masked features are hidden from the height dataset.
The network in (a), (c) and (e) was solvent-deposited, while the network
in (b), (d) and (f) was surfactant-deposited. The maximum possible
substrate height of each network is masked in (a) and (b), the maximum
possible substrate height excluding all carbon nanotubes is masked in
(c) and (d), while artifacts below the substrate surface have been
masked in (e) and (f).}

\end{figure}

In Figure~\ref{fig-afm-morphology}, deep streaks or scars with rounded
edges can often be found in the close vicinity of larger nanotube
bundles and junctions. These features are imaging artifacts, which
sometimes sit at heights well below the actual substrate surface. The
artifacts result from the use of line levelling when large features are
present in the atomic force microscope image \autocite{Sinha2009}.
Imaging artifacts are highlighted using a mask in
Figure~\ref{fig-cnt-mask} (e)-(f), with heights of \(6.4 \pm 0.4\) nm
and \(2.7 \pm 0.2\) nm for the solvent-deposited and
surfactant-deposited films respectively. By adding the median substrate
height from Figure~\ref{fig-afm-substrate} to the artifact heights,
average background values of \(8.3 \pm 0.7\) nm and \(4.6 \pm 0.5\) nm
are obtained for each film, both within the margin of error for the
median substrate heights obtained using Figure~\ref{fig-cnt-mask}. It
can be stated with confidence that the median values taken from the
masks in Figure~\ref{fig-cnt-mask} (a)-(d) is representative of the
actual average substrate level relative to the carbon nanotubes present,
unaffected by image artifact outliers. These median values can be used
to compare the height of carbon nanotube bundles to the average height
of the surrounding substrate, following the same approach as Vobornik
\emph{et al.} \autocite{Vobornik2023}.

Five successive diameter measurements of 30 carbon nanotube bundles
distributed evenly across the network were collected using Gwyddion.
Measurements were not taken at bundle junctions. The average of five
values for each carbon nanotube bundle was then taken and the
corresponding median background height of the substrate subtracted. The
30 height values were sorted into five equal-sized bins, as shown in
Figure~\ref{fig-cnt-histogram} (b) and (d). Each histogram appears to
follow a moderately positive-skewed normal distribution, different to
the skew-free normal distribution found in previous works
\autocite{LeMieux2008,Liu2013,Vobornik2023}. The skew therefore appears
to be a distortion, introduced by the imaging process. The force of the
atomic force microscope tip is known to cause larger bundles to undergo
some degree of compression, and the resulting systematic underestimation
of their height may be responsible for the distribution skewness
\autocite{Vobornik2023}. The fitted skew-normal distribution in
Figure~\ref{fig-cnt-histogram} (b) has values of \(\alpha = 2.7\)
(slant, indicates skew), \(\xi = 4.3\) nm (location, indicates mean),
and \(\omega = 5.9\) nm (scale, indicates standard deviation), while the
distribution in Figure~\ref{fig-cnt-histogram} (d) has values of
\(\alpha = 2.5\), \(\xi = 2.2\) nm, and \(\omega = 2.6\) nm. The
probability density for the carbon nanotube bundle histogram drops to
approximately zero at 0 nm, an indication that the dataset is
representative of the actual film.

\begin{figure}

\begin{minipage}[t]{0.03\linewidth}

{\centering 

\raisebox{-\height}{

\includegraphics{figures/(a).png}

}

}

\end{minipage}%
%
\begin{minipage}[t]{0.01\linewidth}

{\centering 

~

}

\end{minipage}%
%
\begin{minipage}[t]{0.45\linewidth}

{\centering 

\raisebox{-\height}{

\includegraphics{figures/ch5/Ned_NTQ24_20220125_00235.png}

}

}

\end{minipage}%
%
\begin{minipage}[t]{0.01\linewidth}

{\centering 

~

}

\end{minipage}%
%
\begin{minipage}[t]{0.03\linewidth}

{\centering 

\raisebox{-\height}{

\includegraphics{figures/(b).png}

}

}

\end{minipage}%
%
\begin{minipage}[t]{0.01\linewidth}

{\centering 

~

}

\end{minipage}%
%
\begin{minipage}[t]{0.45\linewidth}

{\centering 

\raisebox{-\height}{

\includegraphics{figures/ch5/NTQ24_20220125_00235_cnt_histogram.png}

}

}

\end{minipage}%
%
\begin{minipage}[t]{0.01\linewidth}

{\centering 

~

}

\end{minipage}%
\newline
\begin{minipage}[t]{0.03\linewidth}

{\centering 

\raisebox{-\height}{

\includegraphics{figures/(c).png}

}

}

\end{minipage}%
%
\begin{minipage}[t]{0.01\linewidth}

{\centering 

~

}

\end{minipage}%
%
\begin{minipage}[t]{0.45\linewidth}

{\centering 

\raisebox{-\height}{

\includegraphics{figures/ch5/Ned_NTQ8C7_w4_pristine_00084_20210428(2).png}

}

}

\end{minipage}%
%
\begin{minipage}[t]{0.01\linewidth}

{\centering 

~

}

\end{minipage}%
%
\begin{minipage}[t]{0.03\linewidth}

{\centering 

\raisebox{-\height}{

\includegraphics{figures/(d).png}

}

}

\end{minipage}%
%
\begin{minipage}[t]{0.01\linewidth}

{\centering 

~

}

\end{minipage}%
%
\begin{minipage}[t]{0.45\linewidth}

{\centering 

\raisebox{-\height}{

\includegraphics{figures/ch5/NTQ8C7_w4_pristine_cnt_histogram.png}

}

}

\end{minipage}%
%
\begin{minipage}[t]{0.01\linewidth}

{\centering 

~

}

\end{minipage}%

\caption[Histogram height distributions for each carbon nanotube network
with corresponding kernel density estimate
plots.]{\label{fig-cnt-histogram}Histogram height distributions for each
carbon nanotube network with corresponding kernel density estimate plots
collected via the morphology analysis method outlined by Vobornik
\emph{et al.} \autocite{Vobornik2023}. The histogram for the
solvent-deposited network in (a) is shown in (b) and the histogram for
the surfactant-deposited network in (c) is shown in (d). The red dotted
line represents the lower bound for multi-tube bundles at 2.9 nm,
assuming the minimum nanotube diameter is 1.45 nm.}

\end{figure}

Analysis of carbon nanotube network morphology can be simplified by
assuming the component nanotubes are cylinders, follow 2D packing and
are of equal diameter \autocite{Murugathas2018}.
Table~\ref{tbl-circle-packing} shows the relationship between the
diameter of a bundle of 2D packed cylinders and their constituent
diameters. Assuming an average carbon nanotube diameter of 1.45 nm, it
is possible to use Table~\ref{tbl-circle-packing} to find the
approximate number of nanotubes, \emph{n}, present in the median bundle
diameter of a film, where the median bundle diameter is taken from the
distributions in Figure~\ref{fig-cnt-histogram}. An estimate of the
relative proportion of multi-tube bundles in a film can be found by
taking the integral of each probability distribution with a lower bound
of 2.9 nm, the minimum multi-tube bundle size for 1.45 nm diameter
nanotubes. Since the area under the curve represents the probability a
bundle will have a particular diameter, the ratio of the bounded
integral to the unbounded integral gives a reasonable estimate of the
proportion of multi-tube bundles relative to single nanotubes. The
proportion of substrate covered by nanotubes can be determined by
comparing the projected surface area of the masks in
Figure~\ref{fig-cnt-mask} (a)-(d) with the full image area.

\hypertarget{tbl-circle-packing}{}
\begin{longtable}[]{@{}
  >{\raggedright\arraybackslash}p{(\columnwidth - 16\tabcolsep) * \real{0.1053}}
  >{\raggedright\arraybackslash}p{(\columnwidth - 16\tabcolsep) * \real{0.1228}}
  >{\raggedright\arraybackslash}p{(\columnwidth - 16\tabcolsep) * \real{0.0965}}
  >{\raggedright\arraybackslash}p{(\columnwidth - 16\tabcolsep) * \real{0.0965}}
  >{\raggedright\arraybackslash}p{(\columnwidth - 16\tabcolsep) * \real{0.1228}}
  >{\raggedright\arraybackslash}p{(\columnwidth - 16\tabcolsep) * \real{0.1228}}
  >{\raggedright\arraybackslash}p{(\columnwidth - 16\tabcolsep) * \real{0.1228}}
  >{\raggedright\arraybackslash}p{(\columnwidth - 16\tabcolsep) * \real{0.1228}}
  >{\raggedright\arraybackslash}p{(\columnwidth - 16\tabcolsep) * \real{0.0877}}@{}}
\caption{\label{tbl-circle-packing}The first eight optimised ratios of
2D packed circle diameter to encompassing circle diameter, given to 3
s.f. (encompassing circle diameter = \(d\), number of packed circles =
\(n\), approximate packed circle diameter = \(d_n\)).\\
}\tabularnewline
\toprule\noalign{}
\endfirsthead
\endhead
\bottomrule\noalign{}
\endlastfoot
\(n\) & \text{2} & \text{3} & \text{4} & \text{5} & \text{6} & \text{7}
& \text{8} & \text{9} \\
\(d\)/\(d_n\) & \text{2.00} & 2.15 & 2.41 & \text{2.70} & \text{3.00} &
\text{3.00} & \text{3.30} & 3.61 \\
\end{longtable}

\hypertarget{tbl-histogram-parameters}{}
\begin{longtable}[t]{>{\raggedright\arraybackslash}p{3.5cm}>{\centering\arraybackslash}p{2.2cm}>{\centering\arraybackslash}p{2.2cm}>{\centering\arraybackslash}p{2.2cm}>{\centering\arraybackslash}p{2.2cm}}
\caption{\label{tbl-histogram-parameters}The median of histogram distributions for solvent-deposited and
surfactant-deposited carbon nanotube films, alongside rough estimates
for the number of nanotubes present per median bundle, the estimated
proportion of multi-tube bundles present across the network and the
proportion of substrate covered by the nanotubes. }\tabularnewline

\toprule
 & Median Bundle Diameter (nm) & Tubes per Median Bundle & \% Multi-Tube Bundles & \% Nanotube Coverage\\
\midrule
Solvent-deposited & 8.3 ± 4.5 & 24 & > 97\% & < 66\%\\
Surfactant-deposited & 3.9 ± 1.9 & 4 & > 77\% & < 40\%\\
\bottomrule
\end{longtable}

These estimates are all shown in Table~\ref{tbl-histogram-parameters},
with the caveat that median bundle diameter is likely being
underestimated due to tip compression \autocite{Vobornik2023}.
Alternative estimates for median bundle diameter can be obtained by
finding the median heights of the combined unmasked regions from
Figure~\ref{fig-cnt-mask} (a) and (c) and from Figure~\ref{fig-cnt-mask}
(b) and (d), which are \(12.7 \pm 2.6\) nm and \(8.0 \pm 0.9\) nm
respectively. By subtracting the median substrate height of each image,
\(8.4 \pm 0.5\) nm and \(4.3 \pm 0.1\) nm, alternative median bundle
diameters of \(4.3 \pm 3.1\) nm and \(3.7 \pm 1.0\) nm are found, within
the error margin of the values in Table~\ref{tbl-histogram-parameters}.
Both the carbon nanotube bundle diameter median and standard deviation
are small for surfactant-deposited films when compared to the median and
standard deviation of solvent-deposited films. However, despite the
presence of surfactant, it is apparent both from
Figure~\ref{fig-cnt-histogram} and Table~\ref{tbl-histogram-parameters}
that most surfactant-dispersed carbon nanotubes are not deposited
individually. It is possible that the bundling of surfactant-dispersed
carbon nanotubes occurs as nanotubes are deposited onto the surface, as
this bonding process may disrupt the repulsive forces from the
surfactant coating and allow attractive forces to temporarily dominate.

\hypertarget{dense-nanotube-networks}{%
\subsubsection*{Dense Nanotube Networks}\label{dense-nanotube-networks}}
\addcontentsline{toc}{subsubsection}{Dense Nanotube Networks}

The solvent-deposited and surfactant-deposited networks in
Figure~\ref{fig-afm-morphology} (a)-(d) are relatively sparse, while the
steam-deposited film seen in Figure~\ref{fig-afm-morphology} (e)-(f)
shows a dense network of single and less-bundled nanotubes, along with
some larger bundles. It was unclear whether the method of Vobornik
\emph{et al.} could be used for these networks due to difficulty
distinguishing small nanotubes from the substrate, but a similar
approach was attempted. The dual-masking technique seen earlier was used
to estimate the substrate height. The first mask covers all exposed
substrate, including small bundles, as shown in
Figure~\ref{fig-dense-network} (a). The second mask is set empty,
assuming that almost the entire surface is densely covered with carbon
nanotubes. The median height of the masked region is \(2.2 \pm 0.3\) nm.
Artifact regions with a height of \(0.7 \pm 0.2\) nm are highlighted in
the top-left corner of Figure~\ref{fig-dense-network} (b). Assuming the
actual substrate surface is visible in atomic force microscope image,
its height is estimated by adding the artifact height to the median
substrate height of \(1.9 \pm 0.3\) nm from
Figure~\ref{fig-afm-substrate}, which yields \(2.6 \pm 0.5\) nm. This
estimate aligns with the masking result found previously, indicating it
gives a reasonable estimate of substrate height even for dense networks.

\begin{figure}

\begin{minipage}[t]{0.03\linewidth}

{\centering 

\raisebox{-\height}{

\includegraphics{figures/(a).png}

}

}

\end{minipage}%
%
\begin{minipage}[t]{0.01\linewidth}

{\centering 

~

}

\end{minipage}%
%
\begin{minipage}[t]{0.45\linewidth}

{\centering 

\raisebox{-\height}{

\includegraphics{figures/ch5/Ned_NGQ14D2_W4_pristine_20220713_00567_1.png}

}

}

\end{minipage}%
%
\begin{minipage}[t]{0.01\linewidth}

{\centering 

~

}

\end{minipage}%
%
\begin{minipage}[t]{0.03\linewidth}

{\centering 

\raisebox{-\height}{

\includegraphics{figures/(b).png}

}

}

\end{minipage}%
%
\begin{minipage}[t]{0.01\linewidth}

{\centering 

~

}

\end{minipage}%
%
\begin{minipage}[t]{0.45\linewidth}

{\centering 

\raisebox{-\height}{

\includegraphics{figures/ch5/Ned_NGQ14D2_W4_pristine_20220713_00567_2.png}

}

}

\end{minipage}%
%
\begin{minipage}[t]{0.01\linewidth}

{\centering 

~

}

\end{minipage}%
\newline
\begin{minipage}[t]{0.03\linewidth}

{\centering 

\raisebox{-\height}{

\includegraphics{figures/(c).png}

}

}

\end{minipage}%
%
\begin{minipage}[t]{0.01\linewidth}

{\centering 

~

}

\end{minipage}%
%
\begin{minipage}[t]{0.45\linewidth}

{\centering 

\raisebox{-\height}{

\includegraphics{figures/ch5/NT14D2_W4_pristine_cnt_histogram.png}

}

}

\end{minipage}%
%
\begin{minipage}[t]{0.01\linewidth}

{\centering 

~

}

\end{minipage}%
%
\begin{minipage}[t]{0.03\linewidth}

{\centering 

\raisebox{-\height}{

\includegraphics{figures/(d).png}

}

}

\end{minipage}%
%
\begin{minipage}[t]{0.01\linewidth}

{\centering 

~

}

\end{minipage}%
%
\begin{minipage}[t]{0.45\linewidth}

{\centering 

\raisebox{-\height}{

\includegraphics{figures/ch5/Ned_NGQ14D2_W4_pristine_20220713_00567_3.png}

}

}

\end{minipage}%
%
\begin{minipage}[t]{0.01\linewidth}

{\centering 

~

}

\end{minipage}%

\caption[An atomic force microscope image of a dense carbon nanotube
network which has been masked below various threshold heights, along
with corresponding histogram height distribution and kernel density
estimate plots.]{\label{fig-dense-network}The dense steam-deposited
network is covered with a mask showing the maximum possible height of
the substrate in (a) and covered with a mask showing median substrate
height in (b). A histogram of bundle measurements above the median
substrate height is shown in (c), while a mask covering all regions with
height at or greater than the median carbon nanotube diameter is shown
in (d).}

\end{figure}

Taking binned diameter measurements of the larger bundles, as seen
earlier, gives the histogram shown in Figure~\ref{fig-dense-network}
(c). Unlike the skew-normal distributions seen earlier, the distribution
in Figure~\ref{fig-dense-network} (c) falls to zero at \(\sim\) 1 nm.
This unphysical result comes from undercounting the large population of
smaller nanotubes, which in turn results from less-bundled nanotubes
being difficult to distinguish from the substrate background. It is
still possible to use masking to obtain an estimate of the median bundle
height. The median height of the combined unmasked regions in
Figure~\ref{fig-dense-network} (a)-(b) is \(4.1 \pm 0.2\) nm, with all
regions in the network at or above this height shown in
Figure~\ref{fig-dense-network} (d). The median substrate height,
\(2.2 \pm 0.3\) nm, can be subtracted from the unmasked region height
for a median bundle diameter of \(1.9 \pm 0.5\) nm. This estimate is
approximately half that of the median bundle diameter in the film
deposited with surfactant but without steam present. This result implies
the presence of steam disrupts the bundling of carbon nanotubes during
deposition. However, it is also possible that no substrate is visible in
Figure~\ref{fig-dense-network}, and the two films have similar bundle
sizes once the resulting height offset is taken into account.

The discussion in this section gives us a new understanding of the
histograms shown in Figure~\ref{fig-afm-morphology}. It appears that
these histograms are linear combinations of skewed normal distributions.
These distributions include a negatively-skewed distribution
corresponding to the substrate surface and a positively-skewed
distribution corresponding to the carbon nanotube bundles. The
trough-like image artifacts as well as X and Y junctions between
overlapping nanotubes may also form similarly skewed normal
distributions as part of the complete image histogram
\autocite{Murugathas2018}. The linear combination of histograms could be
modelled mathematically in order to rapidly extract key parameters from
atomic force microscope images \autocite{Marchenko2010}, but
implementing this approach is outside of the scope of this thesis.
Introducing steam when depositing with surfactant visibly promotes even
surface coverage, but also leads to increased adsorption of water and
possibly surfactant onto the substrate and network. Finding a method of
removing the unwanted surfactant or moisture present may be required for
robust biosensors \autocite{Kane2014,Barnett2018}.

\hypertarget{sec-pristine-raman}{%
\subsection{Raman Spectroscopy}\label{sec-pristine-raman}}

Raman spectroscopy was also used to analyse and compare the deposited
carbon nanotube networks. Raman measurements were collected from a
solvent-deposited carbon nanotube film and a steam-assisted
surfactant-deposited film, both on SiO\(_2\), in the manner described in
\textbf{?@sec-raman-characterisation}. These spectra were then processed
using the Python script mentioned in Section~\ref{sec-raman-analysis}.
For each location, spectra over two wavenumber ranges were collected. A
peak corresponding to the bulk Si/SiO\(_2\) substrate, found in the
range between 100 cm\(^{-1}\) and 650 cm\(^{-1}\), was used as a
reference peak for normalisation of spectral intensity \autocite{E2015};
these normalised spectra are shown in Figure~\ref{fig-pristine-raman}.
In nanotube networks, the properties of individual nanotubes influence
the frequency of vibrational modes excited by the inelastic scattering
of incident laser light, and therefore the location of peaks in the
resulting Raman spectrum. In carbon nanotube network spectra, a D-band
comprising a single D-peak is typically observed at \(\sim\) 1320
cm\(^{-1}\), and a G-band comprising two G-peaks, G\(^-\) and G\(^+\),
is observed between \(\sim\) 1525 cm\(^{-1}\) and \(\sim\) 1650
cm\(^{-1}\), as seen in Figure~\ref{fig-pristine-raman} (a)-(b)
\autocite{Dresselhaus2005,King2014,E2015}.

\begin{figure}

\begin{minipage}[t]{0.11\linewidth}

{\centering 

~

}

\end{minipage}%
%
\begin{minipage}[t]{0.03\linewidth}

{\centering 

\raisebox{-\height}{

\includegraphics{figures/(a).png}

}

}

\end{minipage}%
%
\begin{minipage}[t]{0.01\linewidth}

{\centering 

~

}

\end{minipage}%
%
\begin{minipage}[t]{0.70\linewidth}

{\centering 

\raisebox{-\height}{

\includegraphics{figures/ch5/bundled_raman.png}

}

}

\end{minipage}%
%
\begin{minipage}[t]{0.15\linewidth}

{\centering 

~

}

\end{minipage}%
\newline
\begin{minipage}[t]{0.11\linewidth}

{\centering 

~

}

\end{minipage}%
%
\begin{minipage}[t]{0.03\linewidth}

{\centering 

\raisebox{-\height}{

\includegraphics{figures/(b).png}

}

}

\end{minipage}%
%
\begin{minipage}[t]{0.01\linewidth}

{\centering 

~

}

\end{minipage}%
%
\begin{minipage}[t]{0.70\linewidth}

{\centering 

\raisebox{-\height}{

\includegraphics{figures/ch5/singletube_raman.png}

}

}

\end{minipage}%
%
\begin{minipage}[t]{0.15\linewidth}

{\centering 

~

}

\end{minipage}%
\newline
\begin{minipage}[t]{0.10\linewidth}

{\centering 

~

}

\end{minipage}%
%
\begin{minipage}[t]{0.03\linewidth}

{\centering 

\raisebox{-\height}{

\includegraphics{figures/(c).png}

}

}

\end{minipage}%
%
\begin{minipage}[t]{0.01\linewidth}

{\centering 

~

}

\end{minipage}%
%
\begin{minipage}[t]{0.72\linewidth}

{\centering 

\raisebox{-\height}{

\includegraphics{figures/ch5/comparison-raman.png}

}

}

\end{minipage}%
%
\begin{minipage}[t]{0.14\linewidth}

{\centering 

~

}

\end{minipage}%

\caption[Raman spectra at different locations across two different 40 µm
\(\times\) 100 µm carbon nanotube film regions, along with a histogram
of D-peak/G\(^+\)-peak spectral ratios corresponding to each
film.]{\label{fig-pristine-raman}A series of nine Raman spectra at
different locations across two different 40 µm \(\times\) 100 µm carbon
nanotube film regions, where (a) shows spectra from a film deposited
using solvent while (b) shows spectra from a film deposited with
surfactant in the presence of steam. (c) shows a histogram of the range
of D-peak/G\(^+\)-peak spectral ratios corresponding to each film.}

\end{figure}

Closer inspection of the D-peak and G-peaks in
Figure~\ref{fig-pristine-raman} can give us important information about
the composition of the nanotube network. G\(^-\) is a minor peak found
at \(\sim\) 1570 cm\(^{-1}\), while G\(^+\) is a larger feature at
\(\sim\) 1590 cm\(^{-1}\). The G\(^+\) feature is associated with
semiconducting nanotubes and describes the in-plane vibration of bonds
along the nanotube length, while the G\(^-\) feature is associated with
metallic nanotubes and describes the in-plane vibration of bonds about
the nanotube circumference \autocite{King2014,Swiniarski2021}. The
G\(^+\) feature has a sharp Lorentzian lineshape, while the G\(^-\)
feature has an asymmetric Breit-Wigner-Fano (BWF) lineshape
\autocite{Blackburn2006,Swiniarski2021}. The splitting between the
wavenumber location of the G\(^-\) and G\(^+\) local maxima is lower in
Figure~\ref{fig-pristine-raman} (b) than in
Figure~\ref{fig-pristine-raman} (a), which may indicate more metallic
nanotubes are present in the surfactant-deposited network
\autocite{Swiniarski2021}. The D-peak is a ring-breathing mode which
indicates defects are present in the carbon nanotube atomic structure;
the ring-breathing mode results from the defects disrupting in-plane
lattice vibration
\autocite{King2014,CuentasGallegos2011,Swiniarski2021}.

Carbon nanotube structural disorder can be quantified by comparing the
relative magnitudes of the D-peak and G\(^+\)-peak
\autocite{Dresselhaus2005,King2014}. Figure~\ref{fig-pristine-raman} (c)
compares the ratio between D-peak and G\(^+\)-peak intensity across all
nine positions across the solvent-deposited and the surfactant-deposited
films. The spread of values for the \(I_D/I_G\) ratio is lower for the
steam-deposited network. This spatially homogeneous vibrational
behaviour reflects the steam-deposited film being more evenly
distributed, as discussed in Section~\ref{sec-pristine-morphology}. It
is also observed that \(I_D/I_G\) is larger for the steam-deposited
films than for the solvent-deposited films, indicating more defects are
present in the steam-deposited film. These defects may be charge
impurities introduced by adsorbed water or surfactant
\autocite{Christensen2022}. The size of the G\(^-\) feature relative to
the G\(^+\) feature can also be used to detect charge polarisation due
to surfactant presence, though since the G\(^-\) feature is also
sensitive to bundle size, it cannot be used for a direct comparison of
surfactant presence for the films in Figure~\ref{fig-pristine-raman}.
However, if the defects are surfactant, the \(I_{G^-}/I_{G^+}\) ratio
could be useful for comparing films of similar morphology before and
after cleaning steps aimed at removing surfactant, with possible
cleaning approaches discussed in \textbf{?@sec-future-work-fabrication}.

\hypertarget{sec-pristine-electrical-characterisation}{%
\section{Electrical Characteristics of Pristine
Devices}\label{sec-pristine-electrical-characterisation}}

\hypertarget{sec-python-analysis}{%
\subsection{Python Analysis}\label{sec-python-analysis}}

Analysis of electrical measurements was performed using the three Python
modules described in Section~\ref{sec-field-effect-transistor-analysis}.
For all three modules, the type of device being analysed (carbon
nanotube or graphene) can be set in the config file. The config file can
also be used to add annotations to any plot, to set the procedure for
normalising a sensing dataset, and the size of the filter window used
for the filters described below. Shorted or non-conducting channels can
also be removed from the analysis procedure by setting a maximum and
minimum allowable \(I_d\) for working channels in the config file. In
this work, a filter window of 40 datapoints was used.

The first of the three modules is for processing sensing datasets. Types
of plots produced include normalised plots, plots with fitted curves,
plots with the linear baseline drift removed, plots of signal against
analyte addition, and plots with various noise filters applied. This
module uses the scipy.optimize.curve\_fit function when fitting
exponential and linear curves to regions of the sensing data. It then
returns .csv spreadsheets containing the results of these analyses,
including the standard deviation for all calculated parameters. For a
linear fit \(c_1t + c_2\), the initial estimates used for \(c_1\) and
\(c_2\) were straightforward: \(c_1=1\) and \(c_2=0\). For an
exponential fit \(I_0\exp{(-t/\tau)} + I_C\), rough approximations were
used for the initial parameters: \(I_C\) was set as the final current
measurement of the region of interest, \(I_0\) was set as the initial
current measurement minus \(I_C\), and \(\tau\) was set as the time
where measured current drops to \(e^{-1}I_0 + I_C\).

``Despiked'' plots had spurious datapoints removed through the use of an
interquartile range rolling filter. Datapoints in each window with a
\(z\)-score above \(\pm 3\) were removed from the processed data.
``Filtered'' plots had noise reduced using a moving median filter. The
moving median filter is more effective at removing noise than a simple
moving average. It also has advantages over other filters, such as the
Savitzky-Golay or Butterworth filters, when removing noise from sensing
data, where sharp edges are often present. Median filtering can also be
used for baseline drift compensation, though this approach was not used
in this thesis \autocite{Stone2011}. A simple difference calculation
between the mean of the filtered current before an addition and the mean
of the filtered current after the addition was performed for plots of
signal against analyte addition. As seen in
Figure~\ref{fig-spaa-plot-comparison}, this process gave reasonably
consistent results regardless of whether baseline drift corrections were
used. This method of calculating signal is therefore robust even with
significant drift present.

\begin{figure}

\begin{minipage}[t]{0.11\linewidth}

{\centering 

~

}

\end{minipage}%
%
\begin{minipage}[t]{0.03\linewidth}

{\centering 

\raisebox{-\height}{

\includegraphics{figures/(a).png}

}

}

\end{minipage}%
%
\begin{minipage}[t]{0.01\linewidth}

{\centering 

~

}

\end{minipage}%
%
\begin{minipage}[t]{0.70\linewidth}

{\centering 

\raisebox{-\height}{

\includegraphics{figures/ch5/NTQ31C1_mean_simple_difference_before_and_after_step_filtered_concentrations.png}

}

}

\end{minipage}%
%
\begin{minipage}[t]{0.15\linewidth}

{\centering 

~

}

\end{minipage}%
\newline
\begin{minipage}[t]{0.11\linewidth}

{\centering 

~

}

\end{minipage}%
%
\begin{minipage}[t]{0.03\linewidth}

{\centering 

\raisebox{-\height}{

\includegraphics{figures/(b).png}

}

}

\end{minipage}%
%
\begin{minipage}[t]{0.01\linewidth}

{\centering 

~

}

\end{minipage}%
%
\begin{minipage}[t]{0.70\linewidth}

{\centering 

\raisebox{-\height}{

\includegraphics{figures/ch5/NTQ31C1_mean_simple_difference_before_and_after_step_filtered_concentrations_detrend.png}

}

}

\end{minipage}%
%
\begin{minipage}[t]{0.15\linewidth}

{\centering 

~

}

\end{minipage}%

\caption[A comparison of signal with analyte addition plots with
different filtering approaches used.]{\label{fig-spaa-plot-comparison}A
comparison of signal with analyte addition plots with different
filtering approaches used taken from a PBS dilution sensing dataset. In
(a), a simple difference before and after addition calculation was
performed on unfiltered data, while in (b), baseline drift performed on
data with the baseline drift removed.}

\end{figure}

The second module creates combined and individual plots of transfer data
collected from eight channels on a single device. Various parameters
from the transfer characteristics are saved to a .csv file along with
standard error. These parameters include on current, off current,
subthreshold slope and threshold voltage for the carbon nanotube network
devices, and major Dirac point voltage for graphene devices. The third
module gives a comparison of transfer measurements taken of the same
channel before and after some modification. It also calculates the shift
in either threshold voltage or major Dirac voltage of the device.

\hypertarget{graphene-devices}{%
\subsection{Graphene Devices}\label{graphene-devices}}

Graphene field-effect transistor devices were electrically characterised
in the manner described in \textbf{?@sec-electrical-characterisation}
and analysed using the Python code discussed in
Section~\ref{sec-field-effect-transistor-analysis}.
Figure~\ref{fig-pristine-graphene} (a) and (b) show the liquid-gated
transfer characteristics of two graphene devices, which have an average
on-off ratio of \(5.0 \pm 0.6\). These devices were fabricated prior to
Jun 2021. Both devices exhibit the ambipolar characteristics typical of
liquid-gated graphene devices
\autocite{Heller2009a,Heller2010,Xia2010,Kireev2017}. As with the carbon
nanotube network devices, leakage current remained below
\(\sim 1 \times 10^{-7}\) V across both the forward and reverse sweep.
Hysteresis between the forward and reverse sweep is caused by trapping
of charge within and on the surface of the SiO\(_{2}\) dielectric
\autocite{Bartolomeo2011}. In this work, the global minimum of the
transfer characteristic is referred to as the ``major'' Dirac point
while the second Dirac point is referred to as the ``minor'' Dirac
point. The major Dirac point for these devices is slightly to the right
of \(V\) = 0 V, which indicates \(p\)-doping of the channel. This slight
\(p\)-doping is a result of adsorption of oxygen and water from the air
or from residual photoresist \autocite{Cheng2011,Shin2012,Kireev2017}.
Some devices exhibited double Dirac point behaviour, as seen in
Figure~\ref{fig-pristine-graphene} (b).

\begin{figure}

\begin{minipage}[t]{0.03\linewidth}

{\centering 

\raisebox{-\height}{

\includegraphics{figures/(a).png}

}

}

\end{minipage}%
%
\begin{minipage}[t]{0.01\linewidth}

{\centering 

~

}

\end{minipage}%
%
\begin{minipage}[t]{0.45\linewidth}

{\centering 

\raisebox{-\height}{

\includegraphics{figures/ch5/JG098_pristine_TXLG01_5mVstep_220920_norinse.png}

}

}

\end{minipage}%
%
\begin{minipage}[t]{0.01\linewidth}

{\centering 

~

}

\end{minipage}%
%
\begin{minipage}[t]{0.03\linewidth}

{\centering 

\raisebox{-\height}{

\includegraphics{figures/(b).png}

}

}

\end{minipage}%
%
\begin{minipage}[t]{0.01\linewidth}

{\centering 

~

}

\end{minipage}%
%
\begin{minipage}[t]{0.45\linewidth}

{\centering 

\raisebox{-\height}{

\includegraphics{figures/ch5/JGQ00D6_pristine_TXLG01_5mVstep_220914_norinse.png}

}

}

\end{minipage}%
%
\begin{minipage}[t]{0.01\linewidth}

{\centering 

~

}

\end{minipage}%
\newline
\begin{minipage}[t]{0.03\linewidth}

{\centering 

\raisebox{-\height}{

\includegraphics{figures/(c).png}

}

}

\end{minipage}%
%
\begin{minipage}[t]{0.01\linewidth}

{\centering 

~

}

\end{minipage}%
%
\begin{minipage}[t]{0.45\linewidth}

{\centering 

\raisebox{-\height}{

\includegraphics{figures/ch5/JG098_ch1_absolute_values_with_gate_current.png}

}

}

\end{minipage}%
%
\begin{minipage}[t]{0.01\linewidth}

{\centering 

~

}

\end{minipage}%
%
\begin{minipage}[t]{0.03\linewidth}

{\centering 

\raisebox{-\height}{

\includegraphics{figures/(d).png}

}

}

\end{minipage}%
%
\begin{minipage}[t]{0.01\linewidth}

{\centering 

~

}

\end{minipage}%
%
\begin{minipage}[t]{0.45\linewidth}

{\centering 

\raisebox{-\height}{

\includegraphics{figures/ch5/JGQ00D6_ch3_absolute_values_with_gate_current.png}

}

}

\end{minipage}%
%
\begin{minipage}[t]{0.01\linewidth}

{\centering 

~

}

\end{minipage}%

\caption[Liquid-gated transfer characteristics of channels from two
encapsulated graphene devices, alongside the change in transfer
characteristics upon various degrees of exposure to \(1 \times\)
PBS.]{\label{fig-pristine-graphene}Figures (a) and (b) show liquid-gated
transfer characteristics of channels from two AZ\(^\circledR\) 1518
encapsulated graphene devices. The transfer characteristics of channel 1
in (a) and channel 5 in (b) after various degrees of exposure to
\(1 \times\) PBS are shown in (c) and (d) respectively, with each
transfer sweep numbered in the order the sweeps were taken. The dashed
lines correspond to measurements of gate leakage current.}

\end{figure}

Figure~\ref{fig-pristine-graphene} (c) and (d) show the effect of
\(1 \times\) PBS on a single graphene channel. The channels were
measured on exposure to \(1 \times\) PBS, after exposure to \(1 \times\)
PBS for one hour, and after two successive rinse steps. A slight
negative shift of the major Dirac point was observed. Across six
channels from two devices, the graphene channels were found to start
with an average Dirac point of \(0.28 \pm 0.04\) V and finish with an
average Dirac point of \(0.28 \pm 0.02\) V after the rinse steps,
indicating that the shift due to \(1 \times\) PBS exposure is
negligible. Any observed shift may result from gate bias stress, where
successive transfer sweeps introduce charge traps to the graphene layer
and alter the current level at a given gate voltage
\autocite{Bargaoui2018,Noyce2019}. This effect may also be due to the
removal of \(p\)-doped contamination after rinsing, such as adsorbed
oxygen or resist residue, causing interfacial charge traps to empty,
which then results in a negative shift of the major Dirac point
\autocite{Bartolomeo2011,Kireev2017,Peng2018}. Apart from the
previously-mentioned slight negative shift of the major Dirac point,
these values were highly consistent before and after exposure to
\(1 \times\) PBS.

\hypertarget{sec-cnt-devices}{%
\subsection{Carbon Nanotube Network Devices}\label{sec-cnt-devices}}

Each carbon nanotube network device fabricated was electrically
characterised as described in
\textbf{?@sec-electrical-characterisation}, and electrical data was
analysed using the Python code discussed in
Section~\ref{sec-field-effect-transistor-analysis}. Devices with a 100
nm or 300 nm SiO\(_2\) layer were used for liquid gated measurements,
and devices with a 100 nm SiO\(_2\) layer were used for backgated
measurements. Figure~\ref{fig-pristine-cnt-characteristics} displays
multi-channel measurements of representative devices fabricated as
described in \textbf{?@sec-fabrication}. To ensure a consistent
comparison, all device measurements shown here are from devices
fabricated and encapsulated using AZ\(^\circledR\) 1518 photoresist
exclusively. The channels which did not exhibit reliable transistor
characteristics are not shown. These ``non-working'' channels were
either shorted, due to metal remaining on the channel after lift-off, or
were very low current, due to a very sparse carbon nanotube network.
Devices shown here with a solvent-deposited carbon nanotube network were
fabricated prior to Jan 2022; devices with a surfactant-deposited
network without steam present were fabricated prior to Jun 2021; devices
with a surfactant-deposited network without steam were fabricated prior
to Sep 2022.

\begin{figure}

\begin{minipage}[t]{0.03\linewidth}

{\centering 

\raisebox{-\height}{

\includegraphics{figures/(a).png}

}

}

\end{minipage}%
%
\begin{minipage}[t]{0.01\linewidth}

{\centering 

~

}

\end{minipage}%
%
\begin{minipage}[t]{0.45\linewidth}

{\centering 

\raisebox{-\height}{

\includegraphics{figures/ch5/NTQ22C2_solvent_backgate.png}

}

}

\end{minipage}%
%
\begin{minipage}[t]{0.01\linewidth}

{\centering 

~

}

\end{minipage}%
%
\begin{minipage}[t]{0.03\linewidth}

{\centering 

\raisebox{-\height}{

\includegraphics{figures/(b).png}

}

}

\end{minipage}%
%
\begin{minipage}[t]{0.01\linewidth}

{\centering 

~

}

\end{minipage}%
%
\begin{minipage}[t]{0.45\linewidth}

{\centering 

\raisebox{-\height}{

\includegraphics{figures/ch5/NTQ24C8_pristine_TXLG01_220211_solvent_gate.png}

}

}

\end{minipage}%
%
\begin{minipage}[t]{0.01\linewidth}

{\centering 

~

}

\end{minipage}%
\newline
\begin{minipage}[t]{0.03\linewidth}

{\centering 

\raisebox{-\height}{

\includegraphics{figures/(c).png}

}

}

\end{minipage}%
%
\begin{minipage}[t]{0.01\linewidth}

{\centering 

~

}

\end{minipage}%
%
\begin{minipage}[t]{0.45\linewidth}

{\centering 

\raisebox{-\height}{

\includegraphics{figures/ch5/Q5C10_nosteam_backgate.png}

}

}

\end{minipage}%
%
\begin{minipage}[t]{0.01\linewidth}

{\centering 

~

}

\end{minipage}%
%
\begin{minipage}[t]{0.03\linewidth}

{\centering 

\raisebox{-\height}{

\includegraphics{figures/(d).png}

}

}

\end{minipage}%
%
\begin{minipage}[t]{0.01\linewidth}

{\centering 

~

}

\end{minipage}%
%
\begin{minipage}[t]{0.45\linewidth}

{\centering 

\raisebox{-\height}{

\includegraphics{figures/ch5/NTQ5C3_pristine_TXLG01_210602_nosteam_gate.png}

}

}

\end{minipage}%
%
\begin{minipage}[t]{0.01\linewidth}

{\centering 

~

}

\end{minipage}%
\newline
\begin{minipage}[t]{0.03\linewidth}

{\centering 

\raisebox{-\height}{

\includegraphics{figures/(e).png}

}

}

\end{minipage}%
%
\begin{minipage}[t]{0.01\linewidth}

{\centering 

~

}

\end{minipage}%
%
\begin{minipage}[t]{0.45\linewidth}

{\centering 

\raisebox{-\height}{

\includegraphics{figures/ch5/Q18C6_steam_backgate.png}

}

}

\end{minipage}%
%
\begin{minipage}[t]{0.01\linewidth}

{\centering 

~

}

\end{minipage}%
%
\begin{minipage}[t]{0.03\linewidth}

{\centering 

\raisebox{-\height}{

\includegraphics{figures/(f).png}

}

}

\end{minipage}%
%
\begin{minipage}[t]{0.01\linewidth}

{\centering 

~

}

\end{minipage}%
%
\begin{minipage}[t]{0.45\linewidth}

{\centering 

\raisebox{-\height}{

\includegraphics{figures/ch5/NTQ31C6_pristine_TXLG01_230330_steam_gate.png}

}

}

\end{minipage}%
%
\begin{minipage}[t]{0.01\linewidth}

{\centering 

~

}

\end{minipage}%

\caption[Back-gated and liquid-gated transfer characteristics of
encapsulated carbon nanotube network field-effect transistors with
thin-films deposited using various
methods.]{\label{fig-pristine-cnt-characteristics}Back-gated (left) and
liquid-gated (right) transfer characteristics of AZ\(^\circledR\) 1518
encapsulated carbon nanotube network field-effect transistors, where the
film was deposited with solvent in (a) and (b), deposited with
surfactant in (c) and (d), and deposited with surfactant in the presence
of steam in (e) and (f). A step size of 100 mV was used for the
backgated sweeps in (a), (c) and (e), while a step size of 20 mV was
used for the liquid-gated sweeps in (b), (d) and (f). Gate current
(leakage current) is shown with a dashed line. The source-drain voltage
used for all sweeps was \(V_{ds} = 100 \textrm{mV}\), and \(1 \times\)
PBS was used as the buffer for the liquid-gated measurements.}

\end{figure}

When characterising devices using the vapour delivery system chip
carrier, the setup arrangement meant all measurements were taken using a
backgate. Figure~\ref{fig-pristine-cnt-characteristics} (a),
Figure~\ref{fig-pristine-cnt-characteristics} (c) and
Figure~\ref{fig-pristine-cnt-characteristics} (e) show that backgated
devices exhibit \emph{p}-type transistor behaviour. Gate current leakage
was negligible, as shown by the dashed line staying close to zero across
the sweep. The hysteresis observed was much greater than for the
corresponding liquid-gated sweeps on the left. This hysteresis can be
explained by the presence of defects or charge traps within and on the
surface of the gate-insulating SiO\(_2\) and at interfaces between
SiO\(_2\) and carbon nanotubes. The occupancy of these charge traps
evolves over time with applied gate voltage
\autocite{Lee2007,Lee2012,Ha2014}. The devices fabricated with a
solvent-based deposition had a significantly lower off-current than the
surfactant-deposited devices, which may result from very few metallic
nanotubes being present in the less-dense network used to create the
device with characteristics shown in
Figure~\ref{fig-pristine-cnt-characteristics} (a) \autocite{Rouhi2011}.

Transfer measurements were taken to determine whether backgated
measurements could be taken of an unencapsulated device in the vapour
sensor chamber with \(1 \times\) PBS covering the channels. The use of a
backgated configuration with channels in a liquid environment is
generally considered less than ideal, since the sensitivity of the
device is greatly reduced \autocite{Li2023}.
Figure~\ref{fig-buffer-effect-on-backgate} (a) and (b) show the
behaviour of an unencapsulated backgated device before and after being
covered by 50 µL of \(1 \times\) PBS. The on-off ratio and hysteresis of
the channels both increase significantly. The presence of water
increases hysteresis through introducing charge traps at the SiO\(_2\)
surface around the carbon nanotubes and at the surface of the nanotubes
themselves \autocite{Kim2003,Lee2007,Franklin2012,Ha2014}. There is also
a significant increase in current leakage to the backgate for larger
applied voltages, despite the PBS having no visible physical contact
with the Si backgate or Cu plane. This leakage current may simply be due
to an increase in humidity around the device \autocite{Conseil2014}.

\begin{figure}

\begin{minipage}[t]{0.03\linewidth}

{\centering 

\raisebox{-\height}{

\includegraphics{figures/(a).png}

}

}

\end{minipage}%
%
\begin{minipage}[t]{0.01\linewidth}

{\centering 

~

}

\end{minipage}%
%
\begin{minipage}[t]{0.45\linewidth}

{\centering 

\raisebox{-\height}{

\includegraphics{figures/ch5/Q35C3_nobuffer.png}

}

}

\end{minipage}%
%
\begin{minipage}[t]{0.01\linewidth}

{\centering 

~

}

\end{minipage}%
%
\begin{minipage}[t]{0.03\linewidth}

{\centering 

\raisebox{-\height}{

\includegraphics{figures/(b).png}

}

}

\end{minipage}%
%
\begin{minipage}[t]{0.01\linewidth}

{\centering 

~

}

\end{minipage}%
%
\begin{minipage}[t]{0.45\linewidth}

{\centering 

\raisebox{-\height}{

\includegraphics{figures/ch5/Q35C3_buffer.png}

}

}

\end{minipage}%
%
\begin{minipage}[t]{0.01\linewidth}

{\centering 

~

}

\end{minipage}%

\caption[Backgated transfer sweeps of an single unencapsulated device
before and after exposure to 50 µL \(1 \times\)
PBS.]{\label{fig-buffer-effect-on-backgate}Backgated transfer sweeps of
an single unencapsulated device with a 300 nm SiO\(_2\) layer and steam
assisted surfactant-deposited carbon nanotube network channels before
and after being covered in 50 µL \(1 \times\) PBS.}

\end{figure}

The liquid-gated devices in
Figure~\ref{fig-pristine-cnt-characteristics} (b),
Figure~\ref{fig-pristine-cnt-characteristics} (d) and
Figure~\ref{fig-pristine-cnt-characteristics} (f) each exhibited
ambipolar characteristics, commonly observed in liquid-gated carbon
nanotube network FETs
\autocite{Kauffman2008,Heller2009,JongYu2009,Derenskyi2014,Murugathas2018,Albarghouthi2022}.
When devices were appropriately configured, leakage current (shown by
the dashed traces) did not exceed \(\sim 1 \times 10^{-7}\) V across the
forward and reverse sweeps. The devices shown which used steam-deposited
carbon nanotube films consistently showed the least hysteresis.
Section~\ref{sec-pristine-AFM} demonstrates that the median diameter of
the bundles in surfactant-deposited films is \(\sim\) 4 nm less than
those in films deposited in solvent. Hysteresis is known to scale
roughly linearly with bundle diameter, due to trapped charge increasing
as bundle density of states is increased \autocite{Pop2009}.
Steam-deposited devices also showed significantly less
channel-to-channel variation in electrical characteristics more
generally. Channel 1 in Figure~\ref{fig-pristine-cnt-characteristics}
(b) has a much higher off-current than the other channels of the same
device, which appears to be due to a unusually high proportion of
metallic carbon nanotubes present in the network conduction pathways of
this channel \autocite{Rouhi2011,Zaumseil2015}.

A summary of key parameters of pristine liquid-gated devices is shown in
Figure~\ref{fig-sweep-parameters}. The full dataset consists of three
sets of 21 liquid-gated transfer characteristics of working channels,
with each set corresponding to the use of a particular network
deposition method. Measurements from at least three devices are included
in each set. Each entry in the summary corresponds to the average of the
specific parameter in the forward and reverse sweep direction.
Steam-deposited devices showed highly consistent channel-to-channel
electrical properties. Since the carbon nanotube films on these devices
are relatively dense, as seen in Table~\ref{tbl-histogram-parameters},
the network should be well above the percolation threshold. As many
carbon nanotube pathways connect across the channel in parallel, small
variations in the network morphology have less of an impact on the
overall channel behaviour \autocite{Murugathas2018}. The highly
consistent and reproducible subthreshold regime behaviour between
channels seen for steam-deposited devices is a desirable attribute for
reliable real-time multiplexed biosensing
\autocite{Kauffman2008,Heller2009,Gao2010}.

\begin{figure}

\begin{minipage}[t]{0.03\linewidth}

{\centering 

\raisebox{-\height}{

\includegraphics{figures/(a).png}

}

}

\end{minipage}%
%
\begin{minipage}[t]{0.01\linewidth}

{\centering 

~

}

\end{minipage}%
%
\begin{minipage}[t]{0.45\linewidth}

{\centering 

\raisebox{-\height}{

\includegraphics{figures/ch5/onoff_CNT.png}

}

}

\end{minipage}%
%
\begin{minipage}[t]{0.01\linewidth}

{\centering 

~

}

\end{minipage}%
%
\begin{minipage}[t]{0.03\linewidth}

{\centering 

\raisebox{-\height}{

\includegraphics{figures/(b).png}

}

}

\end{minipage}%
%
\begin{minipage}[t]{0.01\linewidth}

{\centering 

~

}

\end{minipage}%
%
\begin{minipage}[t]{0.45\linewidth}

{\centering 

\raisebox{-\height}{

\includegraphics{figures/ch5/trans.png}

}

}

\end{minipage}%
%
\begin{minipage}[t]{0.01\linewidth}

{\centering 

~

}

\end{minipage}%
\newline
\begin{minipage}[t]{0.03\linewidth}

{\centering 

\raisebox{-\height}{

\includegraphics{figures/(c).png}

}

}

\end{minipage}%
%
\begin{minipage}[t]{0.01\linewidth}

{\centering 

~

}

\end{minipage}%
%
\begin{minipage}[t]{0.45\linewidth}

{\centering 

\raisebox{-\height}{

\includegraphics{figures/ch5/threshold_V.png}

}

}

\end{minipage}%
%
\begin{minipage}[t]{0.01\linewidth}

{\centering 

~

}

\end{minipage}%
%
\begin{minipage}[t]{0.03\linewidth}

{\centering 

\raisebox{-\height}{

\includegraphics{figures/(d).png}

}

}

\end{minipage}%
%
\begin{minipage}[t]{0.01\linewidth}

{\centering 

~

}

\end{minipage}%
%
\begin{minipage}[t]{0.45\linewidth}

{\centering 

\raisebox{-\height}{

\includegraphics{figures/ch5/SS.png}

}

}

\end{minipage}%
%
\begin{minipage}[t]{0.01\linewidth}

{\centering 

~

}

\end{minipage}%

\caption[Boxplots showing the statistical distribution of the on-off
ratio, transconductance, threshold voltage and subthreshold slope of
liquid-gated encapsulated carbon nanotube network transistor channels
with thin-films deposited using various
methods.]{\label{fig-sweep-parameters}These boxplots show the (a) the
on-off ratio, (b) the transconductance at \(V_g\) = 0 V, (c) the
threshold voltage and (d) the subthreshold slope of 21 liquid-gated
carbon nanotube network transistor channels taken from a set of three
devices all fabricated using either solvent-deposited,
surfactant-deposited or steam-deposited methods. The boxes indicate the
25th and 75th percentile of the distribution.}

\end{figure}

Channels from surfactant-deposited film devices showed a larger on-off
ratio and subthreshold slope than those from solvent-deposited devices.
Increasing the ratio of gate-sensitive semiconducting carbon nanotubes
to metallic nanotubes tends to increase the on-off ratio
\autocite{LeMieux2008,Rouhi2011,Zaumseil2015,Murugathas2018}. Increasing
network density is expected to increase the proportion of metallic
nanotubes present \autocite{Rouhi2011}, and
Section~\ref{sec-pristine-raman} indicates there are more metallic
nanotubes present in the surfactant-deposited films than in the
solvent-deposited films. However, percolating conduction pathways
dominate device behaviour, and nanotube pathways across the channel with
a lower degree of bundling are less likely to contain metallic tubes
\autocite{Murugathas2018}. Therefore, the larger on-off ratio for
surfactant-deposited film devices is likely a result of their reduced
nanotube bundle size and reduced bundle size variation relative to
solvent-deposited films, as discussed in
Section~\ref{sec-pristine-morphology}. The increased on-off ratio also
results in a larger subthreshold slope, due to the lowering of
off-current across orders of magnitude. A large on-off ratio and
subthreshold slope both indicate reduced device power consumption. The
relatively large on-off ratio and subthreshold slope of steam-deposited
devices are therefore desirable for improved sensor performance
\autocite{Kauffman2008,Heller2009,Gao2010}.

All channels characterised had a positive threshold voltage (\(V_{t}\)).
The threshold voltage was largest and most consistent for steam-assisted
surfactant-deposited films. The steam-assisted surfactant-deposited
devices had an average threshold voltage of 0.37 V, significantly higher
than the solvent-deposited and surfactant-only network devices, with
average threshold voltages of 0.27 V and 0.22 V respectively. This
increased threshold voltage corresponds to increased \(p\)-doping of the
network \autocite{Kang2005,Heller2008,Murugathas2018}. As seen from
Figure~\ref{fig-afm-substrate} (e)-(f) and
Figure~\ref{fig-pristine-raman} (c), the steam deposition process leads
to the presence of \(p\)-dopant contamination due to trapped water
vapour or surfactant on the carbon nanotubes. It has been previously
established that residual surfactant can also enhance the \(p\)-doping
from adsorbed oxygen and water
\autocite{Kane2014,Nonoguchi2018,Christensen2022}. The analysis by Kane
\emph{et al.} shows that the thermal annealing used in this work is
likely inadequate for removing residual surfactant.

The solvent-deposited devices had a slightly larger transconductance at
\(V_g\) = 0 V, the operating voltage used for sensing. As the carbon
nanotube networks of surfactant-deposited film devices are denser than
those with solvent-deposited films, they should exhibit increased
ballistic transport of charge carriers, increased mobility and therefore
increased transconductance \autocite{Rouhi2011}. The observation of
slightly reduced transconductance in surfactant-deposited devices
(\(\sim\) 1 µS) relative to solvent-deposited devices (\textgreater{} 2
µS) is therefore surprising. However, the transconductance behaviour at
\(V_g\) = 0 V for each device morphology partly depends on threshold
voltage. The position on the transfer curve corresponding to maximum
transconductance depends on threshold voltage. If \(V_g\) = 0 V is close
to the point of maximum transconductance, slight shifts in threshold
voltage may dramatically lower transconductance. Surfactant deposited
devices deposited without steam had a lower threshold voltage on average
than the solvent-deposited devices, while the steam-assisted
surfactant-deposited devices had a higher threshold voltage on average.
As a larger transconductance is desirable for enhanced sensitivity and
low power operation, restoring the threshold voltage by removing
surfactant may be useful for device optimisation.

\hypertarget{sec-dummy-sensing}{%
\section{Aqueous Sensing of Phosphate Buffered Saline
Concentration}\label{sec-dummy-sensing}}

\hypertarget{sec-baseline-drift}{%
\subsection{Control Series and Baseline
Drift}\label{sec-baseline-drift}}

To verify the sensitivity of the fabricated field-effect transistors and
therefore test their suitability for sensing, control measurements
mimicking a typical sensing experiment were taken before functionalising
the channels of a carbon nanotube network device. The first step to
verifying device suitability was ensuring the device showed no response
to \(1 \times\) PBS, unwanted behaviour which has previously been
observed during sensing runs using similar devices
\autocite{Cassie2023}. This sequence is referred to in this thesis as
the ``PBS control series''. The PDMS well contained 80 µL \(1 \times\)
PBS at 0 s. The PBS control series ran over the first 1800 s, with 20 µL
\(1 \times\) PBS additions at 100 s, 200 s and 300 s, and 20 µL
subtractions at 400 s, 500 s and 600 s. The device was left untouched
over the next 1200 s to allow the current level to settle, seen to be
sufficient for similar devices \autocite{Cassie2023}. The gate voltage
was held at \(V_g\) = 0 V.

Figure~\ref{fig-salt-conc-control-series} (a) shows the PBS control
series corresponding to each device channel alongside gate current. In
both series, there is no clear stepwise response to any addition or
subtraction of \(1 \times\) PBS. Gate leakage current remains negligible
across the entire control series, with no change in response to
\(1 \times\) PBS additions. The current has a period of short-term decay
followed by much longer term baseline drift, similar to observations by
Lin \emph{et al.} and more recently Noyce \emph{et al.} for parallel
arrangements of single carbon nanotubes in air or vacuum
\autocite{Lin2006,Noyce2019}. This effect results from hysteretic
changes in the occupancy of charge traps in and around the substrate and
carbon nanotubes. The magnitude of baseline drift is lower for our
devices than for those characterised by Noyce \emph{et al.}, which may
be a result of numerous device and setup differences which affect the
presence of charge traps. These differences include the use of
liquid-gating instead of back-gating, a network of carbon nanotubes
instead of single nanotubes, a different channel length, a 300 nm
instead of 90 nm SiO\(_2\) layer, a gate voltage of 0.0 V instead of
-15.0 V, and an asymmetric, liquid-gated transfer sweep over a shorter
voltage range when characterising devices before each control series
\autocite{Noyce2019}.

\begin{figure}

\begin{minipage}[t]{0.11\linewidth}

{\centering 

~

}

\end{minipage}%
%
\begin{minipage}[t]{0.03\linewidth}

{\centering 

\raisebox{-\height}{

\includegraphics{figures/(a).png}

}

}

\end{minipage}%
%
\begin{minipage}[t]{0.01\linewidth}

{\centering 

~

}

\end{minipage}%
%
\begin{minipage}[t]{0.70\linewidth}

{\centering 

\raisebox{-\height}{

\includegraphics{figures/ch5/NTQ31C1_pristine_saltconc_sample_230324_control.png}

}

}

\end{minipage}%
%
\begin{minipage}[t]{0.15\linewidth}

{\centering 

~

}

\end{minipage}%
\newline
\begin{minipage}[t]{0.11\linewidth}

{\centering 

~

}

\end{minipage}%
%
\begin{minipage}[t]{0.03\linewidth}

{\centering 

\raisebox{-\height}{

\includegraphics{figures/(b).png}

}

}

\end{minipage}%
%
\begin{minipage}[t]{0.01\linewidth}

{\centering 

~

}

\end{minipage}%
%
\begin{minipage}[t]{0.70\linewidth}

{\centering 

\raisebox{-\height}{

\includegraphics{figures/ch5/NTQ31C1_pristine_saltconc_sample_230324_linear_fit_exp.png}

}

}

\end{minipage}%
%
\begin{minipage}[t]{0.15\linewidth}

{\centering 

~

}

\end{minipage}%

\caption[Salt concentration control series across six device channels,
with linear and exponential fits to the baseline drift of each
channel.]{\label{fig-salt-conc-control-series}The figure in (a) shows
the behaviour of six channels on a single liquid-gated steam-assisted
surfactant-deposited carbon nanotube network device, alongside the
current through the liquid gate and linear fits to the baseline drift
from \(1200-1800\) s. The source-drain voltage \(V_{ds}\) was 100 mV,
and gate voltage \(V_{g}\) was 0 V, while (b) shows the exponential fits
to the data from each channel from \(0-1800\) s with linear fits
subtracted.}

\end{figure}

As a first-order approximation to the longer time constant exponentials
discussed by Noyce \emph{et al.} \autocite{Noyce2019}, linear fits were
performed on each PBS control series from \(1200-1800\) s. These fits
are tangent to the curve of the sum of the larger time constant
exponentials, and are a close approximation to this curve when higher
order terms in the series expansion are approximately zero. This is only
the case when \(t\ll\tau_i\), where the time interval of interest \(t\)
is much shorter than the time constants of the larger time constant
exponentials, \(\tau_i\). These linear fits are shown by the dashed
black lines in Figure~\ref{fig-salt-conc-control-series} (a). The
parameters from each fit in Figure~\ref{fig-salt-conc-control-series}
(a) are shown in Table~\ref{tbl-linear-fits}, where \(I = c_1t + c_2\).
Intriguingly, the fits for channels 1, 5 and 7 are all parallel within
error. Furthermore, the gradient value for each fit in
Figure~\ref{fig-salt-conc-control-series} (a) is consistent within a 2.6
pA/s range across all channels. The current data from channel 1 is
closely approximated by the linear trendline across the entire control
series. No short-term decay is present for this channel, indicating the
channel has low net trapped charge. It is unclear why trapped charge is
significantly lower for channel 1 relative to the other channels.

\hypertarget{tbl-linear-fits}{}
\begin{longtable}[]{@{}lllllll@{}}
\caption{\label{tbl-linear-fits}The coefficients of linear fits to the
PBS control series of each channel between \(1200-1800\) s, where
\(c_1\) is the gradient and \(c_2\) is the constant term.\\
}\tabularnewline
\toprule\noalign{}
Channels & CH1 & CH2 & CH3 & CH5 & CH6 & CH7 \\
\midrule\noalign{}
\endfirsthead
\toprule\noalign{}
Channels & CH1 & CH2 & CH3 & CH5 & CH6 & CH7 \\
\midrule\noalign{}
\endhead
\bottomrule\noalign{}
\endlastfoot
\(c_1\) (pA/s) & -5.1±0.2 & -7.2±0.1 & -6.5±0.1 & -5.0±0.1 & -7.6±0.1 &
-5.1±0.2 \\
\(c_2\) (\(\mu\)A) & 0.316 & 0.316 & 0.308 & 0.218 & 0.364 & 0.332 \\
\end{longtable}

The long-term linear fits were next subtracted from the raw control
series data. Figure~\ref{fig-salt-conc-control-series} (b) shows
exponential fits to the remaining curve from \(0 - 1800\) s, which was
successful for all channels except channels 1 and 5. The parameters from
each fit are shown in Table~\ref{tbl-exp-fits}, where
\(I = I_0\exp(-t/\tau)\). Any constant term \(I_C\) resulting from the
fit was negligible and so could be neglected. The exponential fits had
characteristic time constants \(\tau\) ranging between \(280 - 610\) s.
Note that the value of peak-to-peak noise is above 5\% of the initial
current value for all channels. This result indicates that 3 time
constants is a sufficient length of time for this short-term baseline
drift to decay almost completely for each channel. At most, a time
period of \(1830 \pm 150\) s is required to minimise the drift present
when sensing, fulfilled here by the length selected for the control
series.

\hypertarget{tbl-exp-fits}{}
\begin{longtable}[]{@{}lllll@{}}
\caption{\label{tbl-exp-fits}The coefficients of exponential fits to the
PBS control series of each channel between \(0-1200\) s, after the
linear fit has been subtracted, where \(I_0\) is the gradient and
\(\tau\) is the time constant.\\
}\tabularnewline
\toprule\noalign{}
Channels & CH2 & CH3 & CH6 & CH7 \\
\midrule\noalign{}
\endfirsthead
\toprule\noalign{}
Channels & CH2 & CH3 & CH6 & CH7 \\
\midrule\noalign{}
\endhead
\bottomrule\noalign{}
\endlastfoot
\(I_0\) (nA) & \(6.07\pm0.08\) & \(7.19\pm0.11\) & \(5.75\pm0.12\) &
\(9.68\pm0.41\) \\
\(\tau\) (s) & \(450\pm10\) & \(610\pm30\) & \(280\pm10\) &
\(350\pm30\) \\
\end{longtable}

From this analysis it appears that the baseline drift for the
liquid-gated carbon nanotube network devices can generally be
approximated as a combination of a linear and exponential term. The
exponential term corresponds to short-term, fast decaying baseline
drift, while the linear term is an approximation to longer-term, slow
decaying exponential baseline drift. The lack of response at all of the
six PBS addition and removal times gives us confidence that this is a
stable baseline which can be used for reliable sensing. Furthermore, the
baseline drift can reasonably be approximated as linear after
\(\sim 1800\) s. This linear drift has a small gradient of less than -10
pA/s. The relatively small size of this drift makes it easier to
distinguish responses due to analyte addition from baseline drift. It
can therefore be concluded that the 1800 s length of the PBS control
series is appropriate for further experimental work.

\hypertarget{sec-salt-conc-series}{%
\subsection{Sensing Series}\label{sec-salt-conc-series}}

A salt concentration sensing series was performed from 1800 s onwards,
directly after the PBS control series. The responses to successive
dilutions of the PBS gate were recorded to confirm the fabricated
devices were sensitive to small environmental changes in their pristine
state, to check for spurious signals, and to ensure gate current leakage
or other confounding factors were not contributing to sensing responses.
The PDMS well contained 80 µL \(1 \times\) PBS at 1800 s. During the
series, successive additions of deionised water were made to reduce the
concentration of PBS in the well. An initial \(1 \times\) PBS addition
was performed at 2100s, to confirm no changes occurred during the PBS
control series that would interfere with sensing. All additions to the
well in the sensing series and resulting changes to the PBS
concentration in the well are shown in Table~\ref{tbl-salt-conc-series}.

\hypertarget{tbl-salt-conc-series}{}
\begin{longtable}[t]{lcccccc}
\caption{\label{tbl-salt-conc-series}A summary of the times at which 20 µL additions were made to the PDMS
well, also showing the concentration in the well and the change in
concentration after each addition. The well contained 80 µL of
\(1 \times\) PBS at 1800 s. The major component in PBS is NaCl, which
has a concentration of 137 mM in \(1 \times\) PBS. }\tabularnewline

\toprule
\multicolumn{1}{c}{ } & \multicolumn{1}{c}{1× PBS} & \multicolumn{5}{c}{DI Water Additions} \\
\cmidrule(l{3pt}r{3pt}){2-2} \cmidrule(l{3pt}r{3pt}){3-7}
Addition \# & 1 & 2 & 3 & 4 & 5 & 6\\
\midrule
Time (s) & 2100 & 2400 & 2700 & 3000 & 3300 & 3600\\
Final PBS volume (µL) & 100 & 120 & 140 & 160 & 180 & 200\\
Final PBS concentration & 1× & 0.83× & 0.71× & 0.63× & 0.56× & 0.50×\\
Δ PBS concentration & - & -0.17× & -0.12× & -0.09× & -0.07× & -0.06×\\
\bottomrule
\end{longtable}

Figure~\ref{fig-salt-conc-sensing} (a) shows a multiplexed salt
concentration sensing series from the channels of a single
AZ\(^\circledR\) 1518 encapsulated device, measured with the NI-PXIe.
The gate voltage used was 0 V, which meant current measurements were
well above the magnitude of the subthreshold device current. Gate
current measurements did not exceed 10 nA at any point. At each of the
deionised water addition times, the current traces for at least two out
of six channels showed a sharp, transient increase in current followed
by a return to an increased baseline. It is well established that
changing the salt concentration of the liquid gate has an electrostatic
gating effect on the carbon nanotubes or graphene, and changes the
transfer characteristics of the channel. This shift in transfer
characteristic leads to a real-time signal response to each addition
\autocite{Heller2009,Heller2010,Kireev2017}.

\begin{figure}

\begin{minipage}[t]{0.11\linewidth}

{\centering 

~

}

\end{minipage}%
%
\begin{minipage}[t]{0.03\linewidth}

{\centering 

\raisebox{-\height}{

\includegraphics{figures/(a).png}

}

}

\end{minipage}%
%
\begin{minipage}[t]{0.01\linewidth}

{\centering 

~

}

\end{minipage}%
%
\begin{minipage}[t]{0.70\linewidth}

{\centering 

\raisebox{-\height}{

\includegraphics{figures/ch5/NTQ31C1_pristine_saltconc_sample_230324.png}

}

}

\end{minipage}%
%
\begin{minipage}[t]{0.15\linewidth}

{\centering 

~

}

\end{minipage}%
\newline
\begin{minipage}[t]{0.11\linewidth}

{\centering 

~

}

\end{minipage}%
%
\begin{minipage}[t]{0.03\linewidth}

{\centering 

\raisebox{-\height}{

\includegraphics{figures/(b).png}

}

}

\end{minipage}%
%
\begin{minipage}[t]{0.01\linewidth}

{\centering 

~

}

\end{minipage}%
%
\begin{minipage}[t]{0.70\linewidth}

{\centering 

\raisebox{-\height}{

\includegraphics{figures/ch5/NTQ31C1_pristine_saltconc_sample_230324_detrend_trunc_arrows_normalised.png}

}

}

\end{minipage}%
%
\begin{minipage}[t]{0.15\linewidth}

{\centering 

~

}

\end{minipage}%
\newline
\begin{minipage}[t]{0.11\linewidth}

{\centering 

~

}

\end{minipage}%
%
\begin{minipage}[t]{0.03\linewidth}

{\centering 

\raisebox{-\height}{

\includegraphics{figures/(c).png}

}

}

\end{minipage}%
%
\begin{minipage}[t]{0.01\linewidth}

{\centering 

~

}

\end{minipage}%
%
\begin{minipage}[t]{0.70\linewidth}

{\centering 

\raisebox{-\height}{

\includegraphics{figures/ch5/NTQ31C1_pristine_saltconc_sample_230324_filtered_detrend_trunc_arrows_normalised.png}

}

}

\end{minipage}%
%
\begin{minipage}[t]{0.15\linewidth}

{\centering 

~

}

\end{minipage}%

\caption[Salt concentration sensing series across eight device channels,
shown alongside the same dataset after the use of various filtering
approaches.]{\label{fig-salt-conc-sensing}A multiplexed salt
concentration sensing series across eight channels of a steam-assisted
surfactant-deposited carbon nanotube network device. The source-drain
voltage \(V_{ds}\) was 100 mV, and gate voltage \(V_g\) was 0 V. In (a),
the raw current measurements for each channel are shown alongside gate
current. The same measurements after despiking, removal of baseline
drift and normalisation to initial current are shown in (b), (c) shows
the data in (b) after being processed with a moving median filter.}

\end{figure}

Using the data in Table~\ref{tbl-linear-fits}, the linear term
approximating baseline drift (\(c_1t\)) for each channel can be
subtracted from the data in Figure~\ref{fig-salt-conc-sensing} (a) to
account for the downward drift. The mean current level just before 1800
s then becomes roughly constant. Next, each channel is normalised
relative to their initial mean current level \(I_{0}\). Artifacts
resulting from PXIe-2737 module lag, single datapoints which fall well
below the current level of the immediately preceding and succeeding
datapoints, are also removed. This ``despike'' process uses an
interquartile range filter, which is described in
Section~\ref{sec-python-analysis}. The resulting dataset is shown in
Figure~\ref{fig-salt-conc-sensing} (b). This figure shows that the
signal-to-noise ratio remains roughly similar across all channels of the
device. However, the behaviour of the initial transient increase with
each addition is highly variable across channels and between additions
for a single channel.

As measurement of the highly variable initial transient is not useful
for robust sensing purposes, a moving median filter was applied, with
the implementation of this filter discussed in
Section~\ref{sec-python-analysis}. The filtered data is shown in
Figure~\ref{fig-salt-conc-sensing} (c). Noise and initial transients are
removed completely, while the clearly defined step to a new current
baseline is retained. Using the real-time data in
Figure~\ref{fig-salt-conc-sensing} (c), a plot of signal against
addition can be created using the method described in
Section~\ref{sec-python-analysis}, shown in
Figure~\ref{fig-salt-conc-signal} (a). This approach illustrates the
increase at each step relative to \(I_{0}\).

\begin{figure}

\begin{minipage}[t]{0.11\linewidth}

{\centering 

~

}

\end{minipage}%
%
\begin{minipage}[t]{0.03\linewidth}

{\centering 

\raisebox{-\height}{

\includegraphics{figures/(a).png}

}

}

\end{minipage}%
%
\begin{minipage}[t]{0.01\linewidth}

{\centering 

~

}

\end{minipage}%
%
\begin{minipage}[t]{0.70\linewidth}

{\centering 

\raisebox{-\height}{

\includegraphics{figures/ch5/NTQ31C1_mean_simple_difference_before_and_after_step_filtered_concentrations.png}

}

}

\end{minipage}%
%
\begin{minipage}[t]{0.15\linewidth}

{\centering 

~

}

\end{minipage}%
\newline
\begin{minipage}[t]{0.14\linewidth}

{\centering 

~

}

\end{minipage}%
%
\begin{minipage}[t]{0.03\linewidth}

{\centering 

\raisebox{-\height}{

\includegraphics{figures/(b).png}

}

}

\end{minipage}%
%
\begin{minipage}[t]{0.01\linewidth}

{\centering 

~

}

\end{minipage}%
%
\begin{minipage}[t]{0.65\linewidth}

{\centering 

\raisebox{-\height}{

\includegraphics{figures/ch5/salt_conc_box_plot.png}

}

}

\end{minipage}%
%
\begin{minipage}[t]{0.17\linewidth}

{\centering 

~

}

\end{minipage}%

\caption[Signal changes from the previous figure, alongside a fit to the
median change in signal for each
addition.]{\label{fig-salt-conc-signal}The signal changes in
Figure~\ref{fig-salt-conc-sensing} (c) are shown in (a). This signal
data is then shown in box plot format in (b) alongside a fit to the
median change in signal for each addition, where \(R^2\) = 0.86.}

\end{figure}

Intriguingly, even though the largest change in PBS concentration
occurred at the first deionised water addition (see
Table~\ref{tbl-salt-conc-series}), there was very little signal change
across all channels, while a relatively large change occurred at the
second addition. The logarithm of final salt concentration has
previously been shown to be proportional to conductance change in the
linear on-regime \autocite{Heller2010}.
Figure~\ref{fig-salt-conc-signal} (b) shows the signal change presented
in terms of this logarithmic relationship. The median values of the
first two additions do not line up well with the overall logarithmic
trend; insufficient mixing in the tightly enclosed PDMS well environment
for the first few additions may be responsible for this result.
Subsequent additions may improve mixing in the well, leading to the
change in concentration at the surface of the channel being more
representative of the overall concentration in the well.

In Figure~\ref{fig-salt-conc-sensing} (b) and
Figure~\ref{fig-salt-conc-sensing} (c), from around the second deionised
water addition onwards, the drift behaviours of individual channels
begin to significantly diverge. This deviation from the baseline drift
subtracted from the raw data occurs either because the linear fit is
only a first-order approximation which weakens with time, or because the
additions themselves affect the drift behaviour. Displaying the data as
discrete signal changes, as in Figure~\ref{fig-salt-conc-signal} (a), is
one way of excluding these deviations (see
Section~\ref{sec-python-analysis}). An alternative way of presenting the
signal changes, by normalising relative to both \(I_{0}\) and the final
current reading with the formula \((I - I_{0})/(I_{f} - I_{0})\), is
shown in Figure~\ref{fig-salt-conc-sensing-2}. This approach is useful
for filtering out remaining unaccounted-for drift behaviour in order to
compare the short-term transient responses to additions across the
device channels. Furthermore, it lets us better understand how the
short-term transient responses affect the longer-term step responses
discussed earlier.

\begin{figure}

\begin{minipage}[t]{0.11\linewidth}

{\centering 

~

}

\end{minipage}%
%
\begin{minipage}[t]{0.03\linewidth}

{\centering 

\raisebox{-\height}{

\includegraphics{figures/(a).png}

}

}

\end{minipage}%
%
\begin{minipage}[t]{0.01\linewidth}

{\centering 

~

}

\end{minipage}%
%
\begin{minipage}[t]{0.70\linewidth}

{\centering 

\raisebox{-\height}{

\includegraphics{figures/ch5/NTQ31C1_pristine_saltconc_sample_230324_detrend_trunc_arrows_normalised_2.png}

}

}

\end{minipage}%
%
\begin{minipage}[t]{0.15\linewidth}

{\centering 

~

}

\end{minipage}%
\newline
\begin{minipage}[t]{0.11\linewidth}

{\centering 

~

}

\end{minipage}%
%
\begin{minipage}[t]{0.03\linewidth}

{\centering 

\raisebox{-\height}{

\includegraphics{figures/(b).png}

}

}

\end{minipage}%
%
\begin{minipage}[t]{0.01\linewidth}

{\centering 

~

}

\end{minipage}%
%
\begin{minipage}[t]{0.70\linewidth}

{\centering 

\raisebox{-\height}{

\includegraphics{figures/ch5/NTQ31C1_pristine_saltconc_sample_230324_filtered_detrend_trunc_arrows_normalised_2.png}

}

}

\end{minipage}%
%
\begin{minipage}[t]{0.15\linewidth}

{\centering 

~

}

\end{minipage}%
\newline
\begin{minipage}[t]{0.11\linewidth}

{\centering 

~

}

\end{minipage}%
%
\begin{minipage}[t]{0.03\linewidth}

{\centering 

\raisebox{-\height}{

\includegraphics{figures/(c).png}

}

}

\end{minipage}%
%
\begin{minipage}[t]{0.01\linewidth}

{\centering 

~

}

\end{minipage}%
%
\begin{minipage}[t]{0.70\linewidth}

{\centering 

\raisebox{-\height}{

\includegraphics{figures/ch5/NTQ31C1_pristine_saltconc_sample_230324_single_step.png}

}

}

\end{minipage}%
%
\begin{minipage}[t]{0.15\linewidth}

{\centering 

~

}

\end{minipage}%

\caption[Salt concentration sensing series presented using an
alternative normalisation approach.]{\label{fig-salt-conc-sensing-2}The
processed data shown in Figure~\ref{fig-salt-conc-sensing} (b) and
Figure~\ref{fig-salt-conc-sensing} (c) is normalised to \(I_{0}\), but
an alternative normalisation can more effectively filter out remaining
drift present. This normalisation presents data relative to both
\(I_{0}\) and the final current reading \(I_{f}\) using the formula
\((I - I_0)/(I_f - I_0)\). Using this normalisation, the data in
Figure~\ref{fig-salt-conc-sensing} (b) and
Figure~\ref{fig-salt-conc-sensing} (c) can be displayed instead as (a)
and (b) respectively. (c) shows a magnified version of the step at
addition 2 in (a).}

\end{figure}

Figure~\ref{fig-salt-conc-sensing-2} (a) and
Figure~\ref{fig-salt-conc-sensing-2} (c) show that the transient
responses to DI water additions vary significantly across the surface of
the device. For example, Figure~\ref{fig-salt-conc-sensing-2} (c) shows
that in response to the second DI water addition, channel 7 gives a
large initial transient response about twice the size of the step
increase between 2600 and 2800 s. Meanwhile, channels 1 and 2 show no
transient response above the step increase.
Figure~\ref{fig-salt-conc-sensing-2} (c) indicates transient size is
based on location across the device, with neighbouring channels showing
the most similarities. This spatially-dependent behaviour may indicate
transient responses are determined by the location of the channel
relative to either the location of water additions or the
slightly-variable location of the liquid gate. Larger and longer-lasting
transient responses are not entirely removed by the moving median
filter, as shown by comparing Figure~\ref{fig-salt-conc-sensing-2} (a)
to Figure~\ref{fig-salt-conc-sensing} (c), and so careful placement of
additions is important when sensing to minimise this effect. However,
even the longest-lasting transients appear to decay to zero within about
200 s, demonstrating that a 200 s spacing between additions at minimum
is necessary for reliable real-time liquid-gated sensing using this
setup.

To explore the effect of gate voltage on signal-to-noise ratio in the
subthreshold regime, two salt concentration sensing series were
performed using the same channel at different gate voltages, shown in
Figure~\ref{fig-salt-conc-SNR} (a). Figure~\ref{fig-salt-conc-SNR} (b)
shows the sensing series between 1800 s and 2700 s.

\begin{figure}

\begin{minipage}[t]{0.03\linewidth}

{\centering 

\raisebox{-\height}{

\includegraphics{figures/(a).png}

}

}

\end{minipage}%
%
\begin{minipage}[t]{0.01\linewidth}

{\centering 

~

}

\end{minipage}%
%
\begin{minipage}[t]{0.36\linewidth}

{\centering 

\raisebox{-\height}{

\includegraphics{figures/ch5/Q2C10ch8custom.png}

}

}

\end{minipage}%
%
\begin{minipage}[t]{0.01\linewidth}

{\centering 

~

}

\end{minipage}%
%
\begin{minipage}[t]{0.03\linewidth}

{\centering 

\raisebox{-\height}{

\includegraphics{figures/(b).png}

}

}

\end{minipage}%
%
\begin{minipage}[t]{0.01\linewidth}

{\centering 

~

}

\end{minipage}%
%
\begin{minipage}[t]{0.55\linewidth}

{\centering 

\raisebox{-\height}{

\includegraphics{figures/ch5/saltconc_initial_additions.png}

}

}

\end{minipage}%

\caption[Figure demonstrating change in signal-to-noise ratio resulting
from adjusting gate voltage.]{\label{fig-salt-conc-SNR}The transfer
characteristics of a single steam-deposited carbon nanotube field-effect
transistor channel are shown in (a). \(V_{gap}\) is the gate voltage
corresponding to the center of the transistor bandgap, found at the
minimum of the characteristic curve. The signal-to-noise ratio of the
channel response to a deionised water addition after a suitable control
series is shown in (b). The blue current trace in (b) was performed
gating the device 150 mV away from \(V_{gap}\), while the red current
was performed gating the device 200 mV away from \(V_{gap}\).}

\end{figure}

Previous work on the signal-to-noise ratio for liquid-gated,
encapsulated carbon nanotube network devices suggests that gating
devices close to \(V_{gap}\) should give a larger signal-to-noise ratio
for salt concentration changes \autocite{Heller2009}. However, in this
case, a clear response to DI water was observed at a gate voltage
further removed from \(V_{gap}\), while gating with a voltage closer to
\(V_{gap}\) led to a response almost indistinguishable from noise. This
discrepancy could be a result of the use of a carbon nanotube network
rather than a single nanotube, where gating may have less of an impact
on noise. Alternatively, it could be a result of a lack of mixing in the
well setup used, leading to inconsistent signal sizes across different
experimental series. Heller \emph{et al.} used a flow cell during their
signal-to-ratio work \autocite{Heller2009}. By using a flow cell with
our devices, better understanding of the role of mixing in the setup
used here could be achieved.

\hypertarget{conclusion}{%
\section{Conclusion}\label{conclusion}}

To ensure fabricated transistors were suitable for biosensing purposes,
the morphology and electrical properties of the pristine carbon nanotube
and graphene transistors were investigated.

The morphology of the carbon nanotube networks was found to have a
significant impact on the electrical characteristics of the devices,
determined by comparison of the skew-normal height profile of the carbon
nanotube network and the key electrical parameters of a selection of
carbon nanotube network devices. When carbon nanotubes were deposited in
solvent, the resulting networks were highly bundled (\(>97\) \% bundled)
and there were large variations in bundle diameter. Liquid-gated devices
created using these carbon nanotube films had highly variable on-off
ratios due to the large variation in the conductive pathways available.
In contrast, devices using films fabricated using surfactant present
showed lower bundling and less variation in bundle diameter.
Steam-deposited devices in particular showed improved device-to-device
reproducibility, particularly with respect to on-off ratio. These
devices also exhibited lower hysteresis, due to the relatively
consistent bundle diameters and high density of these networks. When
performing multiplexed sensing, consistent channel behaviour is highly
desirable since comparing sensing behaviour between channels is more
straightforward.

However, steam-deposited networks had the most surface contamination
present. Atomic force microscopy indicated the presence of contamination
due to steam-assisted deposition. Raman spectroscopy showed that the
steam deposition method was associated with increased defects in the
network, with a significantly increased D-peak to G\(^+\)-peak height
ratio. The average threshold voltage for steam-assisted
surfactant-deposited network devices was significantly higher than the
average threshold voltage for solvent-deposited devices, indicating
these defects \(p\)-dope the network. The shift in threshold voltage was
found to reduce device transconductance, a parameter which should be
maximised for optimal sensing. Furthermore, the presence of surfactant
or even trapped water could potentially have negative impacts on device
functionalisation, discussed further in
\textbf{?@sec-noncovalent-functionalisation}. Techniques to remove
contaminants may need to be explored in more detail in future works,
which is discussed further in \textbf{?@sec-future-work-fabrication}.
Raman spectroscopy could be used to verify the effectiveness of such
cleaning methods by comparing the \(I_{G^-}/I_{G^+}\) intensity peak
ratio before and after cleaning.

Constant voltage real-time measurements of the carbon nanotube
field-effect transistor devices had a characteristic drift that could be
modelled using both a linear and exponential term. This was true for
both liquid-gated and back-gated devices. The linear term of
liquid-gated baseline drift had a reasonably consistent gradient between
device channels, indicating that similar drift behaviour should be
reproducible between devices fabricated in the same manner. The
exponential behaviour was less consistent, indicating that this
hysteresis behaviour is less closely correlated with the fabrication
process. A control series length of 1800 s was used, as this period was
sufficient to minimise exponential drift during sensing for all the
channels tested.

A real-time PBS dilution sensing series was used to verify that the
carbon nanotube transistor devices were highly sensitive to changes in
an aqueous environment. The expected logarithmic signal response
behaviour was observed when detecting dilutions of PBS in a liquid-gate
environment. Signal size relative to baseline drift was highly
consistent between channels, which is promising for multiplexing work.
Deviations from the logarithmic trend possibly indicated insufficient
mixing within the PDMS well during sensing, which may be addressed in
future aqueous sensing work by using a flow cell. This experiment is a
useful control for sensing using functionalised devices. It is important
to confirm that the devices are highly sensitive, while also being able
to discern whether a signal response is due to interaction with an
attached biomolecule or whether the carbon nanotubes themselves are
responding.

Graphene field-effect transistor devices were often found to possess a
double-minima feature, which appears to be the result of a lack of
doping from the metal contacts in the center of the device channels.
These double Dirac points are unlikely to have any significant effect on
the sensing behaviour of graphene devices. The graphene device
characteristics were found to be highly consistent after 1 hour exposure
to PBS with minimal drift. There was some indications from the transfer
characteristics that \(p\)-dopants were present on the graphene surface.
Salt concentration sensing with graphene FETs is not shown in this
thesis, but it is important to perform similar device verification and
control testing when using a batch of graphene devices for
biofunctionalised sensing.

\cleardoublepage
\phantomsection
\addcontentsline{toc}{part}{Appendices}
\appendix

\hypertarget{vapour-system-hardware}{%
\chapter{Vapour System Hardware}\label{vapour-system-hardware}}

\hypertarget{tbl-vapour-sensor-components}{}
\begin{longtable}[]{@{}
  >{\raggedright\arraybackslash}p{(\columnwidth - 4\tabcolsep) * \real{0.5930}}
  >{\raggedright\arraybackslash}p{(\columnwidth - 4\tabcolsep) * \real{0.2209}}
  >{\raggedright\arraybackslash}p{(\columnwidth - 4\tabcolsep) * \real{0.1860}}@{}}
\caption{\label{tbl-vapour-sensor-components}Major components used in
construction of the vapour delivery system described in this
thesis.}\tabularnewline
\toprule\noalign{}
\begin{minipage}[b]{\linewidth}\raggedright
Description
\end{minipage} & \begin{minipage}[b]{\linewidth}\raggedright
Part No.
\end{minipage} & \begin{minipage}[b]{\linewidth}\raggedright
Manufacturer
\end{minipage} \\
\midrule\noalign{}
\endfirsthead
\toprule\noalign{}
\begin{minipage}[b]{\linewidth}\raggedright
Description
\end{minipage} & \begin{minipage}[b]{\linewidth}\raggedright
Part No.
\end{minipage} & \begin{minipage}[b]{\linewidth}\raggedright
Manufacturer
\end{minipage} \\
\midrule\noalign{}
\endhead
\bottomrule\noalign{}
\endlastfoot
Mass flow controller, 20 sccm full scale & GE50A-013201SBV020 & MKS
Instruments \\
Mass flow controller, 200 sccm full scale & GE50A-013202SBV020 & MKS
Instruments \\
Mass flow controller, 500 sccm full scale & FC-2901V & Tylan \\
Analogue flowmeter, 240 sccm max. flow & 116261-30 & Dwyer \\
Micro diaphragm pump & P200-B3C5V-35000 & Xavitech \\
Analogue flow controller, for micro diaphragm pump & X3000450 &
Xavitech \\
10 mL Schott bottle & 218010802 & Duran \\
PTFE connection cap system & Z742273 & Duran \\
Baseline VOC-TRAQ flow cell, purple & 043-950 & Ametek Mocon \\
Baseline VOC-TRAQ flow cell, red & 043-951 & Ametek Mocon \\
Humidity and temperature sensor & T9602-5-A & Telaire \\
\end{longtable}

\hypertarget{python-code-for-data-analysis}{%
\chapter{Python Code for Data
Analysis}\label{python-code-for-data-analysis}}

\hypertarget{code-repository}{%
\section{Code Repository}\label{code-repository}}

The code used for general analysis of field-effect transistor devices in
this thesis was written with Python 3.8.8. Contributors to the code used
include Erica Cassie, Erica Happe, Marissa Dierkes and Leo Browning. The
code is located on GitHub and the research group OneDrive, as well as
being publicly available on Figshare \autocite{Treacher2024}.

\hypertarget{sec-histogram-analysis}{%
\section{Atomic Force Microscope Histogram
Analysis}\label{sec-histogram-analysis}}

The purpose of this code is to analyse atomic force microscope (AFM)
images of carbon nanotube networks in .xyz format taken using an atomic
force microscope and processed in Gwyddion (see
\textbf{?@sec-afm-characterisation}). It was originally designed by
Erica Happe in Matlab, and adapted by Marissa Dierkes and myself for use
in Python. The code imports the .xyz data and sorts it into bins 0.15 nm
in size for processing. To perform skew-normal distribution fits, both
\emph{scipy.optimize.curve\_fit} and \emph{scipy.stats.skewnorm} modules
are used in this code.

\hypertarget{sec-raman-analysis}{%
\section{Raman Spectroscopy Analysis}\label{sec-raman-analysis}}

The purpose of this code is to analyse a series of Raman spectra taken
at different points on a single film (see
\textbf{?@sec-raman-characterisation}). Data is imported in a series of
tab-delimited text files, with the low wavenumber spectrum (100
cm\(^{-1} - 650\) cm\(^{-1}\)) and high wavenumber spectrum (1300
cm\(^{-1} - 1650\) cm\(^{-1}\)) imported in separate datafiles for each
scan location.

\hypertarget{sec-field-effect-transistor-analysis}{%
\section{Field-Effect Transistor
Analysis}\label{sec-field-effect-transistor-analysis}}

The purpose of this code is to analyse electrical measurements taken of
field-effect transistor (FET) devices. Electrical measurements were
either taken from the Keysight 4156C Semiconductor Parameter Analyser,
National Instruments NI-PXIe or Keysight B1500A Semiconductor Device
Analyser as discussed in \textbf{?@sec-electrical-characterisation}; the
code is able to analyse data in .csv format taken from all three
measurement setups. The main Python file in the code base consists of
three related but independent modules: the first analyses and plots
sensing data from the FET devices, the second analyses and plots
transfer characteristics from channels across a device, and the third
compares individual channel characteristics before and after a
modification or after each individual modification in a series of
modifications. The code base also features a separate config file and
style sheet which govern the behaviour of the main code. The code base
was designed collaboratively by myself and Erica Cassie over GitHub
using the Sourcetree Git GUI.

\hypertarget{sec-vapour-delivery-analysis}{%
\section{Vapour Delivery System
Analysis}\label{sec-vapour-delivery-analysis}}

The purpose of this code is to display electrical measurements taken of
field-effect transistor devices inside the vapour delivery system
chamber alongside data collected from the photoionisation detector and
mass flow controllers. Electrical measurements were taken with the
Keysight 4156C Semiconductor Parameter Analyser or Keysight B1500A
Semiconductor Device Analyser, using the vapour sensing chamber chip
carrier described in \textbf{?@sec-electrical-characterisation}.
Electrical data is imported from a .csv file, while photoionisation
detector data is imported from a tab-delimited text file and mass flow
controller data is imported from a .lvm file.

\hypertarget{references}{%
\chapter*{References}\label{references}}
\addcontentsline{toc}{chapter}{References}

\markboth{References}{References}

\begingroup
\raggedright

\printbibliography[heading=none]

\endgroup

\newpage{}

\hfill\break

\thispagestyle{empty}

\mbox{~} \clearpage \newpage


\backmatter

\end{document}
