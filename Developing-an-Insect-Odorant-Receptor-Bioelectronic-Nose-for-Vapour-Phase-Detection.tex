% Options for packages loaded elsewhere
\PassOptionsToPackage{unicode}{hyperref}
\PassOptionsToPackage{hyphens}{url}
%
\documentclass[
  a4paper,
]{scrbook}

\usepackage{amsmath,amssymb}
\usepackage{iftex}
\ifPDFTeX
  \usepackage[T1]{fontenc}
  \usepackage[utf8]{inputenc}
  \usepackage{textcomp} % provide euro and other symbols
\else % if luatex or xetex
  \usepackage{unicode-math}
  \defaultfontfeatures{Scale=MatchLowercase}
  \defaultfontfeatures[\rmfamily]{Ligatures=TeX,Scale=1}
\fi
\usepackage{lmodern}
\ifPDFTeX\else  
    % xetex/luatex font selection
  \setmainfont[]{Latin Modern Roman}
  \setsansfont[]{Latin Modern Roman}
\fi
% Use upquote if available, for straight quotes in verbatim environments
\IfFileExists{upquote.sty}{\usepackage{upquote}}{}
\IfFileExists{microtype.sty}{% use microtype if available
  \usepackage[]{microtype}
  \UseMicrotypeSet[protrusion]{basicmath} % disable protrusion for tt fonts
}{}
\makeatletter
\@ifundefined{KOMAClassName}{% if non-KOMA class
  \IfFileExists{parskip.sty}{%
    \usepackage{parskip}
  }{% else
    \setlength{\parindent}{0pt}
    \setlength{\parskip}{6pt plus 2pt minus 1pt}}
}{% if KOMA class
  \KOMAoptions{parskip=half}}
\makeatother
\usepackage{xcolor}
\setlength{\emergencystretch}{3em} % prevent overfull lines
\setcounter{secnumdepth}{5}
% Make \paragraph and \subparagraph free-standing
\ifx\paragraph\undefined\else
  \let\oldparagraph\paragraph
  \renewcommand{\paragraph}[1]{\oldparagraph{#1}\mbox{}}
\fi
\ifx\subparagraph\undefined\else
  \let\oldsubparagraph\subparagraph
  \renewcommand{\subparagraph}[1]{\oldsubparagraph{#1}\mbox{}}
\fi


\providecommand{\tightlist}{%
  \setlength{\itemsep}{0pt}\setlength{\parskip}{0pt}}\usepackage{longtable,booktabs,array}
\usepackage{calc} % for calculating minipage widths
% Correct order of tables after \paragraph or \subparagraph
\usepackage{etoolbox}
\makeatletter
\patchcmd\longtable{\par}{\if@noskipsec\mbox{}\fi\par}{}{}
\makeatother
% Allow footnotes in longtable head/foot
\IfFileExists{footnotehyper.sty}{\usepackage{footnotehyper}}{\usepackage{footnote}}
\makesavenoteenv{longtable}
\usepackage{graphicx}
\makeatletter
\def\maxwidth{\ifdim\Gin@nat@width>\linewidth\linewidth\else\Gin@nat@width\fi}
\def\maxheight{\ifdim\Gin@nat@height>\textheight\textheight\else\Gin@nat@height\fi}
\makeatother
% Scale images if necessary, so that they will not overflow the page
% margins by default, and it is still possible to overwrite the defaults
% using explicit options in \includegraphics[width, height, ...]{}
\setkeys{Gin}{width=\maxwidth,height=\maxheight,keepaspectratio}
% Set default figure placement to htbp
\makeatletter
\def\fps@figure{htbp}
\makeatother

\usepackage{booktabs}
\usepackage{longtable}
\usepackage{array}
\usepackage{multirow}
\usepackage{wrapfig}
\usepackage{float}
\usepackage{colortbl}
\usepackage{pdflscape}
\usepackage{tabu}
\usepackage{threeparttable}
\usepackage{threeparttablex}
\usepackage[normalem]{ulem}
\usepackage{makecell}
\usepackage{xcolor}
\usepackage{fancyhdr}
\usepackage{titling}
\setlength{\droptitle}{-2cm}
\preauthor{
  \begin{center}
  \Large
  \vspace{10mm}
  by
  \vspace{20mm}
}
\postauthor{
  \end{center}
  \vfill
}

\predate{
  \begin{center}
  A thesis 
  submitted in partial fulfilment of the \\
  requirements of the degree of \\
  Doctor of Philosophy in Physics\\               % Degree
  School of Physical and Chemical Sciences\\          % Department
  Te Herenga Waka - Victoria University of Wellington\\                       % University 
  \vspace{5mm}
}
\postdate{
  \\
  \includegraphics[width=3in,height=1.5in]{figures/VUW-logo.png}\\
  \end{center}
  }

\renewcommand{\topfraction}{.8}
\renewcommand{\floatpagefraction}{.8}
\clubpenalty=9996
\widowpenalty=9999
\makeatletter
\makeatother
\makeatletter
\@ifpackageloaded{bookmark}{}{\usepackage{bookmark}}
\makeatother
\makeatletter
\@ifpackageloaded{caption}{}{\usepackage{caption}}
\AtBeginDocument{%
\ifdefined\contentsname
  \renewcommand*\contentsname{Table of contents}
\else
  \newcommand\contentsname{Table of contents}
\fi
\ifdefined\listfigurename
  \renewcommand*\listfigurename{List of Figures}
\else
  \newcommand\listfigurename{List of Figures}
\fi
\ifdefined\listtablename
  \renewcommand*\listtablename{List of Tables}
\else
  \newcommand\listtablename{List of Tables}
\fi
\ifdefined\figurename
  \renewcommand*\figurename{Figure}
\else
  \newcommand\figurename{Figure}
\fi
\ifdefined\tablename
  \renewcommand*\tablename{Table}
\else
  \newcommand\tablename{Table}
\fi
}
\@ifpackageloaded{float}{}{\usepackage{float}}
\floatstyle{ruled}
\@ifundefined{c@chapter}{\newfloat{codelisting}{h}{lop}}{\newfloat{codelisting}{h}{lop}[chapter]}
\floatname{codelisting}{Listing}
\newcommand*\listoflistings{\listof{codelisting}{List of Listings}}
\makeatother
\makeatletter
\@ifpackageloaded{caption}{}{\usepackage{caption}}
\@ifpackageloaded{subcaption}{}{\usepackage{subcaption}}
\makeatother
\makeatletter
\@ifpackageloaded{tcolorbox}{}{\usepackage[skins,breakable]{tcolorbox}}
\makeatother
\makeatletter
\@ifundefined{shadecolor}{\definecolor{shadecolor}{rgb}{.97, .97, .97}}
\makeatother
\makeatletter
\makeatother
\makeatletter
\makeatother
\ifLuaTeX
  \usepackage{selnolig}  % disable illegal ligatures
\fi
\usepackage[citestyle = ieee,urldate = iso8601]{biblatex}
\addbibresource{references.bib}
\IfFileExists{bookmark.sty}{\usepackage{bookmark}}{\usepackage{hyperref}}
\IfFileExists{xurl.sty}{\usepackage{xurl}}{} % add URL line breaks if available
\urlstyle{same} % disable monospaced font for URLs
\hypersetup{
  pdftitle={Developing an Insect Odorant Receptor Bioelectronic Nose for Vapour-Phase Detection},
  pdfauthor={Eddyn Oswald Perkins Treacher},
  hidelinks,
  pdfcreator={LaTeX via pandoc}}

\title{Developing an Insect Odorant Receptor Bioelectronic Nose for
Vapour-Phase Detection}
\author{Eddyn Oswald Perkins Treacher}
\date{Nov 2024}

\begin{document}
\frontmatter

\maketitle

\clearpage
\newpage
\thispagestyle{empty} % Hide header and footer on this page
\mbox{~}
\clearpage
\newpage

%----------------------------------------------
%   Abstract
%----------------------------------------------

\thispagestyle{plain}

\begin{flushleft}
% Manually add a section to the table of contents
\pagenumbering{roman}
\addcontentsline{toc}{chapter}{Abstract}
\huge\textbf{Abstract}
\end{flushleft}

\vspace*{\baselineskip}

The ability to detect volatile organic compounds in a highly sensitive and selective manner is desirable for applications as varied as diagnosis of illnesses at a remote clinic, monitoring of air in an industrial setting, or identification of invasive organisms at a biosecurity checkpoint. Historically, animal noses have been used for such tasks, as their combined sensitivity and selectivity are superior to traditional artificial sensors. However, training and deploying animals in such situations is both time and cost intensive. In recent years, an improved understanding of \textit{in vivo} biological sensing has driven efforts to mimic these highly efficient processes in an artificial sensor format. \\[5pt] To this end, a “bioelectronic nose” was developed. This sensor uses an artificial transducer to amplify responses of an insect odorant receptor protein to specific volatile compounds. Thin-film transistors were used as the amplifier element, given their low cost, small size and extreme sensitivity. Various thin-film morphologies were compared, and their suitability for bioelectronic nose development assessed. Transducers made using a novel steam-assisted thin-film deposition technique were found to have highly consistent device-to-device electrical properties relative to other films. Films made using this process typically showed more surface contamination than other morphologies, but their high sensitivity was nonetheless confirmed with a non-specific sensing series in an aqueous environment. \\[5pt] One of the major challenges encountered in this thesis was variability in the quality of sensor functionalisation. Raman spectroscopy and fluorescence microscopy confirmed an existing non-covalent attachment method could successfully immobilise nanodiscs onto the transistor channel region. However, various sensors functionalised using the same procedure often exhibited no sensing activity. Extensive electrical characterisation indicated the presence of an unidentified contamination layer preventing electrical interaction between the insect odorant receptors and transducer thin-film. It was shown that this layer was unlikely to be directly associated with the thin-film morphology used for the transducer. \\[5pt] Subsequently, an alternative biotin-based non-covalent method was used for functionalisation of the proteins, which eliminated several possible contamination sources. This alternative biotin-based method was used to demonstrate successful aqueous sensing of femtomolar concentrations of methyl salicylate by an iOR10a-functionalised device. When tested in a custom-built vapour delivery system, a similar bioelectronic sensor was shown to be highly sensitive to the target vapour. However, consistent reproduction of the biotin-based method was challenging due to the harsh cleaning method involved. It was therefore difficult to determine conclusively whether sensor responses were selective. By finding new, systematic approaches to address the major barriers to sensor success carefully identified in this work, there are promising signs that a highly reliable vapour-phase bioelectronic nose can be produced.

%\fancyhf{} %clear all headers and footers fields
%\thispagestyle{fancy} % Change header and footer on this page
%\renewcommand{\headrulewidth}{0pt}
%\fancyhead[L]{\textit{Abstract}} % Set header content
%\fancyfoot[L]{\thepage} %prints the page number on the right side of the header

\clearpage
\newpage
\thispagestyle{empty} % Hide header and footer on this page
\mbox{~}
\clearpage
\newpage


%----------------------------------------------
%   Acknowledgement
%----------------------------------------------

\thispagestyle{plain}

\begin{flushleft}
% Manually add a section to the table of contents
\addcontentsline{toc}{chapter}{Acknowledgements}
\huge\textbf{Acknowledgements}
\end{flushleft}

\vspace*{\baselineskip}

I would first like to acknowledge the lands of my ancestors, and the lands of the sovereign first peoples to which my ancestors travelled. We each come from the land, live off the land and return to the land.\\[5pt]
\textit{Noon of Essex to Warrang, on the Friends, Autumn 1811} \\[5pt]
\textit{Cave of Cambridgeshire to Warrang, on the Royal Charlotte, Autumn 1825} \\[5pt]
\textit{Boyce of Suffolk to Warrang, 1832} \\[5pt] 
\textit{Charlton of Northumberland to Warrang, on the Clyde, Spring 1834} \\[5pt]
\textit{Prouse of Devonshire to Pito-one, on the Duke of Roxburgh, Summer 1840} \\[5pt]
\textit{Ebden of Devonshire to Pito-one, on the Tyne, Winter 1841} \\[5pt]
\textit{Collis of Hampshire to Pito-one, on the Birman, Autumn 1842} \\[5pt]
\textit{Swann of Loch Garman to Te Whanganui-a-Tara, 1844} \\[5pt] 
\textit{Blythe of Berkshire to Whakatū, circa 1846} \\[5pt]
\textit{Innes of Berkshire to Naarm, on the Sacramento, Autumn 1853} \\[5pt]
\textit{Sheppard of Gloucestershire to Naarm, 1853} \\[5pt] 
\textit{Bruce of London to Naarm, on the Omega, Autumn 1855} \\[5pt]
\textit{Quennell of Surrey to Warrang, on the Asiatic, Winter 1855} \\[5pt]
\textit{Barr of Glasgow to Kōpūtai, on the Sir Edward Paget, Winter 1856} \\[5pt] 
\textit{Perkins of London to Te Whanganui-a-Tara, on the Matoaka, Spring 1859} \\[5pt]
\textit{McKee of Antrim to Tāmaki Makaurau, on the Indian Empire, Spring 1862} \\[5pt]
\textit{Sandilands of Peeblesshire to Ōtepoti, circa 1864} \\[5pt] 
\textit{Treacher of Berkshire to Te Whanganui-a-Tara, on the Wild Duck, Summer 1865} \\[5pt]
\textit{McTaggart of Argyllshire to Kōpūtai, on the Edward P. Bouverie, Autumn 1869} \\[5pt] 
\textit{Chapman of Kent to Whakatū, on the Adamant, Winter 1874} \\[5pt]
\textit{Cheel of London to Whakatū, on the Queen Bee, Winter 1877} \\[5pt]  
\textit{Hutchison of Aberdeen to Tarntanya, before 1882.} \\[5pt] 
I chose to start my doctoral studies just a few months into a global pandemic. Completing a challenging project with a worldwide crisis in the background might have been impossible without the supervision of AProf. Natalie Plank. Her ability to adapt to and overcome any problem has taught me that there is no situation which is truly unmanageable. I am deeply grateful for her leadership throughout a time of particular chaos. \newpage
\fancyhf{} %clear all headers and footers fields
\thispagestyle{fancy} % Change header and footer on this page
\renewcommand{\headrulewidth}{0pt}
\fancyhead[L]{\textit{Acknowledgements}} % Set header content
\fancyfoot[L]{\thepage} %prints the page number on the left side of the header 
I started this project with minimal formal training in biological science, coming from a primarily physics and engineering background. \\[5pt] The immense support I received from Melissa Jordan and Colm Carraher from the Institute for Plant and Food Research (PFR) to complete this project meant that this was not an issue, and I thank them both immensely for this. \\[5pt] I would not have been able to begin this thesis without the financial backing and support I received from PFR and the Better Border Biosecurity (B3) programme. In particular, I am very grateful to Andrew Kralicek, formerly with PFR and now at Scentian Bio, and the ex-Director of B3, David Teulon, for helping to secure funding for my project. I would also like to thank the donor of the Ernest Marsden Scholarship in Physics for their significant financial support. \\[5pt] There are many incredibly supportive people who I worked alongside during my project. I would like to start off by thanking Rifat Ullah, whose mentoring and kindness encouraged me to pursue further study. His work on the initial design and setup of the vapour delivery system was invaluable to me throughout this project. I am also especially grateful to Alex Puglisi, for constructing the mechanical elements of the vapour delivery system and giving me extensive feedback on the system design. I would like to thank Peter Coard, for his advice and guidance when constructing the electrical elements of the vapour delivery system. I would also like to thank Selvan Murugathas for his advice on constructing the insect odorant receptor sensors, as well as Damon Colbert and Valentina Lucarelli, who provided the insect odorant receptor nanodiscs used in this work. \\[5pt] Thank you to AProf. Ben Ruck, my supportive secondary supervisor, and to AProf. Franck Natali, for always asking about my thesis in the tearoom. Thank you to Gideon Gouws for his friendly encouragement and advice. For their substantial technical assistance and mentoring during this project, I would also like to thank Alan Rennie, Grant Franklin, Chris Lepper, Rashika Gunasekara, Pete Jebson and Sushila Pillai from VUW, Andrew Chan from PFR, AProf. Charles Unsworth from the University of Auckland, and Prof. Simon Brown and his nanomaterials group from the University of Canterbury. \\[5pt] I was lucky enough to start my doctoral program just as a group of supportive and talented senior students were finishing, and finished just as a group of enthusiastic and talented new doctoral students were starting. A special thanks to Jenna Nyugen, Erica Happe and Erica Cassie for teaching me the fabrication processes and characterisation procedures that made this thesis happen; and a special thanks to Marissa Dierkes, Danica Fontein, Sangar Begzaad and Alireza Zare, for their incredible support throughout the thesis writing process. I would further like to thank everyone else I shared an office with and worked alongside, including Jackson, Will, Roshni, Ali, Kira, Catherine, Martin, Janani, Ted, Kiri and Joe. \\[5pt] A massive thank you to Openstar Technologies. It has been an honour to work on a cutting-edge plasma physics project right here in Te Whanganui-a-Tara. A particularly big thank you to Ratu, Darren and Thomas for having me as part of the plasma physics team. Thank you also to the other Openstar interns, in particular the other plasma physics interns, Valentina, Benjy and Chris. I wish you success in all your dipole-confined plasma related endeavours. \\[5pt] I want to thank Shodokan Aikido New Zealand for their support throughout this thesis, in particular for the once-in-a-lifetime opportunity to travel to Osaka to be graded for first-dan by Nariyama Shihan. Thanks for all the training and support, Ian. Thank you to all the friends and family, old and new, who have supported me over these wild past few years. You know who you are. \\[5pt] Thank you to my brother, Keeson, and to my parents, Hilary and Phillip. Your support means everything to me, and I would not be where I am today without you. Our Friday lunchtime cafe visits inspired and motivated me throughout the doctoral program. Thank you, thank you, thank you for your love, your compassion, and for being there for me. \\[5pt] Finally, thank you Nina. Your love has kept me going through the most difficult and most wonderful times over the last three and a half years. You are the light of my life, and I am so happy to have taken on this challenge with you by my side. \\[5pt] Arohanui and peace to you all, Eddyn (Ned)

\fancyhf{} %clear all headers and footers fields
\thispagestyle{fancy} % Change header and footer on this page
\renewcommand{\headrulewidth}{0pt}
\fancyhead[R]{\textit{Acknowledgements}} % Set header content
\fancyfoot[R]{\thepage} %prints the page number on the right side of the header

\clearpage
\newpage
\thispagestyle{empty} % Hide header and footer on this page
\mbox{~}
\clearpage
\newpage

\pagestyle{headings}

\ifdefined\Shaded\renewenvironment{Shaded}{\begin{tcolorbox}[frame hidden, boxrule=0pt, sharp corners, interior hidden, borderline west={3pt}{0pt}{shadecolor}, enhanced, breakable]}{\end{tcolorbox}}\fi

\renewcommand*\contentsname{Table of Contents}
{
\setcounter{tocdepth}{2}
\addcontentsline{toc}{chapter}{Table of Contents}
\tableofcontents
}
\listoffigures
\addcontentsline{toc}{chapter}{List of Figures}
\listoftables
\addcontentsline{toc}{chapter}{List of Tables}

\clearpage
\newpage
\thispagestyle{empty} % Hide header and footer on this page
\mbox{~}
\clearpage
\newpage

%----------------------------------------------
%   List of Abbreviations
%----------------------------------------------

\thispagestyle{plain}

\begin{flushleft}
% Manually add a section to the table of contents
\addcontentsline{toc}{chapter}{List of Abbreviations}
\huge\textbf{List of Abbreviations}
\end{flushleft}

\vspace*{\baselineskip}

\begin{table}[H]
  \begin{tabular}{@{}p{0.25\textwidth} p{0.75\textwidth}@{}}  % Adjust the width as needed
    2D  & 2-Dimensional  \\[5pt]
    Ab  & Antibody  \\[5pt]
    AB  & Amyl Butyrate  \\[5pt]
    AB-NTA  & N$\alpha$,N$\alpha$-Bis(carboxymethyl)-\textit{L}-lysine hydrate  \\[5pt]
    AFM  & Atomic Force Microscope/Microscopy  \\[5pt]
    AH  & Absolute Humidity  \\[5pt]
    Avi-tag  & Avidin-tag  \\[5pt]
    BMIM  & 1-butyl-3-methylimidazolium bis(trifluoromethylsulfonyl)imide  \\[5pt]
    CAD  & Computer Aided Design \\[5pt]
    CNT  & Carbon Nanotube  \\[5pt]
    CVD  & Chemical Vapour Deposition  \\[5pt]
    Cy3  & Cyanine 3  \\[5pt]
    DAN  & 1,5-diaminonaphthalene  \\[5pt]
    DAQ  & Data Acquisition Input/Output Module  \\[5pt]
    DCB  & 1,2-dichlorobenzene  \\[5pt]
    DI  & Deionised  \\[5pt]
    DMF  & Dimethylformamide   \\[5pt]
    DMSO  & Dimethylsulfoxide   \\[5pt]
    DMT-MM   & 4-(4,6-dimethoxy-1,3,5-triazin-2-yl)-4 methylmorpholinium chloride \\[5pt]
    DMMP  & Dimethyl Methylphosphonate  \\[5pt]
    DNA  & Deoxyribonucleic Acid  \\[5pt]
    E2Hex  & \textit{trans}-2-hexan-1-al  \\[5pt]
    EB  & Ethyl Butyrate  \\[5pt]
    EDC  & 1-Ethyl-3-(3-dimethylaminopropyl)carbodiimide  \\[5pt]
    EDL  & Electric Double Layer  \\[5pt]
    EIS  & Electrochemical Impedance Spectroscopy  \\[5pt]
    EtHex  & Ethyl Hexanoate  \\[5pt]
    EtOH  & Ethanol  \\[5pt]
    FET  & Field-Effect Transistor  \\[5pt]
  \end{tabular}
\end{table}

\newpage
\fancyhf{} %clear all headers and footers fields
\thispagestyle{fancy} % Change header and footer on this page
\renewcommand{\headrulewidth}{0pt}
\fancyhead[L]{\textit{List of Abbreviations}} % Set header content
\fancyfoot[L]{\thepage} %prints the page number on the right side of the header
\begin{table}[H]
  \begin{tabular}{@{}p{0.25\textwidth} p{0.75\textwidth}@{}}  % Adjust the width as needed
    FITC  & Fluorescein isothiocyanate  \\[5pt]
    GA  & Glutaraldehyde  \\[5pt]
    GFET  & Graphene Field-Effect Transistor  \\[5pt]
    GFP  & Green Fluorescent Protein  \\[5pt]
    GPCR  & G-protein Coupled Receptor  \\[5pt]
    HEK  & Human Embryonic Kidney  \\[5pt]
    His-tag  & Histidine-tag  \\[5pt]
    hOR  & Human Odorant Receptor  \\[5pt]
    HPLC  & High-performance Liquid Chromatography   \\[5pt]
    iOR  & Insect Odorant Receptor  \\[5pt]
    IPA  & Isopropanol  \\[5pt]
    LOD  & Limit of Detection  \\[5pt]
    m-CNT  & Metallic Carbon Nanotube   \\[5pt]
    MeOH  & Methanol   \\[5pt]
    MeSal  & Methyl Salicylate   \\[5pt]
    MFC  & Mass Flow Controller   \\[5pt]
    mOR  & Mouse Odorant Receptor  \\[5pt]
    MOSFET  & Metal-Oxide-Semiconductor Field-Effect Transistor  \\[5pt]
    MSP  & Membrane Scaffold Protein  \\[5pt]
    MWCNT  & Multi-Walled Carbon Nanotube  \\[5pt]
    ND  & Nanodisc  \\[5pt]
    NHS  & N-Hydroxysuccinimide  \\[5pt]
    NMR  & Nuclear Magnetic Resonance  \\[5pt]
    NSB  & Non-Specific Binding   \\[5pt]
    NTA  & Nitrilotriacetic Acid   \\[5pt]
    OBP  & Odorant Binding Protein  \\[5pt]
    OR  & Odorant Receptor  \\[5pt]
    ORCO  & Odorant Receptor Co-Receptor  \\[5pt]
    PBA  & 1-Pyrenebutyric Acid  \\[5pt]
    PBASE  & 1-Pyrenebutanoic Acid N-hydroxysuccinimide Ester  \\[5pt]
    PBS  & Phosphate-Buffered Saline  \\[5pt]
    PCB  & Printed Circuit Board   \\[5pt]
  \end{tabular}
\end{table}

\newpage
\fancyhf{} %clear all headers and footers fields
\thispagestyle{fancy} % Change header and footer on this page
\renewcommand{\headrulewidth}{0pt}
\fancyhead[R]{\textit{List of Abbreviations}} % Set header content
\fancyfoot[R]{\thepage} %prints the page number on the right side of the header
\begin{table}[H]
  \begin{tabular}{@{}p{0.25\textwidth} p{0.75\textwidth}@{}}  % Adjust the width as needed
    PDL & Poly-\textit{D}-lysine  \\[5pt]
    PDMS  & Polydimethylsiloxane   \\  [5pt]
    PEG  & Polyethylene Glycol  \\[5pt] 
    PID  & Photoionisation Detector  \\[5pt] 
    PLL  & Poly-\textit{L}-lysine  \\[5pt]
    PPB  & Pyrene-PEG-Biotin  \\[5pt]
    PPF  & Pyrene-PEG-FITC  \\[5pt]
    PPN  & Pyrene-PEG-NTA  \\[5pt]
    PPR  & Pyrene-PEG-Rhodamine  \\[5pt]
    PTFE  & Polytetrafluoroethylene (Teflon™)  \\[5pt]
    PVC  & Polyvinyl chloride  \\[5pt]
    QCM  & Quartz Crystal Microbalance  \\[5pt]
    RH  & Relative Humidity  \\[5pt]
    RHI  & Relative Humidity and Temperature Indicator  \\[5pt] 
    RNA  & Ribonucleic Acid   \\[5pt]
    SAW  & Surface Acoustic Wave   \\[5pt]
    s-CNT  & Semiconducting Carbon Nanotube   \\[5pt]
    SEM  & Scanning Electron Microscope/Microscopy   \\[5pt]
    SMU  & Source Measure Unit   \\[5pt]
    SPR  & Surface Plasmon Resonance   \\[5pt]
    Sulfo-NHS  & N-hydroxysulfosuccinimide   \\[5pt]
    SWCNT  & Single-Walled Carbon Nanotube   \\[5pt]
    TFTFET  & Thin-Film Field-Effect Transistor  \\[5pt]
    TMAH  & Tetramethylammonium hydroxide  \\[5pt]
    TX  & Transfer Characteristics  \\[5pt]
    UV  & Ultraviolet  \\[5pt]
    VI  & Virtual Instrument  \\[5pt]
    VUAA1  & N-(4-Ethylphenyl)-2-{[4-ethyl-5-(pyridin-3-yl)-4H-1,2,4-triazol-3-yl]sulfanyl}acetamide  \\[5pt] 
  \end{tabular}
\end{table}

\clearpage
\newpage
\thispagestyle{empty} % Hide header and footer on this page
\mbox{~}
\clearpage
\newpage

% Adjust the top and bottom margins of float pages to center floats
\makeatletter
\setlength{\@fptop}{0pt plus 1fil}
\setlength{\@fpbot}{0pt plus 1fil}
\makeatother

\pagestyle{headings}
\mainmatter
\bookmarksetup{startatroot}

\hypertarget{introduction}{%
\chapter{Introduction}\label{introduction}}

\hypertarget{background}{%
\section{Background}\label{background}}

The `bioelectronic nose', an electronic transducer modified with
elements of the animal olfactory system, has the potential to allow
specific detection of airborne volatile compounds at concentrations as
low as parts per trillion
\autocite{Glatz2011,Kwon2015,Dung2018,Kim2022a}. An ideal transducer
platform is the thin-film transistor (TFT) which is particularly
portable, simple to use, small and robust
\autocite{Kauffman2008,Khan2020}. The thin films used in these
field-effect transistors (FETs) include carbon nanotube networks and
graphene, low-dimensional nanomaterials which are both highly sensitive
and biocompatible \autocite{Shkodra2021}. The implications of successful
development of such a portable and robust bioelectronic nose are
significant. Applications could be found in high-importance fields such
as biosecurity, medicine, environmental protection and food safety
\autocite{Dung2018,Arakawa2019,Yang2017,Son2017}. For example, it has
been demonstrated that it is possible to detect invasive brown
marmorated stinkbugs based on their volatile trace \autocite{Moser2020}.
A bioelectronic nose could potentially accomplish this biosecurity task
far more cheaply and efficiently than trained sniffer dogs
\autocite{Lee2010,Moon2020,Terutsuki2020}. There has been rapid progress
in the development of bioelectronic noses using carbon nanotube
field-effect transistors (CNT FETs) and graphene field-effect
transistors (GFETs) over the past 15-20 years
\autocite{Yoon2009,Lee2010,Yang2018}.

Insect odorant receptors (iORs) enable simple invertebrates, such as the
vinegar fruit fly \emph{Drosophila melanogaster}, to distinguish between
a huge number of specific volatile compounds
\autocite{Hallem2004,Smart2008,Wicher2008,Munch2016,Bohbot2020}. Within
the past five years, a variety of \emph{Drosophila melanogaster} iORs
have been successfully coupled with highly sensitive low-dimensional
thin-film transistors (TFTs) for specific detection of fruit-like odors
in an aqueous environment \autocite{Murugathas2019a,Murugathas2020}.
iORs have also been used for sensitive and selective volatile detection
in a lipid bilayer format, but not in a portable bioelectronic nose
format \autocite{Yamada2021}. In this thesis, my aim was to investigate
whether a bioelectronic nose capable of odorant detection in a
vapour-phase environment could be constructed by coupling iORs with
TFTs. Alongside practical applications, development of a vapour-phase
bioelectronic nose using iORs may give us a greater understanding of the
mechanisms underlying insect olfaction \autocite{Lee2010}. The
transduction mechanism of nanomaterial-based iOR sensors is still
unknown, and I hope to shed further light on the biological and
electronic processes underpinning this mechanism
\autocite{Murugathas2020,Khadka2019,Cheema2021}.

\hypertarget{thesis-outline}{%
\section{Thesis Outline}\label{thesis-outline}}

This thesis consists of nine chapters. The first three chapters,
including this one, are background chapters introducing the general
topics of this thesis. The fourth and fifth chapters are methods
chapters, while the next three chapters (sixth, seventh and eight)
describe the results obtained. The ninth chapter concludes the thesis
and discusses possible next steps for future research.

\textbf{Chapter 2} gives a broad description of carbon nanotube and
graphene field-effect transistors with a focus on their use in sensing
applications. The chapter begins by looking at the general structure and
properties of thin-film transistors, where key figures of merit such as
transconductance, on-off ratio, gate current and hysteresis are
described. Graphene field-effect transistors (GFETs) and carbon nanotube
network field-effect transistors (CNT FETs) are then discussed in
greater detail. These descriptions include the chemical composition of
each nanomaterial, their conduction behaviour and their unique sensor
properties when integrated into a field-effect transistor as a
thin-film.

\textbf{Chapter 3} investigates existing odorant receptor-coupled
thin-film field-effect transistors in the literature. First, the
biological structure of odorant receptors and membrane formats for their
protection in vitro are discussed. Details are then provided regarding
the construction and operation of existing vertebrate odorant receptor
TFT biosensors. The structure and function of the insect odorant
receptor is then contrasted with the vertebrate odorant receptor, and
existing insect odorant receptor TFT biosensors in the literature are
discussed. The chapter finishes with a brief discussion of non-specific
binding and its role in hindering biosensor activity.

\textbf{Chapter 4} describes the fabrication of the CNT FET and GFET
transducers used in this thesis and the characterisation techniques used
to probe their behaviour. The chapter starts with an introduction to
photolithography for thin-film transistor device fabrication. Various
techniques are described for random deposition of carbon nanotube
networks to act as channels for these thin-film transistors.
Characterisation techniques described in this chapter include atomic
force microscopy (AFM), fluorescence microscopy, Raman spectroscopy and
electrical characterisation with various semiconductor device analysers.

\textbf{Chapter 5} outlines the development of a vapour delivery system
for characterisation of the insect odorant receptor functionalised TFT
biosensors in a vapour-phase environment. The vapour delivery system was
upgraded from an existing system to include new mass flow controllers,
to have greater control of flow through the system, and off the shelf
vapour sensors, to collect vapour flow data that could be used for
comparison against biosensor activity. The chapter also describes the
design and construction of an electronic interface to monitor and
control the components of the vapour delivery system, and calibration of
the system.

\textbf{Chapter 6} presents the results obtained from the use of
characterisation techniques on the pristine GFETs and CNT FETs. Various
carbon nanotube (CNT) network morphologies are displayed and analysed.
The Raman spectra and electrical device parameters of these CNT network
morphologies are then discussed, along with electrical parameters from
graphene devices. The sensitivity of a dense CNT network morphology
device is then verified in the aqueous phase.

\textbf{Chapter 7} explores the non-covalent functionalisation of GFETs
and CNT FETs with various linker molecules for insect odorant receptor
attachment. The linker molecules tested were 1-pyrenebutanoic acid
N-hydroxysuccinimide ester (PBASE) and 1-pyrenebutyric acid (PBA) with
1-Ethyl-3-(3-dimethylaminopropyl)carbodiimide (EDC). Pyrene-NTA and
pyrene-biotin were also investigated as other possible linker molecules.
The quality of various functionalisation approaches was then explored
with various fluorescently-tagged linker molecules and biomolecules. In
this process, various potential obstacles to successful biosensor
functionalisation were identified.

\textbf{Chapter 8} maps out progress made towards the creation of an
insect odorant receptor functionalised TFT biosensor for use in a
vapour-phase environment. Two different approaches are described that
gave rise to working aqueous-phase biosensors. The first
functionalisation approach, which used PBASE in methanol, led to
irreproducible results when biosensing. Possible factors causing the
unreliability of this method were then investigated. A second approach
was then designed to avoid the malign influence of any identified
factors.

\textbf{Chapter 9} details the use of the vapour delivery system for
testing the functionalised biosensors in the vapour phase. First, the
flow behaviour of volatile organic vapours through the system was
validated using onboard reference sensors. The response of a pristine
carbon nanotube device to two volatile compounds is then compared to the
response of carbon nanotube devices functionalised using the second
functionalisation approach discussed in the previous chapter.

\textbf{Chapter 10} summarises the conclusions drawn from this work, and
proposes various related studies which can be undertaken to continue the
work described in this thesis.

\bookmarksetup{startatroot}

\hypertarget{sec-iOR-sensors}{%
\chapter{Biosensing with Insect Odorant
Receptors}\label{sec-iOR-sensors}}

\hypertarget{sec-biosensing-transducers}{%
\section{Introduction}\label{sec-biosensing-transducers}}

In \textbf{?@sec-thin-film-transistors}, it was established that as
carbon nanotubes and graphene are extremely sensitive and are easily
modified with biomaterials, they are a highly suitable platform for
biosensing \autocite{Kauffman2008,Ohno2010}. In the early 2000s, it was
established that sensitive and selective biosensors could be created by
modifying a carbon nanotube field-effect transistor channel with protein
receptors \autocite{Chen2003,Kauffman2008}. In the following two
decades, a wide range of other biological receptors have been attached
to carbon nanotube FETs and graphene FETs for the creation of
biosensors, including enzymes \autocite{Lee2009,Zhang2015a,Dudina2019},
antibodies \autocite{Kim2008,Jin2015,Tsang2019} and aptameric DNA
\autocite{Maehashi2007,Gao2016,Nguyen2021}. These miniaturised `lab on a
chip' devices are of particular interest due to their reliability, low
cost, rapid use time, simple operation and small size compared with
alternative spectroscopic or fluorescence-based sensing methods
\autocite{Khan2020}. It is hoped that such sensors could be deployed
outside the laboratory in a range of front-line settings which require
rapid and reliable detection \autocite{Dung2018,Yang2018,Kim2022a}.

Rapid developments in this biosensor technology and parallel
developments in the understanding of animal olfaction led to these
transistors being used in bioelectronic nose applications from the late
2000s onwards \autocite{Yoon2009,Jin2012,Lee2012b,Park2012}.
`Bioelectronic nose' is a general term dating back to 1961, which refers
to the use of an biologically-modified electronic array to detect
specific odor traces in a highly selective and sensitive manner. As the
name suggests, the signals from this system mimic the electrochemical
signals received by olfactory neurons in an animal nose
\autocite{Glatz2011,Dung2018}. A biomimetic approach to bioelectronic
nose development couples the CNT FET or GFET signal-amplifying
transducer element with sensitive components of the animal olfactory
system. These components include olfactory cells \autocite{Wang2007},
odorant binding proteins (OBPs) \autocite{Larisika2015,Kotlowski2018}
and odorant receptor proteins (olfactory receptors, ORs)
\autocite{Yang2018,Murugathas2020}. An bioelectronic nose can discern
specific volatile odors in the air at low parts-per-trillion
concentrations \autocite{Lee2010,Moon2020}. This performance is far
superior to commercially-available vapour-phase sensors, which are at
best only responsive down to parts-per-billion concentrations
\autocite{GasDetector1,GasDetector2}. The aim for novel olfactory-based
electronic biosensors is to match or surpass this level of selective
accuracy to volatile organic vapours
\autocite{Glatz2011,Kwon2015,Dung2018,Bohbot2020,Kim2022a}.

\hypertarget{sec-odorant-receptors-biosensors}{%
\section{Odorant Receptors in Field-Effect Transistor
Biosensors}\label{sec-odorant-receptors-biosensors}}

\hypertarget{sec-odorant-receptors}{%
\subsection{Odorant Receptors}\label{sec-odorant-receptors}}

Odorant receptors (ORs) are an essential part of the olfactory systems
of most animals, including humans. ORs let us distinguish between
thousands of odors \autocite{Buck1991,Dung2018,Yang2018,Kim2022a}.
Vertebrate odorant receptors are part of a group of seven-transmembrane
proteins known as G-protein coupled receptors (GPCRs)
\autocite{Buck1991,Glatz2011,Dung2018,Wicher2021}. Compounds entering a
vertebrate nose selectively bind to specific odorant receptors, which
undergo a change in conformation \autocite{Dung2018,Kim2022a}. The
binding process leads to activation and dissociation of the G-protein
within the neuronal cell. Intracellular signalling events triggered by
G-protein dissociation are converted to an action potential which is
then transmitted to the brain \autocite{Buck1991,Glatz2011,Zhang2021}.
The combination of activated receptors is then interpreted as a specific
odor \autocite{Sato2014,Kwon2015,Hurot2020,Kim2022a}. Odorant receptors
may be activated by a few or many target (or `agonist') compounds. The
target compounds are determined by subtle differences in OR amino acid
composition \autocite{Carraher2015,Yang2018,Goodwin2021}. An
`antagonist' compound may inhibit the response of a receptor to other
compounds \autocite{Lee2012,Carraher2015}. Compounds which trigger a
strong signal from a specific odorant receptor are often referred to as
`positive ligands' for that receptor, while those that cause no response
are `negative ligands'
\autocite{Murugathas2019a,Murugathas2020,Yoo2022}.

\hypertarget{sec-artificial-membranes}{%
\subsection{Artificial Membranes}\label{sec-artificial-membranes}}

Odorant receptors are transmembrane proteins, which are insoluble and
tend to aggregate and oligomerise in solution \autocite{Nath2007}. They
therefore require stabilisation from either a specific detergent
environment or a membrane layer to preserve their structure and function
when solubilised \autocite{Fruh2011,Dung2018}. Odorant receptors can be
expressed and isolated using heterologous cell systems, where a host
cell replicates a protein from transfected RNA or DNA material
\autocite{Glatz2011,Dung2018}. The most commonly used expression cells
are human embryonic kidney (HEK) cells \autocite{Lim2014,Ahn2020},
\emph{E. Coli} bacteria \autocite{Yang2017,Yang2018} and \emph{S.
cerevisiae} (baker's yeast) \autocite{Bohbot2020}. The cell membrane can
then be used directly in a sensor \autocite{Dung2018}. Alternatively,
odorant receptors can be embedded in an artificial lipid membrane format
that mimics the native cell membrane \autocite{Nath2007}. These
membranes can be produced in large numbers and remain in storage for
much longer than live cells \autocite{Goldsmith2011,Lim2015}. Lipid
membranes are constructed from phospholipid molecules, which comprise of
a small, hydrophilic, polar `head' and long, hydrophobic, non-polar
`tail' \autocite{Bose2021,Ramadon2022}. These artificial membranes
include detergent micelles, nanovesicles (including liposomes), and
nanodiscs \autocite{Yang2018,Moon2020}. The small size of these
artificial membranes makes them appropriate for use with
nanomaterial-based transducers \autocite{Lim2015,Dung2018}.

\begin{figure}

{\centering \includegraphics[width=0.53\textwidth,height=\textheight]{figures/ch3/OSC_Microbio_07_03_micelle_edit.png}

}

\caption{\label{fig-micelle}Liposomes and micelles are made up of a
lipid membrane, which acts as a substitute for the cell membrane
\emph{in vitro}. The polar, hydrophilic `heads' and non-polar,
hydrophobic `tails' of the component phospholipids are indicated.
Adapted from \autocite{Micelle}.}

\end{figure}

A nanovesicle is a nanoscopic spherical bilayer fluid sac. There are
various types of artificial nanovesicles, including liposomes,
ethosomes, transfersomes, niosomes and phytosomes. The type of
nanovesicle depends on its chemical makeup \autocite{Ramadon2022}. For
example, a liposome is made up of phospholipid and cholesterol, and can
consist of one or more concentric amphiphilic bilayers. The liposome can
contain hydrophobic compounds within the bilayer due to hydrophobic
interactions, while hydrophilic compounds are held within the vesicle
core or interior \autocite{Nath2007,Ramadon2022}. A nanovesicle can be
used solely as a format to protect membrane proteins
\autocite{Murugathas2020}, or with the addition of integrated ion
channels, can mimic the operation of a cell \emph{in vivo}, with
intracellular signalling pathways triggered by the membrane proteins
\autocite{Lim2015}. Nanomicelles (or simply micelles) are also
nanoscopic and spherical, but unlike nanovesicles have no inner fluid
sac \autocite{Nath2007,Bose2021}. Micelles self-assemble when
phospholipid is mixed with detergent. The surface of the micelle is made
up of the hydrophilic detergent and phospholipid heads, while the
internal core is made up of the hydrophobic phospholipid tail
\autocite{Nath2007}. Hydrophobic compounds can be contained within the
core of the micelle \autocite{Bose2021}. Figure~\ref{fig-micelle}
illustrates the difference between the liposome and micelle structures.

\begin{figure}

{\centering \includegraphics[width=0.45\textwidth,height=\textheight]{figures/ch3/iOR_nanodisc.png}

}

\caption{\label{fig-msp-iOR-nanodisc}A lipid nanodisc wrapped around an
insect odorant receptor transmembrane protein. The nanodisc consists of
a lipid bilayer and membrane scaffold protein (MSP). The amine group
shown is the odorant receptor N-terminus. Reproduced with permission
from \autocite{Murugathas2019a}.}

\end{figure}

Nanodiscs have emerged as a model membrane candidate which has many
advantages over the more traditional nanovesicle and micelle formats.
The nanodisc is a disc-shaped lipid bilayer encompassed by an membrane
scaffold protein (MSP) \autocite{Nath2007,Bayburt2010,Yang2018}. The
amphiphilic membrane scaffold protein protects the exposed, strongly
hydrophobic side chains of the nanodisc in an aqueous environment
\autocite{Fruh2011,Yang2018}. Unlike liposomes and micelles, there is
little variation between the size of individual nanodiscs due to
constraints placed on the bilayer by the encompassing scaffold protein
used, meaning greater consistency within and between membrane batches
\autocite{Nath2007,Fruh2011}. Nanodiscs have also been found to be
significantly less prone to non-specific binding (see
Section~\ref{sec-non-specific-binding}) than micelles
\autocite{Fruh2011}. Another advantage of nanodiscs is that the membrane
scaffold protein can be attached to biosensor surfaces at specific
affinity tags, for example, the scaffold protein hexahistidine tag
(his-tag) \autocite{Bayburt2010,Fruh2011}. Depending on the type of MSP
used, a nanodisc measures between 10-20 nm across and can hold either a
single or several odorant receptors \autocite{Nath2007,Bayburt2010}. The
protein coating of the nanodisc makes it particularly stable. The
stability of nanodiscs means they can be used to produce particularly
reliable and long-lived biosensor devices
\autocite{Goldsmith2011,Yang2018,Moon2020,Cheema2021}.

\hypertarget{sec-sensor-types}{%
\subsection{Sensor Functionalisation}\label{sec-sensor-types}}

For a bioelectronic nose to operate, sufficient coupling must exist
between the bioreceptor element and the graphene or carbon nanotube
channel of the field-effect transistor. Odorant receptors can be
directly attached by physical adsorption; however, this approach is
difficult to control, and can lead to weak coupling between the odorant
receptors and the transducer \autocite{Kwon2015,Dung2018,Bohbot2020}.
Alternatively, a bifunctional linker element may mediate the attachment
between functional groups of the bioreceptor and the transducer in a
biochemical process referred to as ``functionalisation''
\autocite{Star2003a}. Functionalisation may involve covalent or
non-covalent bonding to the carbon-ring transducer surface. Covalent
bonding is stronger than non-covalent bonding, and therefore gives a
more permanent attachment between linker molecules and the transducer.
Unlike covalent attachment, non-covalent attachment preserves the
polycyclic \(sp^2\) bonding of carbon atoms in graphene and carbon
nanotubes and therefore the electrical properties of the channel
\autocite{DiCrescenzo2014,Yao2021,Shkodra2021,Li2023}. For example, one
group found covalent bonding of diazonium linker caused a \(\sim 50\) \%
drop in graphene channel mobility \autocite{Lerner2014}. In comparison,
only a \(\sim 5\) \% drop in mobility was seen for attachment of a
mixture of linkers containing pyrene to a graphene channel via
non-covalent \(\pi\) stacking \autocite{Thodkar2021}. The relative
advantages and disadvantages of each type of receptor immobilisation can
be found in Table~\ref{tbl-functionalisation-types}.

\begin{figure}

{\centering \includegraphics[width=0.75\textwidth,height=\textheight]{figures/ch3/pyrene-cnt.png}

}

\caption{\label{fig-pi-interaction-cnt}Attachment of pyrene-based linker
molecule pyrene-X and receptor Y to a carbon nanotube, representing the
transducer element of a field-effect transistor. Source: Adapted from
\autocite{Carbonnanotube}.}

\end{figure}

\(\pi\)-stacking or \(\pi-\pi\) interaction\footnote{It has been argued
  that this label is unhelpfully specific and a misrepresentation of
  what can be simply classed as a type of Van Der Waals bonding
  \autocite{Martinez2012,Perez2015}. However, as the use of the term is
  widespread in the literature, it is also used here for the sake of
  clarity.} is a specific type of non-covalent bonding which occurs due
to dispersion forces between unsaturated polycyclic molecules
\autocite{Perez2015}. A wide range of linker molecules with aromatic
moieties, such as pyrene, have been used for modification of polycyclic
carbon nanotubes and graphene via \(\pi\)-stacking
\autocite{Hermanson2013-16,Perez2015,Zhou2019,Mishyn2022}.
Figure~\ref{fig-pi-interaction-cnt} demonstrates the relationship
between a pyrene-based linker molecule to the transducer and receptor
elements. Pyrene-based \(\pi\)-stacking underlies all the
functionalisation processes used in this thesis. In this thesis,
biomolecules are attached to the linker element via covalent bonding to
the amino functional group, but there are many other nucleophilic
functional groups available for binding, including carboxyls, hydroxyls,
thiols/sulfhydryls, phenols, imidazoles and so on
\autocite{Fruh2011,Dung2018}.

\hypertarget{tbl-functionalisation-types}{}
\begin{longtable}[t]{>{\raggedright\arraybackslash}p{5.4cm}>{\raggedright\arraybackslash}p{1.45cm}>{\raggedright\arraybackslash}p{1.3cm}>{\raggedright\arraybackslash}p{1.45cm}>{\raggedright\arraybackslash}p{1.3cm}>{\raggedright\arraybackslash}p{1.3cm}}
\caption{\label{tbl-functionalisation-types}A comparison of the advantages and disadvantages of different approaches
for immobilising odorant receptors onto carbon nanotube or graphene
transducers. Simplicity \(-\) minimal cost, time and effort required for
functionalisation; Synergy \(-\) receptor attachment which does not
negatively impact transducer operation or receptor activity; Stability
\(-\) successful operation over a long time and under a range of
conditions; Specificity \(-\) receptor attachment in a controlled and
directional manner; Strength \(-\) strong receptor-transducer binding. }\tabularnewline

\toprule
Attachment Type & Simplicity & Synergy & Specificity & Stability & Strength\\
\midrule
Direct Adsorption & High & Medium & Low & Low & Low\\
Linker, covalently tethered & Medium & Low & High & High & High\\
Linker, non-covalently tethered & Medium & High & Medium & Medium & Medium\\
\bottomrule
\end{longtable}

Table~\ref{tbl-or-biosensors} summarises all published odorant-receptor
functionalised carbon nanotube and graphene field-effect
transistor-based sensors to date. The majority of published works on
this topic come from the Tai Hyun Park group at Seoul National
University. The Park group has mainly focused on CNT FETs functionalised
with human odorant receptors, but has used a range of different covalent
and non-covalent transducer immobilisation techniques when producing the
sensors. Dog and mouse odorant receptors have also been used, by the
Park group and by Goldsmith \emph{et al.} respectively. As far as the
author knows, the Plank group at Te Herenga Waka \(-\) Victoria
University of Wellington is the only group to have produced carbon
nanotube and graphene field-effect transistors functionalised with
insect odorant receptors. The behaviour of insect odorant receptors is
significantly different to that of vertebrate odorant receptors, and
their behaviour in sensor applications is currently not well understood.
The distinction between vertebrate and insect odorant receptors is
discussed in more depth in Section~\ref{sec-insect-OR-biosensors}.

Three functionalisation linkers were used by both the Park group and a
secondary research group: non-covalently attached glutaraldehyde
(GA)-conjugated 1,5-diaminonaphthalene (DAN)
\autocite{Kwon2015,Goodwin2021}, non-covalently attached
1-pyrenebutanoic acid N-hydroxysuccinimide ester (PBASE)
\autocite{Murugathas2020,Yoo2022}, and covalently attached
nickel-nitrilotriacetic acid (Ni-NTA) modified diazonium salt
\autocite{Goldsmith2011,Son2017}. The bonding between the linker
molecule and receptor element is typically covalent, regardless of the
type of bonding that exists between linker and transducer.
Interestingly, no single paper compares multiple possible
functionalisation techniques directly, making it difficult to assess the
relative quality of various attachment methods. The limit of detection
(LOD) could be used as a rough measure of quality. The functionalisation
procedure resulting in the lowest limit of detection used was
non-covalent \autocite{Park2012}. However, the quoted LOD is highly
variable across the non-covalently functionalised devices. Furthermore,
non-covalent functionalisation of odorant receptors has never been used
for vapour sensing. The next section further explores the sensing
behaviour of biosensors functionalised with the most commonly-used
protocols.

\newpage
\thispagestyle{empty}
\KOMAoptions{paper=landscape,pagesize}

\hypertarget{tbl-or-biosensors}{}
\begin{longtable}[]{@{}lllllll@{}}
\caption{\label{tbl-or-biosensors}Summary of past fabrication methods
for odorant receptor-functionalised carbon nanotube and graphene
biosensors.}\tabularnewline
\toprule\noalign{}
Attachment & Attachment Method & References & Transducer & OR Type & OR
Format & LOD \\
\midrule\noalign{}
\endfirsthead
\toprule\noalign{}
Attachment & Attachment Method & References & Transducer & OR Type & OR
Format & LOD \\
\midrule\noalign{}
\endhead
\bottomrule\noalign{}
\endlastfoot
Non-covalent & Vacuum-drying & Kim, 2009. \cite{Kim2009a} & CNTFET &
Human & Cell membrane & 100 fM \\
& DMT-MM & Yoon, 2009. \cite{Yoon2009} & CNTFET & Human & Cell membrane
& 10 fM \\
& PDL & Jin, 2012. \cite{Jin2012} & CNTFET & Human & Nanovesicles & 1
fM \\
& & Park, 2012. \cite{Park2012a} & CNTFET & Dog & Nanovesicles & 1 fM \\
& & Lim, 2014. \cite{Lim2014} & CNTFET & Human & Nanovesicles & 10 fM \\
& & Lim, 2015. \cite{Lim2015} & CNTFET & Human & Nanovesicles & 1 fM \\
& & Son, 2015. \cite{Son2015} & CNTFET & Human & Nanovesicles & 10
ng/L \\
& & Ahn, 2015. \cite{Ahn2015} & CNTFET & Human & Nanovesicles & 1 fM \\
& GA-conjugated DAN & Park, 2012. \cite{Park2012} & GFET & Human & Cell
membrane & 0.04 fM \\
& & Lee, 2012. \cite{Lee2012b} & CNTFET & Human & Cell membrane & 1
fM \\
& & Kwon, 2015. \cite{Kwon2015} & GFET & Human & Cell membrane & 0.1
fM \\
& & Goodwin, 2021. \cite{Goodwin2021} & GFET & Human & Cell membrane &
0.5 pM \\
& PBASE & Murugathas, 2019. \cite{Murugathas2019a} & CNTFET &
\textit{Insect} & Nanodiscs & 1 fM \\
& & Murugathas, 2020. \cite{Murugathas2020} & GFET & \textit{Insect} &
Nanovesicles, Nanodiscs & 1 fM \\
& & Ahn, 2020. \cite{Ahn2020} & GFET & Human & Nanovesicles & 100 fM \\
& & Yoo, 2022. \cite{Yoo2022} & CNTFET & Human & Micelles & 1 fM \\
Covalent & Diazonium salt/Ni-NTA & Goldsmith, 2011. \cite{Goldsmith2011}
& CNTFET & Mouse & Micelles, Nanodiscs & \textasciitilde7 ppb \\
& & Son, 2017. \cite{Son2017} & CNTFET & Human & Micelles & 10 fM \\
& Half-v5 mouse Ab & Lee, 2018. \cite{Lee2018} & CNTFET & Human &
Nanodiscs & 1 fM \\
\end{longtable}

\newpage
\KOMAoptions{paper=portrait,pagesize}

\hypertarget{sec-biosensor-methods}{%
\subsection{Sensing Behaviour}\label{sec-biosensor-methods}}

\begin{figure}

\begin{minipage}[t]{0.03\linewidth}

{\centering 

\raisebox{-\height}{

\includegraphics{figures/(a).png}

}

}

\end{minipage}%
%
\begin{minipage}[t]{0.01\linewidth}

{\centering 

~

}

\end{minipage}%
%
\begin{minipage}[t]{0.45\linewidth}

{\centering 

\raisebox{-\height}{

\includegraphics{figures/ch3/ion-channel-nanovesicle-lim2015.png}

}

}

\end{minipage}%
%
\begin{minipage}[t]{0.01\linewidth}

{\centering 

~

}

\end{minipage}%
%
\begin{minipage}[t]{0.03\linewidth}

{\centering 

\raisebox{-\height}{

\includegraphics{figures/(b).png}

}

}

\end{minipage}%
%
\begin{minipage}[t]{0.01\linewidth}

{\centering 

~

}

\end{minipage}%
%
\begin{minipage}[t]{0.45\linewidth}

{\centering 

\raisebox{-\height}{

\includegraphics{figures/ch3/afm-nanovesicle-lim2015.png}

}

}

\end{minipage}%
%
\begin{minipage}[t]{0.01\linewidth}

{\centering 

~

}

\end{minipage}%
\newline
\begin{minipage}[t]{0.03\linewidth}

{\centering 

\raisebox{-\height}{

\includegraphics{figures/(c).png}

}

}

\end{minipage}%
%
\begin{minipage}[t]{0.01\linewidth}

{\centering 

~

}

\end{minipage}%
%
\begin{minipage}[t]{0.45\linewidth}

{\centering 

\raisebox{-\height}{

\includegraphics{figures/ch3/nanovesicle-amyl-butyrate-1-lim2015.png}

}

}

\end{minipage}%
%
\begin{minipage}[t]{0.01\linewidth}

{\centering 

~

}

\end{minipage}%
%
\begin{minipage}[t]{0.03\linewidth}

{\centering 

\raisebox{-\height}{

\includegraphics{figures/(d).png}

}

}

\end{minipage}%
%
\begin{minipage}[t]{0.01\linewidth}

{\centering 

~

}

\end{minipage}%
%
\begin{minipage}[t]{0.45\linewidth}

{\centering 

\raisebox{-\height}{

\includegraphics{figures/ch3/lim2015-potassium1.png}

}

}

\end{minipage}%
%
\begin{minipage}[t]{0.01\linewidth}

{\centering 

~

}

\end{minipage}%
\newline
\begin{minipage}[t]{0.03\linewidth}

{\centering 

\raisebox{-\height}{

\includegraphics{figures/(e).png}

}

}

\end{minipage}%
%
\begin{minipage}[t]{0.01\linewidth}

{\centering 

~

}

\end{minipage}%
%
\begin{minipage}[t]{0.45\linewidth}

{\centering 

\raisebox{-\height}{

\includegraphics{figures/ch3/nanovesicle-amyl-butyrate-2-lim2015.png}

}

}

\end{minipage}%
%
\begin{minipage}[t]{0.01\linewidth}

{\centering 

~

}

\end{minipage}%
%
\begin{minipage}[t]{0.03\linewidth}

{\centering 

\raisebox{-\height}{

\includegraphics{figures/(f).png}

}

}

\end{minipage}%
%
\begin{minipage}[t]{0.01\linewidth}

{\centering 

~

}

\end{minipage}%
%
\begin{minipage}[t]{0.45\linewidth}

{\centering 

\raisebox{-\height}{

\includegraphics{figures/ch3/lim2015-potassium2.png}

}

}

\end{minipage}%
%
\begin{minipage}[t]{0.01\linewidth}

{\centering 

~

}

\end{minipage}%

\caption{\label{fig-lim-ion-channel}Schematics detailing the
nanovesicle-based carbon nanotube field-effect biosensor of Lim \emph{et
al.} (a) shows a schematic of the different elements and signalling
pathways present in the sensor, (b) shows an atomic force microscope
image of the functionalised device. Real-time conductance changes
resulting from amyl butyrate additions to the electrolyte gate are
compared against relevant controls in (c) and (d), while the
dose-dependent response patterns to amyl butyrate corresponding to (c)
and (d) are shown in (e) and (f) respectively. The key in (c)-(f)
indicates the concentration of ions available to flow through
nanovesicle ion channels in each experimental series. Reproduced with
permission from \autocite{Lim2015}.}

\end{figure}

Vertebrate odorant receptors have previously been coupled with
nanovesicle ion channels for selective analyte detection
\autocite{Lim2015,Dung2018}. Lim \emph{et al.} synthesised nanovesicles
featuring the human odorant receptor hOR2AG1, covalently coupled with a
potassium ion channel and placed alongside an endogenous calcium ion
channel. This configuration is shown in Figure~\ref{fig-lim-ion-channel}
(a). These nanovesicles were attached to the carbon nanotube network of
a thin-film device via charge-charge interaction with poly-D-lysine,
demonstrated with the atomic force microscope image in
Figure~\ref{fig-lim-ion-channel} (b). Binding of amyl butyrate to
hOR2AG1 causes the OR to change conformation, opening the coupled
potassium ion channel and causing ions to flow into the nanovesicle,
resulting in transistor channel gating. The real-time signal responses
associated with channel gating due to ion flow are shown in
Figure~\ref{fig-lim-ion-channel} (c)-(f). Intracellular signalling by
the odorant receptors means that target binding also activates the
calcium ion channel, and so the presence of calcium ions is sufficient
for a sensing response. In all electrolytes used to obtain a signal
response, ions are present in high concentrations relative to analyte,
ensuring ions are readily available for signalling processes. Without
either potassium or calcium ions present, ion inflow cannot occur, so no
conductance change is observed \autocite{Lim2015}.

\begin{figure}

\begin{minipage}[t]{0.03\linewidth}

{\centering 

\raisebox{-\height}{

\includegraphics{figures/(a).png}

}

}

\end{minipage}%
%
\begin{minipage}[t]{0.04\linewidth}

{\centering 

~

}

\end{minipage}%
%
\begin{minipage}[t]{0.90\linewidth}

{\centering 

\raisebox{-\height}{

\includegraphics{figures/ch3/kwon2015-OR-graphene.png}

}

}

\end{minipage}%
%
\begin{minipage}[t]{0.04\linewidth}

{\centering 

~

}

\end{minipage}%
\newline
\begin{minipage}[t]{0.03\linewidth}

{\centering 

\raisebox{-\height}{

\includegraphics{figures/(b).png}

}

}

\end{minipage}%
%
\begin{minipage}[t]{0.01\linewidth}

{\centering 

~

}

\end{minipage}%
%
\begin{minipage}[t]{0.45\linewidth}

{\centering 

\raisebox{-\height}{

\includegraphics{figures/ch3/kwon2015_transfer.png}

}

}

\end{minipage}%
%
\begin{minipage}[t]{0.01\linewidth}

{\centering 

~

}

\end{minipage}%
%
\begin{minipage}[t]{0.03\linewidth}

{\centering 

\raisebox{-\height}{

\includegraphics{figures/(c).png}

}

}

\end{minipage}%
%
\begin{minipage}[t]{0.01\linewidth}

{\centering 

~

}

\end{minipage}%
%
\begin{minipage}[t]{0.45\linewidth}

{\centering 

\raisebox{-\height}{

\includegraphics{figures/ch3/kwon2015-amyl-butyrate.png}

}

}

\end{minipage}%
%
\begin{minipage}[t]{0.01\linewidth}

{\centering 

~

}

\end{minipage}%
\newline
\begin{minipage}[t]{0.03\linewidth}

{\centering 

\raisebox{-\height}{

\includegraphics{figures/(d).png}

}

}

\end{minipage}%
%
\begin{minipage}[t]{0.01\linewidth}

{\centering 

~

}

\end{minipage}%
%
\begin{minipage}[t]{0.45\linewidth}

{\centering 

\raisebox{-\height}{

\includegraphics{figures/ch3/kwon2015-helional.png}

}

}

\end{minipage}%
%
\begin{minipage}[t]{0.01\linewidth}

{\centering 

~

}

\end{minipage}%
%
\begin{minipage}[t]{0.03\linewidth}

{\centering 

\raisebox{-\height}{

\includegraphics{figures/(e).png}

}

}

\end{minipage}%
%
\begin{minipage}[t]{0.01\linewidth}

{\centering 

~

}

\end{minipage}%
%
\begin{minipage}[t]{0.45\linewidth}

{\centering 

\raisebox{-\height}{

\includegraphics{figures/ch3/kwon2015-dose.png}

}

}

\end{minipage}%
%
\begin{minipage}[t]{0.01\linewidth}

{\centering 

~

}

\end{minipage}%

\caption{\label{fig-kwon-multiplexed}Schematics showing the odorant
receptor-functionalised graphene field-effect biosensor of Kwon \emph{et
al.} (a) shows the functionalisation of odorant receptors onto graphene
using non-covalently attached GA-modified DAN linker; (b) compares
transfer characteristics of the device with graphene only (GM), graphene
with DAN linker (GM/DAN), and after modification with one of two
different ORs (hOR2AG1, hOR3A1); (c) shows the real-time responses of
the liquid-gated hOR2AG1-modified transistor (sub-SB2) to various
concentrations of amyl butyrate (AB) analyte; (d) shows the real-time
responses of the hOR3A1-modified transistor (sub-SB3) to various
concentrations of helional (HE) analyte; and (e) shows the
dose-dependent response curve corresponding to the sub-SB2 and sub-SB3
sensors. Reproduced with permission from \autocite{Kwon2015}.}

\end{figure}

\begin{figure}

\begin{minipage}[t]{0.03\linewidth}

{\centering 

\raisebox{-\height}{

\includegraphics{figures/(a).png}

}

}

\end{minipage}%
%
\begin{minipage}[t]{0.01\linewidth}

{\centering 

~

}

\end{minipage}%
%
\begin{minipage}[t]{0.45\linewidth}

{\centering 

\raisebox{-\height}{

\includegraphics{figures/ch3/yoo2022-micelle-cnt.png}

}

}

\end{minipage}%
%
\begin{minipage}[t]{0.01\linewidth}

{\centering 

~

}

\end{minipage}%
%
\begin{minipage}[t]{0.03\linewidth}

{\centering 

\raisebox{-\height}{

\includegraphics{figures/(b).png}

}

}

\end{minipage}%
%
\begin{minipage}[t]{0.01\linewidth}

{\centering 

~

}

\end{minipage}%
%
\begin{minipage}[t]{0.45\linewidth}

{\centering 

\raisebox{-\height}{

\includegraphics{figures/ch3/yoo2022-AFM.png}

}

}

\end{minipage}%
%
\begin{minipage}[t]{0.01\linewidth}

{\centering 

~

}

\end{minipage}%
\newline
\begin{minipage}[t]{0.03\linewidth}

{\centering 

\raisebox{-\height}{

\includegraphics{figures/(c).png}

}

}

\end{minipage}%
%
\begin{minipage}[t]{0.01\linewidth}

{\centering 

~

}

\end{minipage}%
%
\begin{minipage}[t]{0.45\linewidth}

{\centering 

\raisebox{-\height}{

\includegraphics{figures/ch3/yoo2022-TX.png}

}

}

\end{minipage}%
%
\begin{minipage}[t]{0.01\linewidth}

{\centering 

~

}

\end{minipage}%
%
\begin{minipage}[t]{0.03\linewidth}

{\centering 

\raisebox{-\height}{

\includegraphics{figures/(d).png}

}

}

\end{minipage}%
%
\begin{minipage}[t]{0.01\linewidth}

{\centering 

~

}

\end{minipage}%
%
\begin{minipage}[t]{0.45\linewidth}

{\centering 

\raisebox{-\height}{

\includegraphics{figures/ch3/yoo2022-DMMP.png}

}

}

\end{minipage}%
%
\begin{minipage}[t]{0.01\linewidth}

{\centering 

~

}

\end{minipage}%
\newline
\begin{minipage}[t]{0.03\linewidth}

{\centering 

\raisebox{-\height}{

\includegraphics{figures/(e).png}

}

}

\end{minipage}%
%
\begin{minipage}[t]{0.01\linewidth}

{\centering 

~

}

\end{minipage}%
%
\begin{minipage}[t]{0.45\linewidth}

{\centering 

\raisebox{-\height}{

\includegraphics{figures/ch3/yoo2022-DMMP-2.png}

}

}

\end{minipage}%
%
\begin{minipage}[t]{0.01\linewidth}

{\centering 

~

}

\end{minipage}%
%
\begin{minipage}[t]{0.03\linewidth}

{\centering 

\raisebox{-\height}{

\includegraphics{figures/(f).png}

}

}

\end{minipage}%
%
\begin{minipage}[t]{0.01\linewidth}

{\centering 

~

}

\end{minipage}%
%
\begin{minipage}[t]{0.45\linewidth}

{\centering 

\raisebox{-\height}{

\includegraphics{figures/ch3/yoo2022-DMMP-3.png}

}

}

\end{minipage}%
%
\begin{minipage}[t]{0.01\linewidth}

{\centering 

~

}

\end{minipage}%

\caption{\label{fig-yoo-micelle}Schematics of the micelle-based carbon
nanotube field-effect transistor of Yoo \emph{et al.} (a) shows the
functionalisation of detergent micelles onto the carbon nanotube network
channel; (b) shows the same height profile across an atomic force
microscope image of the carbon nanotube network before and after
functionalisation with micelles using PBASE; (c) shows the liquid-gated
transfer characteristic of a device channel before and after
functionalisation with PBASE and micelles; (d) shows real-time responses
of the functionalised, liquid-gated channel to additions of dimethyl
methylphosphonate (DMMP) analyte; (e) shows the dose-dependent response
curve to DMMP; and (f) shows a control series demonstrating the
selective behaviour of the sensor (EB is ethyl butyrate, AB is amyl
butyrate). Reproduced with permission from \autocite{Yoo2022}.}

\end{figure}

Odorant receptors can also be expressed in the native cell membrane and
attached directly to the biosensor channel. Here, the changes in odorant
receptor conformation that result from analyte binding cause affects the
distance between charges on the odorant receptor and the transducer
channel, gating the channel \autocite{Kwon2015,Dung2018}. Kwon \emph{et
al.} functionalised graphene field-effect transistors with human odorant
receptors hOR2AG1 and hOR3A1 using non-covalently attached
1,5-diaminonaphthalene (DAN) modified with glutaraldehyde (GA) as a
linker, as shown in Figure~\ref{fig-kwon-multiplexed} (a). The odorant
receptors attach to the GA-modified DAN via a Schiff-base reaction
\autocite{Subasi2022}. OR attachment was demonstrated by SEM imaging as
well as a significant change in device resistance, shown in
Figure~\ref{fig-kwon-multiplexed} (b). Both hOR2AG1 and hOR3A1 showed
real-time responses to their corresponding target analyte at
sub-femtomolar concentrations, as shown in
Figure~\ref{fig-kwon-multiplexed} (c) and
Figure~\ref{fig-kwon-multiplexed} (d) respectively. No responses were
seen from linker-modified graphene to the same analyte additions. The
dose-dependent response curve of both these odorant receptor sensors is
shown in Figure~\ref{fig-kwon-multiplexed} (e). As in
Figure~\ref{fig-lim-ion-channel} (d), the response behaviour follows a
curve which can be described using a Langmuir surface adsorption
isotherm, with saturation behaviour observed with picomolar additions.

Biosensors have also been produced where odorant receptors are held in a
detergent micelle format instead of the native cell membrane. The
mechanism behind sensing is the same as for odorant receptors in the
cell membrane, where a conformational change in the odorant receptors
leads to channel gating \autocite{Dung2018,Yoo2022}. Yoo \emph{et al.}
functionalised random-network CNT FETs with detergent micelles which
contained human odorant receptor hOR2T7. PBASE was used as the linker
molecule, which attaches to the odorant receptor via its amine group and
non-covalently tethers it to the transducer, illustrated in
Figure~\ref{fig-yoo-micelle} (a). Successful immobilisation was
demonstrated by a raised atomic force microscope height profile after
receptor attachment (Figure~\ref{fig-yoo-micelle} (b)) and an on-current
drop in the liquid-gated transfer characteristics of the device
(Figure~\ref{fig-yoo-micelle} (c)). The sensor showed sharp real-time
responses to the addition of DMMP concentrations, as seen in
Figure~\ref{fig-yoo-micelle} (d). The dose dependence curve for DMMP
responses is shown in in Figure~\ref{fig-yoo-micelle} (e), again showing
a Langmuir-type response curve to successive DMMP additions. Various
analytes with a similar scent to DMMP were added at high concentrations
to the liquid-gate, shown in Figure~\ref{fig-yoo-micelle} (f). No
response was seen to any these additions, demonstrating the selectivity
of the sensor.

\begin{figure}

\begin{minipage}[t]{0.03\linewidth}

{\centering 

\raisebox{-\height}{

\includegraphics{figures/(a).png}

}

}

\end{minipage}%
%
\begin{minipage}[t]{0.01\linewidth}

{\centering 

~

}

\end{minipage}%
%
\begin{minipage}[t]{0.40\linewidth}

{\centering 

\raisebox{-\height}{

\includegraphics{figures/ch3/afm-nanodisc-goldsmith.png}

}

}

\end{minipage}%
%
\begin{minipage}[t]{0.01\linewidth}

{\centering 

~

}

\end{minipage}%
%
\begin{minipage}[t]{0.03\linewidth}

{\centering 

\raisebox{-\height}{

\includegraphics{figures/(b).png}

}

}

\end{minipage}%
%
\begin{minipage}[t]{0.01\linewidth}

{\centering 

~

}

\end{minipage}%
%
\begin{minipage}[t]{0.50\linewidth}

{\centering 

\raisebox{-\height}{

\includegraphics{figures/ch3/eugenol-goldsmith.png}

}

}

\end{minipage}%
%
\begin{minipage}[t]{0.01\linewidth}

{\centering 

~

}

\end{minipage}%

\caption{\label{fig-eugenol-responses}The functionalisation of mOR174-9
nanodiscs onto single-CNT field effect transistor vapour sensing use is
demonstrated with an atomic force microscope image in (a), while (b)
shows real-time responses of the sensor to 2 ppm eugenol vapour. The
response to eugenol on day 69 (red triangles) indicates that the device
retains the ability to respond to eugenol 10 weeks after
functionalisation. Reproduced with permission from
\autocite{Goldsmith2011}.}

\end{figure}

In the first study of this kind, Goldsmith \emph{et al.} demonstrated
that a single-CNT device functionalised with mOR174-9 odorant receptors
in either a micelle or nanodisc format could be used as a vapour-phase
biosensor. Micelle immobilisation was confirmed using atomic force
microscopy, as shown in Figure~\ref{fig-eugenol-responses} (a). The mOR
CNT FETs were exposed to nitrogen flow at 50\% relative humidity. The
conductance across the channel was measured while a specific
concentration of the positive ligand eugenol was added to the constant
flow for 100 s, then removed from the flow for 100 s. This cycle was
repeated five times. Figure~\ref{fig-eugenol-responses} (b) shows that a
\(\sim\) 9\% increase in current was observed during each cycle of
exposure to eugenol. The device still responded to eugenol cycles after
69 days of storage in 25\% (v/v) ethanol at 4°C. This persistent
activity may result from the long-lived nanodisc format used
\autocite{Goldsmith2011}. As far as the author knows, there is no
existing study which investigates whether this behaviour can be
replicated for insect odorant receptor devices. It is not clear that the
vertebrate odorant receptors used here can simply be substituted for
iORs for vapour-phase sensing. The difference between receptors is
discussed further in the subsequent section.

\hypertarget{sec-insect-OR-biosensors}{%
\section{Insect Odorant Receptor Field-Effect Transistor
Biosensors}\label{sec-insect-OR-biosensors}}

\hypertarget{insect-odorant-receptors}{%
\subsection{Insect Odorant Receptors}\label{insect-odorant-receptors}}

\begin{figure}

{\centering \includegraphics[width=0.65\textwidth,height=\textheight]{figures/ch3/OR_diagram.png}

}

\caption{\label{fig-iOR-membrane}The tuning OR and odorant receptor
coreceptor (ORCO) on the native cell membrane, with C-terminus and
N-terminus indicated. The red arrow indicates the location of ion
transport across the membrane. Adapted from
\autocite{Brito2016,Wicher2021}.}

\end{figure}

Insect odorant (or olfactory) receptors (iORs) are a diverse range of
odorant-sensitive seven-transmembrane proteins located in the dendrite
cells of insect sensory hairs, known as sensilla
\autocite{Clyne1999,Carraher2015,Brito2016,Wicher2021}. When volatile
compounds enter the sensilla, they are carried by odorant binding
proteins (OBPs) through an aqueous environment to the dendrite cells
\autocite{Carraher2015,Brito2016,Wicher2021}. These cells possess a
insect-specific set of `tuning' iORs alongside a generic co-receptor
known as `ORCO' (Odorant Receptor Co-Receptor)
\autocite{Carraher2015,Butterwick2018,Khadka2019,Wicher2021}. The ORCO
co-receptor is insensitive to target compounds (aside from synthetic
compounds like VUAA1). Instead, it couples with the tuning iOR to form a
non-selective, permeable ion channel
\autocite{Butterwick2018,Wicher2021}. When a compound binds to a tuning
iOR, the ion channel opens to allow cations to travel across the cell
membrane, activating intracellular signalling
\autocite{Smart2008,Wicher2008,Sato2008,Carraher2015,Brito2016,Butterwick2018,Wicher2021}.
The combination of resulting OR signals is sent to the insect brain for
interpretation as an odor \autocite{Hallem2004,Carraher2015,Wicher2021}.
The tuning iORs respond to (or are inhibited by) a huge variety of odors
\autocite{Munch2016}. The non-trivial binding behaviour of iORs with
analyte means artificial neural network processing may be required for
accurate readouts from a self-contained, lab-on-a-chip iOR biosensor
\autocite{Bachtiar2016}.

Vertebrate odorant receptor proteins are terminated with an amine group
outside the cell membrane, known as the N-terminus, and terminated with
a carboxyl group inside the cell membrane, known as the C-terminus.
Initially, iORs were thought to be similar in structure to vertebrate
GPCRs \autocite{Clyne1999}, but is now known that iORs have a completely
different topology and mechanism, despite also being a
seven-transmembrane protein. The terminus configuration is inverted: the
C-terminus of the iOR is extracellular, and the N-terminus is
intracellular
\autocite{Smart2008,Glatz2011,Carraher2015,Brito2016,Wicher2021}.
Furthermore, there is no sequence similarity between iORs and GPCRs.
Evolutionarily, insect odorant receptors are thought to be closely
related to insect gustatory receptors (GRs), while they bear no relation
to GPCRs \autocite{Glatz2011,Carraher2015,Wicher2021}. However, despite
iORs not being GPCRs, some interaction between the iOR complex and the
G-protein of the olfactory cell plays a role in odor detection \emph{in
vivo} \autocite{Wicher2008,Wicher2021}. The \emph{in vivo} configuration
of the odorant receptor on the cell membrane, showing the terminus
configuration and location of ORCO ion channel, is illustrated in
Figure~\ref{fig-iOR-membrane}. By testing sensors which incorporate
insect odorant receptors, new information may emerge which helps us to
better understand their atypical structure.

\hypertarget{sensor-functionalisation}{%
\subsection{Sensor Functionalisation}\label{sensor-functionalisation}}

\begin{figure}

\begin{minipage}[t]{0.03\linewidth}

{\centering 

\raisebox{-\height}{

\includegraphics{figures/(a).png}

}

}

\end{minipage}%
%
\begin{minipage}[t]{0.01\linewidth}

{\centering 

~

}

\end{minipage}%
%
\begin{minipage}[t]{0.45\linewidth}

{\centering 

\raisebox{-\height}{

\includegraphics{figures/ch3/pristine_graphene_murugathas.png}

}

}

\end{minipage}%
%
\begin{minipage}[t]{0.01\linewidth}

{\centering 

~

}

\end{minipage}%
%
\begin{minipage}[t]{0.03\linewidth}

{\centering 

\raisebox{-\height}{

\includegraphics{figures/(b).png}

}

}

\end{minipage}%
%
\begin{minipage}[t]{0.01\linewidth}

{\centering 

~

}

\end{minipage}%
%
\begin{minipage}[t]{0.45\linewidth}

{\centering 

\raisebox{-\height}{

\includegraphics{figures/ch3/OR22a_graphene_murugathas.png}

}

}

\end{minipage}%
%
\begin{minipage}[t]{0.01\linewidth}

{\centering 

~

}

\end{minipage}%
\newline
\begin{minipage}[t]{0.03\linewidth}

{\centering 

\raisebox{-\height}{

\includegraphics{figures/(c).png}

}

}

\end{minipage}%
%
\begin{minipage}[t]{0.01\linewidth}

{\centering 

~

}

\end{minipage}%
%
\begin{minipage}[t]{0.45\linewidth}

{\centering 

\raisebox{-\height}{

\includegraphics{figures/ch3/pristine_CNT_murugathas.png}

}

}

\end{minipage}%
%
\begin{minipage}[t]{0.01\linewidth}

{\centering 

~

}

\end{minipage}%
%
\begin{minipage}[t]{0.03\linewidth}

{\centering 

\raisebox{-\height}{

\includegraphics{figures/(d).png}

}

}

\end{minipage}%
%
\begin{minipage}[t]{0.01\linewidth}

{\centering 

~

}

\end{minipage}%
%
\begin{minipage}[t]{0.45\linewidth}

{\centering 

\raisebox{-\height}{

\includegraphics{figures/ch3/OR22a_CNT_murugathas.png}

}

}

\end{minipage}%
%
\begin{minipage}[t]{0.01\linewidth}

{\centering 

~

}

\end{minipage}%

\caption{\label{fig-functionalisation-AFM-literature}Atomic force
microscope images of (a) a pristine graphene monolayer, (b) a OR22a
nanodisc-functionalised graphene monolayer, (c) a pristine carbon
nanotube network, and (d) an OR22a nanodisc-functionalised carbon
nanotube network. Reproduced with permission from
\autocite{Murugathas2019a,Murugathas2020}.}

\end{figure}

Murugathas \emph{et al.} attached a variety of insect odorant receptors
to carbon nanotubes and graphene field-effect transistors using a
nanodisc format. Atomic force microscope images of a graphene monolayer
before and after immobilisation of OR22a nanodiscs with PBASE linker are
shown in Figure~\ref{fig-functionalisation-AFM-literature} (a) and (b)
respectively, while atomic force microscope images of a randomly
deposited carbon nanotube network before and after OR22a nanodisc
immobilisation with PBASE are shown in
Figure~\ref{fig-functionalisation-AFM-literature} (c) and (d)
respectively. Features are seen across the surface of the
post-functionalisation image which are tens of nanometers in height. On
the nanotube network, these features are seen directly next to nanotube
bundles, indicating selective attachment to the nanotubes over the
SiO\(_2\) substrate. As nanodiscs are only 10-20 nm in height, it
appears that these features are large agglomerates of nanodiscs
\autocite{Nath2007,Bayburt2010,Murugathas2019a,Murugathas2020}. As seen
previously for a carbon nanotube network FET in
Section~\ref{sec-odorant-receptors-biosensors}, functionalisation occurs
by non-covalent attachment of PBASE to the channel, and covalent
attachment of the PBASE linker to the odorant receptor amine group. The
nanodisc membrane also possesses amine residues \autocite{Bayburt2010},
so in some cases immobilisation may be directly between the PBASE linker
and nanodisc membrane.

Functionalisation of a FET device channel with iORs significantly alters
the transfer characteristics of that channel. Murugathas \emph{et al.}
found that successful functionalisation of a CNT FET device with iORs
would typically increase the device on-current, increase its on-off
ratio and cause a significant negative shift in threshold voltage, as
shown in Figure~\ref{fig-functionalisation-literature} (a)
\autocite{Murugathas2019a}. Meanwhile, successful functionalisation of a
graphene device with iORs would typically dramatically decrease the
device on-current and cause a negative shift in Dirac voltage, as seen
in Figure~\ref{fig-functionalisation-literature} (b)
\autocite{Murugathas2020}. These changes are not simply the result of
linker attachment to the channel surface \autocite{Murugathas2019a}. It
is thought that the negative shift of both threshold and Dirac voltages
are caused by the N-terminus amine groups on the odorant receptors or
amine groups on the nanodisc membrane scaffold proteins donating
electrons to the device channel, which has a similar effect to doping
the channel with impurities
\autocite{Bradley2004,Murugathas2019a,Murugathas2020}. Note that very
similar changes occur when functionalising with empty nanodiscs which
contain no odorant receptors, shown in
Figure~\ref{fig-functionalisation-literature} (c) and
Figure~\ref{fig-functionalisation-literature} (d). Unless the odorant
receptors attach preferentially to the network over nanodiscs, it
appears the gating effect is predominantly due to the large-scale
attachment of nanodisc membranes.

\begin{figure}

\begin{minipage}[t]{0.03\linewidth}

{\centering 

\raisebox{-\height}{

\includegraphics{figures/(a).png}

}

}

\end{minipage}%
%
\begin{minipage}[t]{0.01\linewidth}

{\centering 

~

}

\end{minipage}%
%
\begin{minipage}[t]{0.45\linewidth}

{\centering 

\raisebox{-\height}{

\includegraphics{figures/ch3/OR22a_ND_GFET.png}

}

}

\end{minipage}%
%
\begin{minipage}[t]{0.01\linewidth}

{\centering 

~

}

\end{minipage}%
%
\begin{minipage}[t]{0.03\linewidth}

{\centering 

\raisebox{-\height}{

\includegraphics{figures/(b).png}

}

}

\end{minipage}%
%
\begin{minipage}[t]{0.01\linewidth}

{\centering 

~

}

\end{minipage}%
%
\begin{minipage}[t]{0.45\linewidth}

{\centering 

\raisebox{-\height}{

\includegraphics{figures/ch3/OR22a_ND_CNTFET.png}

}

}

\end{minipage}%
%
\begin{minipage}[t]{0.01\linewidth}

{\centering 

~

}

\end{minipage}%
\newline
\begin{minipage}[t]{0.03\linewidth}

{\centering 

\raisebox{-\height}{

\includegraphics{figures/(c).png}

}

}

\end{minipage}%
%
\begin{minipage}[t]{0.01\linewidth}

{\centering 

~

}

\end{minipage}%
%
\begin{minipage}[t]{0.45\linewidth}

{\centering 

\raisebox{-\height}{

\includegraphics{figures/ch3/Empty_ND_GFET.png}

}

}

\end{minipage}%
%
\begin{minipage}[t]{0.01\linewidth}

{\centering 

~

}

\end{minipage}%
%
\begin{minipage}[t]{0.03\linewidth}

{\centering 

\raisebox{-\height}{

\includegraphics{figures/(d).png}

}

}

\end{minipage}%
%
\begin{minipage}[t]{0.01\linewidth}

{\centering 

~

}

\end{minipage}%
%
\begin{minipage}[t]{0.45\linewidth}

{\centering 

\raisebox{-\height}{

\includegraphics{figures/ch3/Empty_ND_CNTFET.png}

}

}

\end{minipage}%
%
\begin{minipage}[t]{0.01\linewidth}

{\centering 

~

}

\end{minipage}%

\caption{\label{fig-functionalisation-literature}Transfer characteristic
curves before and after functionalisation of (a) an OR22a
nanodisc-functionalised graphene FET, (b) an OR22a
nanodisc-functionalised CNT network FET, (c) an empty
nanodisc-functionalised graphene FET and (d) an empty
nanodisc-functionalised CNT network FET. Reproduced with permission from
\autocite{Murugathas2019a,Murugathas2020}.}

\end{figure}

\hypertarget{sensing-behaviour}{%
\subsection{Sensing Behaviour}\label{sensing-behaviour}}

\begin{figure}

\begin{minipage}[t]{0.03\linewidth}

{\centering 

\raisebox{-\height}{

\includegraphics{figures/(a).png}

}

}

\end{minipage}%
%
\begin{minipage}[t]{0.01\linewidth}

{\centering 

~

}

\end{minipage}%
%
\begin{minipage}[t]{0.45\linewidth}

{\centering 

\raisebox{-\height}{

\includegraphics{figures/ch3/OR22a_realtime_normalised_CNTFET_1.png}

}

}

\end{minipage}%
%
\begin{minipage}[t]{0.01\linewidth}

{\centering 

~

}

\end{minipage}%
%
\begin{minipage}[t]{0.03\linewidth}

{\centering 

\raisebox{-\height}{

\includegraphics{figures/(b).png}

}

}

\end{minipage}%
%
\begin{minipage}[t]{0.01\linewidth}

{\centering 

~

}

\end{minipage}%
%
\begin{minipage}[t]{0.45\linewidth}

{\centering 

\raisebox{-\height}{

\includegraphics{figures/ch3/OR22a_realtime_normalised_GFET_1.png}

}

}

\end{minipage}%
%
\begin{minipage}[t]{0.01\linewidth}

{\centering 

~

}

\end{minipage}%
\newline
\begin{minipage}[t]{0.03\linewidth}

{\centering 

\raisebox{-\height}{

\includegraphics{figures/(c).png}

}

}

\end{minipage}%
%
\begin{minipage}[t]{0.01\linewidth}

{\centering 

~

}

\end{minipage}%
%
\begin{minipage}[t]{0.45\linewidth}

{\centering 

\raisebox{-\height}{

\includegraphics{figures/ch3/OR22a_realtime_normalised_CNTFET_2.png}

}

}

\end{minipage}%
%
\begin{minipage}[t]{0.01\linewidth}

{\centering 

~

}

\end{minipage}%
%
\begin{minipage}[t]{0.03\linewidth}

{\centering 

\raisebox{-\height}{

\includegraphics{figures/(d).png}

}

}

\end{minipage}%
%
\begin{minipage}[t]{0.01\linewidth}

{\centering 

~

}

\end{minipage}%
%
\begin{minipage}[t]{0.45\linewidth}

{\centering 

\raisebox{-\height}{

\includegraphics{figures/ch3/OR22a_realtime_normalised_GFET_2.png}

}

}

\end{minipage}%
%
\begin{minipage}[t]{0.01\linewidth}

{\centering 

~

}

\end{minipage}%

\caption{\label{fig-iOR-sensing-literature}Real-time responses to
concentrations of methyl hexanoate in \(1 \times\) phosphate buffer
saline (PBS) with 1\% v/v DMSO by (a) an OR22a nanodisc-functionalised
CNT network FET and (b) an OR22a nanodisc-functionalised graphene FET,
alongside the normalised signal response curves corresponding to (c) CNT
network FETs and (d) graphene FETs. The response curves show the
cumulative responses of OR22a-functionalised devices to both the
positive ligand methyl hexanoate (green) and negative ligand
\emph{trans}-2-hexan-1-al (black). They also show the cumulative
response of a empty nanodisc functionalised device to methyl hexanoate
(purple). Reproduced with permission from
\autocite{Murugathas2019a,Murugathas2020}.}

\end{figure}

Figure~\ref{fig-iOR-sensing-literature} (a) and (b) show the respective
responses of the OR22a-functionalised CNT FET and graphene FET to
various concentrations of methyl hexanoate in real-time. This result
demonstrates that iOR-FETs are sensitive down to the femtomolar scale in
an aqueous environment. Figure~\ref{fig-iOR-sensing-literature} (c) and
(d) compares the dose dependent responses to methyl hexanoate from
multiple OR22a-functionalised devices to that of relevant controls. It
was verified that the OR22a-functionalised devices would not respond to
\emph{trans}-2-hexan-1-al, the negative ligand for OR22a; it was also
verified that empty nanodiscs would not respond non-selectively to the
positive ligand \autocite{Murugathas2019a,Murugathas2020}. It is notable
that unlike iORs \emph{in vivo}, ORCO does not appear to be required for
the bioelectronic nose to function
\autocite{Murugathas2019a,Murugathas2020,Khadka2019,Cheema2021}.
Furthermore, G-protein signaling pathways are not required
\autocite{Sato2014}. It has been proposed that the signal response
results from the positive ligand binding to the iOR protein, causing a
change in conformation, the same mechanism underpinning the behaviour of
many of the vertebrate odorant receptor sensors seen in
Section~\ref{sec-odorant-receptors-biosensors}. Cheema \emph{et al.}
used neutron reflectometry to demonstrate that OR22a nanodiscs undergo a
1 nm height change after ethyl hexanoate exposure, likely resulting from
a structural change \autocite{Cheema2021}.

This change most likely affects the channel in one of two ways. The
first involves transfer of charge from the iOR to the channel, reducing
I\(_{d}\) and causing a negative threshold voltage (or Dirac point)
shift. Another could be a more indirect electrostatic gating effect from
movement of charge within the Debye screening length of the channel. The
Debye length of \(1 \times\) PBS buffer is typically much shorter than
the height of a single nanodisc \autocite{Murugathas2019a}. However, if
structural changes in the iOR were primarily occurring at its base, it
is still possible that the electrostatic gating could be the primary
sensing mechanism. From further development and examination of iOR-based
biosensors, new insights into the mechanisms underlying the nanodisc
signal transduction may emerge \autocite{Glatz2011}. As discussed here,
the literature has primarily focused on the operation of iOR carbon
nanotube and graphene FET biosensors in an aqueous environment
\autocite{Murugathas2019a,Murugathas2020}. It is as yet unknown whether
insect odorant receptors can operate in a vapour-phase environment, but
this possibility is explored in this thesis.

\hypertarget{sec-non-specific-binding}{%
\section{Non-Specific Binding}\label{sec-non-specific-binding}}

Non-specific binding (NSB) refers to any attachment within the sensing
environment not related to the specific analyte of interest
\autocite{Lichtenberg2019,Shkodra2021}. Non-specific binding is
particularly significant for protein-functionalised devices. Proteins
may be spontaneously adsorbed onto carbon nanotube or graphene surfaces
during functionalisation in a manner which is not linker-mediated
\autocite{Bradley2004,Star2003a,Chen2004}. Non-covalently bound proteins
may detach and reattach to available surfaces in a non-specific manner
when exposed to a high ionic strength electrolyte post-functionalisation
\autocite{Dung2018}. Non-specific binding may also result from
protein-protein interactions, misoriented attachment of proteins,
attachment to a sticky substrate \autocite{Chen2004,Lichtenberg2019}. It
can also result from electrostatic binding to any charged surface
present, such as the gold electrodes \autocite{Garcia-Aljaro2010} or
Ag/AgCl reference electrode
\autocite{Chen2004,Minot2007,Lichtenberg2019}. Liquid-gated graphene and
carbon nanotube devices are highly sensitive to the approach of charge
within the Debye length of the device channel, and so non-specific
adsorption can lead to spurious signals when sensing
\autocite{Star2003a,Chen2004,Lichtenberg2019,Shkodra2021}. A variety of
measures can be taken to prevent NSB from occurring. Once bioreceptors
have been attached to the channel, remaining exposed carbon nanotubes
can be passivated with chemical coatings such as Tween-20
\autocite{Chen2004}, PEG
\autocite{Star2003a,Lee2012b,Gao2016,Filipiak2018}, and ethanolamine
\autocite{Maehashi2007,Das2011}.

\hypertarget{summary}{%
\section{Summary}\label{summary}}

Odorant receptors can be used to fabricate highly sensitive and
selective biosensors using carbon nanotube and graphene field-effect
transistors as the transducer element. Both vertebrate and insect
odorant receptors are seven-transmembrane proteins, but each has a
different sequence and have inverted terminus positions relative to the
cell membrane wall \emph{in vivo}. Insect odorant receptor detection
\emph{in vivo} differs significantly from vertebrate OR detection, with
an ORCO-mediated ion channel involved. ORs can be held in the native
cell membrane for sensor applications, but artificial lipid bilayer
formats such as micelles, nanovesicles or nanodiscs are generally
preferred due to their enhanced stability. Mammalian odorant receptors
have been thoroughly explored in carbon nanotube and graphene
field-effect transistor sensing applications, with both non-covalent and
covalent functionalisation mechanisms used to create sensors which
detect analyte at sub-femtomolar concentrations. The mechanisms behind
sensing rely on transistor gating either due to ion flow into a
nanovesicle format, or a conformational change in the odorant receptor.
Covalently-attached mammalian odorant receptors have also been used for
vapour-phase detection with a single carbon nanotube field-effect
transistor device at concentrations down to \(\sim\) 7 ppb by Goldsmith
\emph{et al.}

Femtomolar detection of analyte has also been achieved with an insect
odorant receptor functionalised device. However, the exact mechanism
behind detection is unclear, as the presence of ORCO is not required for
successful sensor behaviour. It is possible that the mechanism results
from a change in conformation of the odorant receptor, similar to the
mammalian odorant receptor. Due to the possible difference in mechanism
between mammalian odorant receptor detection and insect odorant receptor
detection, it is not clear that vapour-phase detection can be achieved
by simply reproducing the work of Goldsmith \emph{et al.} using insect
odorant receptors. \textbf{?@sec-noncovalent-functionalisation} looks
further at various non-covalent functionalisation approaches for the
creation of a insect odorant receptor-based field-effect transistor
biosensor, while \textbf{?@sec-biosensing-iORs} tests sensor behaviour
in both aqueous and vapour-phase environments.

\cleardoublepage
\phantomsection
\addcontentsline{toc}{part}{Appendices}
\appendix

\hypertarget{vapour-system-hardware}{%
\chapter{Vapour System Hardware}\label{vapour-system-hardware}}

\hypertarget{tbl-vapour-sensor-components}{}
\begin{longtable}[]{@{}
  >{\raggedright\arraybackslash}p{(\columnwidth - 4\tabcolsep) * \real{0.5930}}
  >{\raggedright\arraybackslash}p{(\columnwidth - 4\tabcolsep) * \real{0.2209}}
  >{\raggedright\arraybackslash}p{(\columnwidth - 4\tabcolsep) * \real{0.1860}}@{}}
\caption{\label{tbl-vapour-sensor-components}Major components used in
construction of the vapour delivery system described in this
thesis.}\tabularnewline
\toprule\noalign{}
\begin{minipage}[b]{\linewidth}\raggedright
Description
\end{minipage} & \begin{minipage}[b]{\linewidth}\raggedright
Part No.
\end{minipage} & \begin{minipage}[b]{\linewidth}\raggedright
Manufacturer
\end{minipage} \\
\midrule\noalign{}
\endfirsthead
\toprule\noalign{}
\begin{minipage}[b]{\linewidth}\raggedright
Description
\end{minipage} & \begin{minipage}[b]{\linewidth}\raggedright
Part No.
\end{minipage} & \begin{minipage}[b]{\linewidth}\raggedright
Manufacturer
\end{minipage} \\
\midrule\noalign{}
\endhead
\bottomrule\noalign{}
\endlastfoot
Mass flow controller, 20 sccm full scale & GE50A-013201SBV020 & MKS
Instruments \\
Mass flow controller, 200 sccm full scale & GE50A-013202SBV020 & MKS
Instruments \\
Mass flow controller, 500 sccm full scale & FC-2901V & Tylan \\
Analogue flowmeter, 240 sccm max. flow & 116261-30 & Dwyer \\
Micro diaphragm pump & P200-B3C5V-35000 & Xavitech \\
Analogue flow controller, for micro diaphragm pump & X3000450 &
Xavitech \\
10 mL Schott bottle & 218010802 & Duran \\
PTFE connection cap system & Z742273 & Duran \\
Baseline VOC-TRAQ flow cell, purple & 043-950 & Ametek Mocon \\
Baseline VOC-TRAQ flow cell, red & 043-951 & Ametek Mocon \\
Humidity and temperature sensor & T9602-5-A & Telaire \\
\end{longtable}

\hypertarget{python-code-for-data-analysis}{%
\chapter{Python Code for Data
Analysis}\label{python-code-for-data-analysis}}

\hypertarget{code-repository}{%
\section{Code Repository}\label{code-repository}}

The code used for general analysis of field-effect transistor devices in
this thesis was written with Python 3.8.8. Contributors to the code used
include Erica Cassie, Erica Happe, Marissa Dierkes and Leo Browning. The
code is located on GitHub and the research group OneDrive, and is
available on request.

\hypertarget{sec-histogram-analysis}{%
\section{Atomic Force Microscope Histogram
Analysis}\label{sec-histogram-analysis}}

The purpose of this code is to analyse atomic force microscope (AFM)
images of carbon nanotube networks in .xyz format taken using an atomic
force microscope and processed in Gwyddion (see
\textbf{?@sec-afm-characterisation}). It was originally designed by
Erica Happe in Matlab, and adapted by Marissa Dierkes and myself for use
in Python. The code imports the .xyz data and sorts it into bins 0.15 nm
in size for processing. To perform skew-normal distribution fits, both
\emph{scipy.optimize.curve\_fit} and \emph{scipy.stats.skewnorm} modules
are used in this code.

\hypertarget{sec-raman-analysis}{%
\section{Raman Spectroscopy Analysis}\label{sec-raman-analysis}}

The purpose of this code is to analyse a series of Raman spectra taken
at different points on a single film (see
\textbf{?@sec-raman-characterisation}). Data is imported in a series of
tab-delimited text files, with the low wavenumber spectrum (100
cm\(^{-1} - 650\) cm\(^{-1}\)) and high wavenumber spectrum (1300
cm\(^{-1} - 1650\) cm\(^{-1}\)) imported in separate datafiles for each
scan location.

\hypertarget{sec-field-effect-transistor-analysis}{%
\section{Field-Effect Transistor
Analysis}\label{sec-field-effect-transistor-analysis}}

The purpose of this code is to analyse electrical measurements taken of
field-effect transistor (FET) devices. Electrical measurements were
either taken from the Keysight 4156C Semiconductor Parameter Analyser,
National Instruments NI-PXIe or Keysight B1500A Semiconductor Device
Analyser as discussed in \textbf{?@sec-electrical-characterisation}; the
code is able to analyse data in .csv format taken from all three
measurement setups. The main Python file in the code base consists of
three related but independent modules: the first analyses and plots
sensing data from the FET devices, the second analyses and plots
transfer characteristics from channels across a device, and the third
compares individual channel characteristics before and after a
modification or after each individual modification in a series of
modifications. The code base also features a separate config file and
style sheet which govern the behaviour of the main code. The code base
was designed collaboratively by myself and Erica Cassie over GitHub
using the Sourcetree Git GUI.

\hypertarget{references}{%
\chapter*{References}\label{references}}
\addcontentsline{toc}{chapter}{References}

\markboth{References}{References}

\printbibliography[heading=none]


\backmatter

\end{document}
