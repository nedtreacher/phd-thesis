% Options for packages loaded elsewhere
\PassOptionsToPackage{unicode}{hyperref}
\PassOptionsToPackage{hyphens}{url}
%
\documentclass[
  a4paper,
]{scrbook}

\usepackage{amsmath,amssymb}
\usepackage{iftex}
\ifPDFTeX
  \usepackage[T1]{fontenc}
  \usepackage[utf8]{inputenc}
  \usepackage{textcomp} % provide euro and other symbols
\else % if luatex or xetex
  \usepackage{unicode-math}
  \defaultfontfeatures{Scale=MatchLowercase}
  \defaultfontfeatures[\rmfamily]{Ligatures=TeX,Scale=1}
\fi
\usepackage{lmodern}
\ifPDFTeX\else  
    % xetex/luatex font selection
  \setmainfont[]{Latin Modern Roman}
  \setsansfont[]{Latin Modern Roman}
\fi
% Use upquote if available, for straight quotes in verbatim environments
\IfFileExists{upquote.sty}{\usepackage{upquote}}{}
\IfFileExists{microtype.sty}{% use microtype if available
  \usepackage[]{microtype}
  \UseMicrotypeSet[protrusion]{basicmath} % disable protrusion for tt fonts
}{}
\makeatletter
\@ifundefined{KOMAClassName}{% if non-KOMA class
  \IfFileExists{parskip.sty}{%
    \usepackage{parskip}
  }{% else
    \setlength{\parindent}{0pt}
    \setlength{\parskip}{6pt plus 2pt minus 1pt}}
}{% if KOMA class
  \KOMAoptions{parskip=half}}
\makeatother
\usepackage{xcolor}
\setlength{\emergencystretch}{3em} % prevent overfull lines
\setcounter{secnumdepth}{5}
% Make \paragraph and \subparagraph free-standing
\ifx\paragraph\undefined\else
  \let\oldparagraph\paragraph
  \renewcommand{\paragraph}[1]{\oldparagraph{#1}\mbox{}}
\fi
\ifx\subparagraph\undefined\else
  \let\oldsubparagraph\subparagraph
  \renewcommand{\subparagraph}[1]{\oldsubparagraph{#1}\mbox{}}
\fi


\providecommand{\tightlist}{%
  \setlength{\itemsep}{0pt}\setlength{\parskip}{0pt}}\usepackage{longtable,booktabs,array}
\usepackage{calc} % for calculating minipage widths
% Correct order of tables after \paragraph or \subparagraph
\usepackage{etoolbox}
\makeatletter
\patchcmd\longtable{\par}{\if@noskipsec\mbox{}\fi\par}{}{}
\makeatother
% Allow footnotes in longtable head/foot
\IfFileExists{footnotehyper.sty}{\usepackage{footnotehyper}}{\usepackage{footnote}}
\makesavenoteenv{longtable}
\usepackage{graphicx}
\makeatletter
\def\maxwidth{\ifdim\Gin@nat@width>\linewidth\linewidth\else\Gin@nat@width\fi}
\def\maxheight{\ifdim\Gin@nat@height>\textheight\textheight\else\Gin@nat@height\fi}
\makeatother
% Scale images if necessary, so that they will not overflow the page
% margins by default, and it is still possible to overwrite the defaults
% using explicit options in \includegraphics[width, height, ...]{}
\setkeys{Gin}{width=\maxwidth,height=\maxheight,keepaspectratio}
% Set default figure placement to htbp
\makeatletter
\def\fps@figure{htbp}
\makeatother

\usepackage{booktabs}
\usepackage{longtable}
\usepackage{array}
\usepackage{multirow}
\usepackage{wrapfig}
\usepackage{float}
\usepackage{colortbl}
\usepackage{pdflscape}
\usepackage{tabu}
\usepackage{threeparttable}
\usepackage{threeparttablex}
\usepackage[normalem]{ulem}
\usepackage{makecell}
\usepackage{xcolor}
\usepackage{titling}
\setlength{\droptitle}{-2cm}
\preauthor{
  \begin{center}
  \Large
  \vspace{10mm}
  by

  \vspace{20mm}
}
\postauthor{
  \end{center}
  \vfill
}

\predate{
  \begin{center}
  A thesis 
  submitted in fulfilment of the \\
  requirements of the degree of \\
  Doctor of Philosophy in Physics\\               % Degree
  School of Physical and Chemical Sciences\\          % Department
  Te Herenga Waka - Victoria University of Wellington\\                       % University 
  \vspace{5mm}
}
\postdate{
  \\
  \includegraphics[width=3in,height=1.5in]{figures/VUW-logo.png}\\
  \end{center}
  }
\makeatletter
\makeatother
\makeatletter
\@ifpackageloaded{bookmark}{}{\usepackage{bookmark}}
\makeatother
\makeatletter
\@ifpackageloaded{caption}{}{\usepackage{caption}}
\AtBeginDocument{%
\ifdefined\contentsname
  \renewcommand*\contentsname{Table of contents}
\else
  \newcommand\contentsname{Table of contents}
\fi
\ifdefined\listfigurename
  \renewcommand*\listfigurename{List of Figures}
\else
  \newcommand\listfigurename{List of Figures}
\fi
\ifdefined\listtablename
  \renewcommand*\listtablename{List of Tables}
\else
  \newcommand\listtablename{List of Tables}
\fi
\ifdefined\figurename
  \renewcommand*\figurename{Figure}
\else
  \newcommand\figurename{Figure}
\fi
\ifdefined\tablename
  \renewcommand*\tablename{Table}
\else
  \newcommand\tablename{Table}
\fi
}
\@ifpackageloaded{float}{}{\usepackage{float}}
\floatstyle{ruled}
\@ifundefined{c@chapter}{\newfloat{codelisting}{h}{lop}}{\newfloat{codelisting}{h}{lop}[chapter]}
\floatname{codelisting}{Listing}
\newcommand*\listoflistings{\listof{codelisting}{List of Listings}}
\makeatother
\makeatletter
\@ifpackageloaded{caption}{}{\usepackage{caption}}
\@ifpackageloaded{subcaption}{}{\usepackage{subcaption}}
\makeatother
\makeatletter
\@ifpackageloaded{tcolorbox}{}{\usepackage[skins,breakable]{tcolorbox}}
\makeatother
\makeatletter
\@ifundefined{shadecolor}{\definecolor{shadecolor}{rgb}{.97, .97, .97}}
\makeatother
\makeatletter
\makeatother
\makeatletter
\makeatother
\ifLuaTeX
  \usepackage{selnolig}  % disable illegal ligatures
\fi
\usepackage[citestyle = ieee,urldate = iso8601]{biblatex}
\addbibresource{references.bib}
\IfFileExists{bookmark.sty}{\usepackage{bookmark}}{\usepackage{hyperref}}
\IfFileExists{xurl.sty}{\usepackage{xurl}}{} % add URL line breaks if available
\urlstyle{same} % disable monospaced font for URLs
\hypersetup{
  pdftitle={Volatile Organic Compound Detection Using Insect Odorant-Receptor Functionalised Field-Effect Transistors},
  pdfauthor={Eddyn Oswald Perkins Treacher},
  hidelinks,
  pdfcreator={LaTeX via pandoc}}

\title{Volatile Organic Compound Detection Using Insect Odorant-Receptor
Functionalised Field-Effect Transistors}
\author{Eddyn Oswald Perkins Treacher}
\date{Mar 2024}

\begin{document}
\frontmatter
\maketitle
\ifdefined\Shaded\renewenvironment{Shaded}{\begin{tcolorbox}[frame hidden, interior hidden, sharp corners, borderline west={3pt}{0pt}{shadecolor}, boxrule=0pt, enhanced, breakable]}{\end{tcolorbox}}\fi

\renewcommand*\contentsname{Table of contents}
{
\setcounter{tocdepth}{2}
\tableofcontents
}
\mainmatter
\bookmarksetup{startatroot}

\hypertarget{acknowledgements}{%
\chapter*{Acknowledgements}\label{acknowledgements}}
\addcontentsline{toc}{chapter}{Acknowledgements}

\markboth{Acknowledgements}{Acknowledgements}

\begin{verbatim}
69450
\end{verbatim}

Rifat, Alex - vapour sensor Erica Cassie - FET sensing setup Rob Keyzers
and Jennie Ramirez-Garcia - NMR spectra Patricia Hunt - Computational
chemistry

\bookmarksetup{startatroot}

\hypertarget{sec-pristine-characteristics}{%
\chapter{Characteristics of Pristine Carbon Nanotube \& Graphene Field
Effect Transistors}\label{sec-pristine-characteristics}}

\hypertarget{introduction}{%
\section{Introduction}\label{introduction}}

A range of methods were followed to fabricate carbon nanotube network
and graphene field-effect transistors for biosensor use. This chapter
therefore looks to use the characterisation techniques outlined in the
previous chapter to compare and contrast the device channel morphologies
and electrical characteristics resulting from various methods.

The three carbon nanotube film types used for devices were the
solvent-deposited, surfactant-deposited and steam-assisted
surfactant-deposited (steam-deposited) films discussed in the previous
chapter. As minor changes were made to fabrication processes throughout
the thesis, the fabrication dates of devices used are stated, which can
be cross-referenced with \textbf{?@sec-fabrication} to identify the
specific process used. Atomic force microscopy and Raman spectroscopy
was performed on the carbon nanotube networks to identify the
distribution of carbon nanotube diameters and the defects present on the
carbon nanotube networks. Electrical characterisation was then used to
see how the morphology of each film type affects the performance of the
completed devices. Both back-gated and liquid-gated transfer
characteristics were compared, as well as key parameters taken from the
liquid-gated characteristics. The electrical behaviour of liquid-gated
graphene devices was also examined, as well as the impact of water on
the performance of back-gated devices for vapour sensing use.

Finally, as a control measurement for liquid-gated sensing and to verify
the behaviour of the pristine device as a sensor, a salt concentration
sensing series was performed with a steam-deposited carbon nanotube
network device. The device characteristics were taken and device drift
was examined and modelled. The sensing series was performed by
successively diluting 1XPBS electrolyte in the polydimethylsiloxane
`well' (electrolyte container) while passing a current through the
device, and measuring the current response to dilutions. Various filters
were applied to the collected data to better understand the signal
change.

\hypertarget{sec-pristine-morphology}{%
\section{Carbon Nanotube Network Morphology and
Composition}\label{sec-pristine-morphology}}

\hypertarget{sec-pristine-AFM}{%
\subsection{Atomic Force Microscopy}\label{sec-pristine-AFM}}

Figure~\ref{fig-afm-morphology} shows a side-by-side comparison of the
surface morphology of carbon nanotube films fabricated using the methods
described in \textbf{?@sec-dep-carbon-nanotubes}. These images were
collected using an atomic force microscope and processed in the manner
described in \textbf{?@sec-afm-characterisation}.
Figure~\ref{fig-bundled-network} shows a film of carbon nanotubes
deposited in solvent, Figure~\ref{fig-dropcast-network} shows a film of
carbon nanotubes dropcast in surfactant, and
Figure~\ref{fig-steaming-network} shows carbon nanotubes dropcast in
surfactant in the presence of steam. As discussed in previous works
using solvent-based deposition techniques for depositing carbon
nanotubes, in each network multi-tube bundles form due to strong mutual
attraction between nanotubes
\autocite{Zheng2017,Murugathas2018,Murugathas2019a,Nguyen2021}. However,
when surfactants are present, they adsorb onto the carbon nanotubes and
form a highly repulsive structure able to overcome the strong attraction
between nanotubes. This repulsion keeps the individual carbon nanotubes
more isolated
\autocite{Wenseleers2004,Gavrel2013,Hermanson2013-16,Shimizu2013,DiCrescenzo2014}.
The diameter range provided by the supplier for the individual carbon
nanotubes used is \(1.2-1.7\) nm, while the length range is \(0.3-5.0\)
\(\mu\)m (Nanointegris).

\begin{figure}

\begin{minipage}[t]{0.47\linewidth}

{\centering 

\raisebox{-\height}{

\includegraphics{figures/ch6/Ned_NTQ24_20220125_00235.png}

}

}

\subcaption{\label{fig-bundled-network}}
\end{minipage}%
%
\begin{minipage}[t]{0.05\linewidth}

{\centering 

~

}

\end{minipage}%
%
\begin{minipage}[t]{0.47\linewidth}

{\centering 

\raisebox{-\height}{

\includegraphics{figures/ch6/Ned_NTQ24_20220125_00235_histogram_initialguess.png}

}

}

\subcaption{\label{fig-bundled-network-histogram}}
\end{minipage}%
\newline
\begin{minipage}[t]{0.47\linewidth}

{\centering 

\raisebox{-\height}{

\includegraphics{figures/ch6/Ned_NTQ8C7_w4_pristine_00084_20210428(2).png}

}

}

\subcaption{\label{fig-dropcast-network}}
\end{minipage}%
%
\begin{minipage}[t]{0.05\linewidth}

{\centering 

~

}

\end{minipage}%
%
\begin{minipage}[t]{0.47\linewidth}

{\centering 

\raisebox{-\height}{

\includegraphics{figures/ch6/Ned_NTQ8C7_w4_pristine_00084_20210428(2)_histogram_initialguess.png}

}

}

\subcaption{\label{fig-dropcast-network-histogram}}
\end{minipage}%
\newline
\begin{minipage}[t]{0.47\linewidth}

{\centering 

\raisebox{-\height}{

\includegraphics{figures/ch6/Ned_NGQ14D2_W4_pristine_20220713_00567.png}

}

}

\subcaption{\label{fig-steaming-network}}
\end{minipage}%
%
\begin{minipage}[t]{0.05\linewidth}

{\centering 

~

}

\end{minipage}%
%
\begin{minipage}[t]{0.47\linewidth}

{\centering 

\raisebox{-\height}{

\includegraphics{figures/ch6/Ned_NGQ14D2_W4_pristine_20220713_00567_histogram_initialguess.png}

}

}

\subcaption{\label{fig-steaming-network-histogram}}
\end{minipage}%

\caption{\label{fig-afm-morphology}2.5 \(\mu\)m \(\times\) 2.5 \(\mu\)m
atomic force microscope (AFM) images of carbon nanotube films deposited
using various methods, shown side-by-side with histogram height
distributions and kernel density estimate (KDE) plots corresponding to
each image. The network shown in (a) with height distribution shown in
(b) was deposited in solvent, the network shown in (c) with height
distribution shown in (d) was dropcast in surfactant, and the network
shown in (e) with height distribution shown in (f) was dropcast in
surfactant with steam present.}

\end{figure}

It has previously been demonstrated that the diameter range of deposited
single-walled carbon nanotubes can be modelled via a normal or Gaussian
distribution \autocite{LeMieux2008,Liu2013,Vobornik2023}. However, when
the height profiles from the 2.5 \(\mu\)m \(\times\) 2.5 \(\mu\)m AFM
images are directly extracted and binned, as plotted in black in
Figure~\ref{fig-afm-morphology}, the histograms obtained do not follow a
normal distribution. One reason for this result is the surface roughness
of the silicon dioxide substrate. The carbon nanotubes do not lie
perfectly level on a perfectly level silicon oxide substrate. In
practice, both the SiO\(_2\) substrate and the surface of the carbon
nanotubes both have a degree of roughness. To find the contribution of
surface roughness to the height profile histogram corresponding to each
network deposition method, silicon dioxide substrates were modified
using the same processes as in Figure~\ref{fig-afm-morphology} but
without carbon nanotubes present in the solutions used. 2.5 \(\mu\)m
\(\times\) 2.5 \(\mu\)m AFM images of the modified surfaces are shown in
Figure~\ref{fig-afm-substrate}.

\begin{figure}

\begin{minipage}[t]{0.47\linewidth}

{\centering 

\raisebox{-\height}{

\includegraphics{figures/ch6/Ned_SiO2_00351_20231016.png}

}

}

\subcaption{\label{fig-sio2-only}}
\end{minipage}%
%
\begin{minipage}[t]{0.05\linewidth}

{\centering 

~

}

\end{minipage}%
%
\begin{minipage}[t]{0.47\linewidth}

{\centering 

\raisebox{-\height}{

\includegraphics{figures/ch6/Ned_SiO2_00351_20231016_histogram_initialguess.png}

}

}

\subcaption{\label{fig-sio2-histogram}}
\end{minipage}%
\newline
\begin{minipage}[t]{0.47\linewidth}

{\centering 

\raisebox{-\height}{

\includegraphics{figures/ch6/Ned_SiO2_s_surfactant_nosteam_00355_20231016.png}

}

}

\subcaption{\label{fig-surfactant-afm}}
\end{minipage}%
%
\begin{minipage}[t]{0.05\linewidth}

{\centering 

~

}

\end{minipage}%
%
\begin{minipage}[t]{0.47\linewidth}

{\centering 

\raisebox{-\height}{

\includegraphics{figures/ch6/Ned_SiO2_s_surfactant_nosteam_00355_20231016_histogram_initialguess.png}

}

}

\subcaption{\label{fig-surfactant-histogram}}
\end{minipage}%
\newline
\begin{minipage}[t]{0.47\linewidth}

{\centering 

\raisebox{-\height}{

\includegraphics{figures/ch6/Ned_SiO2_s_surfactant_steam_00357_20231016.png}

}

}

\subcaption{\label{fig-steamed-surfactant}}
\end{minipage}%
%
\begin{minipage}[t]{0.05\linewidth}

{\centering 

~

}

\end{minipage}%
%
\begin{minipage}[t]{0.47\linewidth}

{\centering 

\raisebox{-\height}{

\includegraphics{figures/ch6/Ned_SiO2_s_surfactant_steam_00357_20231016_histogram_initialguess.png}

}

}

\subcaption{\label{fig-steamed-surfactant-histogram}}
\end{minipage}%

\caption{\label{fig-afm-substrate}2.5 \(\mu\)m \(\times\) 2.5 \(\mu\)m
atomic force microscope (AFM) images of silicon dioxide substrates
alongside histogram height distributions and KDE plots corresponding to
each image. The substrate in (a) and (b) was exposed to solvent, the
substrate in (c) and (d) was exposed to surfactant, and the substrate in
(e) and (f) was exposed to surfactant with steam present.}

\end{figure}

In Figure~\ref{fig-afm-substrate}, it appears that each substrate
surface has a roughness that follows a normal distribution with some
degree of skewness. Figure~\ref{fig-sio2-histogram} and
Figure~\ref{fig-surfactant-histogram} are negatively skewed
distributions. The fitted skew-normal distribution in
Figure~\ref{fig-sio2-histogram} has a skew parameter \(\alpha\) (or
shape parameter) of -3.2, a location parameter \(\xi\) of 2.2 nm and a
scale parameter \(\omega\) of 0.5 nm, while in
Figure~\ref{fig-surfactant-histogram} \(\alpha = -2.2\), \(\xi = 2.2\)
nm and \(\omega = 0.5\) nm. \(\xi\) and \(\omega\) correspond to the
mean and standard deviation of the skew-free normal distribution when
\(\alpha\) is set equal to zero \autocite{Azzalini1999}. The close
correspondence between \(\xi\) and \(\omega\) for these distributions
but not \(\alpha\) implies that the skewness is a variable imaging or
processing artifact rather than a physical property of the surface.
Without distortion, the roughness of a clean SiO\(_2\) surface should
follow a normal distribution \autocite{Velicky2015}.

However, Figure~\ref{fig-steamed-surfactant-histogram} has a pronounced
positive skew with a long tail. The tail appears to result from the
contribution of residual surfactant aggregates to surface morphology,
observed in Figure~\ref{fig-steamed-surfactant} and recently discussed
elsewhere in the literature \autocite{Christensen2022,Vobornik2023}.
Attempting to fit a skew-normal distribution to this histogram fails
when all three variables are allowed to vary due to the presence of the
tail. Instead, previous values obtained for \(\xi\) and \(\omega\) can
be used for the fitting process, with only \(\alpha\) allowed to change.
Fixing \(\xi\) and \(\omega\) at 2.2 nm and 0.5 nm respectively gives
the result shown in Figure~\ref{fig-steamed-surfactant-histogram}. The
fitted distribution has an \(\alpha\) of -2.4. The distribution closely
fits the negative tail of the histogram, but deviates slightly from the
positive tail due to the presence of surfactant. Since this deviation is
small, the quality of the fit is still reasonably high, with an
R-squared value of 0.98. Surfactant contamination could have negative
effects on both sensitivity of carbon nanotubes and also could damage
attached biological elements.

\begin{figure}

\begin{minipage}[t]{0.47\linewidth}

{\centering 

\raisebox{-\height}{

\includegraphics{figures/ch6/Ned_NTQ8C7_w4_pristine_00084_20210428(2)_mask.png}

}

}

\subcaption{\label{fig-mask}}
\end{minipage}%
%
\begin{minipage}[t]{0.05\linewidth}

{\centering 

~

}

\end{minipage}%
%
\begin{minipage}[t]{0.47\linewidth}

{\centering 

\raisebox{-\height}{

\includegraphics{figures/ch6/NTQ24_20220125_00235_cnt_histogram.png}

}

}

\subcaption{\label{fig-solvent-cnt-histogram}}
\end{minipage}%
\newline
\begin{minipage}[t]{0.47\linewidth}

{\centering 

\raisebox{-\height}{

\includegraphics{figures/ch6/NTQ8C7_w4_pristine_cnt_histogram.png}

}

}

\subcaption{\label{fig-surfactant-cnt-histogram}}
\end{minipage}%
%
\begin{minipage}[t]{0.05\linewidth}

{\centering 

~

}

\end{minipage}%
%
\begin{minipage}[t]{0.47\linewidth}

{\centering 

\raisebox{-\height}{

\includegraphics{figures/ch6/NT14D2_W4_pristine_cnt_histogram.png}

}

}

\subcaption{\label{fig-steamed-surfactant-cnt-histogram}}
\end{minipage}%

\caption{\label{fig-cnt-histogram}An masked AFM image is shown in (a),
where the masked carbon nanotube bundles are shaded blue. The mask sets
a height threshold so that masked features are excluded from the height
dataset. Histogram height distributions with corresponding KDE plots
collected via the morphology analysis method outlined by Vobornik
\emph{et al.} \autocite{Vobornik2023} are shown in (b)-(d). The
substrate in (b) was exposed to solvent, the substrate in (c) was
exposed to surfactant, and the substrate in (d) was exposed to
surfactant with steam present.}

\end{figure}

Using the morphology analysis technique outlined by Vobornik \emph{et
al.} \autocite{Vobornik2023}, five successive diameter measurements of
30 carbon nanotube bundles were collected using Gwyddion. Measurements
were not taken at bundle junctions. A height threshold `mask' was
defined in Gwyddion to determine average substrate height, as shown in
Figure~\ref{fig-mask}. This background value was subtracted from our
diameter measurements to determine the actual bundle height. The mean
background height of the solvent-deposited, surfactant-deposited and
steam-assisted surfactant-deposited bundle diameter histograms were
\(8.8 \pm 4.0\) nm, \(4.2 \pm 1.8\) nm and \(3.3 \pm 1.0\) nm
respectively. An increased maximum feature height leads to an increased
mean background height, and by examining the AFM images in
Figure~\ref{fig-afm-morphology} it appears this may be due to deep
artifacts on the surface of the substrate in the vicinity of large
features. The average of the five height-adjusted values for each carbon
nanotube bundle was then calculated, and these 30 averages were sorted
into six equal-sized bins. The binned bundle diameter measurements,
alongside estimated probability density, are shown in
Figure~\ref{fig-cnt-histogram}.

From Figure~\ref{fig-cnt-histogram}, it is clear that each histogram
appears to follow a positively skewed normal distribution, different to
the skew-free normal distribution expected from previous works
\autocite{LeMieux2008,Liu2013,Vobornik2023}. The skew is likely another
artifact from imaging the network with the atomic force microscope. The
force of the atomic force microscope tip is known to cause larger
bundles to undergo some degree of compression, and the resulting
systematic underestimation of their height may be responsible for the
distribution skewness \autocite{Vobornik2023}. The fitted skew-normal
distribution in Figure~\ref{fig-solvent-cnt-histogram} has
\(\alpha = 2.7\) (shape), \(\xi = 4.3\) nm (location/mean),
\(\omega = 5.9\) nm (scale/standard deviation), the distribution in
Figure~\ref{fig-surfactant-cnt-histogram} has \(\alpha = 2.4\),
\(\xi = 2.2\) nm, \(\omega = 2.6\) nm, and the distribution in
Figure~\ref{fig-steamed-surfactant-cnt-histogram} has \(\alpha = 3.6\),
\(\xi = 2.2\) nm and \(\omega = 1.5\) nm. The probability density for
the carbon nanotube bundle histogram drops to approximately zero at or
before 0 nm, which is physically appropriate.

\begin{verbatim}
Warning: package 'kableExtra' was built under R version 4.3.3
\end{verbatim}

\hypertarget{tbl-circle-packing}{}
\begin{longtable}[]{@{}
  >{\raggedright\arraybackslash}p{(\columnwidth - 16\tabcolsep) * \real{0.1053}}
  >{\raggedright\arraybackslash}p{(\columnwidth - 16\tabcolsep) * \real{0.1228}}
  >{\raggedright\arraybackslash}p{(\columnwidth - 16\tabcolsep) * \real{0.0965}}
  >{\raggedright\arraybackslash}p{(\columnwidth - 16\tabcolsep) * \real{0.0965}}
  >{\raggedright\arraybackslash}p{(\columnwidth - 16\tabcolsep) * \real{0.1228}}
  >{\raggedright\arraybackslash}p{(\columnwidth - 16\tabcolsep) * \real{0.1228}}
  >{\raggedright\arraybackslash}p{(\columnwidth - 16\tabcolsep) * \real{0.1228}}
  >{\raggedright\arraybackslash}p{(\columnwidth - 16\tabcolsep) * \real{0.1228}}
  >{\raggedright\arraybackslash}p{(\columnwidth - 16\tabcolsep) * \real{0.0877}}@{}}
\caption{\label{tbl-circle-packing}The first eight optimised ratios of
2D packed circle diameter to encompassing circle diameter, given to 3
s.f. (encompassing circle diameter = \(d\), number of packed circles =
\(n\), approximate packed circle diameter = \(d_n\)).\\
}\tabularnewline
\toprule\noalign{}
\endfirsthead
\endhead
\bottomrule\noalign{}
\endlastfoot
\(n\) & \text{2} & \text{3} & \text{4} & \text{5} & \text{6} & \text{7}
& \text{8} & \text{9} \\
\(d\)/\(d_n\) & \text{2.00} & 2.15 & 2.41 & \text{2.70} & \text{3.00} &
\text{3.00} & \text{3.30} & 3.61 \\
\end{longtable}

Previously, analysis of the morphology of carbon nanotube networks has
been simplified by assuming the component nanotubes are cylinders,
follow 2D packing and are of equal diameter \autocite{Murugathas2018}.
Table~\ref{tbl-circle-packing} shows the relationship between the
diameter of a bundle of 2D packed cylinders and the constituent
diameters of up to nine cylinders within that bundle. From looking up
the relevant \(d\)/\(d_n\) packing ratios, and assuming an average
carbon nanotube diameter of 1.45 nm, it is possible to use to find the
approximate number of nanotubes \emph{n} likely to be present in the
mean bundle size corresponding to each deposition type
\autocite{Graham1998,Specht2023}. These estimates are shown in
Table~\ref{tbl-histogram-parameters}. Also shown in
Table~\ref{tbl-histogram-parameters} is an estimate of the ratio of
single- to multi-tube bundles for each deposition. This estimate was
obtained by taking the integral of each distribution with a lower bound
of 2.9 nm, the minimum multi-tube bundle size for 1.45 nm diameter
nanotubes. As the area under the curve represents the probability a
bundle will have a particular diameter, this integral should give a good
estimate of the relative proportion of multi-tube bundles.
Table~\ref{tbl-histogram-parameters} should be interpreted as
lower-limit estimates of the size and relative proportion of bundles,
recalling that the distribution skewness indicates underestimation of
the true bundle height.

\hypertarget{tbl-histogram-parameters}{}
\begin{longtable}[t]{>{\raggedright\arraybackslash}p{4cm}>{\centering\arraybackslash}p{3cm}>{\centering\arraybackslash}p{3cm}>{\centering\arraybackslash}p{3cm}}
\caption{\label{tbl-histogram-parameters}The mean of histogram distributions for carbon nanotube films deposited
using various methods, alongside estimates for the number of nanotubes
present per mean bundle and the estimated proportion of multi-tubed
bundles present across the network. }\tabularnewline

\toprule
 & Mean Bundle Diameter (nm) & Tubes per Average Bundle & \% Multi-Tube Bundles\\
\midrule
Solvent deposited & 8.8 ± 4.0 & 28 & > 96\%\\
Surfactant deposited & 4.2 ± 1.8 & 5 & > 75\%\\
Surfactant deposited with steam & 3.3 ± 1.0 & 3 & > 65\%\\
\bottomrule
\end{longtable}

Both the carbon nanotube bundle diameter mean and standard deviation are
small for surfactant-deposited films when compared to the mean and
standard deviation of solvent-deposited films. However, despite the
presence of surfactant, it is apparent both from
Figure~\ref{fig-afm-morphology} and Table~\ref{tbl-histogram-parameters}
that not all surfactant-dispersed carbon nanotubes are deposited
individually. Bundling may occur during the process of deposition onto
the substrate, which could disrupt the repulsive forces from the
surfactant coating and allow attractive forces to temporarily dominate.
It is possible that the bundling of surfactant-dispersed carbon
nanotubes is a consequence of dynamics introduced by the coffee-ring
effect \autocite{Deegan1997,VanGaalen2021}. The coffee-ring effect
refers to a build-up of dispersed solid forming around the edges of a
dispersion evaporating on a surface. This process occurs due to the
dispersion edges being fixed by surface forces, leading to capillary
flow outwards to replace liquid evaporating at the edges, bringing solid
material along with it. The presence of vapour is known to disrupt this
capillary effect \autocite{Bishop2020}, which may explain why mean
bundle diameter is lower for the films deposited in surfactant with
steam present relative to films deposited in surfactant without steam.

The discussion in this section gives us a new understanding of the
histograms shown in Figure~\ref{fig-afm-morphology}. It is now apparent
that these histograms are linear combinations of skewed normal
distributions. These distributions include a negatively-skewed
distribution corresponding to the substrate surface and a
positively-skewed distribution corresponding to the carbon nanotube
bundles. X and Y junctions between overlapping nanotubes may also form a
similarly skewed normal distribution as part of the full histogram
\autocite{Murugathas2018}. The complete linear combination could be
modelled mathematically in order to rapidly extract key parameters from
atomic force microscope images \autocite{Marchenko2010}, but
implementing this approach is outside of the scope of this thesis.
Another outcome of this discussion is awareness that carbon nanotube
bundling within a network is lowered by the presence of surfactant
during deposition. Introducing steam when depositing with surfactant
lowers bundling even further, but also leads to residual surfactant
pooling and attaching to the substrate surface. These results may both
be explained by the presence of steam enabling surfactant to follow
carbon nanotubes to the substrate surface, which keeps them from
bundling during the attachment process. The unwanted persistence of
surfactant means that higher temperature vacuum annealing may be
required for robust biosensors \autocite{Kane2014}.

\hypertarget{sec-pristine-raman}{%
\subsection{Raman Spectroscopy}\label{sec-pristine-raman}}

\begin{figure}

\begin{minipage}[t]{0.47\linewidth}

{\centering 

\raisebox{-\height}{

\includegraphics{figures/ch6/bundled_raman.png}

}

}

\subcaption{\label{fig-solvent-deposited}}
\end{minipage}%
%
\begin{minipage}[t]{0.05\linewidth}

{\centering 

~

}

\end{minipage}%
%
\begin{minipage}[t]{0.47\linewidth}

{\centering 

\raisebox{-\height}{

\includegraphics{figures/ch6/singletube_raman.png}

}

}

\subcaption{\label{fig-surfactant-deposited}}
\end{minipage}%
\newline
\begin{minipage}[t]{0.33\linewidth}

{\centering 

~

}

\end{minipage}%
%
\begin{minipage}[t]{0.35\linewidth}

{\centering 

\raisebox{-\height}{

\includegraphics{figures/ch6/comparison-raman.png}

}

}

\subcaption{\label{fig-dg-peak-comparison}}
\end{minipage}%
%
\begin{minipage}[t]{0.33\linewidth}

{\centering 

~

}

\end{minipage}%

\caption{\label{fig-pristine-raman}A series of nine Raman spectra at
different locations across a 40 \(\mu\)m \(\times\) 100 \(\mu\)m carbon
nanotube film region, where (a) shows spectra from a film deposited
using solvent while (b) shows spectra from a film deposited with
surfactant in the presence of steam. (c) shows the spread of the
D-peak/G\(^+\)-peak spectral ratios corresponding to each film.}

\end{figure}

Raman spectroscopy was also used to analyse and compare the deposited
carbon nanotube networks. Raman spectra were collected from a
solvent-deposited carbon nanotube film and a steam-assisted
surfactant-deposited film, both on silicon dioxide, in the manner
described in \textbf{?@sec-raman-characterisation}. These spectra were
then processed using the Python script mentioned in
Section~\ref{sec-raman-analysis}. For each location, spectra over two
wavenumber ranges were collected. A peak corresponding to the silicon
dioxide substrate, found in the range between 100 cm\(^{-1}\) and 650
cm\(^{-1}\), was used as a reference peak for the normalisation of
intensity across the wavenumber range between 1300 cm\(^{-1}\) and 1650
cm\(^{-1}\). These normalised spectra are shown in
Figure~\ref{fig-pristine-raman}. In all spectra, a D-band comprising a
single D-peak is observed at \(\sim\) 1320 cm\(^{-1}\), and a G-band
comprising two G-peaks, G\(^-\) and G\(^+\), is observed between
\(\sim\) 1525 cm\(^{-1}\) and \(\sim\) 1650 cm\(^{-1}\). These features
are characteristic of networks of semiconducting carbon nanotubes
\autocite{Dresselhaus2005,King2014}.

Closer inspection of the D peak and G peaks can give us important
information about network composition. G\(^-\) is a minor peak found at
\(\sim\) 1570 cm\(^{-1}\), while G\(^+\) is a larger feature at \(\sim\)
1590 cm\(^{-1}\). The G\(^+\) feature describes the in-plane vibration
of carbon bonds along the length of the carbon nanotubes, while the
G\(^-\) feature describes the in-plane vibration of bonds about the
nanotube circumference \autocite{King2014,Swiniarski2021}. The splitting
between the wavenumber location of the G\(^-\) and G\(^+\) local maxima
is lower in Figure~\ref{fig-surfactant-deposited} than in
Figure~\ref{fig-solvent-deposited}, indicating more metallic nanotubes
are present in the surfactant-deposited network
\autocite{Swiniarski2021}. The D-peak gives an indication of the defects
present in the carbon nanotube atomic structure
\autocite{King2014,Swiniarski2021}. The size of the normalised D-peak
appears much lower in Figure~\ref{fig-solvent-deposited} than in
Figure~\ref{fig-surfactant-deposited}, indicating the solvent deposition
process introduces less defects to the carbon nanotubes than
surfactant-mediated deposition.

It is also possible to compare the relative magnitude of the D-peak and
G\(^+\)-peak intensity to quantify carbon nanotube structural disorder,
which disrupts in-plane lattice vibration
\autocite{Dresselhaus2005,King2014}. Figure~\ref{fig-dg-peak-comparison}
gives a summary of the ratios between the D-peak and G\(^+\)-peak across
all nine positions for the solvent-deposited and surfactant-deposited
film. It is immediately observed that I\(_{D}\)/I\(_{G}\) is
significantly larger for the steam-assisted, surfactant-deposited films
than for the solvent-deposited films. This is a further indication of
the presence of defects across the steam-deposited network. These
defects are likely introduced through the introduction of charge
impurites by surfactant aggregates present around the carbon nanotubes
\autocite{Christensen2022}. However, at the same time, the range of
values for the I\(_{D}\)/I\(_{G}\) ratio is lower for the
steam-deposited network. This spatially homogeneous vibrational
behaviour implies the steam-deposited network is more evenly distributed
than the solvent-deposited network, which matches the discussion in
Section~\ref{sec-pristine-morphology}.

\hypertarget{sec-pristine-electrical-characterisation}{%
\section{Electrical Characteristics of Pristine
Devices}\label{sec-pristine-electrical-characterisation}}

\hypertarget{sec-python-analysis}{%
\subsection{Python Analysis}\label{sec-python-analysis}}

Analysis of electrical measurements was performed using the three
modules described in Section~\ref{sec-field-effect-transistor-analysis}.
The first of the three modules is for processing sensing datasets. This
module cleans, analyses and filters sensing data and produces a variety
of plots. These plots include normalised plots (type of normalisation
can be set in the code config file), plots with fitted curves, plots
with the linear baseline drift removed, plots of signal with analyte
addition, ``despiked'' plots and ``filtered'' plots. The analysis used
to produce these plots is described further below. It is possible to add
annotations to any of these plots using the config file, and it is
possible to produce a plot with a combination of these modifications.
The module can also fit exponential and linear trendlines to regions of
the sensing data, and find the signal change per analyte addition; the
module then returns spreadsheets containing the results of these
analyses, including the standard deviation for all calculated
parameters.

The scipy.optimize.curve\_fit function is used in the first module to
fit linear and exponential curves to regions of interest of the sensing
data. For a linear fit \(c_1t + c_2\), initial parameters are simply set
as \(c_1=1\) and \(c_2=0\). For an exponential fit
\(I_0\exp{(-t/\tau)} + I_C\), rough approximations are used for the
initial parameters: \(I_C\) is set as the final current measurement of
the region of interest, \(I_0\) is set as the initial current
measurement minus \(I_C\), and \(\tau\) is set as the time where current
has dropped to \(e^{-1}I_0 + I_C\).

``Despiked'' plots have had spurious datapoints removed through the use
of an interquartile range rolling filter. The window size of the rolling
filter used was 40 datapoints, and datapoints in each window with a
z-score above \(\pm 3\) were removed from the plotted/processed data.
``Filtered'' plots had noise reduced using a moving median filter. The
moving median filter is more effective at removing noise than a simple
moving average, and has advantages over other filters (such as the
Savitzky-Golay filter) when removing noise from data with sharp edges,
as is the case for sensing data. Median filtering can also be used for
baseline drift compensation, though this approach was not used in this
thesis \autocite{Stone2011}. The moving median filter used had a window
of 40 datapoints.

Plots of signal with analyte addition were constructed from current data
after first removing baseline drift and applying a moving median filter.
A simple difference calculation between the mean of the filtered current
before an addition and the mean of the filtered current after the
addition was performed at each addition. These differences were then
normalised relative to the initial current. The signal with analyte
addition give reasonably consistent results regardless of whether
baseline drift was removed from the data, as shown in
Figure~\ref{fig-spaa-plot-comparison}. We can therefore be confident
that robust signal with analyte addition plots are robust even in the
presence of significant drift.

\begin{figure}

\begin{minipage}[t]{0.50\linewidth}

{\centering 

\raisebox{-\height}{

\includegraphics{figures/ch6/NTQ31C1_mean_simple_difference_before_and_after_step_filtered_concentrations.png}

}

}

\subcaption{\label{fig-spaa-no-detrend}}
\end{minipage}%
%
\begin{minipage}[t]{0.50\linewidth}

{\centering 

\raisebox{-\height}{

\includegraphics{figures/ch6/NTQ31C1_mean_simple_difference_before_and_after_step_filtered_concentrations_detrend.png}

}

}

\subcaption{\label{fig-spaa-detrend}}
\end{minipage}%

\caption{\label{fig-spaa-plot-comparison}A comparison of signal with
analyte addition plots taken from the same salt concentration sensing
dataset (the same dataset as used in
Figure~\ref{fig-salt-conc-sensing}). In (a), a simple difference
calculation performed on filtered data was used, while in (b) the same
calculation was performed on filtered data with the baseline drift
removed, the method used in the body of the thesis.}

\end{figure}

The second module creates combined and individual plots of transfer data
collected from eight channels on a single device. In combined plots,
channels which are non-working, due to being shorted or non-conducting,
are removed via setting a maximum and minimum possible on-current in the
config file. Various parameters from the transfer characteristics are
saved as a spreadsheet along with standard error. These parameters
include on current, off current, subthreshold slope and threshold
voltage for the carbon nanotube devices, and on current, off current and
major Dirac point voltage for graphene devices. The device type being
analysed can be set in the config file.

The third module allows for comparison of transfer measurements taken of
the same channel before and after some modification. It also calculates
the shift in either threshold voltage or major Dirac voltage of the
device.

\hypertarget{sec-cnt-devices}{%
\subsection{Carbon Nanotube Network Devices}\label{sec-cnt-devices}}

\begin{figure}

\begin{minipage}[t]{0.49\linewidth}

{\centering 

\raisebox{-\height}{

\includegraphics{figures/ch6/NTQ24C8_pristine_TXLG01_220211_solvent_gate.png}

}

}

\subcaption{\label{fig-solvent-tx-lg}}
\end{minipage}%
%
\begin{minipage}[t]{0.02\linewidth}

{\centering 

~

}

\end{minipage}%
%
\begin{minipage}[t]{0.49\linewidth}

{\centering 

\raisebox{-\height}{

\includegraphics{figures/ch6/NTQ22C2_solvent_backgate.png}

}

}

\subcaption{\label{fig-solvent-tx-bg}}
\end{minipage}%
\newline
\begin{minipage}[t]{0.49\linewidth}

{\centering 

\raisebox{-\height}{

\includegraphics{figures/ch6/NTQ5C3_pristine_TXLG01_210602_nosteam_gate.png}

}

}

\subcaption{\label{fig-surf-tx-lg}}
\end{minipage}%
%
\begin{minipage}[t]{0.02\linewidth}

{\centering 

~

}

\end{minipage}%
%
\begin{minipage}[t]{0.49\linewidth}

{\centering 

\raisebox{-\height}{

\includegraphics{figures/ch6/Q5C10_nosteam_backgate.png}

}

}

\subcaption{\label{fig-surf-tx-bg}}
\end{minipage}%
\newline
\begin{minipage}[t]{0.49\linewidth}

{\centering 

\raisebox{-\height}{

\includegraphics{figures/ch6/NTQ31C6_pristine_TXLG01_230330_steam_gate.png}

}

}

\subcaption{\label{fig-steam-tx-lg}}
\end{minipage}%
%
\begin{minipage}[t]{0.02\linewidth}

{\centering 

~

}

\end{minipage}%
%
\begin{minipage}[t]{0.49\linewidth}

{\centering 

\raisebox{-\height}{

\includegraphics{figures/ch6/Q18C6_steam_backgate.png}

}

}

\subcaption{\label{fig-steam-tx-bg}}
\end{minipage}%

\caption{\label{fig-pristine-cnt-characteristics}Liquid-gated (left) and
back-gated (right) transfer characteristics of AZ\(^\circledR\) 1518
encapsulated field-effect transistors, where the film was deposited with
solvent in (a) and (b), deposited with surfactant in (c) and (d), and
deposited with surfactant in the presence of steam in (e) and (f). A
step size of 100 mV was used for the backgated sweeps in (a), (c) and
(e), while a step size of 20 mV was used for the liquid-gated sweeps in
(b), (d) and (f). Gate current (leakage current) is shown with a dashed
line. The source-drain voltage used for all sweeps was
\(V_{ds} = 100 \textrm{mV}\), and 1XPBS was used as the buffer for the
liquid-gated measurements here.}

\end{figure}

Each carbon nanotube device fabricated was electrically characterised as
described in \textbf{?@sec-electrical-characterisation}, and electrical
data was analysed using the Python code discussed in
Section~\ref{sec-field-effect-transistor-analysis}. Devices with a 100
nm or 300 nm SiO\(_2\) layer were used for liquid gated measurements,
and devices with a 100 nm SiO\(_2\) layer were used for backgated
measurements. Figure~\ref{fig-pristine-cnt-characteristics} displays
multi-channel measurements of representative devices fabricated as
described in \textbf{?@sec-fabrication}. To ensure a consistent
comparison, each device here was encapsulated with AZ\(^\circledR\) 1518
encapsulation before measurements were taken. The channels which did not
exhibit reliable transistor characteristics are not shown. These
`non-working' channels were either shorted, due to metal remaining on
the channel after lift-off, or were very low current, due to a very
sparse carbon nanotube network. Devices shown here with a
solvent-deposited carbon nanotube network were fabricated prior to Jan
2022; devices with a surfactant-deposited network without steam present
were fabricated prior to Jun 2021; devices with a surfactant-deposited
network without steam were fabricated prior to Sep 2022.

\begin{figure}

\begin{minipage}[t]{0.47\linewidth}

{\centering 

\raisebox{-\height}{

\includegraphics{figures/ch6/onoff_CNT.png}

}

}

\subcaption{\label{fig-on-off-ratio}}
\end{minipage}%
%
\begin{minipage}[t]{0.05\linewidth}

{\centering 

~

}

\end{minipage}%
%
\begin{minipage}[t]{0.47\linewidth}

{\centering 

\raisebox{-\height}{

\includegraphics{figures/ch6/SS.png}

}

}

\subcaption{\label{fig-subthreshold-slope}}
\end{minipage}%
\newline
\begin{minipage}[t]{0.26\linewidth}

{\centering 

~

}

\end{minipage}%
%
\begin{minipage}[t]{0.47\linewidth}

{\centering 

\raisebox{-\height}{

\includegraphics{figures/ch6/threshold_V.png}

}

}

\subcaption{\label{fig-threshold-voltage}}
\end{minipage}%
%
\begin{minipage}[t]{0.26\linewidth}

{\centering 

~

}

\end{minipage}%

\caption{\label{fig-sweep-parameters}These boxplots illustrate the
statistical distribution of (a) the on-off ratio, (b) the subthreshold
slope, and (c) the threshold voltage of AZ\(^\circledR\) 1518
encapsulated liquid-gated transistor channels corresponding to each type
of carbon nanotube film deposition. For each deposition type, electrical
characteristics were taken of 21 channels of at least three separate
devices. The boxes indicate the 25th and 75th percentile of the
distribution.}

\end{figure}

\hypertarget{liquid-gated-cntfets}{%
\subsubsection*{Liquid-Gated CNTFETs}\label{liquid-gated-cntfets}}
\addcontentsline{toc}{subsubsection}{Liquid-Gated CNTFETs}

The liquid-gated devices in Figure~\ref{fig-solvent-tx-lg},
Figure~\ref{fig-surf-tx-lg} and Figure~\ref{fig-steam-tx-lg} each
exhibited ambipolar characteristics, commonly observed in liquid-gated
carbon nanotube network FETs
\autocite{Kauffman2008,Heller2009,JongYu2009,Derenskyi2014,Murugathas2018,Albarghouthi2022}.
When devices were appropriately configured, leakage current (shown by
the dashed traces) did not exceed \(\sim 1 \times 10^{-7}\) V across the
forward and reverse sweep. The devices shown which used steam-deposited
carbon nanotube films showed the least hysteresis.
Section~\ref{sec-pristine-AFM} demonstrates that the mean diameter of
the bundles in these films is about 0.9 nm less than the mean bundles in
films deposited without steam present, and 5.5 nm less than those in
films deposited in solvent. Hysteresis is known to scale roughly
linearly with bundle diameter, due to trapped charge increasing as
bundle density of states is increased \autocite{Pop2009}.
Steam-deposited devices also showed significantly less
channel-to-channel variation in electrical characteristics more
generally. Channel 1 in Figure~\ref{fig-solvent-tx-lg} has a much higher
off-current than the other channels of the same device, which appears to
be due to a uncommonly high proportion of metallic carbon nanotubes
present in the network conduction pathways of this channel
\autocite{Rouhi2011,Zaumseil2015}.

A summary of key parameters of pristine liquid-gated devices is shown in
Figure~\ref{fig-sweep-parameters}. The full dataset consists of three
sets of 21 liquid-gated transfer characteristics of working channels,
with each set corresponding to the use of a particular method of carbon
nanotube network deposition in the device fabrication. Measurements from
at least three devices are included in each set. Each entry in the
summary corresponds to the average of the specific parameter in the
forward and reverse sweep direction. When steam was used for surfactant
deposition of films, the resulting devices showed highly consistent
channel-to-channel electrical properties. Since the carbon nanotube
films on these devices are relatively dense, as seen in
Figure~\ref{fig-steaming-network}, the network should be well above the
percolation threshold. As many carbon nanotube pathways connect across
the channel in parallel, small variations in the network morphology have
less of an impact on the overall channel behaviour
\autocite{Murugathas2018}. Figure~\ref{fig-cnt-histogram} and
Table~\ref{tbl-histogram-parameters} indicate that the range of bundle
sizes is relatively low in the steam-deposited films used in these
devices, meaning the electrical behaviour of dominant conduction
pathways is more spatially consistent. The repeatable subthreshold
regime behaviour between channels seen for steam-deposited devices is a
desirable attribute for reliable real-time multiplexed biosensing
\autocite{Kauffman2008,Heller2009,Gao2010}.

Channels from surfactant-deposited film devices usually showed a larger
on-off ratio and subthreshold slope than those from solvent-deposited
devices. Decreasing the ratio of gate-sensitive semiconducting carbon
nanotubes to metallic nanotubes tends to decrease the on-off ratio
\autocite{LeMieux2008,Rouhi2011,Zaumseil2015,Murugathas2018}.
Section~\ref{sec-pristine-raman} seems to indicate there are more
metallic nanotubes present in the surfactant-deposited films than in the
solvent-deposited films. However, percolating conduction pathways
dominate device behaviour and nanotube pathways across the channel with
a lower degree of bundling are less likely to contain metallic tubes
\autocite{Murugathas2018}. Therefore, the larger on-off ratio for
surfactant-deposited film devices is likely a result of their reduced
nanotube bundle size and reduced bundle size variation relative to other
films, as discussed in Section~\ref{sec-pristine-morphology}. The larger
subthreshold slope is likely due to increased mobility from a denser
nanotube network in surfactant-deposited films \autocite{Rouhi2011}, as
seen in Figure~\ref{fig-steaming-network}. A larger on-off ratio and
subthreshold slope results in a larger change in conductance in response
to changes in the transfer characteristic curve. Therefore, the larger
on-off ratio and subthreshold slope of steam-deposited devices is
desirable for improved sensor performance
\autocite{Kauffman2008,Heller2009,Gao2010}.

All channels characterised had a positive threshold voltage
(\(V_{th}\)). The threshold voltage was largest and most consistent for
steam-assisted surfactant-deposited films. The relatively high values of
\(V_{th}\) which correspond to channel measurements from steam-assisted
surfactant-deposited devices indicates increased \(p\)-doping of the
network relative to networks deposited via alternative processes
\autocite{Kang2005,Heller2008,Murugathas2018}. As seen from
Figure~\ref{fig-steamed-surfactant}-f and
Figure~\ref{fig-dg-peak-comparison}, the steam deposition process leads
to the presence of significant, persistent surfactant aggregates. It has
been previously established that residual surfactant can \(p\)-dope
carbon nanotubes, alongside enhancing \(p\)-doping from adsorped oxygen
and water \autocite{Kane2014,Nonoguchi2018,Christensen2022}. The
presence of residual surfactant may also explain the lowered
subthreshold slope, and therefore mobility, of the steam-deposited
devices relative to devices with films deposited in surfactant without
steam. The analysis by Kane \emph{et al.} shows that the thermal
annealling at 150°C used in this work to remove residual surfactant is
likely inadequate for this purpose. Oxidation of devices and vacuum
annealling at high temperatures (\textgreater{} 600°C) may be required
for effective desorption of the persistent surfactant
\autocite{Kane2014,Barnett2018}. Devices using films made using the
alternative two methods have the advantage of not requiring careful
treatment to remove surfactant.

\hypertarget{back-gated-cntfets}{%
\subsubsection*{Back-Gated CNTFETs}\label{back-gated-cntfets}}
\addcontentsline{toc}{subsubsection}{Back-Gated CNTFETs}

When characterising devices using the vapour delivery system chip
carrier, the setup arrangement meant all measurements were taken using a
backgate. Figure~\ref{fig-solvent-tx-bg}, Figure~\ref{fig-surf-tx-bg}
and Figure~\ref{fig-steam-tx-bg} show that backgated devices exhibit
\emph{p}-type transistor behaviour. Gate current leakage was negligible,
as shown by the dashed line staying close to zero across the sweep.
Significant hysteresis was observed. The hysteresis can be explained by
the presence of defects or charge traps within and on the surface of the
silicon dioxide and at interfaces between the silicon dioxide and carbon
nanotubes \autocite{Lee2007,Lee2012,Ha2014}. The hysteresis observed was
much greater than for the corresponding liquid-gated sweeps on the
right. The devices fabricated with a solvent-based deposition were
switched off at a lower voltage than the devices which used surfactant
during deposition.

\begin{figure}

\begin{minipage}[t]{0.47\linewidth}

{\centering 

\raisebox{-\height}{

\includegraphics{figures/ch6/Q35C3_nobuffer.png}

}

}

\subcaption{\label{fig-no-buffer}}
\end{minipage}%
%
\begin{minipage}[t]{0.05\linewidth}

{\centering 

~

}

\end{minipage}%
%
\begin{minipage}[t]{0.47\linewidth}

{\centering 

\raisebox{-\height}{

\includegraphics{figures/ch6/Q35C3_buffer.png}

}

}

\subcaption{\label{fig-50uL-buffer}}
\end{minipage}%

\caption{\label{fig-buffer-effect-on-backgate}Backgated transfer sweeps
were taken of an single unencapsulated device with a 300 nm SiO\(_2\)
layer and steam assisted surfactant-deposited carbon nanotube network
channels before and after being covered in \(50 \mu\)L 1XPBS
electrolyte.}

\end{figure}

Transfer measurements were taken to determine whether backgated
measurements could be taken of an unencapsulated device in the vapour
sensor chamber with 1XPBS covering the channels.
Figure~\ref{fig-buffer-effect-on-backgate} shows the behaviour of an
unencapsulated backgated device with a 300 nm SiO\(_2\) layer before and
after being covered by 50 \(\mu\)L of 1XPBS (phosphate buffered saline).
The on-off ratio and hysteresis of the channels increase significantly.
The presence of water increases hysteresis through introducing charge
traps at the silicon dioxide surface around the carbon nanotubes and at
the surface of the nanotubes themselves
\autocite{Kim2003,Lee2007,Franklin2012,Ha2014}. There is also a
significant increase in current leakage to the backgate for larger
applied voltages, despite the electrolyte having no visible physical
contact with the silicon backgate or copper plane. This leakage current
may simply be due to an increase in relative humidity around the device
due to the presence of water \autocite{Conseil2014}. As any variation in
threshold voltage due to hysteresis and significant leakage current are
undesirable for sensing procedures, this configuration was not used for
vapour sensing purposes.

\hypertarget{graphene-devices}{%
\subsection{Graphene Devices}\label{graphene-devices}}

\begin{figure}

\begin{minipage}[t]{0.47\linewidth}

{\centering 

\raisebox{-\height}{

\includegraphics{figures/ch6/JG098_pristine_TXLG01_5mVstep_220920_norinse.png}

}

}

\subcaption{\label{fig-graphene-transfer-1}}
\end{minipage}%
%
\begin{minipage}[t]{0.05\linewidth}

{\centering 

~

}

\end{minipage}%
%
\begin{minipage}[t]{0.47\linewidth}

{\centering 

\raisebox{-\height}{

\includegraphics{figures/ch6/JGQ00D6_pristine_TXLG01_5mVstep_220914_norinse.png}

}

}

\subcaption{\label{fig-graphene-transfer-2}}
\end{minipage}%
\newline
\begin{minipage}[t]{0.47\linewidth}

{\centering 

\raisebox{-\height}{

\includegraphics{figures/ch6/JG098_ch1_absolute_values_with_gate_current.png}

}

}

\subcaption{\label{fig-graphene-transfer-comparison-1}}
\end{minipage}%
%
\begin{minipage}[t]{0.05\linewidth}

{\centering 

~

}

\end{minipage}%
%
\begin{minipage}[t]{0.47\linewidth}

{\centering 

\raisebox{-\height}{

\includegraphics{figures/ch6/JGQ00D6_ch3_absolute_values_with_gate_current.png}

}

}

\subcaption{\label{fig-graphene-transfer-comparison-2}}
\end{minipage}%

\caption{\label{fig-pristine-graphene}These figures show liquid-gated
transfer characteristics of channels from two AZ\(^\circledR\) 1518
encapsulated graphene devices. The characteristics of working device
channels upon initial exposure to 1XPBS are shown in (a) and (b). The
transfer characteristics of channel 1 in (a) and channel 5 in (b) after
various degrees of exposure to 1XPBS are shown in (c) and (d)
respectively, with each transfer sweep numbered in the order the sweeps
were taken. The dashed lines correspond to measurements of gate leakage
current.}

\end{figure}

Graphene field-effect transistor devices were electrically characterised
in the manner described in \textbf{?@sec-electrical-characterisation}
and analysed using the Python code discussed in
Section~\ref{sec-field-effect-transistor-analysis}.

Figure~\ref{fig-pristine-graphene} shows the liquid-gated transfer
characteristics of two graphene devices. These devices were fabricated
prior to Jun 2021. Both devices exhibit the ambipolar characteristics
typical of liquid-gated graphene devices
\autocite{Heller2009a,Heller2010,Xia2010,Kireev2017}. As with the carbon
nanotube network devices, leakage current remained below \(\sim\) 1
\(\times\) \(10^{-7}\) V across both the forward and reverse sweep.
Hysteresis between the forward and reverse sweep is caused by trapping
of charge within and on the surface of the SiO\(_{2}\) dielectric
\autocite{Bartolomeo2011}. The major Dirac point for these devices is
slightly to the right of V\(_{\textrm{Dirac}} \approxeq\) 0 V, which
indicates \(p\)-doping of the channel. This slight \(p\)-doping is
likely a result of a adsorption of oxygen and water from the air and
residue resist from photolithography
\autocite{Cheng2011,Shin2012,Kireev2017}.

Some devices exhibited a double-minima feature, indicating the presence
of two Dirac points. This effect arises due to doping of graphene by the
metal contacts. In shorter length channels, metal doping affects the
entire channel length, leading to a consistent Fermi level and a single
Dirac point. However, for longer channel lengths like ours, the doping
effect from metal contact no longer reaches the entire channel, leading
to a difference in Fermi level between the graphene in the channel and
graphene under the metal contact. The difference in Fermi levels results
in the presence of a second Dirac point
\autocite{Bartolomeo2011,Feng2014,Peng2018}. The global minimum of the
transfer characteristic can be referred to as the `major' Dirac point.

Figure~\ref{fig-pristine-graphene} also shows the effect of 1XPBS on the
graphene channels. The channels were measured on exposure to 1XPBS,
after exposure to 1XPBS for one hour, and after the device surface was
rinsed and 1XPBS was replaced in the well one time, two times and three
times successively. A slight negative shift of the major Dirac point was
observed. This effect is possibly a result of gate bias stress, where
successive transfer sweeps introduce charge traps to the graphene layer
and alters the current level at a given gate voltage
\autocite{Bargaoui2018,Noyce2019}. Alternatively, Kireev \emph{et al.}
found that a series of liquid-gated sweeps also reduced the size of the
second Dirac point, and suggested that it indicated the gate current was
removing atmospheric contaminants from the graphene surface via current
annealing \autocite{Kireev2017}. This could be explained as the removal
of contaminants causing improved contact between the metal and graphene
surface, and thus increasing metal doping and consistency of the Fermi
level across the channel. If the contaminants removed are \(p\)-dopants,
then this effect could also explain the negative shift of the major
Dirac point.

\hypertarget{tbl-graphene-parameters}{}
\begin{longtable}[]{@{}
  >{\raggedright\arraybackslash}p{(\columnwidth - 6\tabcolsep) * \real{0.3288}}
  >{\centering\arraybackslash}p{(\columnwidth - 6\tabcolsep) * \real{0.2192}}
  >{\centering\arraybackslash}p{(\columnwidth - 6\tabcolsep) * \real{0.2603}}
  >{\centering\arraybackslash}p{(\columnwidth - 6\tabcolsep) * \real{0.1918}}@{}}
\caption{\label{tbl-graphene-parameters}Average on-off ratio and major
Dirac point voltage for AZ® 1518 encapsulated liquid-gated graphene
transistor channels at various stages of exposure to 1XPBS. Electrical
characteristics were taken of 6 channels total, three channels from each
of two devices.}\tabularnewline
\toprule\noalign{}
\begin{minipage}[b]{\linewidth}\raggedright
\end{minipage} & \begin{minipage}[b]{\linewidth}\centering
1XPBS: Initial
\end{minipage} & \begin{minipage}[b]{\linewidth}\centering
1XPBS: After 1 hr
\end{minipage} & \begin{minipage}[b]{\linewidth}\centering
1XPBS: Rinse
\end{minipage} \\
\midrule\noalign{}
\endfirsthead
\toprule\noalign{}
\begin{minipage}[b]{\linewidth}\raggedright
\end{minipage} & \begin{minipage}[b]{\linewidth}\centering
1XPBS: Initial
\end{minipage} & \begin{minipage}[b]{\linewidth}\centering
1XPBS: After 1 hr
\end{minipage} & \begin{minipage}[b]{\linewidth}\centering
1XPBS: Rinse
\end{minipage} \\
\midrule\noalign{}
\endhead
\bottomrule\noalign{}
\endlastfoot
On-Off Ratio (arb.) & 5.1 ± 0.3 & 5.0 ± 0.7 & 5.0 ± 0.6 \\
Dirac Point Voltage (V) & 0.28 ± 0.04 & 0.31 ± 0.03 & 0.28 ± 0.02 \\
\end{longtable}

Table~\ref{tbl-graphene-parameters} shows the on-off ratio and major
Dirac point voltage of the graphene devices. Apart from the
previously-mentioned slight negative shift of the major Dirac point,
these values were highly consistent before and after exposure to 1XPBS.

\hypertarget{sec-dummy-sensing}{%
\section{Aqueous Sensing of Phosphate Buffered Saline
Concentration}\label{sec-dummy-sensing}}

\hypertarget{sec-baseline-drift}{%
\subsection{Control Series and Baseline
Drift}\label{sec-baseline-drift}}

\hypertarget{tbl-threshold-voltages}{}
\begin{longtable}[]{@{}lrrrrrr@{}}
\caption{\label{tbl-threshold-voltages}The threshold voltages \(V_{th}\)
of each working channel of a steam-deposited device, and the difference
between each \(V_{th}\) and the mean device threshold voltage
\(V_{th, mean}\).\\
}\tabularnewline
\toprule\noalign{}
Channels & CH1 & CH2 & CH3 & CH5 & CH6 & CH7 \\
\midrule\noalign{}
\endfirsthead
\toprule\noalign{}
Channels & CH1 & CH2 & CH3 & CH5 & CH6 & CH7 \\
\midrule\noalign{}
\endhead
\bottomrule\noalign{}
\endlastfoot
Threshold voltage (mV) & 160 & 150 & 130 & 140 & 180 & 140 \\
Relative to mean (mV) & 10 & 0 & -20 & -10 & 30 & -10 \\
\end{longtable}

To verify the sensitivity of the fabricated field-effect transistors and
therefore verify their suitability for sensing, control measurements
replicating a typical sensing experiment were taken before
functionalising the channels of a carbon nanotube network device. The
first step to verifying device suitability was ensuring the device
showed no response to 1XPBS. This sequence is referred to in this thesis
as the `PBS control series'. The PDMS well contained 80 \(\mu\)L 1X PBS
at 0 s. The PBS control series ran over the first 1800 s, with
20\(\mu\)L phosphate buffer saline (1XPBS) additions at 100 s, 200 s and
300 s, and 20\(\mu\)L subtractions at 400 s, 500 s and 600 s. The device
was left untouched over the next 1200 s to allow the current level to
settle. The gate voltage was held at \(V_g\) = 0 V.

Figure~\ref{fig-transfer-sweeps} shows the transfer sweeps of the six
working channels of a steam-assisted surfactant-deposited carbon
nanotube field-effect transistor measured using the NI PXIe. The device
was fabricated on a substrate with a 300 nm SiO\(_{2}\) layer, the
carbon nanotube film was deposited using the steam-assisted surfactant
method and encapsulated with AZ\(^\circledR\) 1518 before measurement.
The central feature in the transfer characteristics of channels 1 and 7
are absolute-value measurements of negative current. These are
unphysical measurements due to equipment error, and can be considered as
regions where zero current passes through the channel. The threshold
voltages of the channels are shown in
Table~\ref{tbl-threshold-voltages}. Table~\ref{tbl-threshold-voltages}
also shows the difference between the threshold voltage of each channel
and the mean threshold voltage of the device. The mean threshold voltage
was \(V_{th}\) = \(150 \pm 20\) mV. As discussed previously, the
electrical characteristics are highly consistent between the channels
due to the film deposition method used.

\begin{figure}

\begin{minipage}[t]{0.45\linewidth}

{\centering 

\raisebox{-\height}{

\includegraphics{figures/ch6/NTQ31C1_pristine_TXLG02_230322.png}

}

}

\subcaption{\label{fig-transfer-sweeps}}
\end{minipage}%
%
\begin{minipage}[t]{0.05\linewidth}

{\centering 

~

}

\end{minipage}%
%
\begin{minipage}[t]{0.50\linewidth}

{\centering 

\raisebox{-\height}{

\includegraphics{figures/ch6/NTQ31C1_pristine_saltconc_sample_230324_control.png}

}

}

\subcaption{\label{fig-linear-fit}}
\end{minipage}%
\newline
\begin{minipage}[t]{0.28\linewidth}

{\centering 

~

}

\end{minipage}%
%
\begin{minipage}[t]{0.45\linewidth}

{\centering 

\raisebox{-\height}{

\includegraphics{figures/ch6/NTQ31C1_pristine_saltconc_sample_230324_linear_fit_exp.png}

}

}

\subcaption{\label{fig-exp-fit}}
\end{minipage}%
%
\begin{minipage}[t]{0.28\linewidth}

{\centering 

~

}

\end{minipage}%

\caption{\label{fig-salt-conc-control-series}The transfer
characteristics in (a) were taken of the steam-deposited carbon nanotube
field-effect transistor used here for an example of salt concentration
sensing. The absolute values of measurements are shown, so that negative
values resulting from measurement error can be visualised. Linear fits
to the PBS control series from each channel from 1200 s onwards are
shown in (b), while exponential fits to the PBS control series from
\(0-1200\) with the linear fit subtracted are shown in (c). No
significant response to PBS additions are seen at any of the addition
times from \(100-600\) s.}

\end{figure}

Figure~\ref{fig-linear-fit} shows the PBS control series corresponding
to each device channel alongside gate current. In both series, there is
no clear stepwise response to any addition or subtraction of 1XPBS. Gate
leakage current remains negligible across the entire control series,
with no change in response to 1XPBS additions. The current has a period
of short-term decay followed by much longer term baseline drift, similar
to observations by Lin \emph{et al.} and more recently Noyce \emph{et
al.} for parallel arrangements of single carbon nanotubes in air or
vacuum \autocite{Lin2006,Noyce2019}. This effect results from changes in
the occupancy of charge traps in and around the substrate and carbon
nanotubes. The magnitude of baseline drift is lower for our devices than
for those characterised by Noyce \emph{et al.}, which may be a result of
numerous device and setup differences which affect the presence of
charge traps. These differences include liquid-gating instead of
back-gated, the use of a network of carbon nanotubes instead of single
nanotubes, a different channel length, the use of a 300 nm instead of 90
nm SiO\(_2\) layer, and the use of an asymmetric, liquid-gated transfer
sweep over a shorter voltage range to characterise devices before each
control series was measured \autocite{Noyce2019}.

\hypertarget{tbl-linear-fits}{}
\begin{longtable}[]{@{}lllllll@{}}
\caption{\label{tbl-linear-fits}The coefficients of linear fits to the
PBS control series of each channel between \(1200-1800\) s, where
\(c_1\) is the gradient and \(c_2\) is the constant term.\\
}\tabularnewline
\toprule\noalign{}
Channels & CH1 & CH2 & CH3 & CH5 & CH6 & CH7 \\
\midrule\noalign{}
\endfirsthead
\toprule\noalign{}
Channels & CH1 & CH2 & CH3 & CH5 & CH6 & CH7 \\
\midrule\noalign{}
\endhead
\bottomrule\noalign{}
\endlastfoot
\(c_1\) (pA/s) & -5.1±0.2 & -7.2±0.1 & -6.5±0.1 & -5.0±0.1 & -7.6±0.1 &
-5.1±0.2 \\
\(c_2\) (\(\mu\)A) & 0.316 & 0.316 & 0.308 & 0.218 & 0.364 & 0.332 \\
\end{longtable}

\hypertarget{tbl-exp-fits}{}
\begin{longtable}[]{@{}lllll@{}}
\caption{\label{tbl-exp-fits}The coefficients of exponential fits to the
PBS control series of each channel between \(0-1200\) s, after the
linear fit has been subtracted, where \(I_0\) is the gradient and
\(\tau\) is the time constant.\\
}\tabularnewline
\toprule\noalign{}
Channels & CH2 & CH3 & CH6 & CH7 \\
\midrule\noalign{}
\endfirsthead
\toprule\noalign{}
Channels & CH2 & CH3 & CH6 & CH7 \\
\midrule\noalign{}
\endhead
\bottomrule\noalign{}
\endlastfoot
\(I_0\) (nA) & \(6.07\pm0.08\) & \(7.19\pm0.11\) & \(5.75\pm0.12\) &
\(9.68\pm0.41\) \\
\(\tau\) (s) & \(450\pm10\) & \(610\pm30\) & \(280\pm10\) &
\(350\pm30\) \\
\end{longtable}

As a first-order approximation to the longer time constant exponentials
discussed by Noyce \emph{et al.} \autocite{Noyce2019}, linear fits were
performed on each PBS control series from \(1200-1800\) s. These fits
are tangent to the curve of the sum of the larger time constant
exponentials, and are a close approximation to this curve when higher
order terms in the series expansion are approximately zero. This is only
the case when \(t\ll\tau_i\), where the time interval of interest \(t\)
is much shorter than the time constants of the larger time constant
exponentials, \(\tau_i\). These linear fits are shown by the dashed
yellow lines in Figure~\ref{fig-linear-fit}. The parameters from each
fit in Figure~\ref{fig-linear-fit} are shown in
Table~\ref{tbl-linear-fits}, where \(I = c_1t + c_2\). The fits for
channels 1, 5 and 7 are all in parallel within error. The gradient value
for each fit in Figure~\ref{fig-linear-fit} is consistent within a 2.6
pA/s range across all channels. The current data from channel 1 is
closely approximated by the linear across the entire control series. No
short-term decay is present for this channel, indicating the channel has
low net trapped charge. It is unclear why this short-term exponential
decay behaviour is only absent for channel 1.

The long-term linear fits were next subtracted from the raw control
series data. Figure~\ref{fig-exp-fit} shows exponential fits to the
remaining curve from \(0-1800\) s, which was successful for all channels
except channels 1 and 5. The parameters from each fit are shown in
Table~\ref{tbl-exp-fits}, where \(I = I_0\exp(-t/\tau)\). Any constant
term \(I_C\) resulting from the fit was negligible and so could be
neglected. The exponential fits had characteristic time constants
\(\tau\) ranging between \(280 - 610\) s. Note that the value of
peak-to-peak noise is above 5\% of the initial current value for all
channels. This result indicates that 3 time constants is a sufficient
length of time for this short-term baseline drift to decay almost
completely for each channels. At most, \(1830\pm150\) s is required to
minimise the drift present when sensing is performed, which is fulfilled
by the chosen length of the control series.

From this analysis it appears that the baseline drift for the
liquid-gated carbon nanotube devices can generally be approximated as a
combination of a exponential and linear term. The lack of response to
1XPBS at any of the six PBS addition and removal times gives us
confidence that this is a stable baseline which can be used for reliable
chemical sensing. Furthermore, after \(\sim 1800\) s the baseline drift
can be reasonably approximated as linear, with a small gradient of less
than -10 pA/s. The approximately linear current change means that it
becomes easier to distinguish responses due to analyte addition. It can
therefore be concluded that the 1800 s length of the PBS control series
is appropriate for minimising baseline drift for more reliable sensing.

\hypertarget{sec-salt-conc-series}{%
\subsection{Sensing Series}\label{sec-salt-conc-series}}

A salt concentration sensing series were performed from 1800 s onwards,
directly after the PBS control series. The responses to successive
dilutions of the liquid-gate electrolyte were recorded to confirm the
fabricated devices were sensitive to small environmental changes in
their pristine state, to check for spurious signals, and to ensure gate
current leakage or other confounding factors were not contributing to
sensing responses. The PDMS well contained 80 \(\mu\)L 1X PBS at 1800 s.
During the series, successive additions of deionised water were made to
reduce the concentration of PBS in the well. An initial 1X PBS addition
was performed at 2100s, to confirm no changes occurred during the PBS
control series that would interfere with sensing. All additions to the
well in the sensing series and resulting changes to the PBS
concentration in the well are shown in Table~\ref{tbl-salt-conc-series}.

\hypertarget{tbl-salt-conc-series}{}
\begin{longtable}[t]{lcccccc}
\caption{\label{tbl-salt-conc-series}This table shows the times at which 20 µL additions were made to the
PDMS well, with 300 s between each addition. The concentration in the
well after each addition and the change in concentration after each
addition are also shown. The well contained 80 µL of 1X PBS at 1800 s. }\tabularnewline

\toprule
\multicolumn{1}{c}{ } & \multicolumn{1}{c}{1X PBS Addition} & \multicolumn{5}{c}{DI Water Additions} \\
\cmidrule(l{3pt}r{3pt}){2-2} \cmidrule(l{3pt}r{3pt}){3-7}
 &  &  &  &  &  & \\
\midrule
Time (s) & 2100 & 2400 & 2700 & 3000 & 3300 & 3600\\
Final PBS volume (µL) & 100 & 120 & 140 & 160 & 180 & 200\\
Final PBS concentration & 1X & 0.83X & 0.71X & 0.63X & 0.56X & 0.50X\\
Δ PBS concentration & 0 & -0.17X & -0.12X & -0.09X & -0.07X & -0.06X\\
\bottomrule
\end{longtable}

\begin{figure}

\begin{minipage}[t]{0.50\linewidth}

{\centering 

\raisebox{-\height}{

\includegraphics{figures/ch6/NTQ31C1_pristine_saltconc_sample_230324.png}

}

}

\subcaption{\label{fig-salt-conc-no-norm}}
\end{minipage}%
%
\begin{minipage}[t]{0.50\linewidth}

{\centering 

\raisebox{-\height}{

\includegraphics{figures/ch6/NTQ31C1_pristine_saltconc_sample_230324_detrend_trunc_arrows_normalised.png}

}

}

\subcaption{\label{fig-salt-conc-detrend}}
\end{minipage}%
\newline
\begin{minipage}[t]{0.50\linewidth}

{\centering 

\raisebox{-\height}{

\includegraphics{figures/ch6/NTQ31C1_pristine_saltconc_sample_230324_filtered_detrend_trunc_arrows_normalised.png}

}

}

\subcaption{\label{fig-salt-conc-detrend-filter}}
\end{minipage}%
%
\begin{minipage}[t]{0.50\linewidth}

{\centering 

\raisebox{-\height}{

\includegraphics{figures/ch6/NTQ31C1_mean_simple_difference_before_and_after_step_filtered_concentrations.png}

}

}

\subcaption{\label{fig-salt-conc-signal}}
\end{minipage}%
\newline
\begin{minipage}[t]{0.25\linewidth}

{\centering 

~

}

\end{minipage}%
%
\begin{minipage}[t]{0.50\linewidth}

{\centering 

\raisebox{-\height}{

\includegraphics{figures/ch6/salt_conc_box_plot.png}

}

}

\subcaption{\label{fig-salt-conc-box-plot}}
\end{minipage}%
%
\begin{minipage}[t]{0.25\linewidth}

{\centering 

~

}

\end{minipage}%

\caption{\label{fig-salt-conc-sensing}Various visualisations of a
multiplexed salt concentration sensing series taken from a single
device. The source-drain voltage \(V_{ds}\) was 100 mV, and gate voltage
\(V_g\) was 0 V. In (a), the raw current measurements for each channel
are shown alongside gate current. The same measurements after despiking,
removal of baseline drift and normalisation to initial current are shown
in (b), (c) shows the data in (b) after being processed with a moving
median filter, and (d) shows the signal changes in (c). The signal data
in (d) is shown in box plot format in (e) alongside a fit to the median
change in signal for each addition. The R squared value for the fit was
0.86.}

\end{figure}

Figure~\ref{fig-salt-conc-no-norm} shows a multiplexed salt
concentration sensing series from the channels of a single
AZ\(^\circledR\) 1518 encapsulated device, measured with the NI-PXIe.
The gate voltage used was 0 V, which meant current measurements were
well above the magnitude of the subthreshold device current. Gate
current measurements did not exceed 1 nA for the SU8 encapsulated
devices, and did not exceed 10 nA for the AZ\(^\circledR\) 1518 devices.
At each of the deionised water addition times, the current traces for at
least two out of six channels showed a sharp, transient increase in
current followed by a return to an increased baseline. It is well
established that changing the salt concentration of the liquid gate has
an electrostatic gating effect on the carbon nanotubes or graphene, and
changes the transfer characteristics of the channel. This shift in
transfer characteristic leads to a real-time signal response to each
addition \autocite{Heller2009,Heller2010,Kireev2017}.

Using the data in Table~\ref{tbl-linear-fits}, the linear term
approximating baseline drift (\(c_1t\)) for each channel can be
subtracted from the data in Figure~\ref{fig-salt-conc-no-norm} to
account for the downward drift. The mean current level just before 1800
s then becomes roughly constant. Next, each channel is normalised
relative to their initial mean current level \(I_{0}\). Artifacts
resulting from PXIe-2737 module lag, single datapoints which fall well
below the current level of the immediately preceding and succeeding
datapoints, are also removed. This `despike' process uses an
interquartile range filter, which is described in
Section~\ref{sec-python-analysis}. The resulting dataset is shown in
Figure~\ref{fig-salt-conc-detrend}. This figure shows that the
signal-to-noise ratio remains roughly similar across all channels of the
device. However, the behaviour of the initial transient increase with
each addition is highly variable across channels and between additions
for a single channel.

As measurement of the highly variable initial transient is not useful
for robust sensing purposes, a moving median filter was applied, with
the implementation of this filter discussed in
Section~\ref{sec-python-analysis}. The filtered data is shown in
Figure~\ref{fig-salt-conc-detrend-filter}. Noise and initial transients
are removed completely, while the clearly defined step to a new current
baseline is retained. Using the realtime data in
Figure~\ref{fig-salt-conc-detrend-filter}, a plot of signal against
addition can be created using the method described in
Section~\ref{sec-python-analysis}, shown in
Figure~\ref{fig-salt-conc-signal}. This presentation of the data allows
us to see the increase at each step relative to \(I_{0}\).

Intriguingly, even though the largest change in PBS concentration
occurred at the first deionised water addition (see
Table~\ref{tbl-salt-conc-series}), there was very little signal change
across all channels, while a relatively large change occurred at the
second addition. The logarithm of final salt concentration has
previously been shown to be proportional to conductance change in the
linear on-regime \autocite{Heller2010}.
Figure~\ref{fig-salt-conc-box-plot} shows the signal change presented in
terms of this logarithmic relationship. The median values of the first
two additions do not line up well with the overall logarithmic trend;
insufficient mixing in the tightly enclosed PDMS well environment for
the first few additions may be responsible for this result. Subsequent
additions may improve mixing in the well, leading to the change in
concentration at the surface of the channel being more representative of
the overall concentration in the well.

\begin{figure}

\begin{minipage}[t]{0.50\linewidth}

{\centering 

\raisebox{-\height}{

\includegraphics{figures/ch6/NTQ31C1_pristine_saltconc_sample_230324_detrend_trunc_arrows_normalised_2.png}

}

}

\subcaption{\label{fig-salt-conc-detrend-2}}
\end{minipage}%
%
\begin{minipage}[t]{0.50\linewidth}

{\centering 

\raisebox{-\height}{

\includegraphics{figures/ch6/NTQ31C1_pristine_saltconc_sample_230324_filtered_detrend_trunc_arrows_normalised_2.png}

}

}

\subcaption{\label{fig-salt-conc-detrend-filter-2}}
\end{minipage}%
\newline
\begin{minipage}[t]{0.28\linewidth}

{\centering 

~

}

\end{minipage}%
%
\begin{minipage}[t]{0.45\linewidth}

{\centering 

\raisebox{-\height}{

\includegraphics{figures/ch6/NTQ31C1_pristine_saltconc_sample_230324_single_step.png}

}

}

\subcaption{\label{fig-salt-conc-detrend-filter-single-step}}
\end{minipage}%
%
\begin{minipage}[t]{0.28\linewidth}

{\centering 

~

}

\end{minipage}%

\caption{\label{fig-salt-conc-sensing-2}The processed data shown in
Figure~\ref{fig-salt-conc-detrend} and
Figure~\ref{fig-salt-conc-detrend-filter} is normalised to \(I_{0}\),
but an alternative normalisation can more effectively filter out
remaining drift present. This normalisation presents data relative to
both \(I_{0}\) and the final current reading \(I_{f}\) using the formula
\((I - I_{0})/(I_{f} - I_{0})\). Using this normalisation, the data in
Figure~\ref{fig-salt-conc-detrend} and
Figure~\ref{fig-salt-conc-detrend-filter} can be displayed instead as
(a) and (b) respectively. (c) shows a magnified version of the step at
addition 2 in (a).}

\end{figure}

In Figure~\ref{fig-salt-conc-detrend} and
Figure~\ref{fig-salt-conc-detrend-filter}, from around the second
deionised water addition onwards, the drift behaviours of individual
channels begin to significantly diverge. This deviation from the
baseline drift subtracted from the raw data occurs either because the
linear fit is only a first-order approximation which weakens with time,
or because the additions themselves affect the drift behaviour.
Displaying the data as discrete signal changes, as in
Figure~\ref{fig-salt-conc-signal}, is one way of excluding these
deviations (see Section~\ref{sec-python-analysis}). An alternative way
of presenting the signal changes, by normalising relative to both
\(I_{0}\) and the final current reading with the formula
\((I - I_{0})/(I_{f} - I_{0})\), is shown in
Figure~\ref{fig-salt-conc-sensing-2}. This approach is useful for
filtering out remaining unaccounted-for drift behaviour in order to
compare the short-term transient responses to additions across the
device channels. Furthermore, it lets us better understand how the
short-term transient responses affect the longer-term step responses
discussed earlier.

Figure~\ref{fig-salt-conc-detrend-2} and
Figure~\ref{fig-salt-conc-detrend-filter-single-step} show that the
transient responses to DI water additions vary significantly across the
surface of the device. For example,
Figure~\ref{fig-salt-conc-detrend-filter-single-step} shows that in
response to the second DI water addition, channel 7 gives a large
initial transient response about twice the size of the step increase
between 2600 and 2800 s. Meanwhile, channels 1 and 2 show no transient
response above the step increase.
Figure~\ref{fig-salt-conc-detrend-filter-single-step} indicates
transient size is based on location across the device, with neighbouring
channels showing the most similar behaviour. This spatially-dependent
behaviour may indicate transient responses are determined by the
location of the channel relative to either the location of water
additions or the slightly-variable location of the liquid gate. Larger
and longer-lasting transient responses are not entirely removed by the
moving median filter, as shown by comparing
Figure~\ref{fig-salt-conc-detrend-2} to
Figure~\ref{fig-salt-conc-detrend-filter}, and so careful placement of
additions is important when sensing to minimise this effect. However,
even the longest-lasting transients appear to decay to zero within about
200 s, demonstrating that a 200 s spacing between additions at minimum
is necessary for reliable real-time liquid-gated sensing using this
setup.

\hypertarget{signal-to-noise-ratio}{%
\subsubsection*{Signal-to-Noise Ratio}\label{signal-to-noise-ratio}}
\addcontentsline{toc}{subsubsection}{Signal-to-Noise Ratio}

\begin{figure}

\begin{minipage}[t]{0.45\linewidth}

{\centering 

\raisebox{-\height}{

\includegraphics{figures/ch6/Q2C10ch8custom.png}

}

}

\subcaption{\label{fig-transfer-sweep-2}}
\end{minipage}%
%
\begin{minipage}[t]{0.05\linewidth}

{\centering 

~

}

\end{minipage}%
%
\begin{minipage}[t]{0.50\linewidth}

{\centering 

\raisebox{-\height}{

\includegraphics{figures/ch6/saltconc_initial_additions.png}

}

}

\subcaption{\label{fig-salt-conc-SNR}}
\end{minipage}%

\caption{\label{fig-salt-conc-SNR}The transfer characteristics of a
single steam-deposited carbon nanotube field-effect transistor channel
are shown in (a). V\(_{gap}\) is the gate voltage corresponding to the
center of the transistor bandgap, found at the minimum of the
characteristic curve. The signal-to-noise ratio of the channel response
to a deionised water addition after a suitable control series is shown
in (b). The blue current trace in (b) was performed gating the device
150 mV away from V\(_{gap}\), while the red current was performed gating
the device 200 mV away from V\(_{gap}\).}

\end{figure}

To understand the effect of gate voltages on signal-to-noise ratio, two
PBS control and salt concentration sensing series were performed with
the same channel at different gate voltages. The transfer
characteristics of this channel are shown in
Figure~\ref{fig-transfer-sweep-2}, with coloured dashed lines marking
the voltages used for gating the transistor during each sesning series.
Figure~\ref{fig-salt-conc-SNR} shows the initial PBS and DI water
additions made after 1800 s. Previous work on the signal-to-noise ratio
for liquid-gated, encapsulated carbon nanotube devices suggests that
gating devices close to \(V_{gap}\) should give the largest
signal-to-noise ratio for salt concentration additions
\autocite{Heller2009}. However, this was not what was observed for our
carbon nanotube field-effect transitor, as
Figure~\ref{fig-salt-conc-SNR} shows improved signal-to-noise ratio,
i.e.~the signal step can be more clearly distinguished, when gated at a
voltage further removed from \(V_{gap}\). This discrepancy could be a
result of the use of a network of carbon nanotubes rather than a single
nanotube; gating may have less of an impact on noise when a network
morphology is used. Alternatively, it could be a result of a lack of
mixing in our static well setup leading to inconsistent signal sizes
with concentration change. Heller \emph{et al.} used a flow cell during
their signal-to-ratio work \autocite{Heller2009}. By using a flow cell
with our devices, it would be possible to confirm whether this is the
case, and this might also help us reduce the size of unwanted transient
responses resulting from drop-wise additions.

\hypertarget{sec-pristine-EtHex}{%
\section{Vapour Sensing with Ethyl Hexanoate}\label{sec-pristine-EtHex}}

\hypertarget{sec-vapour-drift}{%
\subsection{Baseline Drift}\label{sec-vapour-drift}}

When sensing vapour in the vapour delivery system, devices have no
liquid gate and are instead backgated when taking measurements.
Therefore, the baseline drift of devices characterised in this manner
should be considered separately to those characterised in an
liquid-gated environment. Device baseline drift of a backgated device in
the vapour sensing chamber is therefore examined here in more detail. A
AZ\(^\circledR\) 1518 encapsulated carbon nanotube network device was
used in this discussion. The device was fabricated on a substrate with a
300 nm SiO\(_2\) layer, and the carbon nanotube film was deposited using
the steam-assisted surfactant method. Before measurements were taken,
the vapour system was purged of vapour, the total dilution flow into the
chamber was set at 200 sccm as read by the Tylan mass flow controller
and flow to the PID was set to 150 sccm on the flowmeter. The transfer
sweep of a channel on this device (channel 6) is shown in
Figure~\ref{fig-backgate-transfer}, measured using the B1500A
semiconductor device analyser.

\begin{figure}

{\centering \includegraphics[width=0.35\textwidth,height=\textheight]{figures/ch6/Q2C6_backgate_characterisation.png}

}

\caption{\label{fig-backgate-transfer}Transfer sweep of a
steam-deposited carbon nanotube network field-effect transistor,
backgated in the vapour delivery system device chamber.}

\end{figure}

\begin{figure}

\begin{minipage}[t]{0.15\linewidth}

{\centering 

~

}

\end{minipage}%
%
\begin{minipage}[t]{0.70\linewidth}

{\centering 

\includegraphics{figures/ch6/Q2C6_fitted_curves_edited.png} {}

}

\end{minipage}%
%
\begin{minipage}[t]{0.15\linewidth}

{\centering 

~

}

\end{minipage}%
\newline
\begin{minipage}[t]{0.15\linewidth}

{\centering 

~

}

\end{minipage}%
%
\begin{minipage}[t]{0.70\linewidth}

{\centering 

\includegraphics{figures/ch6/Q2C6_fitted_curves_exp_edited.png} {}

}

\end{minipage}%
%
\begin{minipage}[t]{0.15\linewidth}

{\centering 

~

}

\end{minipage}%

\caption{\label{fig-bg-baseline-drift}The source-drain and gate current
measured for a backgated device channel across 3600 s, where V\(_{ds}\)
= 100 mV and V\(_g\) = 0 V is shown in (a). A linear fit to the data
from 2400 s onwards has been indicated on (a) with a black dashed line.
The linear fit has then been subtracted from (a) to give the dataset
shown in (b). An exponential fit to the dataset in (b) is also shown in
black.}

\end{figure}

Figure~\ref{fig-bg-baseline-drift} (a) shows 3600 s of baseline drift
from the same channel when the device was backgated at V\(_g\) = 0 V and
a source-drain voltage of V\(_{ds}\) = 100 mV was placed across the
channel. During this period of time, a 200 sccm nitrogen flow was placed
through the device chamber with the dilution mass flow controller. Gate
leakage current remains negligible across the entire control series. As
seen for the liquid-gated device in Section~\ref{sec-baseline-drift},
there is a period of rapidly-disappearing exponential decay followed by
a period of stable, approximately linear baseline drift. The baseline
drift observed here appears to be significantly lower than that seen by
Noyce \emph{et al.} \autocite{Noyce2019}. This observation suggests that
the higher magnitude of drift observed by Noyce \emph{et al.} is not a
result of backgating in air, as previously suggested, but instead due to
the use of a significantly different fabrication process for their
devices.

A linear least-squares fit was performed on the samples taken between
2400 s \(-\) 3600 s, and the fit obtained had an R-squared value of
0.998. The constants obtained for the linear fit, where
\(I = c_1t + c_2\), were \(c_1 = -17.31\pm0.05\) pAs\(^{-1}\) and
\(c_2 = 0.779 \mu\)A. Both linear and constant terms are higher than
that of the average liquid-gated device drift. The linear fit was then
subtracted from the raw data, and an exponential least-squares fit was
performed on the remaining dataset. Figure~\ref{fig-bg-baseline-drift}
(b) shows the exponential fit to this remaining dataset from 0 s \(-\)
3600 s. The constants obtained for the exponential fit
\(I = I_0exp(-t/\tau)\) were \(I_0 = 7.20 \pm 0.05\) nA and
\(\tau = 730 \pm 10\) s. The exponential term is similar in size to
those found for the channels of the liquid-gated device, which may
indicate the magnitude of this decay behaviour is independent of the
type of transistor gating. Three time constants equates to
\(2190 \pm 30\), indicating the length of the control sequence could be
safely reduced to 2400 s without the short-term exponential drift being
present during sensing.

This analysis indicates that the baseline drift for the backgated carbon
nanotube under nitrogen flow can be approximated as a combination of a
exponential, linear and constant term. Furthermore, while only measured
here for a single channel, it appears likely that we can expect
backgated baseline drift behaviour to be similar to the multiplexed
liquid-gated drifts observed in Section~\ref{sec-baseline-drift}, except
possibly with a longer time constant for the exponential term. It seems
that the baseline drift behaviour in these devices is primarily due to
the general nature of the carbon nanotube network. It appears possible
that the type of gating used for device characterisation may affect the
rate at which the exponential term decays. However, further
experimentation may be needed to confirm this relationship, which is
outside the scope of this thesis.

\hypertarget{sec-EtHex-series}{%
\subsection{Sensing Series}\label{sec-EtHex-series}}

Directly after the 3600 s control series, the device was exposed to four
intervals of ethyl hexanoate vapour flow from the carrier line. 5 mL of
ethyl hexanoate was placed into the analyte bottle on the carrier line
before testing for the sensing series. The same settings for the vapour
delivery system were kept from Section~\ref{sec-vapour-drift}. A total
flow of 200 sccm between the two mass flow controllers was kept through
the chamber at all times. During each interval 150 sccm flow was placed
through the carrier line. Apart for the duration of these intervals,
flow through the carrier line was kept at zero. The intervals were of
varying lengths to see how the carbon nanotube device responded to
various concentrations of vapour in the chamber as recorded by the PID.
A 1200 s recovery period was placed between each carrier flow interval,
where 200 sccm flow was placed into the chamber from the dilution line.
A separate test was also performed in an identical manner, except no
ethyl hexanoate was placed into the analyte bottle. The chamber
temperature was 22°C \(\pm\) 2°C for all measurements.

\begin{figure}

\begin{minipage}[t]{0.15\linewidth}

{\centering 

~

}

\end{minipage}%
%
\begin{minipage}[t]{0.70\linewidth}

{\centering 

\includegraphics{figures/ch6/Q2C6_Q3C2_detrend_trunc_arrows_normalised_edited.png}
{}

}

\end{minipage}%
%
\begin{minipage}[t]{0.15\linewidth}

{\centering 

~

}

\end{minipage}%
\newline
\begin{minipage}[t]{0.15\linewidth}

{\centering 

~

}

\end{minipage}%
%
\begin{minipage}[t]{0.70\linewidth}

{\centering 

\includegraphics{figures/ch6/Q2C6_Q3C2_mean_simple_difference_before_and_after_edited.png}
{}

}

\end{minipage}%
%
\begin{minipage}[t]{0.15\linewidth}

{\centering 

~

}

\end{minipage}%

\caption{\label{fig-EtHex-sampling}Device channel responses to intervals
of flow from the carrier line into the vapour delivery system chamber.
Intervals begin at 3600 s, 4850 s, 6150 s and 7500 s. The length of each
interval is indicated above the corresponding normalised current
responses in (a) for both ethyl hexanoate (EtHex) and for no analyte
present in the analyte bottle. The signal changes corresponding to the
current responses to each interval for both ethyl hexanoate and for no
analyte present are shown in (b).}

\end{figure}

Figure~\ref{fig-EtHex-sampling} shows the result of these interval
tests, both with and without ethyl hexanoate placed in the analyte
bottle. The data presented in Figure~\ref{fig-EtHex-sampling} (a) has
been normalised, despiked, filtered and corrected for baseline drift in
the manner described in both Section~\ref{sec-python-analysis} and
Section~\ref{sec-salt-conc-series}. Each interval of exposure to carrier
line flow corresponds to a current increase. These increases have been
labelled with the length of the corresponding interval used. When ethyl
hexanoate is present in the analyte bottle on the carrier line, the
response to each exposure interval is considerably larger than the
response when no analyte is present. It should be noted that some
response to carrier line flow is observed even when the analyte bottle
is empty. This is most likely to be the result of low levels of residual
analyte in the carrier line being pumped into the chamber. It appears
very low levels of analyte persist in the line even after purging the
system lines with a roughing pump. Concentration measurements taken
using the photoionisation detector, shown in
Figure~\ref{fig-EtHex-sampling-PID}, also indicate some low-level,
residual vapour reaches the chamber during each interval even when the
analyte bottle is left empty.

\begin{figure}

{\centering \includegraphics[width=0.7\textwidth,height=\textheight]{figures/ch6/input_time_comparison.png}

}

\caption{\label{fig-EtHex-sampling-PID}Nominal concentration
measurements by the photoionisation detector taken from the device
chamber during device current sampling from 3600 s onwards, both with
and without ethyl hexanoate (EtHex) in the analyte bottle. The maximum
nominal concentration reached during each carrier flow interval is
indicated above each peak.}

\end{figure}

The signal response seen corresponds to a change in conductance of the
exposed carbon nanotubes within the device channel. These conductance
changes occur due to the molecular adsorption of ethyl hexanoate vapour
onto the external and internal surfaces of the nanotubes. The vapour can
dopes the semiconducting carbon nanotubes in the channel, causing a
shift in the channel threshold voltage, and can cause carrier scattering
when adsorped onto the metallic nanotubes present. Binding of analyte to
a gas sensing material can be reversible or irreversible. In general,
adsorption onto carbon nanotube sensors is irreversible. This
irreversibility means after a response to analyte, readings from the
sensor will not return to the original baseline within the same
timescale as the sensing response, even after stopping analyte flow to
the chamber \autocite{Agnihotri2005,Lee2005}. From
Figure~\ref{fig-EtHex-sampling}, it is clear that the carbon nanotube
sensor configuration used here is primarily irreversible, where the
current level does not return to baseline within a period of 1200 s
after analyte exposure.

\begin{figure}

{\centering \includegraphics[width=0.55\textwidth,height=\textheight]{figures/ch6/EtHex-ratio-comparison.png}

}

\caption{\label{fig-EtHex-ratio-comparison}Device response against
maximum concentration measurement corresponding to each interval of
carrier flow. The experimental results have been fitted with two
adsorption isotherm models. The chamber temperature was 22°C ± 2°C
across both datasets.}

\end{figure}

Assuming that the signal response is directly proportional to the degree
of surface coverage by adsorbed analyte on the carbon nanotube network
\autocite{Lee2005}, it should be possible to model the relationship
between signal response and concentration in the device chamber with an
adsorption isotherm \autocite{Agnihotri2005}.
Figure~\ref{fig-EtHex-ratio-comparison} shows the maximum nominal
concentration measured for every peak shown in
Figure~\ref{fig-EtHex-sampling-PID} plotted against the corresponding
signal responses shown in Figure~\ref{fig-EtHex-sampling} (b). The
Freundlich adsorption isotherm (Equation~\ref{eq-freundlich}) models
adsorption onto a heterogeneous surface. \(K_F\) is the adsorption
capacity and \(1/n\) is adsorption intensity. \(1/n\) can be used to
understand the heterogeneity of adsorbate sites
\autocite{Ayawei2017,Sabzehmeidani2021}. The vapour response factor is
denoted as \(k_{RF}\), which is equal to 1.6 for a 10.6 eV
photoionisation detector (PID). As the PID has been run uncalibrated, a
factor \(k_{D}\) has been included to account for linear span drift. As
span drift due to window contamination can cause concentration readings
to be reduced up to 30\% after six months of PID operation, it is
expected that \(k_{D}\) falls within the range of 0.2 \(-\) 1
\autocite{PIDmanual,Ionscience}.
\begin{equation}\protect\hypertarget{eq-freundlich}{}{
q_e = K_F(k_Dk_{RF}C_e)^{1/n}
}\label{eq-freundlich}\end{equation}

The best-fit Freundlich isotherm is shown alongside the experimental
data in Figure~\ref{fig-EtHex-ratio-comparison}, fitted using linear
least-squares methods with an R-squared value of 0.921. The fitted
isotherm had a value for \(1/n\) of \(1/n = 0.5\pm0.2\). The isotherm
has previously been used to model adsorption of volatile organic
molecules onto single-walled carbon nanotubes. A value for \(1/n\) above
\(\sim\) 0.2 indicates the carbon nanotube network morphology of a
sensor has a relatively high maximum adsorption capacity. This results
from using a nanotube network morphology with a relatively high external
surface area \autocite{Agnihotri2005}. The relatively high value of
\(1/n\) here indicates that the morphology used has a high external
surface area and is therefore highly sensitive.

\hypertarget{conclusion}{%
\section{Conclusion}\label{conclusion}}

To ensure fabricated transistors were suitable for biosensing purposes,
the morphology and electrical properties of the pristine carbon nanotube
and graphene transistors were investigated.

The morphology of the carbon nanotube networks were found to have a
significant impact on the electrical characteristics of the devices,
which was determined through comparison of the skew-normal height
profile of the carbon nanotube network and the key electrical parameters
of a range of carbon nanotube devices. When networks were highly bundled
(\(>90\)\%), there was a large range of carbon nanotube bundle diameters
present in the network. This large variation in the size of conducting
pathways resulted in a wide range of on-off ratios and threshold
voltages for the liquid-gated devices created using these carbon
nanotube films. In contrast, devices using films fabricated with a
relatively low percentage of bundling (\(<75\) \%) showed highly
consistent on-off ratios and threshold voltages, along with low
hysteresis, due to the relatively consistent bundle diameters and high
density of these networks. These low-bundling networks were found to
have a mean bundle distribution height of \(3.3 \pm 1.0\) nm. When
performing multiplexed sensing, consistent channel behaviour is highly
desirable since comparing sensing behaviour between channels is more
straightforward.

However, atomic force microscope imaging and Raman spectroscopy also
indicated that less bundled networks had the most surface contamination
present. Aggregated surfactant present on the surface had a height of
more than 4 nm, and introduced significant defects to the carbon
nanotube network. The introduction of \(p\)-dopants to the carbon
nanotubes by surfactant appears to have significantly increased the
threshold voltage of steam-assisted surfactant-deposited network devices
relative to steam-free surfactant-deposited network devices. Since the
presence of surfactant could negatively impact biosensing, techniques to
remove contaminants should be explored in more detail. Oxidation and
thermal annealing of carbon nanotube films at high temperatures could be
used to resolve this issue, and this is discussed further in
\textbf{?@sec-future-work}. The presence of electrolyte on the surface
of a backgated transistor for use in vapour sensing was also found to
significantly adversely affect its electrical characteristics.

Constant voltage real-time measurements of the carbon nanotube
field-effect transistor devices had a characteristic drift that could be
modelled using a exponential and linear term. This was true for both
liquid-gated and back-gated devices. The linear term of liquid-gated
baseline drift had a reasonably consistent gradient between device
channels, with a mean value of \(-6.1 \pm 1.2\) pAs\^{}\{-1\},
indicating that similar drift behaviour should be reproducible between
devices fabricated in the same manner. The time constant of the
exponential term for liquid-gated drift ranged from
\(\tau = 280 \pm 10\) s to \(\tau = 610 \pm 30\) s for the device
characterised. The linear term of back-gated baseline drift in nitrogen
found for a single channel was \(-17.31\pm0.05\) pAs\^{}\{-1\}, higher
than all measurements for liquid-gated linear drift. The exponential
term found was \(\tau = 730 \pm 10\) s, higher than that of the
liquid-gated channels. These results indicate a control series length of
1800 s is appropriate for minimising the effects of baseline drift on
liquid-gated sensing, while a control series length of 2400 s is
sufficient in the case of a backgated device under nitrogen flow.

A PBS dilution sensing series indicated that the carbon nanotube
transistor devices were highly sensitive to environmental changes in an
aqueous environment. Successive additions of deionised water to the
1XPBS present in the well gave signal responses of up to 2.5\% above the
control response. The signal response was found to be proportional to
the logarithm of concentration, giving a fit to the median response
sizes with an \(R^2 = 0.86\). Deviations from this trend can possibly be
explained by the enclosed sensing environment preventing sufficient
mixing of electrolyte concentrations within the PDMS well, which could
possibly be addressed by using a flow cell for sensing work. It was also
seen that the signal size relative to baseline drift was highly
consistent between channels. This is a promising result when it comes to
ensuring consistent multiplexing, but it cannot be guaranteed that this
behaviour carries over to sensing with biofunctionalised devices.

Furthermore, vapour sensing with ethyl hexanoate showed that the
transistor devices were responsive to vapour in a manner that
corresponded to the responses from a photoionisation-based reference
sensor. Exposure of the carbon nanotube device to parts-per-million
concentrations of ethyl hexanoate resulted in irreversible signal
responses of up to 0.8\% above the corresponding control measurement.
Each signal response coincided with a local maximum in chamber vapour
concentration as measured by the photoionisation detector. It was found
that the relationship between signal response to vapour and their
corresponding maximum concentration value could be modelled using a
Freundlich adsorption isotherm, where the R-squared value of the model
with the dataset was 0.921. An adsorption intensity of
\(1/n = 0.5\pm0.2\) was found for the sensor response to ethyl
hexanoate, indicating that the carbon nanotube morphology used is
particularly sensitive. Again, this is promising for the sensitivity of
these devices when biofunctionalised, but poses potential issues in
regard to the selectivity of these devices.

Graphene field-effect transistor devices were often found to possess a
double-minima feature, which appears to be the result of a lack of
doping from the metal contacts in the center of the device channels.
These double Dirac points are unlikely to have an significant effect on
the sensing behaviour of graphene devices. The graphene device
characteristics were found to be consistent after 1 hour exposure to 1X
PBS with minimal drift, with an on-off ratio of 5 and major Dirac point
voltage of 0.3 V. There was some indications from the transfer
characteristics that \(p\)-dopants were present on the graphene surface.
Salt concentration and vapour sensing with graphene FETs is not shown in
this thesis, but it is important to perform this experiment and use
similar analysis techniques if there are any concerns about the
sensitivity of a fabricated batch of graphene devices.

\bookmarksetup{startatroot}

\hypertarget{non-covalent-functionalisation-of-carbon-nanotubes-and-graphene}{%
\chapter{Non-Covalent Functionalisation of Carbon Nanotubes and
Graphene}\label{non-covalent-functionalisation-of-carbon-nanotubes-and-graphene}}

\hypertarget{introduction-1}{%
\section{Introduction}\label{introduction-1}}

In previous chapters, I have discussed methods of fabricating carbon
nanotube and graphene devices and then shown that they are sensitive to
environmental changes in a saline solution. However, for specific
sensing, the devices require (bio)chemical functionalisation. Instead of
responding to stimuli themselves, the sensing signal is picked up by
attached receptors. The devices then act as passive transducers for the
received signal. Receptors previously used with carbon nanotube and
graphene devices include aptamers
\autocite{Khan2021,Nguyen2021,Shkodra2021,Nekrasov2021,Mishyn2022,Cassie2023}
and a range of proteins \autocite{Lerner2014,Ahn2020,Tong2020,Wang2020},
including animal odorant receptors
\autocite{Goldsmith2011,Lee2018,Murugathas2019b,Murugathas2020,Moon2020,Yoo2022}.
A common approach to attaching receptors to the transducer involves the
use of a linker molecule to tether the receptor to the transducer.
Verifying that this linker molecule is bridging between the transducer
and the receptor element is important for a complete understanding of
the behaviour of these sensors. This verification involves providing
evidence for effective attachment of linker molecule to the transducing
device channel, then showing successful tethering of odorant receptors
and other biomolecules to the attached linker molecule.

This chapter therefore takes some time exploring the following selection
of available linker molecules for specific biosensing: 1-Pyrenebutanoic
Acid N-Hydroxysuccinimide Ester (PBASE), 1-Pyrenebutyric Acid (PBA),
Pyrene-PEG-NTA (PPN) and Pyrene-PEG-Biotin (PPB). The linker molecules
used are discussed in detail, and numerous hurdles to successful
functionalisation via linker molecules are identified and addressed.
Next, it looks at verifying that that the odorant receptor proteins of
interest have specifically attached to these linker molecules. The
experimental parameters used for both the attachment of linker molecules
and receptor proteins are also varied, and the impact of these
variations on successful functionalisation is investigated via Raman
spectroscopy, fluorescence microscopy and electrical characterisation.

\hypertarget{non-covalent-bonding-and-pi-stacking}{%
\section{\texorpdfstring{Non-Covalent Bonding and
\(\pi\)-Stacking}{Non-Covalent Bonding and \textbackslash pi-Stacking}}\label{non-covalent-bonding-and-pi-stacking}}

Linker molecules may be attached via covalent or non-covalent bonding to
carbon nanomaterials, such as carbon nanotubes and graphene. Covalent
bonding is stronger than non-covalent bonding, and therefore gives a
more permanent attachment between linker molecules and the transducer.
However, non-covalent bonding has the advantage of having less of an
impact on the structure of a nanomaterial than covalent bonding, meaning
non-covalent bonding is less likely to negatively affect the electrical
properties of the transducer
\autocite{Long2012,DiCrescenzo2014,Wang2020,Khan2021,Mishyn2022}. For
example, one group found covalent bonding of diazonium linker caused a
\(\sim 50\)\% drop in graphene channel mobility \autocite{Lerner2014}.
In comparison, only a \(\sim 5\)\% drop in mobility was seen for
attachment of a mixture of linkers containing pyrene to a graphene
channel via non-covalent \(\pi\) stacking \autocite{Thodkar2021}.

\begin{figure}

{\centering \includegraphics[width=0.8\textwidth,height=\textheight]{figures/ch7/pyrene-cnt.png}

}

\caption{\label{fig-pi-interaction-cnt}Attachment of pyrene-based linker
molecule pyrene-X and receptor Y to a carbon nanotube, representing the
transducer element of a field-effect transistor. Source: Adapted from
\autocite{Carbonnanotube}.}

\end{figure}

\(\pi\)-stacking or \(\pi-\pi\) interaction is often used to describe a
type of non-covalent bonding which occurs due to dispersion forces
between unsaturated polycyclic molecules \autocite{Perez2015}. It has
been argued that this label is unhelpfully specific and a
misrepresentation of what can be simply classed as a type of Van Der
Waals bonding \autocite{Martinez2012,Perez2015}. However, as the use of
the term is widespread in the literature, it is also used here for the
sake of clarity. Carbon nanotubes and graphene consist of a network of
carbon atoms attached to each other by sp\(^{2}\) hybrid orbitals in a
polycyclic structure. They are therefore able to strongly interact with
linker molecules with aromatic moieties, such as pyrene
\autocite{Hermanson2013-16,Perez2015,Mishyn2022}.
Figure~\ref{fig-pi-interaction-cnt} is a visual demonstration of the
relationship between the pyrene-based linker molecule with the
transducer and receptor elements. A wide range of pyrene-based linker
molecules have been used for non-covalent modification of carbon
nanotubes and graphene \autocite{Zhou2019}. \(\pi\)-stacking with pyrene
is the bonding mechanism underlying all the functionalisation processes
in this thesis.

\hypertarget{sec-PBASE}{%
\section{Attachment of 1-Pyrenebutanoic Acid N-Hydroxysuccinimide
Ester}\label{sec-PBASE}}

\hypertarget{comparing-attachment-methods}{%
\subsection{Comparing Attachment
Methods}\label{comparing-attachment-methods}}

\begin{figure}

\begin{minipage}[t]{0.47\linewidth}

{\centering 

\raisebox{-\height}{

\includegraphics{figures/ch7/pbase_stable_1_pyrene.png}

}

}

\subcaption{\label{fig-pbase-stable-1}}
\end{minipage}%
%
\begin{minipage}[t]{0.05\linewidth}

{\centering 

~

}

\end{minipage}%
%
\begin{minipage}[t]{0.47\linewidth}

{\centering 

\raisebox{-\height}{

\includegraphics{figures/ch7/pbase_stable_2_pyrene.png}

}

}

\subcaption{\label{fig-pbase-stable-2}}
\end{minipage}%

\caption{\label{fig-pbase-structure}Two conformations of PBASE molecule
with geometry optimised via \emph{ab initio} calculations performed with
Gaussian 16 software \autocite{g16}. White balls correspond to hydrogen,
grey to carbon, red to oxygen and blue to nitrogen. The pyrene moiety is
highlighted in the image with a red dashed outline.}

\end{figure}

1-pyrenebutanoic acid N-hydroxysuccinimide ester (also known
commercially and in the literature both as 1-pyrenebutyric acid
N-hydroxysuccinimide ester and 1-pyrenebutanoic acid succinimidyl ester;
acronyms include PBASE, PBSE, PyBASE, PASE, PYSE, PSE, Pyr-NHS and
PANHS) is a aromatic molecule commonly used for tethering biomolecules
to the carbon rings of graphene and carbon nanotubes. Using
computational modelling, two locally stable molecular conformations were
found to exist, a straight (Figure~\ref{fig-pbase-stable-1}) and bent
(Figure~\ref{fig-pbase-stable-2}) structure. The conformation in
Figure~\ref{fig-pbase-stable-1} has a Hartree-Fock energy of -3427728.67
kJ/mol, while the conformation in Figure~\ref{fig-pbase-stable-2} has a
Hartree-Fock energy of -3427729.66 kJ/mol. The difference between
computed Hartree-Fock energies is 1.0 kJ/mol, small enough that the
existence of both molecular conformations is physically feasible.
Similar straight and bent structures have previously been modelled for
PBASE attached to graphene \autocite{Oishi2022}.

\newpage
\KOMAoptions{paper=landscape,pagesize}

\hypertarget{tbl-pbase-functionalisation}{}
\begin{longtable}[]{@{}lllllll@{}}
\caption{\label{tbl-pbase-functionalisation}Comparison of PBASE
functionalisation processes used for immobilisation of proteins and
aptamers onto carbon nanotubes and graphene. Experimentally optimised
variables are marked with a star (*). Blank entries indicate there was
no mention of the parameter in a particular paper.}\tabularnewline
\toprule\noalign{}
Solvent & Channel & Conc. (mM) & Incubation type & Time (hr) & Rinse
steps & References \\
\midrule\noalign{}
\endfirsthead
\toprule\noalign{}
Solvent & Channel & Conc. (mM) & Incubation type & Time (hr) & Rinse
steps & References \\
\midrule\noalign{}
\endhead
\bottomrule\noalign{}
\endlastfoot
DMF & CNT & 5 & Immersed & 1 & PBS & Maehashi, 2007.
\cite{Maehashi2007} \\
& & 6 & Immersed & 1 & DMF, PBS & García-Aljaro, 2010.
\cite{Garcia-Aljaro2010} \\
& & 6 & Immersed & 1 & DMF & Chen, 2001. \cite{Chen2001} \\
& & 6 & Immersed & 1 & DMF & Cella, 2010. \cite{Cella2010} \\
& & 6 & Immersed & 1 & DMF & Das, 2011. \cite{Das2011} \\
& & 6 & - & 2 & DMF & Besteman, 2003. \cite{Besteman2003} \\
& Graphene & - & - & 2 & DMF & Tsang, 2019. \cite{Tsang2019} \\
& & - & - & 20 & - & Wiedman, 2017. \cite{Wiedman2017} \\
& & 0.2 & Immersed & 20 & DMF, IPA, DI water & Gao, 2018.
\cite{Gao2018} \\
& & 1 & Dropcast & 6 & DMF, IPA, DI water & Nekrasov, 2021.
\cite{Nekrasov2021} \\
& & 5 & Immersed & 1 & DMF, DI water & Hwang, 2016. \cite{Hwang2016} \\
& & 5* & Immersed & 3* & DMF & Hao, 2020. \cite{Hao2020} \\
& & 5 & Immersed & 4* & DMF, DI water & Mishyn, 2022.
\cite{Mishyn2022} \\
& & 6 & Dropcast & 2 & DMF, DI water & Nur Nasufiya, 2020.
\cite{NurNasyifa2020} \\
& & 10 & Dropcast & 2 & DMF, DI water & Campos, 2019.
\cite{Campos2019} \\
& & 10 & Immersed & 2 & DMF, PBS & Kuscu, 2020. \cite{Kuscu2020} \\
& & 10 & Immersed & 1 & DMF & Xu, 2017. \cite{Xu2017} \\
& & 10 & Immersed & 12 & DMF, EtOH, DI water & Khan, 2020.
\cite{Khan2020} \\
& & 50 & Immersed & 4* & MeOH & Wang, 2020. \cite{Wang2020} \\
2-Methoxyethanol & Graphene & 1 & Immersed & 1 & DI water & Ono, 2020.
\cite{Ono2020} \\
Methanol & CNT & 1 & Immersed & 1 & MeOH, DI water & Zheng, 2016.
\cite{Zheng2016} \\
& & 1 & Immersed & 2 & MeOH & Kim, 2009. \cite{Kim2009} \\
& & 100 & Dropcast & 1 & DI water & Yoo, 2022. \cite{Yoo2022} \\
& Graphene & 5 & Immersed & 2 & - & Sethi, 2020. \cite{Sethi2020} \\
& & 5 & Immersed & 1 & MeOH, PBS & Ohno, 2010. \cite{Ohno2010} \\
DMSO & CNT & 10 & - & 1 & DI water & Lopez, 2015. \cite{Lopez2015} \\
& & 10 & Immersed & 1 & PBS & Strack, 2013. \cite{Strack2013} \\
\end{longtable}

\newpage
\KOMAoptions{paper=portrait,pagesize}

The pyrene moiety, highlighted with a red dashed outline in
Figure~\ref{fig-pbase-stable-1}-b, non-covalently bonds to the carbon
rings of the carbon nanotube and graphene surface. The
N-hydroxysuccinimide (NHS) ester group, seen on the right-hand side of
Figure~\ref{fig-pbase-structure}, is highly reactive with amine groups.
It can undergo a nucleophilic substitution reaction with amines attached
to proteins or aptamers, tethering these biomolecules via an amide or
imide bond
\autocite{Chen2001,Hermanson2013-16,Hermanson2013-3,Mishyn2022}.

The non-covalent functionalisation of proteins onto a single-walled
carbon nanotube using PBASE was first reported by Chen \emph{et al.} in
2001 \autocite{Chen2001}. Two successful methods for protein
functionalisation and immobilisation were reported, with the only
differences being the solvent used to dissolve the PBASE powder (DMF,
methanol) and the final concentration of the resulting solutions (6 mM,
1 mM respectively). PBASE powder appears to dissolve poorly in methanol
at higher concentrations, which might explain the use of different
concentrations of PBASE in each solvent. An extensive comparison of
methods used in the literature for PBASE functionalisation of carbon
nanotube and graphene devices with aptamers and proteins is given in
Table~\ref{tbl-pbase-functionalisation}. Several listed works directly
cite Chen \emph{et al.} when discussing functionalisation with PBASE
\autocite{Besteman2003,Cella2010,Campos2019,Zheng2016,Ohno2010}. The
other works listed do not explicitly reference Chen \emph{et al.} in
their methodology; however, the frequency of methods detailing the use
of 6 mM PBASE in dimethylformamide (DMF) and 1 mM PBASE in methanol
indicate that these processes are largely copying the process used by
Chen \emph{et al.}.

However, it is also apparent from
Table~\ref{tbl-pbase-functionalisation} that there is a large degree of
variation in the methods used for PBASE functionalisation. Various
electrical characterisation, microscopy and spectroscopy techniques have
been used to demonstrate successful functionalisation. Until recently,
there has been little justification provided for the selection of
variables used in the functionalisation procedure (e.g.~length of time
submerged in solvent containing PBASE), despite the wide-ranging use of
this process in the literature \autocite{Hinnemo2017,Zhen2018,Wang2020}.
This is surprising, given that the sensitivity of functionalised devices
is considered to be closely related to the density of surface
functionalisation \autocite{White2008,Hermanson2013-3,Chen2014}.
Furthermore, a detailed investigation of PBASE functionalisation process
variables has only been undertaken for graphene-based devices
\autocite{Zhen2018,Hao2020,Wang2020,Mishyn2022}.

Zhen \emph{et al.} \autocite{Zhen2018}, Wang \emph{et al.}
\autocite{Wang2020} and Mishyn \emph{et al.} \autocite{Mishyn2022} have
all claimed that carefully tuning the surface concentration of PBASE is
required to avoid multilayer coverage of the graphene surface, as this
negatively impacts sensing. Mishyn \emph{et al.} \autocite{Mishyn2022}
used cyclic voltammetry to demonstrate that less receptor attachment to
the graphene surface occurs when multiple layers of PBASE are present.
However, none of these groups have presented analyte sensing results
from their functionalised graphene devices. In contrast, Hao \emph{et
al.} \autocite{Hao2020} found that maximising the PBASE surface coverage
of a channel resulted in more sensitive aptameric sensing, thereby
reaching the opposite conclusion. The inconsistency in these recent
findings mean more work is needed to understand the PBASE
functionalisation process to achieve optimal biosensor sensitivity. It
may also be the case that a specific functionalisation process is
required for optimal sensitivity with the use of a specific type of
receptor.

\begin{figure}

\begin{minipage}[t]{\linewidth}

{\centering 

\raisebox{-\height}{

\includegraphics{figures/ch7/labelled_modified_sigma_pbase_nmr.png}

}

}

\subcaption{\label{fig-sigma-nmr}}
\end{minipage}%
\newline
\begin{minipage}[t]{\linewidth}

{\centering 

\raisebox{-\height}{

\includegraphics{figures/ch7/labelled_modified_setareh_pbase_nmr.png}

}

}

\subcaption{\label{fig-setareh-nmr}}
\end{minipage}%
\newline
\begin{minipage}[t]{\linewidth}

{\centering 

\raisebox{-\height}{

\includegraphics{figures/ch7/labelled_modified_dmso_nmr.png}

}

}

\subcaption{\label{fig-dmso-nmr}}
\end{minipage}%

\caption{\label{fig-pbase-nmr}\(^{1}\)H Nuclear Magnetic Resonance (NMR)
spectra, performed with DMSO-d\(_6\) used as the NMR solvent. (a) and
(b) show NMR spectrum for commercially purchased PBASE, from
Sigma-Aldrich and Setareh Biotech respectively, while (c) shows the
blank spectrum taken with only DMSO-d\(_6\) present. Spectra were taken
by Jennie Ramirez-Garcia, School of Chemical and Physical Sciences, Te
Herenga Waka - Victoria University of Wellington. Unlabelled peaks
correspond to sample impurities.}

\end{figure}

Once fastened to a bioreceptor via an amide or imide bond, the
attachment to the linker molecule is not easily broken. However, prior
to use in functionalisation processes, the NHS ester may react with any
water present (hydrolysis). This reaction converts PBASE to
1-pyrenebutyric acid (PBA), leaving it unavailable to react further with
amine groups \autocite{Hermanson2013-3,Hermanson2013-5,Mishyn2022}. If
the amine group functionalisation is performed within a \(\sim1\) hour
period, with a high concentration of bioreceptor used at close to
neutral pH, competing hydrolysis should not have a significantly adverse
impact on the functionalisation process \autocite{Hermanson2013-3}.
However, if PBASE is exposed to water during storage over a significant
length of time, the presence of
1-Ethyl-3-(3-dimethylaminopropyl)carbodiimide (EDC) can be used to
restore the NHS ester and enable the substitution reaction to take place
(see discussion of PBA/EDC in Section~\ref{sec-PBA}).

\hypertarget{examining-1-pyrenebutanoic-acid-n-hydroxysuccinimide-ester-purity}{%
\subsection{Examining 1-Pyrenebutanoic Acid N-Hydroxysuccinimide Ester
Purity}\label{examining-1-pyrenebutanoic-acid-n-hydroxysuccinimide-ester-purity}}

I purchased PBASE from two suppliers, Sigma-Aldrich and Setareh Biotech.
Sigma recommended DMF and methanol as suitable solvents for dissolving
PBASE, alongside chloroform and dimethyl sulfoxide (DMSO). Setareh
Biotech indicated methanol can be used for dissolving PBASE. The two
suppliers had conflicting information for suitable storage of PBASE,
with Sigma recommending room temperature storage while Setareh Biotech
recommends storage of -5 to -30°C and protection from light and
moisture. I used nuclear magnetic resonance (NMR) spectroscopy to verify
the purity of PBASE from various suppliers. As water can react with
PBASE to form unwanted byproducts, it appears that protection from
moisture is particularly important. A particular emphasis was placed on
detecting water presence in the received samples, considering the long
travel time of the PBASE with uncertain storage conditions.

Figure~\ref{fig-pbase-nmr} compares the shapes of hydrogen NMR spectra
of PBASE from each supplier when dissolved in deuterated DMSO, alongside
a blank deuterated DMSO spectrum. Both PBASE samples possessed
characteristic chemical shift features between \(2.1-2.2\) ppm,
\(2.8-2.9\) ppm, and \(3.4-3.5\) ppm. These chemical shifts roughly
correspond to those seen in previous NMR spectra for PBASE
\autocite{NMR2}. The feature at 2.50 ppm represents the deuterated DMSO
solvent, while the single peak between \(3.3-3.4\) ppm represents the
water present in the sample. By comparing the area of these peaks, a
rough estimate of the amount of water originally present in the PBASE
sample can be obtained. The H\(_{2}\)O:DMSO ratio is 1:7 in the blank
spectrum, but \(\sim\) 1:3 in the provided samples, possibly indicating
the introduction of water to the PBASE during production or storage.
However, DMSO is strongly hygroscopic and slight differences in DMSO
storage time, as well as differences in humidity during sample
preparation, may have had a significant impact on this result
\autocite{Lebel1962}. Other impurities are also seen on both PBASE
spectra, though their small size indicates they make up only a small
percentage of each sample. Strack \emph{et al.} \autocite{Strack2013}
recommend leaving frozen PBASE at room temperature for 15 minutes before
exposing it to air to prevent condensation near the PBASE, as this can
cause unnecessary \(H_2O\) contamination.

\hypertarget{sec-PBASE-electrical-characterisation}{%
\subsection{Electrical
Characterisation}\label{sec-PBASE-electrical-characterisation}}

\begin{figure}

\begin{minipage}[t]{0.50\linewidth}

{\centering 

\raisebox{-\height}{

\includegraphics{figures/ch7/Q23D5_ch1_MeOHonly.png}

}

}

\subcaption{\label{fig-meoh-only-tx}}
\end{minipage}%
%
\begin{minipage}[t]{0.50\linewidth}

{\centering 

\raisebox{-\height}{

\includegraphics{figures/ch7/Q23D12_ch1_DMSOonly.png}

}

}

\subcaption{\label{fig-dmso-only-tx}}
\end{minipage}%
\newline
\begin{minipage}[t]{0.50\linewidth}

{\centering 

\raisebox{-\height}{

\includegraphics{figures/ch7/Q2C6_ch2_MeOHPBASE.png}

}

}

\subcaption{\label{fig-meoh-pbase-tx}}
\end{minipage}%
%
\begin{minipage}[t]{0.50\linewidth}

{\centering 

\raisebox{-\height}{

\includegraphics{figures/ch7/Q23D7_ch3_DMSOPBASE.png}

}

}

\subcaption{\label{fig-dmso-pbase-tx}}
\end{minipage}%

\caption{\label{fig-PBASE-vs-solvent-only}The electrical transfer
characteristics of carbon nanotube transistors (\(V_{ds}\) = 100 mV)
before and after being submerged in methanol (a) or dimethyl sulfoxide
(b) for one hour and subsequently rinsed with deionised water. The
change in characteristics of similar transistor channels after being
submerged in these same solvents containing 1 mM PBASE for one hour and
then rinsed are shown in (c) and (d) respectively. Average threshold
voltages for each transfer characteristic curve are also shown (taking
the average of forward and reverse sweep values).}

\end{figure}

The electrical characteristics of the carbon nanotube or graphene
transistor are often used to verify successful functionalisation and
make a statement about the effect of chemical modification on the
channel. However, this verification usually does not account for the
effect of the solvent on the transistor channel.
Figure~\ref{fig-meoh-only-tx} and Figure~\ref{fig-dmso-only-tx} show
that by exposing a steam-deposited carbon nanotube network channel to
solvents commonly used in PBASE functionalisation processes
(Table~\ref{tbl-pbase-functionalisation}), such as methanol (MeOH) or
dimethyl sulfoxide (DMSO), a significant negative shift in channel
threshold voltage occurs even after thorough rinsing with deionised
water. It appears that the carbon nanotubes have adsorped solvent which
persists even after rinsing the device. From the shape of the change in
the transfer curve, it seems the residual polar solvent molecules
capacitively gate the channel \autocite{Artyukhin2006,Heller2008}.
Besteman \emph{et al.} reported observing a similar effect from
prolonged exposure of a single carbon nanotube to dimethylformamide
(DMF) \autocite{Besteman2003}. The ongoing presence of solvent after
device cleaning is also supported by atomic force microscopy.
Figure~\ref{fig-afm-solvent} shows that after 1 hour exposure to DMSO,
the feature height of the network is significantly increased.

\begin{figure}

\begin{minipage}[t]{0.47\linewidth}

{\centering 

\raisebox{-\height}{

\includegraphics{figures/ch7/Danica_NTQ14D6_90sCNT_20220915_00596.png}

}

}

\subcaption{\label{fig-pristine-afm}}
\end{minipage}%
%
\begin{minipage}[t]{0.05\linewidth}

{\centering 

~

}

\end{minipage}%
%
\begin{minipage}[t]{0.47\linewidth}

{\centering 

\raisebox{-\height}{

\includegraphics{figures/ch7/Danica_NTQ14D6_DMSO_W4_20220915_00607.png}

}

}

\subcaption{\label{fig-DMSO-afm}}
\end{minipage}%

\caption{\label{fig-afm-solvent}2.5 \(\mu\)m x 2.5 \(\mu\)m atomic force
microscope images of a surfactant-deposited carbon nanotube film before
(a) and after (b) 1 hour submersed in dimethyl sulfoxide (DMSO). Image
blurring in (b) appears to be due to the presence of adsorped DMSO.}

\end{figure}

Using the same characterisation process as in this work, Murugathas
\emph{et al.} \autocite{Murugathas2019b} showed that the attachment of
PBASE to a solvent-deposited carbon nanotube network had little effect
on channel threshold voltage, implying the presence of PBASE had not
significantly influenced channel gating \autocite{Murugathas2019b}.
However, they did observe a slight increase in channel conductance after
PBASE functionalisation. In Figure~\ref{fig-PBASE-vs-solvent-only}, a
slight increase in channel conductance post-functionalisation is
observed for both Figure~\ref{fig-meoh-pbase-tx} and
Figure~\ref{fig-dmso-pbase-tx} when compared to the solvent-only case in
Figure~\ref{fig-meoh-only-tx} and Figure~\ref{fig-dmso-only-tx}. This
result implies that the presence of PBASE molecules increases channel
mobility and therefore conductance \autocite{Heller2008}.

Capactive gating results from dense coverage of adsorped molecules on
the carbon nanotube surface which have a low permittivity relative to
the surrounding electrolyte \autocite{Heller2008}. The relative
permittivity of MeOH and DMSO are \(\sim\) 33 \autocite{Mohsen-Nia2010}
and \(\sim\) 47 \autocite{Hunger2010} respectively, which are both much
lower than the relative permittivity of phosphate buffer saline,
\(\sim\) 80 \autocite{Shkodra2021}. From Figure~\ref{fig-meoh-only-tx}
and Figure~\ref{fig-dmso-only-tx}, the threshold shift values found
resulting from exposure to each solvent, taking the average of forward
and reverse sweep values from a single device, were \(\Delta\)V =
\(-0.15 \pm 0.02\) V and \(\Delta\)V = \(-0.15 \pm 0.01\) V for MeOH and
DMSO respectively. The average threshold shift value for a second device
exposed to MeOH was \(\Delta\)V = \(-0.16 \pm 0.02\) V, indicating that
this threshold shift result is reproducible. The threshold voltage
shifts in Figure~\ref{fig-meoh-pbase-tx} and
Figure~\ref{fig-dmso-pbase-tx} from the pristine are small compared with
the devices exposed to solvent only - this is likely due to the effect
of increased conductance from the PBASE competing with the gating effect
from the residual solvent.

The absorption of organic solvent by the carbon nanotube network has
unknown but potentially negative implications for biosensor
functionalisation. Use of organic solvents in functionalisation can also
attack the encapsulation layer of devices, promoting gate current
leakage. In light of these issues, recent work has begun to explore
alternative aqueous-based methods for functionalisation of biosensors
\autocite{Khan2021}. The discussion here also illustrates the importance
of considering each substance used when electrical characterising a
device to verify if functionalisation has worked. The qualitative
presence of a change in characteristics (or lack of one) over the full
process is not sufficient to make conclusive remarks regarding
successful functionalisation. A full set of electrical control
measurements are required for an understanding of electronic changes
occuring during the functionalisation process, in the manner of Besteman
\emph{et al.} \autocite{Besteman2003}.

\newpage
\KOMAoptions{paper=landscape,pagesize}

\hfill\break
\hfill\break

\hypertarget{tbl-pba-functionalisation}{}
\begin{longtable}[]{@{}llllllll@{}}
\caption{\label{tbl-pba-functionalisation}Comparison of 1-pyrenebutyric
acid (PBA) functionalisation processes used for immobilisation of
proteins, enzymes and aptamers onto carbon nanotubes and graphene.
1-ethyl-3-(3-dimethylaminopropyl) carbodiimide hydrochloride (EDC) and
NHS were co-mingled in buffer/electrolyte solution or DI water in each
process - some papers used N-hydroxysulfosuccinimide instead of
N-hydroxysuccinimide, and both compounds are abbreviated as NHS in this
table for simplicity. Device exposure times to each solution are shown
next to the solution concentration. Blank entries indicate there was no
mention of the parameter in a particular paper. \(^†\)PEG or PEG pyrene
were used to reduce non-specific binding. \(^{††}\)Several pyrene-based
linkers were compared and PBA gave an optimal functionalisation
result.}\tabularnewline
\toprule\noalign{}
Solvent & Channel & PBA (mM) & Time (hr) & EDC (mM) & NHS (mM) & Time
(min) & References \\
\midrule\noalign{}
\endfirsthead
\toprule\noalign{}
Solvent & Channel & PBA (mM) & Time (hr) & EDC (mM) & NHS (mM) & Time
(min) & References \\
\midrule\noalign{}
\endhead
\bottomrule\noalign{}
\endlastfoot
DMF & Graphene & 0.6 & 1 & - & - & 120 & Gao, 2016\(^†\).
\cite{Gao2016} \\
& & 5 & 2 & 2 & 5 & 30 & Mishyn, 2022. \cite{Mishyn2022} \\
& CNT & 100 & 3 & 200 & - & 30 & Min, 2012. \cite{Min2012} \\
& Graphene, CNT & 7.6 & 2 & 8 & 20 & 120 & Xu, 2014. \cite{Xu2014} \\
DI water & CNT & - & - & 32 & 12 & Overnight & Pacios, 2012\(^†\).
\cite{Pacios2012} \\
Ethanol & CNT & 1 & 1 & 100 & 100 & 20 & Filipiak, 2018\(^†\).
\cite{Filipiak2018} \\
Acetonitrile & Graphene & 1 & 1 & 400 & 100 & 60 & Tong, 2020\(^{††}\).
\cite{Tong2020} \\
Borax & CNT & 2 & 24 & 2.5 & - & 1080 & Liu, 2011\(^†\).
\cite{Liu2011} \\
DMSO & Graphene & 5 & 1 & 50 & 50 & 90 & Fenzl, 2017.
\cite{Fenzl2017} \\
\end{longtable}

\newpage
\KOMAoptions{paper=portrait,pagesize}

\hypertarget{sec-PBA}{%
\section{Attachment of 1-Pyrenebutyric Acid}\label{sec-PBA}}

\hypertarget{comparing-attachment-methods-1}{%
\subsection{Comparing Attachment
Methods}\label{comparing-attachment-methods-1}}

Another linker molecule that can be used to attach receptor molecules to
a carbon nanotube or graphene channel is 1-pyrenebutyric acid (PBA or
PyBA). As with PBASE, the pyrene group of PBA has a \(\pi\) interaction
with the carbon rings of the channel surface. It is possible to react
PBA with 1-ethyl-3-(3-dimethylaminopropyl) carbodiimide hydrochloride
(EDC or EDAC) to form an \emph{O}-acylisourea intermediate, which can
then react with an amine group on a biomolecule and form an amide bond
\autocite{Sehgal1994,Hermanson2013-4}. The water solubility of EDC means
that, unlike PBASE, it is possible to functionalise with EDC dissolved
in water rather than in an organic solvent. However, like PBASE, EDC and
the \emph{O}-acylisourea intermediate are prone to hydrolysis,
especially in acidic conditions. Therefore, like PBASE, it should be
stored at −20°C, and warmed to room temperature to prevent condensation
build-up, since exposure to condensation will hydrolyse the reagent
\autocite{Hermanson2013-4}. Furthermore, by adding N-Hydroxysuccinimide
(NHS) or N-hydroxysulfosuccinimide (sulfo-NHS) to the reaction vessel,
PBASE is formed as an active intermediate, which is less prone to
hydrolysis and increases the PBA/EDC reaction yield
\autocite{Sehgal1994,Hermanson2013-4,Hermanson2013-14}.

A full comparison of functionalisation procedures used for linking
carbon nanotube and graphene devices to aptamers and proteins with PBA
is given in Table~\ref{tbl-pba-functionalisation}. To the best of my
knowledge, this table is as complete a summary as possible of
1-pyrenebutyric acid functionalisation processes for carbon nanotube and
graphene field-effect transistor biochemical sensors. By comparing
Table~\ref{tbl-pbase-functionalisation} and
Table~\ref{tbl-pba-functionalisation}, it is clear that PBASE is more
widely used for non-covalent functionalisation than PBA/EDC. As was the
case for PBASE, there are a wide range of process variables used for the
functionalisation process, with little justification used for variables
chosen. Also notable is the frequent use of polyethylene glycol (PEG) or
pyrene-PEG for prevention of non-specific binding (NSB). Non-specific
binding is discussed further in \textbf{?@sec-non-specific-binding}.
Despite being less widely used, Mishyn \emph{et al.}
\autocite{Mishyn2022} state a preference for the use of PBA/EDC over
PBASE, as they found it was less prone to hydrolysis and gave a larger
reaction yield when binding ferrocene to graphene. A potential downside
of using PBA/EDC for protein immobilisation is that EDC has numerous
ways of interacting with proteins, and not all of these are necessarily
desirable; furthermore, the addition of NHS may also cause other issues,
such as precipitation of the reaction compound
\autocite{Hermanson2013-4}. The greater range of process variables
involved in the functionalisation also adds to the complexity of
reproducing past results.

\hypertarget{raman-spectroscopy}{%
\subsection{Raman Spectroscopy}\label{raman-spectroscopy}}

\begin{figure}

{\centering \includegraphics[width=0.5\textwidth,height=\textheight]{figures/ch7/comparison_raman.png}

}

\caption{\label{fig-linker-raman}This box plot shows the distribution of
D-band peak to G\(^+\)-band peak ratio, I\(_D\)/I\(_G\), across nine
locations for a selection of chemically-modified carbon nanotube films.
The D-band and G-band intensities for all samples were first normalised
to the intensity peak corresponding to the silicon dioxide substrate.}

\end{figure}

Raman spectroscopy was used to verify the attachment of PBA to a carbon
nanotube network film with a silicon dioxide substrate in the manner
outlined in \textbf{?@sec-raman-characterisation}. As highly-bundled
devices were found to have less defects present prior to modification,
as discussed in Section~\ref{sec-pristine-raman}, solvent-deposited
films were used for the verification of pyrene attachment to prevent the
initial presence of defects influencing the analysis. Droplets of DMSO
solution were placed on three (solvent-deposited) carbon nanotube films
taken from the same wafer. The DMSO solution on one film contained 5 mM
PBA, the solution on another film contained 5 mM PBASE, and the DMSO on
the final film contained no linker molecule. After incubation for 1
hour, films were rinsed for 15 s with DMSO, then for 15 s with IPA to
remove excess DMSO while avoiding hydrolysis of the PBASE. After the
first set of Raman spectra was taken, the film initially exposed to PBA
was further exposed to a solution of 20 mM EDC and 40 mM NHS in 1XPBS
electrolyte for 30 minutes, and a second set of Raman spectra was taken
for this film. As in Section~\ref{sec-pristine-raman}, two spectra taken
at each position were processed according to
Section~\ref{sec-raman-analysis}, and the silicon dioxide reference peak
measured in the wavenumber range 100 cm\(^{-1}\) \(-\) 650 cm\(^{-1}\)
was used to normalise the D-band and G-band peaks from the wavenumber
range 1300 cm\(^{-1}\) \(-\) 1650 cm\(^{-1}\). The ratio between the
average intensity of the D-peak and the G\(^+\)-peak at each position
was calculated, and the distribution of ratio values corresponding to
each modified film is shown in Figure~\ref{fig-linker-raman}.

\begin{figure}

\begin{minipage}[t]{0.50\linewidth}

{\centering 

\raisebox{-\height}{

\includegraphics{figures/ch7/NTQ24C10_ch3_comparison_1.png}

}

}

\subcaption{\label{fig-pba-threshold-shift}}
\end{minipage}%
%
\begin{minipage}[t]{0.50\linewidth}

{\centering 

\raisebox{-\height}{

\includegraphics{figures/ch7/NTQ24C10_ch3_comparison_2.png}

}

}

\subcaption{\label{fig-edc-nhs-threshold-shift}}
\end{minipage}%
\newline
\begin{minipage}[t]{0.50\linewidth}

{\centering 

\raisebox{-\height}{

\includegraphics{figures/ch7/NTQ24C10_ch3_comparison_3.png}

}

}

\subcaption{\label{fig-rinse-threshold-shift}}
\end{minipage}%
%
\begin{minipage}[t]{0.50\linewidth}

{\centering 

\raisebox{-\height}{

\includegraphics{figures/ch7/NTQ24C10_ch3_comparison_4.png}

}

}

\subcaption{\label{fig-pba-threshold-shift-comparison}}
\end{minipage}%

\caption{\label{fig-pba-functionalisation-threshold-shift}Electrical
transfer characteristics of a carbon nanotube transistor before
functionalisation alongside the transfer characteristics (a) after being
submerged in DMSO containing 5 mM PBA for 1 hour in red, (b) after being
submerged in 1XPBS containing 20 mM EDC and 40 mM NHS for 30 min in
blue, and (c) after being submerged in fresh 1XPBS for 1 hour in green.
The dashed lines in (d) are linear fits tangent to the subthreshold
slope of each characteristic curve, and are shown alongside the
threshold voltages calculated by finding the intercept of each fit.}

\end{figure}

There is a \(\sim 3 \times\) increase in the intensity ratio
I\(_D\)/I\(_G\) for both the films modified with PBASE and PBA compared
to the film which was only exposed to DMSO. Previous works have found
that a change in the intensity ratio indicates successful
\(\pi\)-stacking on the carbon nanotube surface, as it indicates surface
modification of the carbon nanotubes has occurred
\autocite{Wei2010,Lan2013}. Wei \emph{et al.} \autocite{Wei2010} found
functionalisation with PBASE altered the ratio by a factor of
\(\sim 1.5 \times\), while Lan \emph{et al.} \autocite{Lan2013} found
that functionalisation with PBA altered the ratio by a factor of
\(\sim 0.8 \times\). The reason for the large difference between results
is not immediately clear, but may result from the significant
differences in the pristine composition and morphology of carbon
nanotube networks used in each publication, and differences in the
functionalisation method used. Across all scan locations in
\textbf{?@fig-raman-comparison}, the value found for I\(_D\)/I\(_G\) is
consistently \(\sim 0.095\) for both PBA and PBASE. Furthermore,
subsequent Raman measurements of the PBA-modified film after further
functionalisation with EDC/NHS do not show a significant change in
I\(_D\)/I\(_G\). These results indicate that presence of the NHS ester
has little effect on the Raman shift. It should be clarified that Raman
spectroscopy cannot be used to distinguish between the presence of PBA
and PBASE on the device surface. However, it is clear that
functionalisation of the carbon nanotube network with both the PBA and
PBASE has led to measurable \(pi\)-stacking between the network and the
pyrene group attached to each compound.

\hypertarget{electrical-characterisation}{%
\subsection{Electrical
Characterisation}\label{electrical-characterisation}}

Figure~\ref{fig-pba-functionalisation-threshold-shift} shows the
transfer characteristics of a carbon nanotube transistor channel at
various stages of a PBA/EDC functionalisation, where a excess of
N-hydroxysuccinimide (NHS) was added alongside EDC. A solvent-deposited
carbon nanotube film was used for the device. The PBA was dissolved in
DMSO, and the device channels were exposed to this solution for 1 hour.
The electrical change resulting from PBA exposure is shown in
Figure~\ref{fig-pba-threshold-shift}. The threshold shift with the
addition of 5 mM PBA in DMSO for 1 hour is equivalent to the shift seen
when only DMSO is added, \(\Delta\)V = -0.15 V. The lack of a
significant threshold shift directly attributable to the PBA is a result
of pyrene having a neutral charge state; any contributions from the
charged carboxyl group are screened from the carbon nanotube sidewalls
by surrounding water molecules \autocite{Lerner2012}. However, as in the
case of the addition of PBASE, there also appears to be an increase in
hole mobility, which may be due to the pyrene groups increasing
connectivity within the carbon nanotube network
\autocite{Murugathas2019b}.

Subsequently, the device was rinsed with 1XPBS and exposed to 20 mM EDC
and 40 mM NHS in 1XPBS electrolyte for 30 minutes.
Figure~\ref{fig-edc-nhs-threshold-shift} shows the change resulting from
subsequent EDC/NHS exposure. When EDC/NHS is added, a threshold shift of
\(\Delta\)V \(\sim\) -0.08 V was observed on multiple channels. The
exposure to EDC/NHS negatively shifts the transfer characteristic curve,
most likely due to the PBA present reacting to form positively-charged
\emph{O}-acylisourea esters and negatively gating the attached carbon
nanotube network \autocite{Heller2008,Hermanson2013-4}.
Figure~\ref{fig-rinse-threshold-shift} shows that this shift is not
significantly affected by further exposure of the channel to PBS. This
indicates that hydrolysis over the course of one hour is insufficient to
hydrolyse a significant proportion of the \emph{O}-acylisourea back to
PBA, as PBA is charge neutral. We therefore expect that the a
significant amount of \emph{O}-acylisourea remains active within this
time period and available for reaction with biomolecule amine groups.

\hypertarget{attachment-of-peglyated-pyrene-based-linkers}{%
\section{Attachment of PEGlyated Pyrene-Based
Linkers}\label{attachment-of-peglyated-pyrene-based-linkers}}

\hypertarget{sec-NTA-biotin-PEG}{%
\subsection{Pyrene-NTA, Pyrene-Biotin and
PEGylation}\label{sec-NTA-biotin-PEG}}

Through chemical coupling/conjugation, it is possible to replace the NHS
ester group on PBASE with other groups that can undergo binding
reactions with proteins. Unlike PBASE, these groups do not suffer the
drawback of being readily hydrolysed. For example, PBASE can be modified
with N\(\alpha\),N\(\alpha\)-Bis(carboxymethyl)-L-lysine hydrate (also
known as N-(5-Amino-1-carboxypentyl)iminodiacetic acid, AB-NTA) to
produce pyrene-nitrilotriacetic acid (pyrene-NTA). The attached NTA
group is able to chelate with metal ions such as Cu\(^{2+}\) or
Ni\(^{2+}\), which then can then coordinate with polyhistidine-tags
attached to a protein \autocite{Holzinger2011,Amano2016,Chang2017}. Use
of Cu\(^{2+}\) ions over Ni\(^{2+}\) gives stronger histidine bonding
and less non-specific adsorption \autocite{Chang2017}. Functionalisation
using the NTA-Ni\(^{2+}\) chemistry was successfully used to attach
mammalian odorant receptors to a single carbon nanotube for detection of
eugenol vapour in real-time \autocite{Goldsmith2011}. Pyrene-biotin
(pyrene butanol biotin ester) can also be produced for attaching avidin
or strepavidin \autocite{Holzinger2011}. As avidin and strepavidin are
tetrameric, they can be attached to both pyrene-biotin and biotinylated
avi-tagged proteins simultaneously via strong non-covalent bonding,
therefore linking the transducer and receptor
\autocite{Star2003a,Dundas2013,Hermanson2013-11,Fairhead2015}. As the
presence of his-tags and avi-tags on proteins can be readily controlled,
these methods offer improved specificity and directionality over the
traditional amide bonding seen earlier.

It is also possible to attach polyethylene glycol (PEG) chains to a
pyrene group and modify them with reactive groups such as NTA and biotin
to attach proteins in the manner outlined in the previous paragraph
\autocite{Hermanson2013-18,Meran2018}. Once modified with PEG, the water
solubility of pyrene linkers increases, making it possible to perform a
full functionalisation procedure exclusively in aqueous solution
\autocite{Hermanson2013-18}. By setting the length of the PEG chain, the
size of the linker molecule can be controlled - selection of a short
chain is important for ensuring attached receptors remain within the
Debye length of the transducer \autocite{Shkodra2021}. Functionalisation
of a graphene transducer with pyrene-PEG-biotin has previously been used
to bind streptavidin to a graphene field-effect transistor device
\autocite{Miki2019}. The PEGlyated linkers used in the following
sections were purchased pre-prepared. Pyrene-PEG-NTA (2 kDa) was
purchased from Nanocs, while pyrene-PEG-FITC (2 kDa, 10 kDa),
pyrene-PEG-rhodamine (3.4 kDa), mPEG-Pyrene (10 kDa) and
pyrene-PEG-biotin (10 kDa) were purchased from Creative PEGworks.

\hypertarget{sec-impediments}{%
\section{Identifying Functionalisation Issues using Fluorescence
Microscopy}\label{sec-impediments}}

\hypertarget{sec-fluorescence-remarks}{%
\subsection{General Overview}\label{sec-fluorescence-remarks}}

Various dyes and fluorescent tags were used to investigate approaches
for identifying successful attachment of biomolecules to a carbon
nanotube or graphene surface with fluorescence microscopy. The dyes
included fluorescein isothiocyanate (FITC), Rhodamine B and Cyanine 3
(Cy3). Green fluorescent protein was also used for this testing process.
It is important to note that these dyes and the GFP chromophore all
contain benzene rings which are able to \(\pi\)-stack with carbon rings
to some degree
\autocite{Nakayama-Ratchford2007,Tang2012,Khrenova2019,Qiu2019}.
However, there is also significant variation in the effectiveness of
this \(\pi\)-stacking, as shown by Figure~\ref{fig-FITC-rhodamine-B}.
Here, a clear, specific interaction is seen between Rhodamine B and
graphene, but little interaction between FITC and graphene is observed,
even when a longer exposure time is used. Whether the addition of pyrene
linker groups these dyes, or dye-modified biomolecules, was therefore
investigated in further detail. This process then led to the
identification of multiple issues that could impede a successful device
functionalisation.

\begin{figure}

\begin{minipage}[t]{0.47\linewidth}

{\centering 

\raisebox{-\height}{

\includegraphics{figures/ch7/NGW8D1_FITC_6.5sexposure_10X_221121.png}

}

}

\subcaption{\label{fig-FITC}}
\end{minipage}%
%
\begin{minipage}[t]{0.05\linewidth}

{\centering 

~

}

\end{minipage}%
%
\begin{minipage}[t]{0.47\linewidth}

{\centering 

\raisebox{-\height}{

\includegraphics{figures/ch7/NGW8D4_rhodamineB_cornergraphene_221110.png}

}

}

\subcaption{\label{fig-rhodamine}}
\end{minipage}%

\caption{\label{fig-FITC-rhodamine-B}Four 200 \(\mu\)m \(\times\) 200
\(\mu\)m graphene squares modified with the dyes (a) fluorescein
isothiocyanate (FITC) and (b) Rhodamine B. No pyrene/PEG/pyrene-PEG was
attached to these dyes. In (a), an FITC filter and 6.5 s exposure time
was used, and in (b) a Texas Red filter and 1.4 s exposure time was
used.}

\end{figure}

Both SU8 and AZ\(^\circledR\) 1518 photoresist fluoresced under a
variety of microscope filters, resulting from light interacting with the
photoactive component present in both resists \autocite{Pai2007}. This
background fluorescence was found to drown out fluorescence from a
dye-functionalised device channel, and so photoresist encapsulated
devices were not used for fluorescence imaging. (Consider a photograph
of a dim outdoor lamp; if the photograph was taken on a starless night,
the light from the lamp would show up clearly, but with the sun out the
light would be very difficult to see regardless of how the photograph
was taken.) A different type of encapsulation could potentially be used
to verify linker attachment with fluorescence after a device has been
encapsulated. These alternative encapsulation methods for use with
fluorescence microscopy are discussed in \textbf{?@sec-future-work}.

\hypertarget{sec-photoresist-contamination}{%
\subsection{Photoresist
Contamination}\label{sec-photoresist-contamination}}

\begin{figure}

\begin{minipage}[t]{0.47\linewidth}

{\centering 

\raisebox{-\height}{

\includegraphics{figures/ch7/modified_SU8only_FITCfilter_channel2_350ms_12.6X_221207.png}

}

}

\subcaption{\label{fig-350ms-SU8}}
\end{minipage}%
%
\begin{minipage}[t]{0.05\linewidth}

{\centering 

~

}

\end{minipage}%
%
\begin{minipage}[t]{0.47\linewidth}

{\centering 

\raisebox{-\height}{

\includegraphics{figures/ch7/modified_CNT20_1mMPPF_channel8_350ms_12.6X_221124.png}

}

}

\subcaption{\label{fig-350ms-SU8-FITC}}
\end{minipage}%
\newline
\begin{minipage}[t]{0.47\linewidth}

{\centering 

\raisebox{-\height}{

\includegraphics{figures/ch7/modified_Ccontrol_ch2_mCherry_10sexposure_highcontrast_12.6X.png}

}

}

\subcaption{\label{fig-aptamer-photoresist-1}}
\end{minipage}%
%
\begin{minipage}[t]{0.05\linewidth}

{\centering 

~

}

\end{minipage}%
%
\begin{minipage}[t]{0.47\linewidth}

{\centering 

\raisebox{-\height}{

\includegraphics{figures/ch7/modified_C1_ch5_mCherry_10sexposure_highcontrast_12.6X.png}

}

}

\subcaption{\label{fig-aptamer-photoresist-2}}
\end{minipage}%
\newline
\begin{minipage}[t]{0.47\linewidth}

{\centering 

\raisebox{-\height}{

\includegraphics{figures/ch7/modified_softbake1minacetonerinse_aptamer_ch3_mCherry_30sexposure_highcontrast_ISO200_12.6X.png}

}

}

\subcaption{\label{fig-aptamer-photoresist-3}}
\end{minipage}%
%
\begin{minipage}[t]{0.05\linewidth}

{\centering 

~

}

\end{minipage}%
%
\begin{minipage}[t]{0.47\linewidth}

{\centering 

\raisebox{-\height}{

\includegraphics{figures/ch7/modified_hardbake_aptamer_ch3_mCherry_30sexposure_highcontrast_ISO200_12.6X.png}

}

}

\subcaption{\label{fig-aptamer-photoresist-4}}
\end{minipage}%

\caption{\label{fig-photoresist-contamination}A fluorescence image of a
SU8-encapsulated carbon nanotube device is shown in (a), while (b) shows
the same channel after modification with an solution of 1 mM
Pyrene-PEG-FITC. A 0.35 s exposure time and FITC filter were used for
(a)-(b). The fluorescence image in (c) shows an unencapsulated channel,
while (d) shows the same channel after Cy3-tagged aptamer exposure. A 10
s exposure time and mCherry filter were used for (c)-(d). The
fluorescence images in (e) and (f) show devices pre-coated with a thin
layer of photoresist then submerged in Cy3-tagged aptamer, where the
device in (f) was hardbaked before aptamer exposure. An mCherry filter
and 30 s exposure time were used for for (e)-(f).}

\end{figure}

An functionalisation issue quickly encountered when characterising
pyrene-PEG-FITC (PPF) interaction with sensing channels via fluorescence
microscopy was an unwanted secondary interaction between the linker and
residual photoresist. Figure~\ref{fig-350ms-SU8} and
Figure~\ref{fig-350ms-SU8-FITC} are fluorescence images of SU8
encapsulation (using the pre-2023 mask) before and after being exposed
to PPF. Despite the same microscope settings being used to take the
images (filter, ISO, contrast, exposure time), the SU8 exposed to PPF
appears much brighter than the pristine SU8. This result indicates that
the linker appears to have an extensive interaction with the photoresist
via an unknown mechanism. No fluorescence is seen from the device
channel. The length of exposure time required to see fluorescence from
the modified channel would lead to fluorescence from the modified linker
attached to the photoresist \(-\) as well as the photoresist itself
\(-\) flooding the image with light. Therefore, it is not clear whether
the carbon nanotubes have been functionalised with the dye-modified
linker. However, out of caution we can assume that the presence of this
secondary interaction is not desirable.

A similar interaction was seen between AZ\(^\circledR\) 1518 photoresist
and fluorescent-tagged, amine-terminated aptamer. An unencapsulated
carbon nanotube network device, fabricated using the pre-June 2022
process outlined in \textbf{?@sec-fabrication}, was incubated with 500
nM Cy3-tagged aptamer in Tris buffer at 4°C overnight. The aptamer was
first denatured by heating in a water bath at 95°C for 5 minutes then
cooling in an ice bath for 10 minutes before use.

Figure~\ref{fig-aptamer-photoresist-1} and
Figure~\ref{fig-aptamer-photoresist-2} are fluorescence images of the
device channel region before and after exposure to aptamer. A thick red
ring is visible around the electrodes after functionalisation, despite
no PBASE being used to tether the amine-terminated aptamer. It appears
that these bright patches correspond to patches of residual photoresist
which have not been completely removed from the carbon nanotube square
by the development process. These patches have then interacted with the
aptamer, causing them to appear bright under the fluorescence
microscope. Beyond potentially interfering with functionalisation,
photoresist residue blocking a device channel will prevent interaction
with the buffer and prevent sensing.

To test whether residual resist could be prevented from interacting with
aptamer by crosslinking the resist, two unencapsulated devices were
prepared as follows. Devices were first spincoated with AZ\(^\circledR\)
1518 in the manner described in \textbf{?@sec-fabrication}. Next, the
majority of resist was removed by soaking the device in acetone for 1
minute. This process left a thin coating of photoresist on the devices.
One of these devices was then hardbaked at 200°C for 1 hour. Both
devices were subsequently functionalised in the following manner:

\begin{enumerate}
\def\labelenumi{\arabic{enumi}.}
\item
  Unencapsulated device was submerged in 1 mM PBASE in methanol solution
  for 1 hour.
\item
  The device was then rinsed with methanol and Tris buffer.
\item
  1 \(\mu\)M Cy3-tagged aptamer was denatured by heating in a water bath
  at 95°C for 5 minutes then cooling in an ice bath for 10 minutes
  before use.
\item
  The device was incubated with aptamer in Tris buffer at 4°C overnight.
\end{enumerate}

Fluorescence microscope images of channels from each device are shown in
Figure~\ref{fig-aptamer-photoresist-3} and
Figure~\ref{fig-aptamer-photoresist-4}, where the latter is the device
hardbaked before functionalisation. By comparing the two images, it is
apparent that hardbaking the AZ\(^\circledR\) 1518 photoresist
significantly reduces the amount of fluorescent aptamer attached to the
surface. This result is an indication that sufficient heating of the
photoresist can prevent it interacting with PBASE or amine-tagged
biological material. However, there is still some Cy3-tagged aptamer
fluorescence visible in Figure~\ref{fig-aptamer-photoresist-4}. It
appears that hardbaking has not completely prevented photoresist from
interacting with the aptamer. It is possible that heating from the
bottom of the device is insufficient to hardbake the photoresist layer
completely, an effect that would be amplified for the thick photoresist
layer on encapsulated devices. Therefore, from June 2023 onwards devices
were vacuum annealed for 1 hour at 150°C prior to functionalisation.
This approach was taken to ensure photoresist was heated from above as
well as below and made chemically inert across its surface.

This result demonstrates the use of fluorescence microscopy as a tool to
detect residue and test suitable residue elimination measures. Further
testing showed that performing a 1 minute flood exposure (for positive
resist only) then placing a device in AZ\(^\circledR\) 326 developer for
3 minutes was highly effective at removing photoresist residue. Both
these development and annealing techniques were used for all
functionalised devices in subsequent sections.

\hypertarget{sec-hydrophobicity}{%
\subsection{Hydrophobicity of Carbon Nanotubes and
Graphene}\label{sec-hydrophobicity}}

\begin{figure}

\begin{minipage}[t]{0.47\linewidth}

{\centering 

\raisebox{-\height}{

\includegraphics{figures/ch7/modified_NGW8D9_edgechannel_post10min50Cacetonerinse_221109.png}

}

}

\subcaption{\label{fig-no-attachment}}
\end{minipage}%
%
\begin{minipage}[t]{0.05\linewidth}

{\centering 

~

}

\end{minipage}%
%
\begin{minipage}[t]{0.47\linewidth}

{\centering 

\raisebox{-\height}{

\includegraphics{figures/ch7/modified_NGW8D7_edgechannel_lowerexposuretime_221108.png}

}

}

\subcaption{\label{fig-attachment-post-plasma}}
\end{minipage}%

\caption{\label{fig-hydrophobicity}Fluorescence images of a 1000
\(\mu\)m \(\times\) 100 \(\mu\)m graphene channel after
functionalisation with 1 mM pyrene-PEG-rhodamine in 1XPBS. The graphene
film in (a) was not oxygen plasma cleaned before functionalisation,
while the graphene film in (b) was oxygen plasma cleaned at 5 W for 15 s
at 300 mTorr pressure immediately before functionalisation. Insets show
a 10 \(\mu\)L droplet placed on an unencapsulated carbon nanotube device
before (a) and after (b) the same oxygen plasma treatment procedure.
Images were taken using an Texas Red filter and a 1.8 s exposure time.}

\end{figure}

\begin{figure}

\begin{minipage}[t]{0.47\linewidth}

{\centering 

\raisebox{-\height}{

\includegraphics{figures/ch7/modified_NGW6D4_PyPEGFITC_channel2_1.6sexposure_10X_221122.png}

}

}

\subcaption{\label{fig-FITC-silicon-dioxide}}
\end{minipage}%
%
\begin{minipage}[t]{0.05\linewidth}

{\centering 

~

}

\end{minipage}%
%
\begin{minipage}[t]{0.47\linewidth}

{\centering 

\raisebox{-\height}{

\includegraphics{figures/ch7/modified_SiO2_1mMPPR_2_221109.png}

}

}

\subcaption{\label{fig-rhodamine-silicon-dioxide}}
\end{minipage}%
\newline
\begin{minipage}[t]{0.47\linewidth}

{\centering 

\raisebox{-\height}{

\includegraphics{figures/ch7/modified_NGW6D7_PyPEGFITC_channel3top_postmsurfclean_7.5sexposure_20X_221123.png}

}

}

\subcaption{\label{fig-PPF-PBS-20X}}
\end{minipage}%
%
\begin{minipage}[t]{0.05\linewidth}

{\centering 

~

}

\end{minipage}%
%
\begin{minipage}[t]{0.47\linewidth}

{\centering 

\raisebox{-\height}{

\includegraphics{figures/ch7/modified_NGW6D7_PyPEGFITC_channel1_postmsurfclean_7.75sexposure_40X_221123.png}

}

}

\subcaption{\label{fig-PPF-PBS-40X}}
\end{minipage}%

\caption{\label{fig-silicon-dioxide-interaction}The 1000 \(\mu\)m
\(\times\) 100 \(\mu\)m graphene film in image (a) was functionalised
with 1 mM pyrene-PEG-FITC in 1XPBS after oxygen plasma treatment, taken
using an FITC filter and a 1.6 s exposure time. (b) shows a silicon
dioxide surface which had never been exposed to carbon nanotubes,
graphene or photoresist after exposure to 1 mM pyrene-PEG-rhodamine in
1XPBS, taken using a Texas Red filter and a 1.8 s exposure time.
Graphene films on a substrate functionalised with 1 mM pyrene-PEG-FITC
in 1XPBS after oxygen plasma treatment then cleaned with m-CNT
dispersion surfactant (NanoIntegris) are shown in (c) and (d), where a
FITC filter was used, with 7.5 s and 7.75 s exposure times
respectively.}

\end{figure}

As PEGlyated linker dissolves well in aqueous solution, initial
fluorescence imaging focused on functionalising devices with these
linkers dissolved in 1XPBS. It was hoped that by keeping the device
channels in a pH-controlled environment, the channel surface would be
made more suitable for the attached receptors.
Figure~\ref{fig-no-attachment} shows a graphene film after exposure to
pyrene-PEG-rhodamine (PPR) in 1XPBS solution for 1 hour. The
pyrene-PEG-rhodamine has interacted with the silicon dioxide substrate
(discussed further in Section~\ref{sec-pyrene-interactions}) but not the
graphene film. The graphene has not attached to the pyrene or rhodamine
due to the highly hydrophobic graphene surface repelling the surrounding
solution, preventing \(\pi\)-stacking from occurring. The hydrophobicity
of the graphene surface is not intrinsic to graphene, but instead
results from a hydrocarbonaceous layer which forms on the channel
surface when exposed to air \autocite{Ashraf2014}. A hydrophobic layer
wiil also form on carbon nanotube networks
\autocite{Stando2019,Park2022}. Treatment with oxygen plasma at 5 W for
15 s has previously been found to remove this hydrocarbonaceous layer,
restoring the intrinsic hydrophilicity of graphene \autocite{Shin2010}.
Storing the graphene surface in deionised water rather than air prevents
the return of this hydrocarbon layer \autocite{Ashraf2014}. The use of a
relatively low power plasma ensures damage to the graphene layer is
minimised.

Treatment of an unencapsulated carbon nanotube network device at 5 W for
15 s at 300 mTorr greatly reduced the contact angle of a water droplet
placed on the device surface, shown inset in
Figure~\ref{fig-hydrophobicity} before and after plasma treatment. A
graphene film was then functionalised with pyrene-PEG-rhodamine in 1XPBS
in the same manner as for the film in Figure~\ref{fig-no-attachment},
except with the same plasma treatment performed on the film less than 1
minute before functionalisation. The result is shown in
Figure~\ref{fig-attachment-post-plasma}. The graphene now appears to
interact with the pyrene-PEG-rhodamine. These results both indicate that
the plasma treatment is increasing the hydrophilicity of the device
surface, improving the ability of pyrene-PEG-rhodamine to \(\pi\)-stack
with graphene. The disadvantage of this procedure is that the plasma
cleaning introduces defects to the graphene surface which may be
undesirable for device electrical behaviour. Furthermore, it was often
found that devices functionalised in this manner had their conductance
drop significantly after functionalisation, even though plasma treatment
itself did not significantly alter device conductance. Solvent was
therefore used for the initial linker functionalisation in
Section~\ref{sec-linker-receptor-attachment}, as it did not require a
plasma cleaning step for successful attachment.

\hypertarget{sec-pyrene-interactions}{%
\subsection{Substrate Interaction with Linker
Molecules}\label{sec-pyrene-interactions}}

Another issue that arose when verifying surface functionalisation was
the interaction between pyrene linker and the silicon dioxide substrate.
This interaction meant it was difficult to discern whether the pyrene
group was interacting in a specific manner with the channel film. It was
confirmed that pyrene-PEG was interacting with silicon dioxide, rather
than residual photoresist or nanomaterial, by performing a
pyrene-PEG-rhodamine functionalisation on pristine silicon dioxide, as
shown in Figure~\ref{fig-rhodamine-silicon-dioxide}. The PEGlyated
linker supplier suggested that the surface should be thoroughly rinsed
with surfactant to remove weakly-bound pyrene-PEG-FITC attached to the
silicon dioxide, while preserving the pyrene-PEG-FITC strongly attached
via \(\pi\)-stacking to the graphene or carbon nanotube film
\autocite{CreativePEGworks2022}. The following process was then used to
remove pyrene-PEG-FITC from the silicon dioxide: the film was rinsed
with DI water for 30 s, then placed in m-CNT dispersion solution
(NanoIntegris) for 5 minutes at 70°C while agitating with a pipette, and
finally rinsed with DI water, ethanol, acetone, IPA and nitrogen dried.
The results of this thorough cleaning process are shown in
Figure~\ref{fig-PPF-PBS-20X} and Figure~\ref{fig-PPF-PBS-40X}. The
majority of pyrene-PEG-FITC was removed in regions with no graphene, but
remained where graphene was present, indicating specific,
\(\pi\)-stacking interaction took place between the pyrene-PEG-FITC and
graphene. However, this surfactant rinse step was not used when
performing functionalisation with biological materials, to prevent
damage to the lipid membranes used.

\hypertarget{sec-coffee-ring}{%
\subsection{Coffee-Ring Effect}\label{sec-coffee-ring}}

\begin{figure}

\begin{minipage}[t]{0.47\linewidth}

{\centering 

\raisebox{-\height}{

\includegraphics{figures/ch7/modified_GFPNi2+1_5sexposure_12.6X_mediumcontrast_18_230324.png}

}

}

\subcaption{\label{fig-GFP-coffee-ring-1}}
\end{minipage}%
%
\begin{minipage}[t]{0.05\linewidth}

{\centering 

~

}

\end{minipage}%
%
\begin{minipage}[t]{0.47\linewidth}

{\centering 

\raisebox{-\height}{

\includegraphics{figures/ch7/modified_GFPNi2+1_5sexposure_12.6X_mediumcontrast_20_230324.png}

}

}

\subcaption{\label{fig-GFP-coffee-ring-2}}
\end{minipage}%

\caption{\label{fig-GFP-coffee-ring}Both (a) and (b) show a build-up of
his-tag GFP at the edges of the droplet region where pyrene-PEG-NTA had
been present, taken using an GFP filter and a 5 s exposure time. On the
right hand side of (b), no his-tag GFP is visible on the metal
electrode, as no pyrene-PEG attaches to the metal electrodes.}

\end{figure}

From Table~\ref{tbl-pbase-functionalisation}, full device submersion
appears to be the most common approach for functionalisation with
solution containing linker molecules like PBASE. However, some groups
placed small droplets of solution onto the device channels when
functionalisation, and this approach was tested as part of the
fluorescence verification work. For functionalisation with his-tagged
green fluorescent protein, after plasma cleaning at 5 W for 15 s at 300
mTorr, a 4 \(\mu\)L droplet of 100\(\mu\)M pyrene-PEG-NTA in 1XPBS was
placed on each graphene device channel and left covered in a humid
environment for 15 minutes. The device was then rinsed with 1XPBS,
submerged in 10 mM NiSO\(_4\) in 1XPBS for 1 hour, rinsed in 1XPBS, then
submerged in 10 mL of 100 ng/mL his-tag GFP solution (Thermofisher)
overnight. Fluorescence microscope imaging showed that a ring of
biomaterial would build up around the outer edge of regions where
pyrene-PEG-NTA had been present, as seen in
Figure~\ref{fig-GFP-coffee-ring}.

It appears this is a result of the his-tag GFP attaching to a dense
region of pyrene-PEG-NTA at the edge of the functionalisation droplet.
This accretion of pyrene-PEG-NTA at the edge of the droplet is a result
of the coffee-ring effect, where the evaporation of the droplet leads to
transport of particles to the droplet edges via capillary flow
\autocite{Deegan1997,Shimobayashi2018}. As this gradient in surface
coverage of attached proteins has unknown consequences for sensing, in
subsequent sections devices were functionalised with PBASE or
pyrene-PEG-NTA by submerging them in solution instead of dropcasting.

\hypertarget{sec-linker-receptor-attachment}{%
\section{Verifying Linker-OR Nanodisc Attachment with Fluorescence
Microscopy}\label{sec-linker-receptor-attachment}}

\hypertarget{sec-PBASE-GFP-OR-attachment}{%
\subsection{GFP-OR Nanodisc
Functionalisation}\label{sec-PBASE-GFP-OR-attachment}}

To verify the formation of amide (or imide) bonds between PBASE and the
odorant receptors (ORs) contained within nanodiscs, a fluorescent
biomarker was directly attached to the odorant receptors for detection
with fluorescence microscopy. The biomarker used was the \emph{Aequorea
Victoria} green fluorescent protein (GFP). The functionalisation of
unencapsulated carbon nanotube devices (steam-deposited, fabricated
using post-June 2023 methods outlined in \textbf{?@sec-fabrication})
with PBASE and GFP-OR nanodiscs was performed as follows:

\begin{enumerate}
\def\labelenumi{\arabic{enumi}.}
\item
  The device was exposed to UV light for 1 minute, placed in
  AZ\(^\circledR\) 326 developer for 3 minutes, then rinsed with
  acetone, isopropanol and nitrogen dried.
\item
  The device was vacuum annealed for 1 hour at 150°C (Note: Steps 1 \& 2
  were added to ensure any residual photoresist on the channel was
  removed or passivated before functionalisation, see
  Section~\ref{sec-photoresist-contamination}).
\item
  A solution of 1 mM PBASE (Setareh Biotech) in methanol prepared by
  fully dissolving 2 mg PBASE in 5 mL methanol by vortex mixing at 1000
  rpm in a dark room (Note: PBASE was stored at -18°C for 18 months
  prior to use, and was thawed under vacuum for 15 minutes in dark
  conditions before opening)
\item
  The device was then rinsed with methanol, fully submerged in \(\sim\)
  1 mL of PBASE in methanol solution and left covered with parafilm for
  1 hour, then rinsed with methanol for 15 s, rinsed with 1XPBS for 15 s
  and nitrogen dried to remove residual PBASE.
\item
  The device was left dry and in darkness while collecting the GFP-OR
  nanodiscs from the -80°C freezer. An opaque cover was placed over the
  GFP-OR vial to shield it from light.
\item
  20 \(\mu\)L GFP-OR nanodiscs (batch number ND-GFP-OR43b-0002, prepared
  12 months earlier) were diluted in 2 mL 1XPBS (Note: The full 2 mL was
  used to flush out the nanodisc vial when preparing the nanodisc
  solution, with successive additions and subtractions of 50 \(\mu\)L
  1XPBS into and from the vial).
\item
  The device was submerged in the GFP-OR43b nanodisc solution and left
  covered with parafilm for 1 hour in darkness, then rinsed with 1XPBS
  for 15 s.
\item
  For fluorescence microscopy, the device was briefly rinsed with DI
  water and nitrogen dried to remove dried-down salt residue left by the
  1XPBS.
\end{enumerate}

A control device was prepared using the same process but skipping steps
3 and 4.

\hypertarget{sec-PBASE-GFP-OR-microscopy}{%
\subsection{Fluorescence Microscopy}\label{sec-PBASE-GFP-OR-microscopy}}

\begin{figure}

\begin{minipage}[t]{0.47\linewidth}

{\centering 

\raisebox{-\height}{

\includegraphics{figures/ch7/modified_GFPOR_10sexposure_20X_mediumcontrast_ch6_240208.png}

}

}

\subcaption{\label{fig-GFP-OR-ch6-zoom}}
\end{minipage}%
%
\begin{minipage}[t]{0.05\linewidth}

{\centering 

~

}

\end{minipage}%
%
\begin{minipage}[t]{0.47\linewidth}

{\centering 

\raisebox{-\height}{

\includegraphics{figures/ch7/modified_GFPOR_PBASE_10sexposure_20X_mediumcontrast_ch1_231019_2.png}

}

}

\subcaption{\label{fig-PBASE-GFP-OR-ch1-zoom}}
\end{minipage}%
\newline
\begin{minipage}[t]{0.47\linewidth}

{\centering 

\raisebox{-\height}{

\includegraphics{figures/ch7/modified_GFPOR_10sexposure_40X_mediumcontrast_ch6_240208.png}

}

}

\subcaption{\label{fig-GFP-OR-ch6}}
\end{minipage}%
%
\begin{minipage}[t]{0.05\linewidth}

{\centering 

~

}

\end{minipage}%
%
\begin{minipage}[t]{0.47\linewidth}

{\centering 

\raisebox{-\height}{

\includegraphics{figures/ch7/modified_GFPOR_PBASE_10sexposure_40X_mediumcontrast_ch1_231019_2.png}

}

}

\subcaption{\label{fig-PBASE-GFP-OR-ch1}}
\end{minipage}%
\newline
\begin{minipage}[t]{0.47\linewidth}

{\centering 

\raisebox{-\height}{

\includegraphics{figures/ch7/modified_GFPOR_10sexposure_40X_mediumcontrast_ch3_240208.png}

}

}

\subcaption{\label{fig-GFP-OR-ch3}}
\end{minipage}%
%
\begin{minipage}[t]{0.05\linewidth}

{\centering 

~

}

\end{minipage}%
%
\begin{minipage}[t]{0.47\linewidth}

{\centering 

\raisebox{-\height}{

\includegraphics{figures/ch7/modified_GFPOR_PBASE_10sexposure_40X_mediumcontrast_ch2_231019.png}

}

}

\subcaption{\label{fig-PBASE-GFP-OR-ch2}}
\end{minipage}%

\caption{\label{fig-PBASE-GFP-ORs}The fluorescence images on the left
side \(-\) (a), (c) and (e) \(-\) show unencapsulated carbon nanotube
network channels from a device incubated in GFP-OR43b nanodiscs. The
rectangular dark regions to the left and right of each image are the
gold electrodes. The fluorescence images on the right \(-\) (b), (d) and
(f) \(-\) show the channels of a similar device after successive PBASE
and GFP-OR43b nanodisc incubation. Images (a) and (c) are both of the
same channel on the first device, and images (b) and (d) are of the same
channel on the second, but (c) and (d) were taken using a greater
magnification. The insets in (c)-(f) compare the central channel region
of (c)-(f) more directly. All images were taken with the same microscope
settings (GFP filter and 10 s exposure time), directly after
functionalisation in a dark room.}

\end{figure}

Fluorescence images of the GFP-OR43b and control devices discussed in
Section~\ref{sec-PBASE-GFP-OR-attachment} are shown in
Figure~\ref{fig-PBASE-GFP-ORs}. Fluorescence microscope images were
taken immediately after functionalisation; devices were transported to
the fluorescence microscope room in a foil-wrapped container, and the
fluorescence microscope room was kept dark while images were taken. All
fluorescence images were taken of channels with ungated resistance
measurements within a 50-500 k\(\Omega\) range both before and after
functionalisation.

The silicon dioxide regions in each image appear bright under the GFP
filter, indicating non-specific binding between the GFP-OR43b nanodiscs
and the silicon dioxide substrate. As this device has been annealed, UV
exposed and developed before functionalisation, this non-specific
attachment is unlikely to be interaction with residual photoresist (see
Section~\ref{sec-photoresist-contamination}). The SiO\(_2\) substrate
also appears brighter in the images on the right of
Figure~\ref{fig-PBASE-GFP-ORs}, which are of the device initially
exposed to PBASE. The discussion in
Section~\ref{sec-pyrene-interactions} indicates that the pyrene moiety
of PBASE non-specifically interacts with the silicon dioxide substrate.
The attachment of GFP-OR43b nanodiscs to this PBASE coating appears have
led to more GFP-OR43b nanodiscs attaching to the silicon dioxide, giving
rise to the brighter fluorescence of the silicon dioxide seen for the
PBASE-incubated device on the right of Figure~\ref{fig-PBASE-GFP-ORs}.

A comparison of fluorescence in the channel region between images on the
left of Figure~\ref{fig-PBASE-GFP-ORs} (GFP-OR43b only) and the images
on the right (GFP-OR43b and PBASE) is given by the inset in
Figure~\ref{fig-GFP-OR-ch6}-f.~The inset demonstrates that the channels
not incubated in PBASE are significantly less bright than those that had
been incubated with PBASE. It appears that, as expected from the
discussion in Section~\ref{sec-hydrophobicity}, the GFP-OR43b in 1XPBS
is unable to approach the unmodified channel due to the hydrophobicity
of the carbon nanotubes. However, when the carbon nanotubes are modified
with PBASE, the GFP-OR43b is able to attach to the channel, and so the
channel shows up brightly under the fluorescence microscope GFP filter.
This trend was consistent across all conducting channels on each of the
two devices. As far as I know, this is the first time fluorescence has
been used to verify the successful attachment of odorant receptor
nanodiscs to a carbon nanotube network.

\hypertarget{sec-conclusion}{%
\section{Conclusion}\label{sec-conclusion}}

It has been well-established in the literature that the \(\pi\)-stacking
reaction mechanism between pyrene-based linkers and graphene and carbon
nanotube network field-effect transistors can be used to create working
biosensors. The previous use of various linker molecules for biosensor
functionalisation was investigated. Despite the wide use of
1-pyrenebutanoic acid N-hydroxysuccinimide ester (PBASE) and
1-pyrenebutyric acid (PBA) for functionalisation of biosensors, the
literature shows a significant variation in the methods used for
attachment of linker molecules to a transistor channel. The most common
methods, using 6 mM PBASE dissolved in dimethylformamide or 1 mM PBASE
in methanol, stem directly from the first documented use of PBASE for
functionalisation of carbon nanotube biosensors. In the last 6 years,
more research has been done into optimising the PBASE methodology for
graphene devices, but there is still disagreement in the literature over
whether minimising or maximising PBASE coverage on a graphene device
channel is desirable for sensing. Due to disagreement in the literature
around suitable non-covalent methods for biosensor functionalisation,
several steps were taken to identify a rapid and simple method for
verifying successful functionalisation, and to locate any potential
barriers to a successful functionalisation.

I first compared the advantages and disadvantages of the various linker
molecules under investigation. The use of hydrogen NMR gave indications
that water was present in PBASE samples prepared in DMSO. Concerns
around the impact of the hydrolysis of PBASE on functionalisation mean
that the presence of water is strongly undesirable. An alternative
functionalisation approach less prone to hydrolysis is the reaction of
PBA with EDC in the presence of NHS. However, this process has its own
disadvantages, such as undesirable protein interactions and the
increased amount of steps and process variables involved. Pyrene-NTA is
also less prone to hydrolysis than PBASE but unlike PBASE or PBA/EDC
interacts with a specific protein tag, the histidine tag. PEGlyation of
the pyrene-NTA linker also means that the entire functionalisation
process can be performed in aqueous solution, avoiding the introduction
of non-organic solvents. This approach is desirable, since the
non-aqueous solvents traditionally used for functionalisation may have
negative impacts on device behaviour. For example, carbon nanotube
device channel transfer characteristics were found to undergo a
significant shift of \(\Delta V = -0.15 \pm 0.02\) when exposed to DMSO
or MeOH for 1 hour.

Next, I verified that the pyrene groups of the linker molecules of
interest were attaching successfully to either carbon nanotubes or
graphene. Raman spectroscopy showed that incubating a highly-bundled
carbon nanotube film in 5 mM PBASE or PBA in DMSO for 1 hour increased
I\(_D\)/I\(_G\) by a factor of \(\sim 3\) relative to the DMSO-only
case. Incubating a steam-deposited carbon nanotube device in a 1 mM
concentration of PBASE in methanol or DMSO for 1 hour was found to cause
a significant increase in device on-current relative to the solvent-only
case, and a similar increase in on-current was seen for 5 mM PBA in DMSO
relative to the DMSO-only case. When a PBA-functionalised device was
placed in aqueous solution with 20 mM EDC and 40 mM NHS for 30 minutes,
a further increase in on-current was seen. Fluorescence microscopy was
used to demonstrate the successful attachment of pyrene-PEG to graphene
using an attached FITC probe, where immersing a graphene film in 1 mM
pyrene-PEG in ethanol led to the channels becoming brightly fluorescent
relative to the background using a 1 s exposure time.

Various obstacles to successful functionalisation were encountered and
addressed. Photoresist contamination was addressed with exposure and
development steps before functionalisation (no exposure for SU8
encapsulated devices). Hydrophobicity of graphene films was addressed by
plasma treatment before functionalisation in aqueous solution. A
surfactant rinse was used to distinguish between weak substrate-linker
interaction and \(\pi\)-stacking between linker and the channel.
Finally, coffee-ring distribution of linker was addressed by always
submerging the device in linker when functionalising.

Finally, fluorescence microscopy was used to investigate PBASE
functionalisation of GFP-tagged odorant receptors. An eight-channel
device was modified by submersion in 1 mM PBASE in methanol for 1 hour,
then submersion in 10 \(\mu\)L mL\(^{-1}\) OR43b nanodiscs in 1XPBS for
1 hour. An eight-channel control device was also prepared by submersion
in 10 \(\mu\)L mL\(^{-1}\) OR43b nanodiscs in 1XPBS for 1 hour, but with
no PBASE. The channels of the PBASE-submersed devices showed significant
GFP fluorescence, while the channels of the control devices showed
little to no GFP fluorescence. As far as I am aware, this is the first
time fluorescence has been used to verify successful attachment of
odorant receptors to a carbon nanotube network.

\bookmarksetup{startatroot}

\hypertarget{biosensing-with-insect-odorant-receptor-functionalised-carbon-nanotube-and-graphene-devices}{%
\chapter{Biosensing with Insect Odorant-Receptor Functionalised Carbon
Nanotube and Graphene
Devices}\label{biosensing-with-insect-odorant-receptor-functionalised-carbon-nanotube-and-graphene-devices}}

\hypertarget{introduction-2}{%
\section{Introduction}\label{introduction-2}}

\hypertarget{aqueous-sensing-of-ethyl-hexanoate-with-or22a-functionalised-carbon-nanotube-transistor}{%
\section{Aqueous Sensing of Ethyl Hexanoate with OR22a-functionalised
Carbon Nanotube
Transistor}\label{aqueous-sensing-of-ethyl-hexanoate-with-or22a-functionalised-carbon-nanotube-transistor}}

\hypertarget{or-nanodisc-functionalisation}{%
\subsection{OR Nanodisc
Functionalisation}\label{or-nanodisc-functionalisation}}

A carbon nanotube network field-effect transistor device, fabricated
using post-June 2023 methods as described in \textbf{?@sec-fabrication},
was functionalised with OR22a nanodiscs. The network used for the device
was deposited using the steam-assisted surfactant method, and the device
was encapsulated with AZ\(^\circledR\) 1518 using the post-Jan 2023
photolithography mask. The functionalisation was performed as follows:

\begin{enumerate}
\def\labelenumi{\arabic{enumi}.}
\item
  The device was exposed to UV light for 1 minute, placed in
  AZ\(^\circledR\) 326 developer for 3 minutes, then rinsed with
  acetone, isopropanol and nitrogen dried.
\item
  The device was vacuum annealed for 1 hour at 150°C.
\end{enumerate}

Note: Steps 1 \& 2 were added to ensure any residual photoresist on the
channel was removed or passivated before functionalisation, see
Section~\ref{sec-photoresist-contamination}.

\begin{enumerate}
\def\labelenumi{\arabic{enumi}.}
\setcounter{enumi}{2}
\tightlist
\item
  A solution of 1 mM PBASE (Setareh Biotech) in methanol was prepared by
  fully dissolving 2 mg PBASE in 5 mL methanol by vortex mixing at 1000
  rpm in a dark room.
\end{enumerate}

Note: PBASE was stored at -18°C for 18 months prior to use, and was
thawed under vacuum for 15 minutes in dark conditions before opening.

\begin{enumerate}
\def\labelenumi{\arabic{enumi}.}
\setcounter{enumi}{3}
\item
  The device was then rinsed with methanol, fully submerged in \(\sim\)
  1 mL of PBASE in methanol solution and left covered with parafilm for
  1 hour, then rinsed with methanol for 15 s, rinsed with 1XPBS for 15 s
  and nitrogen dried to remove residual PBASE.
\item
  The device was left dry and in darkness while collecting the OR22a
  nanodiscs from the -80°C freezer.
\item
  10 \(\mu\)L OR22a nanodiscs (batch number ND-OR22a-SB018, 1.9 mg/mL,
  prepared 7 months earlier) were diluted in 1 mL 1XPBS
\end{enumerate}

Note: The full 1 mL was used to flush out the nanodisc vial when
preparing the nanodisc solution, with successive additions and
subtractions of 50 \(\mu\)L 1XPBS into and from the vial.

\begin{enumerate}
\def\labelenumi{\arabic{enumi}.}
\setcounter{enumi}{6}
\tightlist
\item
  The device was submerged in the OR22a nanodisc solution and left
  covered with parafilm for 1 hour, then rinsed with 1XPBS for 15 s and
  gently nitrogen dried.
\end{enumerate}

Liquid-gated electrical characteristics were taken of the sensing
channel (channel 7)before and after functionalisation with OR22a. These
electrical characteristics were taken in using a liquid gate buffer of
1XPBS containing 0.5\% DMSO with the B1500A semiconductor device
analyser. These characteristics are shown in
Figure~\ref{fig-OR22a-TX-comparison}, shown using both a logarithmic and
linear current scale. The device exhibited ambipolar characteristics
before functionalisation, which is typically seen for steam-deposited
carbon nanotube films (Section~\ref{sec-cnt-devices}). However,
\(p\)-type behaviour dominates after device functionalisation due to a
significant drop in \(n\)-type conductance. There was little hysteresis
present, which is also typical for these devices. A slight increase in
hysteresis was observed post-functionalisation. Leakage current (shown
by the dashed traces) never exceeds \(1 \times 10^{-7}\) V, both before
and after functionalisation. The significant change in electrical
characteristics observed could be due to five possible factors \(-\)
adsorption of solvent onto the network, network attachment of PBASE
without subsequent protein attachment, non-specific adsorption of
protein onto the network, PBASE-mediated attachment of the membrane
scaffold protein (MSP) of nanodiscs to the network, and PBASE-mediated
attachment of odorant receptors to the network. Note that as the
nanodisc volume is much larger than that of the odorant receptor, any
direct protein adsorption is highly likely to be adsorption of the
nanodisc membrane scaffold protein onto the carbon nanotube network.
Odorant receptor attachment via PBASE is therefore the only desirable
functionalisation result for sensing purposes.

\begin{figure}

\begin{minipage}[t]{0.50\linewidth}

{\centering 

\includegraphics{figures/ch8/Q1C6_ch7_absolute_values_with_gate_current_edited.png}
{}

}

\end{minipage}%
%
\begin{minipage}[t]{0.50\linewidth}

{\centering 

\includegraphics{figures/ch8/Q1C6_ch7_absolute_values_with_threshold_voltage_shift_without_gate_current_edited.png}
{}

}

\end{minipage}%

\caption{\label{fig-OR22a-TX-comparison}Liquid-gated carbon nanotube
network device transfer characteristics before and after OR22a nanodisc
functionalisation. In (a), the characteristics are shown on a
logarithmic scale, where the gate current for each transfer curve is
shown with a dashed line. In (b), the characteristics are shown on a
linear scale alongside a dashed line tangent to the subthreshold slope
of the characteristic curve. The threshold voltage corresponding to the
intercept of this slope with the x-axis is shown for each transfer
characteristic curve.}

\end{figure}

Only minor changes were observed in the on-off ratio when comparing the
device channel before and after functionalisation. The on-off ratio for
the pristine channel was \(1120\pm220\), fairly typical for a transfer
curve from a steam-assisted surfactant-deposited CNT network device (see
Section~\ref{sec-cnt-devices}). The on-off ratio increased slightly to
\(1830\pm550\) after functionalisation. We expect to see an increase in
on-off ratio for a device successfully functionalised with OR22a, which
may result from an increase in negative charge causing modulation of
Schottky barriers between metallic and semiconducting carbon nanotubes
within the network \autocite{Murugathas2019b}. However, we also expect
increased hole conductance from the attachment of PBASE, even without
proteins being present
(Section~\ref{sec-PBASE-electrical-characterisation}). It is therefore
difficult to determine whether functionalisation has been successful
from the on-off ratio of transfer characteristics alone.

Functionalisation of the channel resulted in a negative shift in
threshold voltage of \(-0.20\pm0.03\) V. This is significantly in excess
of threshold voltage shifts measured for both methanol adsorption
(\(-0.15\pm0.02\) V) and PBASE attachment (\(-0.06\pm0.04\) V),
confirming that protein has attached to the carbon nanotubes. However,
both direct protein adsorption \autocite{Bradley2004,Heller2008} and
empty nanodisc attachment \autocite{Murugathas2019b} also lead to the
same significant negative threshold voltage shift in the liquid-gated
transfer characteristic curve. In all three cases, the voltage shift is
predominantly the result of negative charge transfer from the adsorped
proteins to the semiconducting carbon nanotubes
\autocite{Bradley2004,Heller2008,Murugathas2019b}. It is likely that the
negative shift observed results from some combination of the three types
of attachment. It should be noted that while the size of the
functionalisation-induced threshold voltage shift can be used to
determine whether protein has attached to the nanodisc network, it
cannot be used to specifically determine whether odorant receptors have
attached to the network.

\begin{figure}

\begin{minipage}[t]{0.50\linewidth}

{\centering 

\includegraphics{figures/ch8/Pristine_DF2Q3D10_ 00141_edited.png} {}

}

\end{minipage}%
%
\begin{minipage}[t]{0.50\linewidth}

{\centering 

\includegraphics{figures/ch8/DF2Q3D9_AZ1518ENCAP_PostPBASE_ch6_00313_edited.png}
{}

}

\end{minipage}%
\newline
\begin{minipage}[t]{0.50\linewidth}

{\centering 

\includegraphics{figures/ch8/DF2Q1D6_postNDOR22a_ch7_1_00375_edited.png}
{}

}

\end{minipage}%
%
\begin{minipage}[t]{0.50\linewidth}

{\centering 

\includegraphics{figures/ch8/DF2Q1D6_postNDOR22a_ch7_2sline_2.5um_00379_edited.png}
{}

}

\end{minipage}%

\caption{\label{fig-working-OR22a-AFM}Atomic force microscope images of
the channel region of carbon nanotube network devices before and after
functionalisation. The channel network of a pristine device is shown in
(a), while (b) shows a network after exposure to PBASE in MeOH for 1
hour. The images in (c) and (d) are both of channel 7 from the sensing
device functionalised in this section.}

\end{figure}

Atomic force microscope images of the device channels both before
functionalisation and after sensing with the functionalised device to
confirm the presence of nanodiscs. These are the first atomic force
microscope images taken of OR nanodiscs found on a sensing channel
rather than on a separate carbon nanotube film
\autocite{Murugathas2019b}; the wider 20 \(\mu\)m encapsulation mask
discussed in \textbf{?@sec-encapsulation} made this possible. These
images are shown in Figure~\ref{fig-working-OR22a-AFM}. The position of
nanodiscs relative to the carbon nanotubes is less clearly seen here
than for the CNT films used by Murugathas \emph{et al.}, since the
network morphology used for these devices is much denser
\autocite{Murugathas2019b}. It is therefore difficult to determine
whether nanodiscs have preferentially attached to the carbon nanotubes.
Aggregations of nanodiscs are visible in
Figure~\ref{fig-working-OR22a-AFM} (c)\(-\)(d). These nanodisc clusters
are especially large in the lower half of
Figure~\ref{fig-working-OR22a-AFM} (c), however, these features are
still much smaller than the nanodisc aggregated nanodisc features seen
by Murugathas \emph{et al.}

\begin{figure}

\begin{minipage}[t]{0.50\linewidth}

{\centering 

\includegraphics{figures/ch8/DF2Q3D9_AZ1518ENCAP_PostPBASE_ch6_00313_mask_edited.png}
{}

}

\end{minipage}%
%
\begin{minipage}[t]{0.50\linewidth}

{\centering 

\includegraphics{figures/ch8/DF2Q1D6_postNDOR22a_ch7_1_00375_mask_edited.png}
{}

}

\end{minipage}%

\caption{\label{fig-working-OR22a-masks}The (a) and (b) masks}

\end{figure}

Using the method outlined in Vobornik \emph{et al.} as discussed in
Section~\ref{sec-pristine-AFM}, the average nanotube height of the
carbon nanotube network submersed in PBASE and methanol for 1 hour shown
in Figure~\ref{fig-working-OR22a-AFM} (b) is \(2.2\pm1.3\) nm. This
value is the same as the location parameter \(\xi\) for the skew normal
distribution found for a pristine steam-assisted surfactant-deposited
network (Section~\ref{sec-pristine-AFM}). The substrate mask used to
find the average height is shown in Figure~\ref{fig-working-OR22a-masks}
(a), with this background having an average height of \(8.1\pm0.4\) nm.
The substrate mask of the OR22a-functionalised device shown in
Figure~\ref{fig-working-OR22a-AFM} (c) is shown in
Figure~\ref{fig-working-OR22a-masks} (b). Using this mask, it was found
that the substrate height had an average height of \(4.4 \pm 1.2\) nm.
Subtracting the height of the substrate from the total height of the AFM
images in Figure~\ref{fig-working-OR22a-AFM} (c) and (d), a maximum
height of 21.5 nm for the carbon nanotubes plus the OR22a nanodiscs is
obtained across both images. The average height of the carbon nanotubes
can then be subtracted to find a height range of OR22a aggregates of
10\$-\$19 nm, where \$\sim\$10 nm is the minimum nanodisc height
\autocite{Nath2007,Bayburt2010,Murugathas2020}.

\hypertarget{aqueous-sensing-of-ethyl-hexanoate}{%
\subsection{Aqueous Sensing of Ethyl
Hexanoate}\label{aqueous-sensing-of-ethyl-hexanoate}}

The procedure used for biosensor detection of ethyl hexanoate in liquid
was the same as the procedure outlined in
Section~\ref{sec-dummy-sensing}, except 0.5\% DMSO was present in the
buffer solution (to improve ethyl hexanoate solubility) and dilutions of
ethyl hexanoate in the same 0.5\% DMSO 1XPBS buffer solution were added
during the sensing series. The 0.5\% DMSO 1XPBS was prepared by adding 5
\(\mu\)L of DMSO to 995 \(\mu\)L 1XPBS before device characterisation.
The dilutions of ethyl hexanoate were prepared with the same 1XPBS at
the same time, where 5 \(\mu\)L of 200 fM, 200 pM, 200 nM and 200
\(\mu\)M ethyl hexanoate in DMSO were placed into four individual vials
containing 995 \(\mu\)L 1XPBS each, giving 1mL vials of 1 fM, 1 pM, 1 nM
and 1 \(\mu\)M ethyl hexanoate in 0.5\% DMSO 1XPBS. The ethyl hexanoate
in DMSO dilutions were prepared beforehand as a 1:10 dilution series in
DMSO using 200 mM stock solution, where dilutions ranged from 20 mM to
200 fM. Sampling measurements were taken using the B1500A semiconductor
device analyser, with the transfer measurement in
Figure~\ref{fig-OR22a-TX-comparison} (b) taken directly before sensing.
The full control series plus sensing sequence is shown in
Figure~\ref{fig-EtHex-aqueous-sensing}. Gate current remained negligible
across the entire sensing procedure.

\begin{figure}

{\centering \includegraphics[width=0.7\textwidth,height=\textheight]{figures/ch8/Q1C6.png}

}

\caption{\label{fig-EtHex-aqueous-sensing}The control series (before
1800 s) and ethyl hexanoate sensing series (after 1800 s) of the
OR22a-functionalised device channel. No responses to 0.5\% DMSO 1XPBS
were seen during the control series, while significant responses to
additions of ethyl hexanoate diluted in 0.5\% DMSO 1XPBS were seen at
2400 s, 2700 s, 3000 s and 3600 s.}

\end{figure}

The control series for the sensing series is shown in
Figure~\ref{fig-OR22a-control-series} (a). No clear stepwise response is
seen to buffer additions or subtractions. The functionalised device
shows similar baseline drift behaviour to that of a pristine device,
with a period of short-term decay quickly yielding to a more long-lived
decay behaviour. A linear fit \(I = c_1t + c_2\) to the region
\(1200-1800\) had a gradient of \(c_1 = -1.76\pm0.02\) pA/s. This
gradient is smaller than the range of values found for the linear fit
approximating the longer-term drift of a pristine device
(Section~\ref{sec-baseline-drift}), but of the same order of magnitude.
The linear fit was then subtracted from the control series and an
exponential fit \(I = I_0\exp(-t/\tau)\) was performed on the remaining
dataset, as shown in Figure~\ref{fig-OR22a-control-series} (b). A value
of \(590 \pm 3\) s was found for the exponential time constant, similar
to those found for the channels of the pristine device. This confirms
that the 1800 s control series is sufficient to avoid the presence of
short-term decay during sensing.

\begin{figure}

\begin{minipage}[t]{0.15\linewidth}

{\centering 

~

}

\end{minipage}%
%
\begin{minipage}[t]{0.70\linewidth}

{\centering 

\includegraphics{figures/ch8/Q1C6_with_fitted_curves_edited.png} {}

}

\end{minipage}%
%
\begin{minipage}[t]{0.15\linewidth}

{\centering 

~

}

\end{minipage}%
\newline
\begin{minipage}[t]{0.15\linewidth}

{\centering 

~

}

\end{minipage}%
%
\begin{minipage}[t]{0.70\linewidth}

{\centering 

\includegraphics{figures/ch8/Q1C6_with_fitted_curves_exp_edited.png} {}

}

\end{minipage}%
%
\begin{minipage}[t]{0.15\linewidth}

{\centering 

~

}

\end{minipage}%

\caption{\label{fig-OR22a-control-series}The control series for the
OR22a-functionalised device is shown in (a), alongside an extrapolated
linear fit to the control series from 1200 s onwards. The control series
with the linear approximation subtracted fitted to an exponential curve
is shown in (b).}

\end{figure}

It appears that the exponential fit overestimates current measurements
between 1100 s and 1500 s and underestimates between 1500 s and 1800 s.
This deviation from the fit may result from the linear approximation
used to represent long-term baseline drift being weaker for this channel
than for those discussed previously in Section~\ref{sec-dummy-sensing}
and Section~\ref{sec-pristine-EtHex}. This could result from the
exponential terms for long-term baseline drift having relatively short
time constants, so \(t\ll\tau_i\) no longer holds and higher order terms
in the linear approximation are no longer negligible. This observation
may indicate a relationship exists between device functionalisation and
the long-lived device decay behaviour. However, it may simply result
from the natural variation between randomly-deposited device channels.
Further work may be required to confirm the existence of such a
relationship, though this work is outside the scope of this thesis.

\begin{figure}

\begin{minipage}[t]{0.15\linewidth}

{\centering 

~

}

\end{minipage}%
%
\begin{minipage}[t]{0.70\linewidth}

{\centering 

\includegraphics{figures/ch8/Q1C6_filtered_detrend_trunc_arrows_normalised_edited.png}
{}

}

\end{minipage}%
%
\begin{minipage}[t]{0.15\linewidth}

{\centering 

~

}

\end{minipage}%
\newline
\begin{minipage}[t]{0.15\linewidth}

{\centering 

~

}

\end{minipage}%
%
\begin{minipage}[t]{0.70\linewidth}

{\centering 

\includegraphics{figures/ch8/Q1C6_mean_simple_difference_before_and_after_step_filtered_concentrations_edited.png}
{}

}

\end{minipage}%
%
\begin{minipage}[t]{0.15\linewidth}

{\centering 

~

}

\end{minipage}%

\caption{\label{fig-OR22a-sensing-series}The normalised sensing series
for the OR22a-functionalised device is shown in (a). The current data
has been despiked, with baseline drift removed and a moving median
filter applied. The concentration of each 20 \(\mu\)L addition is
indicated above the time of addition. The signal data corresponding to
the mean difference in current before and after each addition is shown
in (b).}

\end{figure}

Figure~\ref{fig-OR22a-sensing-series} (a) shows the cleaned and filtered
ethyl hexanoate sensing data from the OR22a-functionalised device from
1800 s onwards. The concentration of each 20 \(\mu\)L addition is
indicated above the corresponding addition time. The source-drain
current across the channel decreased rapidly with each addition of ethyl
hexanoate in \(0.5%
\) DMSO 1XPBS solution. This current decrease appears irreversible, as
the current stabilises after each addition at a lower current level than
prior to the addition. This behaviour appears to be a response by OR22a
to its positive ligand ethyl hexanoate, similar to the response by OR22a
to methyl hexanoate seen by Murugathas \emph{et al.}. The presence of
the ORCO coreceptor was not required for responses to be seen. The
device showed responses to ethyl hexanoate over a wide range of
concentrations, beginning with a \(\sim 6%
\) response to 1 fM EtHex in \(0.5%
\) DMSO 1XPBS, while showing no response to 0.5\% 1XPBS buffer.
Interestingly, as seen in Figure~\ref{fig-OR22a-sensing-series} (b), no
clear dose-dependent response was observed. The behaviour seen may be
explained by a decreased sensitivity to subsequent additions seen by
seen by Murugathas \emph{et al.} \autocite{Murugathas2019b} competing
with the logarithmic increases in the concentration around the channel.

\hypertarget{variability-in-biosensor-behaviour}{%
\section{Variability in Biosensor
Behaviour}\label{variability-in-biosensor-behaviour}}

\hypertarget{sec-contamination}{%
\section{Potential Sources of Variability}\label{sec-contamination}}

\hypertarget{sec-cnt-deposition-effects}{%
\subsection{Surfactant Contamination}\label{sec-cnt-deposition-effects}}

\hypertarget{varying-cnt-network-deposition-approach}{%
\subsubsection*{Varying CNT network deposition
approach}\label{varying-cnt-network-deposition-approach}}
\addcontentsline{toc}{subsubsection}{Varying CNT network deposition
approach}

\hypertarget{functionalisation-of-graphene-devices}{%
\subsubsection*{Functionalisation of Graphene
Devices}\label{functionalisation-of-graphene-devices}}
\addcontentsline{toc}{subsubsection}{Functionalisation of Graphene
Devices}

\hypertarget{aggregation-of-odorant-receptor-nanodiscs}{%
\subsection{Aggregation of Odorant Receptor
Nanodiscs}\label{aggregation-of-odorant-receptor-nanodiscs}}

\hypertarget{solvent-contamination}{%
\subsection{Solvent Contamination}\label{solvent-contamination}}

\hypertarget{other-sources-of-variability}{%
\section{Other Sources of
Variability}\label{other-sources-of-variability}}

\hypertarget{vapour-sensing-with-empty-nanodiscs}{%
\section{Vapour Sensing with Empty
Nanodiscs}\label{vapour-sensing-with-empty-nanodiscs}}

\hypertarget{conclusion-1}{%
\section{Conclusion}\label{conclusion-1}}

\cleardoublepage
\phantomsection
\addcontentsline{toc}{part}{Appendices}
\appendix

\hypertarget{sec-vapour-sensor-components}{%
\chapter{Vapour System Hardware}\label{sec-vapour-sensor-components}}

\hypertarget{tbl-vapour-sensor-components}{}
\begin{longtable}[t]{>{\raggedright\arraybackslash}p{5.5cm}>{\raggedright\arraybackslash}p{4.5cm}>{\raggedright\arraybackslash}p{3.75cm}}
\caption{\label{tbl-vapour-sensor-components}Major components used in construction of the vapour delivery system
described in this thesis. }\tabularnewline

\toprule
Description & Part No. & Manufacturer\\
\midrule
Mass flow controller, 20 sccm full scale & GE50A013201SBV020 & MKS Instruments\\
Mass flow controller, 200 sccm full scale & GE50A013202SBV020 & MKS Instruments\\
Mass flow controller, 500 sccm full scale & FC-2901V & Tylan\\
Analogue flowmeter, 240 sccm max. flow & 116261-30 & Dwyer\\
Micro diaphragm pump & P200-B3C5V-35000 & Xavitech\\
\addlinespace
Analogue flow controller, for micro diaphragm pump & X3000450 & Xavitech\\
10 mL Schott bottle & 218010802 & Duran\\
PTFE connection cap system & Z742273 & Duran\\
Baseline VOC-TRAQ flow cell, red & 043-951 & Mocon\\
Humidity and temperature sensor & T9602 & Telaire\\
\addlinespace
Enclosure, for humidity and temperature sensor & MC001189 & Multicomp Pro\\
\bottomrule
\end{longtable}

\hypertarget{sec-python}{%
\chapter{Python Code for Data Analysis}\label{sec-python}}

\hypertarget{code-repository}{%
\section{Code Repository}\label{code-repository}}

The code used for general analysis of field-effect transistor devices in
this thesis was written with Python 3.8.8. Contributors to the code used
include Erica Cassie, Erica Happe, Marissa Dierkes and Leo Browning. The
code is located on GitHub and the research group OneDrive, and is
available on request.

\hypertarget{sec-histogram-analysis}{%
\section{Atomic Force Microscope Histogram
Analysis}\label{sec-histogram-analysis}}

The purpose of this code is to analyse atomic force microscope (AFM)
images of carbon nanotube networks in .xyz format taken using an atomic
force microscope and processed in Gwyddion (see
\textbf{?@sec-afm-characterisation}). It was originally designed by
Erica Happe in Matlab, and adapted by Marissa Dierkes and myself for use
in Python. The code imports the .xyz data and sorts it into bins 0.15 nm
in size for processing. To perform skew-normal distribution fits, both
\emph{scipy.optimize.curve\_fit} and \emph{scipy.stats.skewnorm} modules
are used in this code.

\hypertarget{sec-raman-analysis}{%
\section{Raman Spectroscopy Analysis}\label{sec-raman-analysis}}

The purpose of this code is to analyse a series of Raman spectra taken
at different points on a single film (see
\textbf{?@sec-raman-characterisation}). Data is imported in a series of
tab-delimited text files, with the low wavenumber spectrum (100
cm\(^{-1} - 650\) cm\(^{-1}\)) and high wavenumber spectrum (1300
cm\(^{-1} - 1650\) cm\(^{-1}\)) imported in separate datafiles for each
scan location.

\hypertarget{sec-field-effect-transistor-analysis}{%
\section{Field-Effect Transistor
Analysis}\label{sec-field-effect-transistor-analysis}}

The purpose of this code is to analyse electrical measurements taken of
field-effect transistor (FET) devices. Electrical measurements were
either taken from the Keysight 4156C Semiconductor Parameter Analyser,
National Instruments NI-PXIe or Keysight B1500A Semiconductor Device
Analyser as discussed in \textbf{?@sec-electrical-characterisation}; the
code is able to analyse data in .csv format taken from all three
measurement setups. The main Python file in the code base consists of
three related but independent modules: the first analyses and plots
sensing data from the FET devices, the second analyses and plots
transfer characteristics from channels across a device, and the third
compares individual channel characteristics before and after a
modification or after each of several modifications. The code base also
features a separate config file and style sheet which govern the
behaviour of the main code. The code base was designed collaboratively
by myself and Erica Cassie over GitHub using the Sourcetree Git GUI.


\backmatter
\printbibliography


\end{document}
