% Options for packages loaded elsewhere
\PassOptionsToPackage{unicode}{hyperref}
\PassOptionsToPackage{hyphens}{url}
%
\documentclass[
  a4paper,
]{scrbook}

\usepackage{amsmath,amssymb}
\usepackage{lmodern}
\usepackage{iftex}
\ifPDFTeX
  \usepackage[T1]{fontenc}
  \usepackage[utf8]{inputenc}
  \usepackage{textcomp} % provide euro and other symbols
\else % if luatex or xetex
  \usepackage{unicode-math}
  \defaultfontfeatures{Scale=MatchLowercase}
  \defaultfontfeatures[\rmfamily]{Ligatures=TeX,Scale=1}
  \setmainfont[]{Latin Modern Roman}
  \setsansfont[]{Latin Modern Roman}
\fi
% Use upquote if available, for straight quotes in verbatim environments
\IfFileExists{upquote.sty}{\usepackage{upquote}}{}
\IfFileExists{microtype.sty}{% use microtype if available
  \usepackage[]{microtype}
  \UseMicrotypeSet[protrusion]{basicmath} % disable protrusion for tt fonts
}{}
\makeatletter
\@ifundefined{KOMAClassName}{% if non-KOMA class
  \IfFileExists{parskip.sty}{%
    \usepackage{parskip}
  }{% else
    \setlength{\parindent}{0pt}
    \setlength{\parskip}{6pt plus 2pt minus 1pt}}
}{% if KOMA class
  \KOMAoptions{parskip=half}}
\makeatother
\usepackage{xcolor}
\setlength{\emergencystretch}{3em} % prevent overfull lines
\setcounter{secnumdepth}{5}
% Make \paragraph and \subparagraph free-standing
\ifx\paragraph\undefined\else
  \let\oldparagraph\paragraph
  \renewcommand{\paragraph}[1]{\oldparagraph{#1}\mbox{}}
\fi
\ifx\subparagraph\undefined\else
  \let\oldsubparagraph\subparagraph
  \renewcommand{\subparagraph}[1]{\oldsubparagraph{#1}\mbox{}}
\fi


\providecommand{\tightlist}{%
  \setlength{\itemsep}{0pt}\setlength{\parskip}{0pt}}\usepackage{longtable,booktabs,array}
\usepackage{calc} % for calculating minipage widths
% Correct order of tables after \paragraph or \subparagraph
\usepackage{etoolbox}
\makeatletter
\patchcmd\longtable{\par}{\if@noskipsec\mbox{}\fi\par}{}{}
\makeatother
% Allow footnotes in longtable head/foot
\IfFileExists{footnotehyper.sty}{\usepackage{footnotehyper}}{\usepackage{footnote}}
\makesavenoteenv{longtable}
\usepackage{graphicx}
\makeatletter
\def\maxwidth{\ifdim\Gin@nat@width>\linewidth\linewidth\else\Gin@nat@width\fi}
\def\maxheight{\ifdim\Gin@nat@height>\textheight\textheight\else\Gin@nat@height\fi}
\makeatother
% Scale images if necessary, so that they will not overflow the page
% margins by default, and it is still possible to overwrite the defaults
% using explicit options in \includegraphics[width, height, ...]{}
\setkeys{Gin}{width=\maxwidth,height=\maxheight,keepaspectratio}
% Set default figure placement to htbp
\makeatletter
\def\fps@figure{htbp}
\makeatother

\usepackage{titling}
\setlength{\droptitle}{-2cm}
\preauthor{
  \begin{center}
  \Large
  \vspace{10mm}
  by

  \vspace{20mm}
}
\postauthor{
  \end{center}
  \vfill
}

\predate{
  \begin{center}
  A thesis 
  submitted in fulfilment of the \\
  requirements of the degree of \\
  Doctor of Philosophy in Physics\\               % Degree
  School of Physical and Chemical Sciences\\          % Department
  Te Herenga Waka - Victoria University of Wellington\\                       % University 
  \vspace{5mm}
}
\postdate{
  \\
  \includegraphics[width=3in,height=1.5in]{figures/VUW-logo.png}\\
  \end{center}
  }



 
\makeatletter
\makeatother
\makeatletter
\@ifpackageloaded{bookmark}{}{\usepackage{bookmark}}
\makeatother
\makeatletter
\@ifpackageloaded{caption}{}{\usepackage{caption}}
\AtBeginDocument{%
\ifdefined\contentsname
  \renewcommand*\contentsname{Table of contents}
\else
  \newcommand\contentsname{Table of contents}
\fi
\ifdefined\listfigurename
  \renewcommand*\listfigurename{List of Figures}
\else
  \newcommand\listfigurename{List of Figures}
\fi
\ifdefined\listtablename
  \renewcommand*\listtablename{List of Tables}
\else
  \newcommand\listtablename{List of Tables}
\fi
\ifdefined\figurename
  \renewcommand*\figurename{Figure}
\else
  \newcommand\figurename{Figure}
\fi
\ifdefined\tablename
  \renewcommand*\tablename{Table}
\else
  \newcommand\tablename{Table}
\fi
}
\@ifpackageloaded{float}{}{\usepackage{float}}
\floatstyle{ruled}
\@ifundefined{c@chapter}{\newfloat{codelisting}{h}{lop}}{\newfloat{codelisting}{h}{lop}[chapter]}
\floatname{codelisting}{Listing}
\newcommand*\listoflistings{\listof{codelisting}{List of Listings}}
\makeatother
\makeatletter
\@ifpackageloaded{caption}{}{\usepackage{caption}}
\@ifpackageloaded{subcaption}{}{\usepackage{subcaption}}
\makeatother
\makeatletter
\@ifpackageloaded{tcolorbox}{}{\usepackage[many]{tcolorbox}}
\makeatother
\makeatletter
\@ifundefined{shadecolor}{\definecolor{shadecolor}{rgb}{.97, .97, .97}}
\makeatother
\makeatletter
\makeatother
\ifLuaTeX
  \usepackage{selnolig}  % disable illegal ligatures
\fi
\usepackage[citestyle = ieee]{biblatex}
\addbibresource{references.bib}
\IfFileExists{bookmark.sty}{\usepackage{bookmark}}{\usepackage{hyperref}}
\IfFileExists{xurl.sty}{\usepackage{xurl}}{} % add URL line breaks if available
\urlstyle{same} % disable monospaced font for URLs
\hypersetup{
  pdftitle={Volatile Organic Compound Detection Using Insect Odorant-Receptor Functionalised Field-Effect Transistors},
  pdfauthor={Eddyn Oswald Perkins Treacher},
  hidelinks,
  pdfcreator={LaTeX via pandoc}}

\title{Volatile Organic Compound Detection Using Insect Odorant-Receptor
Functionalised Field-Effect Transistors}
\author{Eddyn Oswald Perkins Treacher}
\date{Apr 2023}

\begin{document}
\frontmatter
\maketitle
\ifdefined\Shaded\renewenvironment{Shaded}{\begin{tcolorbox}[enhanced, borderline west={3pt}{0pt}{shadecolor}, frame hidden, sharp corners, breakable, interior hidden, boxrule=0pt]}{\end{tcolorbox}}\fi

\mainmatter
\bookmarksetup{startatroot}

\hypertarget{acknowledgements}{%
\chapter*{Acknowledgements}\label{acknowledgements}}
\addcontentsline{toc}{chapter}{Acknowledgements}

\markboth{Acknowledgements}{Acknowledgements}

Thanks for all the fish.

\bookmarksetup{startatroot}

\hypertarget{abstract}{%
\chapter*{Abstract}\label{abstract}}
\addcontentsline{toc}{chapter}{Abstract}

\markboth{Abstract}{Abstract}

This is a thesis skeleton written with quarto. Make a copy of this
thesis repo and start to write!

Make a new paragraph by leaving a blank line.

\newpage
\tableofcontents

\bookmarksetup{startatroot}

\hypertarget{introduction}{%
\chapter{Introduction}\label{introduction}}

This is a book created from markdown and executable code.

See for additional discussion of literate programming.

\begin{verbatim}
[1] 2
\end{verbatim}

\bookmarksetup{startatroot}

\hypertarget{carbon-nanotube-and-graphene-field-effect-transistors}{%
\chapter{Carbon Nanotube and Graphene Field-Effect
Transistors}\label{carbon-nanotube-and-graphene-field-effect-transistors}}

\hypertarget{device-functionalisation}{%
\section{Device Functionalisation}\label{device-functionalisation}}

\hypertarget{insect-odorant-receptors}{%
\section{Insect Odorant Receptors}\label{insect-odorant-receptors}}

\bookmarksetup{startatroot}

\hypertarget{carbon-nanotube-and-graphene-field-effect-transistors-as-biosensor-platforms}{%
\chapter{Carbon Nanotube and Graphene Field-Effect Transistors as
Biosensor
Platforms}\label{carbon-nanotube-and-graphene-field-effect-transistors-as-biosensor-platforms}}

\bookmarksetup{startatroot}

\hypertarget{fabrication}{%
\chapter{Fabrication}\label{fabrication}}

Stuff I did to get the results.

\bookmarksetup{startatroot}

\hypertarget{functionalisation-of-carbon-nanotubes-and-graphene-with-odorant-receptors}{%
\chapter{Functionalisation of Carbon Nanotubes and Graphene with Odorant
Receptors}\label{functionalisation-of-carbon-nanotubes-and-graphene-with-odorant-receptors}}

\hypertarget{linker-molecules}{%
\section{Linker molecules}\label{linker-molecules}}

\hypertarget{pyrenebutanoic-acid-n-hydroxysuccinimide-ester-pbase}{%
\subsection{1-Pyrenebutanoic acid N-hydroxysuccinimide ester
(PBASE)}\label{pyrenebutanoic-acid-n-hydroxysuccinimide-ester-pbase}}

\begin{figure}

\begin{minipage}[t]{0.50\linewidth}

{\centering 

\raisebox{-\height}{

\includegraphics{./figures/ch5/pbase_stable_1.JPG}

}

}

\subcaption{\label{fig-pbase-stable-1}Stable form 1}
\end{minipage}%
%
\begin{minipage}[t]{0.50\linewidth}

{\centering 

\raisebox{-\height}{

\includegraphics{./figures/ch5/pbase_stable_2.JPG}

}

}

\subcaption{\label{fig-pbase-stable-2}Stable form 2}
\end{minipage}%

\caption{\label{fig-pbase-structure}Stable forms of PBASE}

\end{figure}

1-Pyrenebutanoic acid N-hydroxysuccinimide ester (variously known
commercially and in the literature as 1-Pyrenebutyric acid
N-hydroxysuccinimide ester, PBASE, PBSE, PASE, Pyr-NHS, PyBASE, PANHS)
is an aromatic molecule commonly used for tethering biomolecules to the
carbon rings of graphene and carbon nanotubes. The use of this
bifunctional molecule for noncovalent functionalisation of proteins onto
a single-walled carbon nanotube was first reported in 2001 by Chen
\emph{et al.} \autocite{Chen2001}. Two methods for protein
functionalisation were successfully used, with the only differences
being the solvent used to dissolve the PBASE powder (DMF, methanol) and
the final concentration of the resulting solutions (6 mM, 1 mM
respectively). The lower concentration may have been used for PBASE in
methanol as PBASE appears to dissolve poorly in methanol at higher
concentrations. Subsequent publications appear to have largely either
chosen or adapted one of these two methods, as demonstrated by the
frequency of the use of 6 mM PBASE in DMF and 1 mM PBASE in methanol in
Table~\ref{tbl-pbase-functionalisation}. Cella \emph{et al.}, Campos
\emph{et al.}, Zheng \emph{et al.} and Ohno \emph{et al.} directly cite
Chen \emph{et al.} when discussing functionalisation with PBASE
\autocite{Cella2010,Campos2019,Zheng2016,Ohno2010}.

However, despite these various methodologies appearing to possess a
common ancestor, there is a large degree of variation in

We purchased PBASE from two suppliers, Sigma-Aldrich and Setareh
Biotech. Sigma recommends DMF and methanol as suitable solvents for
dissolving PBASE alongside chloroform and DMSO. Setareh Biotech
indicates methanol can be used for dissolving PBASE. The two suppliers
have conflicting information for suitable storage of PBASE, with Sigma
recommending room temperature storage while Setareh Biotech recommends
storage of \(-5\) to \(-30 ^\circ \text{C}\) and protection from light
and moisture.

\newpage
\KOMAoptions{paper=landscape,pagesize}

\hypertarget{tbl-pbase-functionalisation}{}
\begin{longtable}[]{@{}llllrll@{}}
\caption{\label{tbl-pbase-functionalisation}Comparison of PBASE
functionalisation processes used for immobilisation of proteins and
aptamers onto liquid-gated CNTFET and graphene FET
sensors}\tabularnewline
\toprule()
Solvent & Channel & Conc. (mM) & Incubation type & Time (hr) & Rinse
steps & References \\
\midrule()
\endfirsthead
\toprule()
Solvent & Channel & Conc. (mM) & Incubation type & Time (hr) & Rinse
steps & References \\
\midrule()
\endhead
DMF & CNTs & 5 & Immersed & 1 & PBS & Maehashi \textit{et al.}
\cite{Maehashi2007} \\
& & 6 & Immersed & 1 & DMF, PBS & García-Aljaro \textit{et al.}
\cite{Garcia-Aljaro2010} \\
& & 6 & Immersed & 1 & DMF & Chen \textit{et al.} \cite{Chen2001} \\
& & 6 & Immersed & 1 & DMF & Cella \textit{et al.} \cite{Cella2010} \\
& & 6 & Immersed & 1 & DMF & Das \textit{et al.} \cite{Das2011} \\
& Graphene & - & - & 2 & DMF & Kwong Hong Tsang \textit{et al.}
\cite{KwongHongTsang2019} \\
& & - & - & 20 & - & Wiedman \textit{et al.} \cite{Wiedman2017} \\
& & 0.2 & Immersed & 20 & DMF, IPA, DI water & Gao \textit{et al.}
\cite{Gao2018} \\
& & 1 & 100 \(\mu\)L droplet & 6 & DMF, IPA, DI water & Nekrasov
\textit{et al.} \cite{Nekrasov2021} \\
& & 5 & Immersed & 1 & DMF, DI water & Hwang \textit{et al.}
\cite{Hwang2016} \\
& & 6 & 6 \(\mu\)L droplet & 2 & DMF, DI water & Nur Nasufiya
\textit{et al.} \cite{NurNasyifa2020} \\
& & 10 & 10 \(\mu\)L droplet & 2 & DMF, DI water & Campos
\textit{et al.} \cite{Campos2019} \\
& & 10 & Immersed & 2 & DMF, PBS & Kuscu \textit{et al.}
\cite{Kuscu2020} \\
& & 10 & Immersed & 1 & DMF & Xu \textit{et al.} \cite{Xu2017} \\
& & 10 & Immersed & 12 & DMF, ethanol, DI water & Khan \textit{et al.}
\cite{Khan2020} \\
2-Methoxyethanol & Graphene & 1 & Immersed & 1 & DI water & Ono
\textit{et al.} \cite{Ono2020} \\
Methanol & CNTs & 1 & Immersed & 1 & Methanol, DI water & Zheng
\textit{et al.} \cite{Zheng2016} \\
& & 1 & Immersed & 2 & Methanol & Kim \textit{et al.} \cite{Kim2009} \\
& Graphene & 5 & Immersed & 2 & - & Sethi \textit{et al.}
\cite{Sethi2020} \\
& & 5 & Immersed & 1 & Methanol, PBS & Ohno \textit{et al.}
\cite{Ohno2010} \\
DMSO & CNTs & 10 & - & 1 & DI water & Lopez \textit{et al.}
\cite{Lopez2015} \\
& & 10 & Immersed & 1 & PBS & Strack \textit{et al.}
\cite{Strack2013} \\
\bottomrule()
\end{longtable}

\newpage
\KOMAoptions{paper=portrait,pagesize}

\bookmarksetup{startatroot}

\hypertarget{results}{%
\chapter{Results}\label{results}}

What I found out.

See for more detailed results

\bookmarksetup{startatroot}

\hypertarget{results-1}{%
\chapter{Results}\label{results-1}}

What I found out.

See for more detailed results

\bookmarksetup{startatroot}

\hypertarget{summary}{%
\chapter{Summary}\label{summary}}

In summary, this book has no content whatsoever.

\begin{verbatim}
[1] 2
\end{verbatim}

\bookmarksetup{startatroot}

\hypertarget{references}{%
\chapter*{References}\label{references}}
\addcontentsline{toc}{chapter}{References}

\markboth{References}{References}


\backmatter
\printbibliography


\end{document}
