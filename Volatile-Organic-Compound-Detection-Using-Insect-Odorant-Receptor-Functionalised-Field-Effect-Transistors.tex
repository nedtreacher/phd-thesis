% Options for packages loaded elsewhere
\PassOptionsToPackage{unicode}{hyperref}
\PassOptionsToPackage{hyphens}{url}
%
\documentclass[
  a4paper,
]{scrbook}

\usepackage{amsmath,amssymb}
\usepackage{lmodern}
\usepackage{iftex}
\ifPDFTeX
  \usepackage[T1]{fontenc}
  \usepackage[utf8]{inputenc}
  \usepackage{textcomp} % provide euro and other symbols
\else % if luatex or xetex
  \usepackage{unicode-math}
  \defaultfontfeatures{Scale=MatchLowercase}
  \defaultfontfeatures[\rmfamily]{Ligatures=TeX,Scale=1}
  \setmainfont[]{Latin Modern Roman}
  \setsansfont[]{Latin Modern Roman}
\fi
% Use upquote if available, for straight quotes in verbatim environments
\IfFileExists{upquote.sty}{\usepackage{upquote}}{}
\IfFileExists{microtype.sty}{% use microtype if available
  \usepackage[]{microtype}
  \UseMicrotypeSet[protrusion]{basicmath} % disable protrusion for tt fonts
}{}
\makeatletter
\@ifundefined{KOMAClassName}{% if non-KOMA class
  \IfFileExists{parskip.sty}{%
    \usepackage{parskip}
  }{% else
    \setlength{\parindent}{0pt}
    \setlength{\parskip}{6pt plus 2pt minus 1pt}}
}{% if KOMA class
  \KOMAoptions{parskip=half}}
\makeatother
\usepackage{xcolor}
\setlength{\emergencystretch}{3em} % prevent overfull lines
\setcounter{secnumdepth}{5}
% Make \paragraph and \subparagraph free-standing
\ifx\paragraph\undefined\else
  \let\oldparagraph\paragraph
  \renewcommand{\paragraph}[1]{\oldparagraph{#1}\mbox{}}
\fi
\ifx\subparagraph\undefined\else
  \let\oldsubparagraph\subparagraph
  \renewcommand{\subparagraph}[1]{\oldsubparagraph{#1}\mbox{}}
\fi


\providecommand{\tightlist}{%
  \setlength{\itemsep}{0pt}\setlength{\parskip}{0pt}}\usepackage{longtable,booktabs,array}
\usepackage{calc} % for calculating minipage widths
% Correct order of tables after \paragraph or \subparagraph
\usepackage{etoolbox}
\makeatletter
\patchcmd\longtable{\par}{\if@noskipsec\mbox{}\fi\par}{}{}
\makeatother
% Allow footnotes in longtable head/foot
\IfFileExists{footnotehyper.sty}{\usepackage{footnotehyper}}{\usepackage{footnote}}
\makesavenoteenv{longtable}
\usepackage{graphicx}
\makeatletter
\def\maxwidth{\ifdim\Gin@nat@width>\linewidth\linewidth\else\Gin@nat@width\fi}
\def\maxheight{\ifdim\Gin@nat@height>\textheight\textheight\else\Gin@nat@height\fi}
\makeatother
% Scale images if necessary, so that they will not overflow the page
% margins by default, and it is still possible to overwrite the defaults
% using explicit options in \includegraphics[width, height, ...]{}
\setkeys{Gin}{width=\maxwidth,height=\maxheight,keepaspectratio}
% Set default figure placement to htbp
\makeatletter
\def\fps@figure{htbp}
\makeatother

\usepackage{titling}
\setlength{\droptitle}{-2cm}
\preauthor{
  \begin{center}
  \Large
  \vspace{10mm}
  by

  \vspace{20mm}
}
\postauthor{
  \end{center}
  \vfill
}

\predate{
  \begin{center}
  A thesis 
  submitted in fulfilment of the \\
  requirements of the degree of \\
  Doctor of Philosophy in Physics\\               % Degree
  School of Physical and Chemical Sciences\\          % Department
  Te Herenga Waka - Victoria University of Wellington\\                       % University 
  \vspace{5mm}
}
\postdate{
  \\
  \includegraphics[width=3in,height=1.5in]{figures/VUW-logo.png}\\
  \end{center}
  }
\makeatletter
\makeatother
\makeatletter
\@ifpackageloaded{bookmark}{}{\usepackage{bookmark}}
\makeatother
\makeatletter
\@ifpackageloaded{caption}{}{\usepackage{caption}}
\AtBeginDocument{%
\ifdefined\contentsname
  \renewcommand*\contentsname{Table of contents}
\else
  \newcommand\contentsname{Table of contents}
\fi
\ifdefined\listfigurename
  \renewcommand*\listfigurename{List of Figures}
\else
  \newcommand\listfigurename{List of Figures}
\fi
\ifdefined\listtablename
  \renewcommand*\listtablename{List of Tables}
\else
  \newcommand\listtablename{List of Tables}
\fi
\ifdefined\figurename
  \renewcommand*\figurename{Figure}
\else
  \newcommand\figurename{Figure}
\fi
\ifdefined\tablename
  \renewcommand*\tablename{Table}
\else
  \newcommand\tablename{Table}
\fi
}
\@ifpackageloaded{float}{}{\usepackage{float}}
\floatstyle{ruled}
\@ifundefined{c@chapter}{\newfloat{codelisting}{h}{lop}}{\newfloat{codelisting}{h}{lop}[chapter]}
\floatname{codelisting}{Listing}
\newcommand*\listoflistings{\listof{codelisting}{List of Listings}}
\makeatother
\makeatletter
\@ifpackageloaded{caption}{}{\usepackage{caption}}
\@ifpackageloaded{subcaption}{}{\usepackage{subcaption}}
\makeatother
\makeatletter
\@ifpackageloaded{tcolorbox}{}{\usepackage[many]{tcolorbox}}
\makeatother
\makeatletter
\@ifundefined{shadecolor}{\definecolor{shadecolor}{rgb}{.97, .97, .97}}
\makeatother
\makeatletter
\makeatother
\ifLuaTeX
  \usepackage{selnolig}  % disable illegal ligatures
\fi
\usepackage[citestyle = ieee,urldate = iso8601]{biblatex}
\addbibresource{references.bib}
\IfFileExists{bookmark.sty}{\usepackage{bookmark}}{\usepackage{hyperref}}
\IfFileExists{xurl.sty}{\usepackage{xurl}}{} % add URL line breaks if available
\urlstyle{same} % disable monospaced font for URLs
\hypersetup{
  pdftitle={Volatile Organic Compound Detection Using Insect Odorant-Receptor Functionalised Field-Effect Transistors},
  pdfauthor={Eddyn Oswald Perkins Treacher},
  hidelinks,
  pdfcreator={LaTeX via pandoc}}

\title{Volatile Organic Compound Detection Using Insect Odorant-Receptor
Functionalised Field-Effect Transistors}
\author{Eddyn Oswald Perkins Treacher}
\date{Jun 2023}

\begin{document}
\frontmatter
\maketitle
\ifdefined\Shaded\renewenvironment{Shaded}{\begin{tcolorbox}[borderline west={3pt}{0pt}{shadecolor}, enhanced, sharp corners, frame hidden, breakable, interior hidden, boxrule=0pt]}{\end{tcolorbox}}\fi

\mainmatter
\bookmarksetup{startatroot}

\hypertarget{acknowledgements}{%
\chapter*{Acknowledgements}\label{acknowledgements}}
\addcontentsline{toc}{chapter}{Acknowledgements}

\markboth{Acknowledgements}{Acknowledgements}

Thanks for all the fish.

\bookmarksetup{startatroot}

\hypertarget{abstract}{%
\chapter*{Abstract}\label{abstract}}
\addcontentsline{toc}{chapter}{Abstract}

\markboth{Abstract}{Abstract}

This is a thesis skeleton written with quarto. Make a copy of this
thesis repo and start to write!

Make a new paragraph by leaving a blank line.

\newpage
\tableofcontents

\bookmarksetup{startatroot}

\hypertarget{introduction}{%
\chapter{Introduction}\label{introduction}}

This is a book created from markdown and executable code.

See for additional discussion of literate programming.

\begin{verbatim}
[1] 2
\end{verbatim}

\bookmarksetup{startatroot}

\hypertarget{carbon-nanotube-and-graphene-field-effect-transistors}{%
\chapter{Carbon Nanotube and Graphene Field-Effect
Transistors}\label{carbon-nanotube-and-graphene-field-effect-transistors}}

\hypertarget{device-functionalisation}{%
\section{Device Functionalisation}\label{device-functionalisation}}

\hypertarget{insect-odorant-receptors}{%
\section{Insect Odorant Receptors}\label{insect-odorant-receptors}}

\bookmarksetup{startatroot}

\hypertarget{carbon-nanotube-and-graphene-field-effect-transistors-as-biosensor-platforms}{%
\chapter{Carbon Nanotube and Graphene Field-Effect Transistors as
Biosensor
Platforms}\label{carbon-nanotube-and-graphene-field-effect-transistors-as-biosensor-platforms}}

\bookmarksetup{startatroot}

\hypertarget{sec-fabrication}{%
\chapter{Fabrication of Carbon Nanotube Network and Graphene
Field-Effect Transistors}\label{sec-fabrication}}

This chapter discusses the fabrication processes for both the carbon
nanotube network and graphene transistors. Experimental optimisation of
the transducer element is critical for biosensor work, and large numbers
of transducers were required for testing various biosensor
functionalisation processes. Therefore, these processes were developed
to rapidly fabricate devices with reproducible device characteristics
appropriate for biosensing work. Also outlined in this chapter are the
characterisation techniques taken to test the quality and
reproducibility of these fabrication processes.

The nitrogen (\(\geq\) 99.99\%) and oxygen (99.7\%) used in fabrication
work was supplied by BOC Limited New Zealand. Deionised (DI) water was
taken from a Synergy\(^\circledR\) UV Water Purification System. The DI
water had a measured conductivity of
\((1.4\pm0.1)\textrm{ } \mu \textrm{S cm}^{-1}\), compared to tap water
with a measured conductivity of
\((7.8\pm0.2)\textrm{ } \mu \textrm{S cm}^{-1}\).

\hypertarget{deposition-of-carbon-nanotubes}{%
\section{Deposition of Carbon
Nanotubes}\label{deposition-of-carbon-nanotubes}}

4-inch \(p\)-type (B-doped) silicon wafers with either a 100 nm or 300
nm SiO\(_2\) layer (WaferPro LLC) were used as the substrate for carbon
nanotube network deposition. A 100 nm SiO\(_2\) layer was the preferred
option for the devices intended for backgated measurements.

\hypertarget{solvent-based}{%
\subsection{Solvent-Based}\label{solvent-based}}

The solvent-based deposition process for the carbon nanotube network in
the second fabrication protocol is as follows. 5 \(\mu\)g of carbon
nanotube bucky paper (NanoIntegris, IsoNanotubes S-99) was dispersed in
10 mL of dichlorobenzene (Sigma Aldrich) by ultrasonication until no
particles were visible to the naked eye. The ultrasonic bath temperature
was kept constant at 25\(^\circ\)C. A 10 mg solution of
2-mercaptopyridine (99\%, Sigma-Aldrich) was dissolved in 1 ml ethanol
and drop-cast over the cleaned SiO\(_2\)/Si surface for 20 minutes,
followed by rinsing in ethanol to remove residual \(2\)-mercaptopyridine
and drying with nitrogen. The substrates were then submerged into the
CNT-DCB suspension for 2 hours, dipped into ethanol for 10 min to remove
excess solvent and any unattached carbon nanotube bundles, and then
dried with nitrogen.

\hypertarget{surfactant-based}{%
\subsection{Surfactant-Based}\label{surfactant-based}}

\hypertarget{simple-dropcasting}{%
\subsubsection*{Simple Dropcasting}\label{simple-dropcasting}}
\addcontentsline{toc}{subsubsection}{Simple Dropcasting}

\hypertarget{steam-assisted-method}{%
\subsubsection*{Steam-assisted Method}\label{steam-assisted-method}}
\addcontentsline{toc}{subsubsection}{Steam-assisted Method}

\bookmarksetup{startatroot}

\hypertarget{functionalisation-of-carbon-nanotubes-and-graphene-with-odorant-receptors}{%
\chapter{Functionalisation of Carbon Nanotubes and Graphene with Odorant
Receptors}\label{functionalisation-of-carbon-nanotubes-and-graphene-with-odorant-receptors}}

\hypertarget{linker-molecules}{%
\section{Linker molecules}\label{linker-molecules}}

\hypertarget{pyrenebutanoic-acid-n-hydroxysuccinimide-ester-pbase}{%
\subsection{1-Pyrenebutanoic acid N-hydroxysuccinimide ester
(PBASE)}\label{pyrenebutanoic-acid-n-hydroxysuccinimide-ester-pbase}}

\begin{figure}

\begin{minipage}[t]{0.47\linewidth}

{\centering 

\raisebox{-\height}{

\includegraphics{./figures/ch5/pbase_stable_1.png}

}

}

\subcaption{\label{fig-pbase-stable-1}Hartree-Fock energy: -3427728.67
kJ/mol (9 s.f.)}
\end{minipage}%
%
\begin{minipage}[t]{0.05\linewidth}

{\centering 

~

}

\end{minipage}%
%
\begin{minipage}[t]{0.47\linewidth}

{\centering 

\raisebox{-\height}{

\includegraphics{./figures/ch5/pbase_stable_2.png}

}

}

\subcaption{\label{fig-pbase-stable-2}Hartree-Fock energy: -3427729.66
kJ/mol (9 s.f.)}
\end{minipage}%

\caption{\label{fig-pbase-structure}Two conformations of PBASE molecule
with geometry optimised via \emph{ab initio} calculation (computed using
Gaussian 16 \autocite{g16}). The difference between computed
Hartree-Fock energies is 1.0 kJ/mol, small enough that the existence of
both molecular conformations is physically possible.}

\end{figure}

1-Pyrenebutanoic acid N-hydroxysuccinimide ester (variously known
commercially and in the literature as 1-Pyrenebutyric acid
N-hydroxysuccinimide ester, PBASE, PBSE, PASE, Pyr-NHS, PyBASE, PANHS)
is a aromatic, bifunctional molecule commonly used for tethering
biomolecules to the carbon rings of graphene and carbon nanotubes. The
optimised molecular structure of PBASE is shown in
Figure~\ref{fig-pbase-structure}.

The non-covalent functionalisation of proteins onto a single-walled
carbon nanotube using PBASE was first reported by Chen \emph{et al.} in
2001 \autocite{Chen2001}. Two methods for protein functionalisation and
immobilisation were successfully used, with the only differences being
the solvent used to dissolve the PBASE powder (DMF, methanol) and the
final concentration of the resulting solutions (6 mM, 1 mM
respectively). The lower concentration may have been used for PBASE in
methanol as PBASE powder appears to dissolve poorly in methanol at
higher concentrations. Cella \emph{et al.}, Campos \emph{et al.}, Zheng
\emph{et al.} and Ohno \emph{et al.} all directly cite Chen \emph{et
al.} when discussing functionalisation with PBASE
\autocite{Cella2010,Campos2019,Zheng2016,Ohno2010}. Other groups using
PBASE for graphene or carbon nanotube functionalisation do not
explicitly reference Chen \emph{et al.} in their methodology, but it is
apparent they often draw on one of these two original methods. This
common ancestry becomes apparent from the high frequency of methods
detailing the use of 6 mM PBASE in DMF and 1 mM PBASE in methanol, as
seen in Table~\ref{tbl-pbase-functionalisation}.

However, despite this shared heritage, it is also apparent from
Table~\ref{tbl-pbase-functionalisation} that there is a large degree of
variation in the methods used for PBASE functionalisation. Various
electrical characterisation, microscopy and spectroscopy techniques have
been used to demonstrate successful functionalisation. However, there
has historically been little justification provided for the exact
parameters used in the procedure. As noted by Zhen \emph{et al.} and
Hinnemo \emph{et al.}, there is more generally a lack of systematic
research into formation of pyrene-derivative monolayers on graphene and
other carbon nanomaterials, despite the wide use of this chemistry in
the literature \autocite{Zhen2018,Hinnemo2017}.

\begin{figure}

\begin{minipage}[t]{\linewidth}

{\centering 

\raisebox{-\height}{

\includegraphics{./figures/ch5/sigma_pbase_nmr.png}

}

}

\subcaption{\label{fig-sigma-nmr}Sigma PBASE in DMSO}
\end{minipage}%
\newline
\begin{minipage}[t]{\linewidth}

{\centering 

\raisebox{-\height}{

\includegraphics{./figures/ch5/setareh_pbase_nmr.png}

}

}

\subcaption{\label{fig-setareh-nmr}Setareh PBASE in DMSO}
\end{minipage}%
\newline
\begin{minipage}[t]{\linewidth}

{\centering 

\raisebox{-\height}{

\includegraphics{./figures/ch5/dmso_nmr.png}

}

}

\subcaption{\label{fig-dmso-nmr}Blank (DMSO only)}
\end{minipage}%

\caption{\label{fig-pbase-nmr}Comparison of NMR spectrum profiles
(arbitrary units)}

\end{figure}

We purchased PBASE from two suppliers, Sigma-Aldrich and Setareh
Biotech. Sigma recommends DMF and methanol as suitable solvents for
dissolving PBASE alongside chloroform and DMSO. Setareh Biotech
indicates methanol can be used for dissolving PBASE. The two suppliers
have conflicting information for suitable storage of PBASE, with Sigma
recommending room temperature storage while Setareh Biotech recommends
storage of \(-5\) to \(-30 ^\circ \text{C}\) and protection from light
and moisture. Figure~\ref{fig-pbase-nmr} compares the shapes of NMR
spectra of PBASE from each supplier dissolved in DMSO, alongside a blank
DMSO spectrum.

\newpage
\KOMAoptions{paper=landscape,pagesize}

\hypertarget{tbl-pbase-functionalisation}{}
\begin{longtable}[]{@{}llllrll@{}}
\caption{\label{tbl-pbase-functionalisation}Comparison of PBASE
functionalisation processes used for immobilisation of proteins and
aptamers onto liquid-gated CNTFET and graphene FET
sensors}\tabularnewline
\toprule()
Solvent & Channel & Conc. (mM) & Incubation type & Time (hr) & Rinse
steps & References \\
\midrule()
\endfirsthead
\toprule()
Solvent & Channel & Conc. (mM) & Incubation type & Time (hr) & Rinse
steps & References \\
\midrule()
\endhead
DMF & CNTs & 5 & Immersed & 1 & PBS & Maehashi \textit{et al.}
\cite{Maehashi2007} \\
& & 6 & Immersed & 1 & DMF, PBS & García-Aljaro \textit{et al.}
\cite{Garcia-Aljaro2010} \\
& & 6 & Immersed & 1 & DMF & Chen \textit{et al.} \cite{Chen2001} \\
& & 6 & Immersed & 1 & DMF & Cella \textit{et al.} \cite{Cella2010} \\
& & 6 & Immersed & 1 & DMF & Das \textit{et al.} \cite{Das2011} \\
& Graphene & - & - & 2 & DMF & Kwong Hong Tsang \textit{et al.}
\cite{KwongHongTsang2019} \\
& & - & - & 20 & - & Wiedman \textit{et al.} \cite{Wiedman2017} \\
& & 0.2 & Immersed & 20 & DMF, IPA, DI water & Gao \textit{et al.}
\cite{Gao2018} \\
& & 1 & 100 \(\mu\)L droplet & 6 & DMF, IPA, DI water & Nekrasov
\textit{et al.} \cite{Nekrasov2021} \\
& & 5 & Immersed & 1 & DMF, DI water & Hwang \textit{et al.}
\cite{Hwang2016} \\
& & 6 & 6 \(\mu\)L droplet & 2 & DMF, DI water & Nur Nasufiya
\textit{et al.} \cite{NurNasyifa2020} \\
& & 10 & 10 \(\mu\)L droplet & 2 & DMF, DI water & Campos
\textit{et al.} \cite{Campos2019} \\
& & 10 & Immersed & 2 & DMF, PBS & Kuscu \textit{et al.}
\cite{Kuscu2020} \\
& & 10 & Immersed & 1 & DMF & Xu \textit{et al.} \cite{Xu2017} \\
& & 10 & Immersed & 12 & DMF, ethanol, DI water & Khan \textit{et al.}
\cite{Khan2020} \\
2-Methoxyethanol & Graphene & 1 & Immersed & 1 & DI water & Ono
\textit{et al.} \cite{Ono2020} \\
Methanol & CNTs & 1 & Immersed & 1 & Methanol, DI water & Zheng
\textit{et al.} \cite{Zheng2016} \\
& & 1 & Immersed & 2 & Methanol & Kim \textit{et al.} \cite{Kim2009} \\
& Graphene & 5 & Immersed & 2 & - & Sethi \textit{et al.}
\cite{Sethi2020} \\
& & 5 & Immersed & 1 & Methanol, PBS & Ohno \textit{et al.}
\cite{Ohno2010} \\
DMSO & CNTs & 10 & - & 1 & DI water & Lopez \textit{et al.}
\cite{Lopez2015} \\
& & 10 & Immersed & 1 & PBS & Strack \textit{et al.}
\cite{Strack2013} \\
\bottomrule()
\end{longtable}

\newpage
\KOMAoptions{paper=portrait,pagesize}

\bookmarksetup{startatroot}

\hypertarget{results}{%
\chapter{Results}\label{results}}

What I found out.

See for more detailed results

\bookmarksetup{startatroot}

\hypertarget{vapour-phase-sensing-with-transistor-biosensors}{%
\chapter{Vapour Phase Sensing with Transistor
Biosensors}\label{vapour-phase-sensing-with-transistor-biosensors}}

\hypertarget{testing-vapour-delivery-system}{%
\section{Testing Vapour Delivery
System}\label{testing-vapour-delivery-system}}

\hypertarget{system-description}{%
\subsection{System Description}\label{system-description}}

\begin{figure}

{\centering \includegraphics[width=0.9\textwidth,height=\textheight]{./figures/ch7/chamber-manifold.png}

}

\caption{Vapour Delivery System - Schematic of device chamber and
manifold}

\end{figure}

\hypertarget{temperature-and-humidity-indicator}{%
\subsection{Temperature and Humidity
Indicator}\label{temperature-and-humidity-indicator}}

\hypertarget{photoionisation-detector}{%
\subsection{Photoionisation Detector}\label{photoionisation-detector}}

\hypertarget{bubbling-vapour}{%
\subsubsection*{Bubbling Vapour}\label{bubbling-vapour}}
\addcontentsline{toc}{subsubsection}{Bubbling Vapour}

First year report: ``\,````First, a 200 sccm flow of N2 gas was sent
through the dilution line to the device chamber until 1000 s. Then, the
flow controller three-way valves were manually adjusted so that the same
200 sccm flow was directed through 50 mL of EtOH analyte in the carrier
line. This continued until 2200 s, where the valves were again manually
adjusted so that 200 sccm clean N2 again flowed through the device
chamber. The resulting current across the device channel was monitored
over this time, and is shown in Figure 19. A response to EtOH exposure
and removal is visible.''``\,''

\bookmarksetup{startatroot}

\hypertarget{summary}{%
\chapter{Summary}\label{summary}}

In summary, this book has no content whatsoever.

\begin{verbatim}
[1] 2
\end{verbatim}

\appendix
\addcontentsline{toc}{part}{Appendices}

\hypertarget{sec-photolithography}{%
\chapter{Photolithography}\label{sec-photolithography}}

This section details some of the standard photolithography procedures
used in the device fabrication processes detailed in
Chapter~\ref{sec-fabrication}. Photoresists, also referred to here as
``resists'', are UV light-sensitive polymeric resins used for
photolithography. Photolithography procedures should be performed under
yellow lighting, as light wavelengths from 320-450 nm can promote photo
reactions in the photoresist used. Aging of photoresist over time can
also significantly affect the photolithography process, and therefore
all processes should be re-optimised regularly over time to give the
desired result \autocite{Microchem}. The range in processing times for
some steps of the processes used here are largely due to the effects of
aging on the photoresist.

\begin{figure}

{\centering \includegraphics{./figures/app1/positive-negative-photolithography.png}

}

\caption{\label{fig-photolithography-types}A side-view comparison of
generic photolithography processes for positive and negative resists in
the ideal case. Photolithography with a positive resist requires a
single softbake step before exposure, while for negative resists a
second baking step is required after exposure (Thicknesses shown not to
scale).}

\end{figure}

Photolithography was performed using both positive and negative
photoresists. Positive resists are made soluble in alkalines by UV light
exposure, meaning exposed areas are removed in the development process.
Conversely, negative resists are cross-linked by exposure and a
post-exposure bake step. All photolithographic exposure was performed on
a Karl Suss MJB3 Contact Aligner with a USHIO super-high pressure 350 W
mercury lamp (USH-350DS, Japan). When performing lithography, the
intensity reading from the aligner was 20.8 - 22.6 mW/cm\(^2\) (Note
however that an external photometer reading at 400 nm found an intensity
output of 17.2 mW/cm\(^2\) when the aligner read 21.0 mW/cm\(^2\)). The
unexposed areas of the negative resist are then removed in the
development process \autocite{Microchem}.
Figure~\ref{fig-photolithography-types} gives a visual representation of
these differences.

\begin{figure}

\begin{minipage}[t]{0.47\linewidth}

{\centering 

\raisebox{-\height}{

\includegraphics{./figures/app1/overcut-profile.png}

}

}

\subcaption{\label{fig-overcut-profile}Overcut profile of a positive
resist}
\end{minipage}%
%
\begin{minipage}[t]{0.05\linewidth}

{\centering 

~

}

\end{minipage}%
%
\begin{minipage}[t]{0.47\linewidth}

{\centering 

\raisebox{-\height}{

\includegraphics{./figures/app1/undercut-profile.png}

}

}

\subcaption{\label{fig-undercut-profile}Undercut profile of a negative
resist}
\end{minipage}%

\caption{\label{fig-photolithography-profiles}Two different resist
profiles seen for different types of photoresist. The undercut profile
is ideal for thin-film metal deposition and subsequent patterned
removal, known as ``lift-off''.}

\end{figure}

The specific photoresist selected for photolithography depends on the
specific use case. The types used in this thesis are positive and
negative AZ\(^\circledR\) photoresists (AZ\(^\circledR\) 1518,
Microchem, Germany; AZ\(^\circledR\) nLOF 2020, Microchem, Germany) and
solid SU-8 (GM 1060, Gersteltec, Switzerland). The AZ\(^\circledR\)
resists used here have a minimum film thickness of
\(1.5\textrm{ } \mu \textrm{m}\) \autocite{Microchem}, while the GM 1060
SU-8 has a minimum film thickness of \(5\textrm{ } \mu \textrm{m}\)
\autocite{SU8}. Positive resists which have not been thermally
crosslinked will soften at higher temperatures (\(\gtrsim 100^\circ\)C
for AZ\(^\circledR\) 1518), leading to a rounded profile. This is not
the case for negative resists, which are more thermally stable
\autocite{Microchem}. Each resist therefore has a different
cross-section profile, as shown in
Figure~\ref{fig-photolithography-profiles}.

The negative resist profile is more suited to metal or metal oxide
deposition and lift-off processes \autocite{Microchem}, though the
process is more sensitive to error due to the extrarequiring more
processing steps than positive resist. Finally, when it is suitably
processed SU-8 is considered to be more biocompatible than other
photoresists. It is especially biocompatible when chemically modified
via processes such as isopropanol sonication and O\(_2\) plasma
treatment \autocite{Chen2021}.

The step-by-step processes for each resist are detailed in the
subsequent sections.

\hypertarget{azcircledr-1518-photoresist}{%
\section{\texorpdfstring{AZ\(^\circledR\) 1518
photoresist}{AZ\^{}\textbackslash circledR 1518 photoresist}}\label{azcircledr-1518-photoresist}}

\begin{enumerate}
\def\labelenumi{\arabic{enumi}.}
\item
  Spincoat the substrate at 4000 rotations per minute (rpm) for 1 minute
  at speed, with an initial acceleration of 500 rpm/s (notes: clean the
  substrate with acetone, isopropanol (IPA) and nitrogen before
  spincoating; use only the minimum amount of photoresist required to
  fully cover the wafer surface)
\item
  Softbake 2-4 minutes at \(95^\circ\)C on the hotplate (2 min for
  individual devices, 4 min for a quarter wafer)
\item
  Mask expose for 10-12 s (note: clean mask with acetone/IPA and N\(_2\)
  dry before use)
\item
  Develop with 3 parts AZ\(^\circledR\) 326 (2.38 \% TMAH metal-ion free
  developer, Microchem, Germany) in 1 part deionised (DI) water for
  30-45 s (note: rinse for 10-15 s in one development solution, then
  perform the rest of the development in clean developer for a cleaner
  profile)
\item
  Rinse device for 30 s in DI water to remove excess developer, then dry
  under nitrogen
\end{enumerate}

\hypertarget{azcircledr-nlof-2020-photoresist}{%
\section{\texorpdfstring{AZ\(^\circledR\) nLOF 2020
photoresist}{AZ\^{}\textbackslash circledR nLOF 2020 photoresist}}\label{azcircledr-nlof-2020-photoresist}}

\begin{enumerate}
\def\labelenumi{\arabic{enumi}.}
\item
  Spincoat at 3000 rotations per minute (rpm) for 1 minute at speed,
  with an initial acceleration of 500 rpm/s (note: clean the substrate
  with acetone, isopropanol (IPA) and nitrogen before spincoating)
\item
  Softbake for \emph{precisely} 60 s at \(110^\circ\)C on the hotplate
\item
  Mask expose for 2.7-3 s (note: clean mask with acetone/IPA and N\(_2\)
  dry before use)
\item
  Post-exposure bake for \emph{precisely} 60 s at \(110^\circ\)C on the
  hotplate to cross-link exposed resist
\item
  Develop with 3 parts AZ\(^\circledR\) 326 in 1 part DI water for 60-70
  s (note: rinse for 30 s in one development solution, then perform the
  rest of the development in clean developer for a cleaner profile)
\item
  Rinse device for 30 s in DI water to remove excess developer, then dry
  under nitrogen
\end{enumerate}

\hypertarget{sec-python}{%
\chapter{Python Code for Data Analysis}\label{sec-python}}

\hypertarget{vapour-delivery-system}{%
\chapter{Vapour Delivery System}\label{vapour-delivery-system}}

\hypertarget{technical-notes}{%
\section{Technical Notes}\label{technical-notes}}

Two LabView Virtual Instruments (VIs) were adapted from pre-existing VIs
for operating the mass flow controllers and monitoring vapour flow into
the device chamber, as well as monitoring temperature and humidity in
the vapour delivery system's manifold. These VIs were named ``\,'' A
third VI was developed in parallel which combined the first two Virtual
Instruments, alongside allowing the sequence of values to control the
mass flow controllers.

From Honours report: ``\,``\,'' Figure 12 gives the right side of the
front panel of the LabView VI sample with vapour.VI, which letsus preset
an autonomously-performed vapour sensing sequence. Each row in each
array module corresponds to a differennest step in this sequence. The
`howManySteps' module lets us set how many of these steps are performed.
The `Durations Array' module determines the length of time in seconds
each step is performed over. The `Carrier Flows Array' and `Dilution
Flows Array' modules let us set the carrier flow and dilution flow,
respectively, in standard cubic centimetres per minute (sccm) through
the gas rig at each step. The carrier flow pushes analyte vapour into
the vapour-sensing device chamber, while dilution flow is used to modify
the flow behaviour of the analyte vapour entering the chamber. The
vapour sensing sequence as depicted in Figure 12 was used for all vapour
sensing runs in this investigation. At the end of the sequence, the data
collected about the vapour sensing process was saved as an .lvm file.
``\,``\,''

\hypertarget{future-improvements}{%
\section{Future Improvements}\label{future-improvements}}


\backmatter
\printbibliography


\end{document}
